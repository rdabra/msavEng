\documentclass[openleft,smallroyalvopaper,10pt,twoside,showtrims]{memoir}

	

\medievalpage[15]
\setlrmargins{*}{1.4cm}{*}
\setulmargins{*}{*}{1}
\checkandfixthelayout



\makepagestyle{myheadings}
\makeevenhead{myheadings}{{\textsc\leftmark}}{}{}
\makeoddhead{myheadings}{}{}{{\textsc\rightmark}}
\makeheadrule{myheadings}{\textwidth}{.2pt}
\makeevenfoot{myheadings}{\small{\thepage}}{}{\footnotesize\textsc{Roberto D Algarte}}
\makeoddfoot{myheadings}{\footnotesize\textsc{A Matem�tica dos Corpos Deform�veis}}{}{\small{\thepage}}

\begin{comment}
\makeatletter
\newcommand\thickhrulefill{\leavevmode \leaders \hrule height 1ex \hfill \kern \z@}
\setlength\midchapskip{10pt}
\makechapterstyle{VZ14}{
	\renewcommand\chapternamenum{}
	\renewcommand\printchaptername{}
	\renewcommand\chapnamefont{\Large\scshape}
	\renewcommand\printchapternum{%
		\chapnamefont\null\thickhrulefill\quad
		\@chapapp\space\thechapter\quad\thickhrulefill}
	\renewcommand\printchapternonum{%
		\par\thickhrulefill\par\vskip\midchapskip
		\hrule\vskip\midchapskip
	}
	\renewcommand\chaptitlefont{\Huge\scshape\centering}
	\renewcommand\afterchapternum{%
		\par\nobreak\vskip\midchapskip\hrule\vskip\midchapskip}
	\renewcommand\afterchaptertitle{%
		\par\vskip\midchapskip\hrule\nobreak\vskip\afterchapskip}
}
\makeatother
\chapterstyle{VZ14}
\end{comment}








\nouppercaseheads

\pagestyle{myheadings}

\usepackage{msav}


\newglossaryentry{eleA}{name={\ensuremath{a}},description={Element of set,}}
	
\newglossaryentry{conjA}{name={\ensuremath{A}},description={Set,}}
	
\newglossaryentry{inte}{name={\ensuremath{\mathbb{Z}}},description={Integer number,}}

\newglossaryentry{pertence}{name={\ensuremath{\in}},description={Belongs to...,}}

\newglossaryentry{notin}{name={\ensuremath{\notin}},description={Doesn't belong to...,}}

\newglossaryentry{forall}{name={\ensuremath{\forall}},description={For all...,}}

\newglossaryentry{emptyset}{name={\ensuremath{\emptyset}},description={Empty set,}}

\newglossaryentry{subset}{name={\ensuremath{\subset}},description={Proper subset of...,}}

\newglossaryentry{nsubset}{name={\ensuremath{\not\subset}},description={Not proper subset of...,}}

\newglossaryentry{subseteq}{name={\ensuremath{\subseteq}},description={Improper subset of...,}}

\newglossaryentry{nsubseteq}{name={\ensuremath{\not\subseteq}},description={Not improper subset of...,}}

\newglossaryentry{cup}{name={\ensuremath{\cup}},description={Union,}}

\newglossaryentry{bigcup}{name={\ensuremath{\bigcup_{i=1}^{n}}},description={Union of $n$ sets,}}

\newglossaryentry{cap}{name={\ensuremath{\cap}},description={Intersection,}}

\newglossaryentry{bigcap}{name={\ensuremath{\bigcap_{i=1}^{n}}},description={Intersection of $n$ sets,}}

\newglossaryentry{inter}{name={\ensuremath{\widehat{A}_3}},description={Interior of set $A_3$,}}

\newglossaryentry{wedge}{name={\ensuremath{\wedge}},description={AND...,}}

\newglossaryentry{vee}{name={\ensuremath{\vee}},description={OR...,}}

\newglossaryentry{setminus}{name={\ensuremath{\setminus}},description={Set difference,}}

\newglossaryentry{css}{name={\ensuremath{\css{C}}},description={Class,}}

\newglossaryentry{complementar}{name={\ensuremath{A'}},description={Complement of $A$,}}

\newglossaryentry{naturais}{name={\ensuremath{\mathbb{N}}},description={Natural numbers or nonnegative integers,}}

\newglossaryentry{racionais}{name={\ensuremath{\mathbb{Q}}},description={Rational numbers,}}

\newglossaryentry{times}{name={\ensuremath{\times}},description={Cartesian product,}}

\newglossaryentry{crt}{name={\ensuremath{\crt{A}{n}}},description={Cartesian product of $n$ sets $A_i$,}}

\newglossaryentry{crtEq}{name={\ensuremath{\con{A}^{n}}},description={Cartesian product of $n$ sets $A$,}}

\newglossaryentry{funcao}{name={\ensuremath{f}},description={Function $f$,}}

\newglossaryentry{valor}{name={\ensuremath{\fua{f}{d}}},description={Value of the function $f$ on $d$,}}

\newglossaryentry{dominio}{name={\ensuremath{\con{D}_{f}}},description={Domain of the function $f$,}}

\newglossaryentry{mapsto}{name={\ensuremath{\mapsto}},description={Mapped to...,}}

\newglossaryentry{real}{name={\ensuremath{\real}},description={Real numbers,}}

\newglossaryentry{imagem}{name={\ensuremath{\con{R}_{f}}},description={Image of function $f$,}}

\newglossaryentry{preimagem}{name={\ensuremath{\con{R}_B^{-1}}},description={Preimage of set $B$,}}

\newglossaryentry{implicabid}{name={\ensuremath{\Leftrightarrow}},description={Bidiretional implication,}}

\newglossaryentry{inversa}{name={\ensuremath{f^{-1}}},description={Inverse function of $f$,}}

\newglossaryentry{identidade}{name={\ensuremath{i_D}},description={Identity function with domain $D$,}}

\newglossaryentry{composto}{name={\ensuremath{f\circ g}},description={Composite function of $f$ and $g$,}}

\newglossaryentry{sum}{name={\ensuremath{\sum_{i=1}^n}},description={Sum of $n$ terms,}}

\newglossaryentry{grupo}{name={\ensuremath{G}},description={Group,}}

\newglossaryentry{inteirosPosNNulo}{name={\ensuremath{\mathbb{N}^+}},description={Natural numbers without zero,}}

\newglossaryentry{campo}{name={\ensuremath{\cam{F}}},description={Field defined by $F$,}}

\newglossaryentry{complexo}{name={\ensuremath{\complexo}},description={Complex numbers,}}

\newglossaryentry{array}{name={\ensuremath{\mat{H}}},description={Array,}}

\newglossaryentry{elArray1}{name={\ensuremath{\mat{H}_{i_1\cdots i_q}}},description={Element of array,}}

\newglossaryentry{defPor}{name={\ensuremath{:=}},description={Defined by...,}}

\newglossaryentry{mTransp}{name={\ensuremath{\mat{A}^\text{T}}},description={Transpose matrix of $\mat{A}$,}}

\newglossaryentry{compConj}{name={\ensuremath{\overline{\mat{F}_{ji}}}},description={Complex conjugate of scalar $\mat{F}_{ji}$,}}

\newglossaryentry{determ}{name={\ensuremath{\det}},description={Determinant of...,}}

\newglossaryentry{traco}{name={\ensuremath{\mathrm{tr}}},description={Trace of...,}}

\newglossaryentry{imag}{name={\ensuremath{\mathrm{i}}},description={Imaginary number,}}

\newglossaryentry{signum}{name={\ensuremath{\mathrm{sgn}}},description={Sign of real number...,}}

\newglossaryentry{adju}{name={\ensuremath{\mathrm{adj}}},description={Adjugate matrix of...,}}

\newglossaryentry{vetor}{name={\ensuremath{\vto{x}}},description={Vector x,}}

\newglossaryentry{espacVet}{name={\ensuremath{V_\cam{F}}},description={Vector space of  $V$ on $\cam{F}$,}}

\newglossaryentry{espacVetConj}{name={\ensuremath{\overline{V_\cam{F}}}},description={Conjugate vector space of  $V$ on $\cam{F}$,}}

\newglossaryentry{sconjGer}{name={\ensuremath{\spn{(U)}}},description={Subset spanned by $U$,}}

\newglossaryentry{dimen}{name={\ensuremath{\dim (V_\cam{F})}},description={Dimension of $V_\cam{F}$,}}

\newglossaryentry{realNNeg}{name={\ensuremath{\real^+}},description={Nonnegative real numbers,}}

\newglossaryentry{realPositiv}{name={\ensuremath{\real^+_*}},description={Positive real numbers,}}

\newglossaryentry{norma}{name={\ensuremath{\| \vto{x}\|}},description={Norm of vector $\vto{x}$,}}

\newglossaryentry{prdint}{name={\ensuremath{\vto{x}\cdot\vto{y}}},description={Inner product of vectors $\vto{x}$ and $\vto{y}$,}}

\newglossaryentry{perpend}{name={\ensuremath{\perp}},description={Orthogonality,}}

\newglossaryentry{unita}{name={\ensuremath{\vun{u}_1}},description={Unitary vector,}}

\newglossaryentry{espacFunc}{name={\ensuremath{V_\cam{F}^U}},description={Function space whose functions map $U$ to $V$,}}

\newglossaryentry{funcCoord}{name={\ensuremath{\vtf{f}_i^{B}}},description={$i$-th coordinate functional of basis $B$,}}

\newglossaryentry{covetor}{name={\ensuremath{\vtf{u}^*}},description={Covector of $\vto{u}$,}}

\newglossaryentry{grGene}{name={\ensuremath{\grc{G}{\cam{F}}{V}}},description={Group of unary invertible operators on domain $V_\cam{F}$,}}

\newglossaryentry{grUni}{name={\ensuremath{\grc{N}{\cam{F}}{V}}},description={Unitary group on domain $V_\cam{F}$,}}

\newglossaryentry{grUniProper}{name={\ensuremath{\grc{U^{+}}{\cam{F}}{Y}}},description={Proper unitary group on domain $Y_\cam{F}$,}}

\newglossaryentry{grUniImProp}{name={\ensuremath{\grc{U^{-}}{\cam{F}}{Y}}},description={Improper unitary group on domain $Y_\cam{F}$,}}


\newglossaryentry{grOrto}{name={\ensuremath{\grc{O}{\cam{F}}{V}}},description={Orthogonal group on domain $V_\cam{F}$,}}


\newglossaryentry{grIsom}{name={\ensuremath{\grc{I}{\cam{F}}{V}}},description={Isometry group on domain $V_\cam{F}$,}}

\newglossaryentry{evl}{name={\ensuremath{\evl{\cam{F}}{U}{V}}},description={Function space of linear functions that map $U$ to $V$,}}

\newglossaryentry{evc}{name={\ensuremath{\evc{\cam{F}}{U}{V}}},description={Function space of continuous functions that map $U_\cam{F}$ to $V_\cam{F}$,}}

\newglossaryentry{evlc}{name={\ensuremath{\evlc{\cam{F}}{Z}{Y}}},description={Function space of continuous linear functions that map $Z$ to $Y$,}}

\newglossaryentry{repVet}{name={\ensuremath{\mav{\vto{u}}{B}}},description={Matrix of vector $\vto{u}$ on basis $B$,}}

\newglossaryentry{repFun}{name={\ensuremath{\maf{\vtf{g}}{B}{C}}},description={Matrix of linear function $\vtf{g}$ on bases $B$ and $C$,}}

\newglossaryentry{transpConj}{name={\ensuremath{\mat{H}^\dagger}},description={Conjugate transpose of matrix $\mat{H}$,}}

\newglossaryentry{parteReal}{name={\ensuremath{\Re}},description={Real part of...,}}

\newglossaryentry{parteImag}{name={\ensuremath{\Im}},description={Imaginary part of...,}}

\newglossaryentry{espacoTens}{name={\ensuremath{\ete{\cam{F}}{U^{\times m}}}},description={Tensor space of order $m$ defined by $(U_i)_\cam{F}$,}}

\newglossaryentry{espacoTensDim}{name={\ensuremath{\eteg{n}{\cam{F}}{U^{\times m}}}},description={Tensor space of order $m$ where $\dim{(U_i)_\cam{F}}=n$,}}

\newglossaryentry{espacoTensConj}{name={\ensuremath{\overline{\ete{\cam{F}}{U^{\times m}}}}},description={Conjugate tensor space of $\ete{\cam{F}}{U^{\times m}}$,}}

\newglossaryentry{tensor}{name={\ensuremath{\tnr{T}}},description={Tensor,}}

\newglossaryentry{tensorConj}{name={\ensuremath{\tnr{T}^c}},description={Element of a conjugate tensor space,}}

\newglossaryentry{espacoTensTipo}{name={\ensuremath{\ete{\cam{F}}{V^{(p,q)}}}},description={Type $(p,q)$ tensor space,}}

\newglossaryentry{baseNatural}{name={\ensuremath{\{\vun{e}_1,\cdots,\vun{e}_n\}}},description={Natural basis,}}

\newglossaryentry{coordNat}{name={\ensuremath{(x_1,\cdots,x_n)}},description={Natural coordinates of vector $\vto{x}$,}}

\newglossaryentry{tensorI}{name={\ensuremath{\tnr{I}}},description={Identity tensor,}}

\newglossaryentry{elevC}{name={\ensuremath{\vtf{c}^\otimes}},description={Tensor function lifted from multiantilinear $\vtf{c}$,}}

\newglossaryentry{contrac}{name={\ensuremath{\vtf{c}^\otimes_p}},description={Contraction of order $p$ of...,}}

\newglossaryentry{tracoRS}{name={\ensuremath{\trt{r}{s}}},description={$r,s$ trace of...,}}

\newglossaryentry{vetorConj}{name={\ensuremath{\vto{v}^c}},description={Element of a conjugate vector space,}}

\newglossaryentry{contracProd}{name={\ensuremath{\tnr{A}\diamond_p\tnr{B}}},description={$p$-th order contractive product of tensors,}}

\newglossaryentry{comContracProd}{name={\ensuremath{\tnr{A}\diamond\tnr{B}}},description={Contractive product of tensors,}}

\newglossaryentry{genInnerProd}{name={\ensuremath{\tnr{A}\odot_p\tnr{B}}},description={$p$-th order partial inner product of tensors,}}

\newglossaryentry{genInnerProdVec}{name={\ensuremath{\tnr{A}\hat{\odot}_p\tnr{B}}},description={vector from a partial inner product of tensors,}}

\newglossaryentry{pCotensor}{name={\ensuremath{\vtf{T}^*_p}},description={$p$-cotensor of $\tnr{T}$,}}

\newglossaryentry{cotensor}{name={\ensuremath{\vtf{T}^*}},description={Cotensor of $\tnr{T}$,}}

\newglossaryentry{substFunc}{name={\ensuremath{\frt{T}}},description={Representative function of $\tnr{T}$,}}

\newglossaryentry{scalP}{name={\ensuremath{\mat{A:B}}},description={Scalar product of arrays $\mat{A}$ and $\mat{B}$,}}

\newglossaryentry{prodArr}{name={\ensuremath{\mat{A*_qB}}},description={$q$-product of arrays $\mat{A}$ and $\mat{B}$,}}

\newglossaryentry{eteS}{name={\ensuremath{\ets{\cam{F}}{V^m}}},description={Symmetric tensor space,}}

\newglossaryentry{eteN}{name={\ensuremath{\etn{\cam{F}}{V^m}}},description={Antisymmetric tensor space,}}

\newglossaryentry{isoTE}{name={\ensuremath{\eto{\cam{F}}{N^{\times m}}}},description={Isotropic tensor space,}}

\newglossaryentry{isoTO}{name={\ensuremath{\etom{\cam{F}}{N^{\times m}}}},description={Anti isotropic tensor space,}}

\newglossaryentry{umAum}{name={\ensuremath{\exists!}},description={There is one and only one,}}

\newglossaryentry{actVect}{name={\ensuremath{\vto{u}\oplus a}},description={Point defined by the action of vector $\vto{u}$ on point $a$,}}

\newglossaryentry{affSub}{name={\ensuremath{\eamd{U}{F}{m}}},description={$m$-dimensional affine subspace defined by vector space $U_\cam{F}$,}}

\newglossaryentry{affSubPar}{name={\ensuremath{\eamsd{S}{F}{a}{n}\parallel\eamsd{V}{F}{b}{r}}},description={Affine subspaces parallel to each other,}}

\newglossaryentry{affSubPerp}{name={\ensuremath{\eamsd{W}{F}{c}{r}\perp\eamsd{S}{F}{a}{n}}},description={Affine subspaces perpendicular to each other,}}

\newglossaryentry{coordSys}{name={\ensuremath{(o,B)}},description={Affine coordinate system defined by point $o$ and basis $B$,}}

\newglossaryentry{vecPoint}{name={\ensuremath{ \vec{v} }},description={Vector with tail at origin $o$ and head at point $v$,}}

\newglossaryentry{ctrAff}{name={\ensuremath{\mathit{f}_a}},description={Affinity $\mathit{f}$ centered at point $a$,}}

\newglossaryentry{transLat}{name={\ensuremath{\mathit{t}_\vto{u}}},description={Translation defined by $\vto{u}$,}}

\newglossaryentry{dilat}{name={\ensuremath{\mathit{s}_a}},description={Dilation centered at point $a$,}}

\newglossaryentry{crossProd}{name={\ensuremath{\vto{x}\times\vto{y}}},description={Cross product of vectors $\vto{x}$ and $\vto{y}$,}}

\newglossaryentry{reprFun}{name={\ensuremath{\tnr{\mathit{g}}}},description={Representative function of second order tensor $\tnr{G}$,}}

\newglossaryentry{tangent}{name={\ensuremath{\tnr{\psi_\mathit{d}}}},description={Tangent function to $\tnr{\psi}$,}}

\newglossaryentry{derivative}{name={\ensuremath{\tnr{\psi}'}},description={Derivative of $\tnr{\psi}$,}}

\newglossaryentry{derivativeAt}{name={\ensuremath{ \fua{D\tnr{\psi}}{\tnr{X}_0} }},description={Value of $\tnr{\psi}'(\tnr{X}_0)$ at $\tnr{X}_0$,}}

\newglossaryentry{partialDer}{name={\ensuremath{\partial_{\tnr{X}_r}\tnr{\varphi}}},description={Partial derivative of $\tnr{\psi}$ with respect to $\tnr{X}_r$,}}

\newglossaryentry{gradi}{name={\ensuremath{\nabla\tnr{\psi}}},description={Gradient of $\tnr{\psi}$,}}

\newglossaryentry{diverg}{name={\ensuremath{\dvt{\tnr{\varphi}}}},description={Divergence of $\tnr{\psi}$,}}

\newglossaryentry{laplac}{name={\ensuremath{\Delta\tnr{\psi}}},description={Laplacian of $\tnr{\psi}$,}}

\newglossaryentry{curl}{name={\ensuremath{\crl{\tnr{\varphi}}}},description={Curl of $\tnr{\hat{\psi}}$,}}

\newglossaryentry{derivHigh}{name={\ensuremath{\tnr{\psi}^{(k)}}},description={Derivative of order $k$ of $\tnr{\psi}$,}}

\newglossaryentry{vectValued}{name={\ensuremath{\tnr{\hat{\psi}}}},description={Covector valued $\tnr{\psi}$ as vector valued,}}

\newglossaryentry{gradVect}{name={\ensuremath{\fua{\grad{\tnr{\hat{\psi}}}}{\vto{x}_0}}},description={Vector correspondent to covector $\fua{\nabla\tnr{\hat{\psi}}}{\vto{x}_0}$,}}

\newglossaryentry{derivScalar}{name={\ensuremath{\dfrac{d\tnr{\hat{\psi}}}{dx}}},description={Derivative of vector valued scalar $\tnr{\hat{\psi}}$,}}

\newglossaryentry{curlVect}{name={\ensuremath{\nabla\times\tnr{\hat{\varphi}}}},description={Covector valued $\crl{\tnr{{\varphi}}}$ as vector valued,}}

\newglossaryentry{bdy}{name={\ensuremath{\mathfrak{B}}},description={Body,}}

\newglossaryentry{prd}{name={\ensuremath{\tempo}},description={Period of time,}}

\newglossaryentry{shp}{name={\ensuremath{ \mathcal{B}_t }},description={Shape of a body at instant $t$,}}

\newglossaryentry{confi}{name={\ensuremath{ \chi }},description={Configuration of a body,}}

\newglossaryentry{defInst}{name={\ensuremath{ \tnr{\chi}_t }},description={Deformation at instant $t$,}}

\newglossaryentry{defGrad}{name={\ensuremath{ \tnr{F}_{\overline{\vto{u}}} }},description={Deformation gradient at point $\overline{\vto{u}}$ and instant $t$,}}


\newglossaryentry{leftStretch}{name={\ensuremath{ \tnr{V}_{\overline{\vto{u}}} }},description={Left stretch tensor at point $\overline{\vto{u}}$,}}

\newglossaryentry{rightStretch}{name={\ensuremath{ \tnr{U}_{\overline{\vto{u}}} }},description={Right stretch tensor at point $\overline{\vto{u}}$,}}

\newglossaryentry{leftCauchy}{name={\ensuremath{ \tnr{B}_{\overline{\vto{u}}} }},description={Left Cauchy-Green tensor at point $\overline{\vto{u}}$,}}

\newglossaryentry{rightCauchy}{name={\ensuremath{ \tnr{C}_{\overline{\vto{u}}} }},description={Right Cauchy-Green tensor at point $\overline{\vto{u}}$,}}

\newglossaryentry{doyle}{name={\ensuremath{ \tnr{E}_{\tnr{S}}^{(k)} }},description={Doyle-Ericksen Tensor,}}

\newglossaryentry{inftyDef}{name={\ensuremath{  \tnr{\epsilon}_{\overline{\vto{u}}} }},description={Infinitesimal Strain Tensor,}}

\newglossaryentry{displa}{name={\ensuremath{  \tnr{\mathit{u}}_t }},description={Displacement function,}}

\newglossaryentry{antiSymPart}{name={\ensuremath{  [\tnr{W}_{\overline{\vto{u}}}] }},description={The antissymetric part of $[\fua{\nabla\tnr{\mathit{u}}_t}{\overline{\vto{u}}}]$,}}

\newglossaryentry{traject}{name={\ensuremath{  T_{\overline{\vto{u}}} }},description={Trajectory of point $\overline{\vto{u}}$,}}

\newglossaryentry{streak}{name={\ensuremath{  \fua{\tnr{p}}{\vto{x},t} }},description={Point at place $\vto{x}$ and instant $t$,}}

\newglossaryentry{streakline}{name={\ensuremath{  S_{{\vto{u}}} }},description={Streakline of place ${\vto{u}}$,}}

\newglossaryentry{eulerDesc}{name={\ensuremath{  \bullet_E }},description={Eulerian description of...,}}

\newglossaryentry{lagDesc}{name={\ensuremath{  \bullet_L }},description={Lagrangian description of...,}}

\newglossaryentry{lagTimeDeriv}{name={\ensuremath{  {\bar{\partial}}_t^k\bullet }},description={Lagrangian time derivative of order $k$ of...,}}

\newglossaryentry{eulerTimeDeriv}{name={\ensuremath{  {\tilde{\partial}}_t^k\bullet }},description={Eulerian time derivative of order $k$ of...,}}

\newglossaryentry{timeDeriv}{name={\ensuremath{  \dot{\bullet} }},description={Time derivative of...,}}

\newglossaryentry{timeDerivTwo}{name={\ensuremath{  \ddot{\bullet} }},description={Time derivative of order 2 of...,}}

\newglossaryentry{velocity}{name={\ensuremath{  {\tnr{\nu}}(\overline{\vto{x}},t) }},description={Velocity of point $\overline{\vto{x}}$ at $t$,}}

\newglossaryentry{accel}{name={\ensuremath{  {\tnr{a}}(\overline{\vto{x}},t) }},description={Acceleration of point $\overline{\vto{x}}$ at $t$,}}

\newglossaryentry{eulerVelGrad}{name={\ensuremath{   \tnr{L}_{\vto{u}} }},description={Eulerian velocity gradient at place $\vto{u}$,}}

\newglossaryentry{lagraVelGrad}{name={\ensuremath{   \tnr{M}_{\overline{\vto{u}}} }},description={Lagrangian velocity gradient at point $\overline{\vto{u}}$,}}

\newglossaryentry{strainRate}{name={\ensuremath{   \tnr{\dot{\epsilon}}_{\overline{\vto{u}}} }},description={Strain rate tensor at point $\overline{\vto{u}}$,}}


\newglossaryentry{spin}{name={\ensuremath{   \tnr{\dot{W}}_{\overline{\vto{u}}} }},description={Spin tensor at point $\overline{\vto{u}}$,}}

\newglossaryentry{vorticity}{name={\ensuremath{   \tnr{\dot{\mathit{w}}}_{{\vto{u}}} }},description={Vorticity vector at place ${\vto{u}}$,}}


\makeglossaries

\newtheorem{teo}{Teorema}
\newtheorem{coro}{Corol�rio}[teo]
\newtheorem{axi}{Axioma}
\newtheorem{prp}{Proposi��o}

\makeindex

\author{Roberto Dias Algarte}
\title{Entendendo a Teoria Matem�tica da Mec�nica dos S�lidos Deform�veis}
\date{Vers�o 1.1.1 - Outubro de 2006}



\begin{document}
	
	
%%%%%%%%%%%%%%%%%%%%% T�tulo dos Cap�tulos
\newsavebox{\ChpNumBox}
%\definecolor{ChapBlue}{rgb}{0.00,0.65,0.65}
\definecolor{ChapBlue}{rgb}{0.70,0.70,0.70}
\makeatletter
\newcommand*{\thickhrulefill}{%
	\leavevmode\leaders\hrule height 1\p@ \hfill \kern \z@}
\newcommand*\BuildChpNum[2]{%
	\begin{tabular}[t]{@{}c@{}}
		\makebox[0pt][c]{#1\strut} \\[.5ex]
		\colorbox{ChapBlue}{%
			\rule[-10em]{0pt}{0pt}%
			\rule{1ex}{0pt}\color{black}#2\strut
			\rule{1ex}{0pt}}%
\end{tabular}}
\makechapterstyle{BlueBox}{%
	\renewcommand{\chapnamefont}{\large\scshape}
	\renewcommand{\chapnumfont}{\Huge\bfseries}
	\renewcommand{\chaptitlefont}{\raggedright\Huge\bfseries}
	\setlength{\beforechapskip}{20pt}
	\setlength{\midchapskip}{26pt}
	\setlength{\afterchapskip}{40pt}
	\renewcommand{\printchaptername}{}
	\renewcommand{\chapternamenum}{}
	\renewcommand{\printchapternum}{%
		\sbox{\ChpNumBox}{%
\BuildChpNum{\chapnamefont\@chapapp}%
{\chapnumfont\thechapter}}}
\renewcommand{\printchapternonum}{%
\sbox{\ChpNumBox}{%
\BuildChpNum{\chapnamefont\vphantom{\@chapapp}}%
{\chapnumfont\hphantom{\thechapter}}}}
\renewcommand{\afterchapternum}{}
\renewcommand{\printchaptertitle}[1]{%
\usebox{\ChpNumBox}\hfill
\parbox[t]{\hsize-\wd\ChpNumBox-1em}{%
\vspace{\midchapskip}%
\thickhrulefill\par
\chaptitlefont ##1\par}}%
}
\chapterstyle{BlueBox}
%%%%%%%%%%%%%%%%%%%%% Fim T�tulo dos Cap�tulos




%%\pagenumbering{roman}
%\pagestyle{fancy}
%\renewcommand{\chaptermark}[1]{\markboth{#1}{}}
%\renewcommand{\sectionmark}[1]{\markright{\thesection\ #1}}
%\fancyhf{} % delete current setting for header and footer
%\fancyhead[LE,RO]{\bfseries\thepage}
%\fancyhead[LO]{\bfseries\rightmark}
%\fancyhead[RE]{\bfseries\leftmark}
%
%
%%COMENTAR O COMANDO ABAIXO
%%\fancyfoot[RE,RO]{\begin{scriptsize}\copyright\textbf{2004-2007 Roberto Dias
%%Algarte}\end{scriptsize}}
%
%\renewcommand{\headrulewidth}{0.5pt}
%
%% COMENTAR O COMANDO ABAIXO
%%\renewcommand{\footrulewidth}{0.5pt}
%
%\addtolength{\headheight}{0.5pt} \fancypagestyle{plain}{
%\fancyhead{}\renewcommand{\headrulewidth}{0pt} }
%
%\frontmatter%%%%%%%%%%%%%%%%%%%%%%%%%%%%%%%%%%%%%%%%%%%%%%%%%%%%%%

%\maketitle
%
%%%%%%%%%%%%%%%%%%%%%%% dedic.tex %%%%%%%%%%%%%%%%%%%%%%%%%%%%%%%%%
%
% sample dedication
%
% Use this file as a template for your own input.
%
%%%%%%%%%%%%%%%%%%%%%%%% Springer-Verlag %%%%%%%%%%%%%%%%%%%%%%%%%%

\thispagestyle{empty}
\vspace*{3.5cm}
\begin{flushright}

% write your text here
{\large �queles que dedicaram a vida aos seus: meus pais.}

\end{flushright}




%\include{pref}

\tableofcontents*


\tcblistof[\chapter*]{mteo}{Lista de Teoremas e Corol�rios}

%\chapter*{Lista de Proposi��es}
%\listtheorems{prp}

%linha de teste de retorno de vers�o


%\listoffigures

\mainmatter%%%%%%%%%%%%%%%%%%%%%%%%%%%%%%%%%%%%%%%%%%%%%%%%%%%%%%%

    %\pagenumbering{arabic}
{\let\newpage\relax \part{T�picos de �lgebra Abstrata}\label{par:matematica}}


\noindent\emph{Na
abordagem matem�tica que baseia a Teoria da Elasticidade, h� uma
grande variedade de metodologias e nota��es. Didaticamente, �
conveniente de\-fi\-nir\--se pre\-li\-mi\-nar\-men\-te, de forma
clara, os fun\-da\-men\-tos ma\-te\-m�\-ti\-cos essenciais a serem
utilizados: suas restri��es e suas formas. Para fins de clareza e
facilidade de compreens�o, esta parte centraliza todas as
defini��es e conceitos matem�ticos utilizados ao longo do texto.
Em resumo, apresenta-se os conceitos elementares da �lgebra
Abstrata, os fundamentos da �lgebra Linear e aspectos geom�tricos
envolvidos, �lgebra e C�lculo Tensorial. O escopo e o n�vel de
profundidade dos temas tratados abrange o necess�rio. Os t�picos
apresentados podem ser aprofundados consultando-se a bibliografia
utilizada, apresentada ao final desta parte. Ao leitor j�
familiarizado com os temas tratados, � recomend�vel passar
rapidamente pelos cap�tulos, ficando a par do enfoque e nota��o utilizados.}


    \chapter{O Que � �lgebra?}
\begin{comment}
An algebraic system can be described as a set of objects together with some operations for combining them.

Algebra is a branch of mathematics dealing with symbols and the rules for manipulating those symbols.

Algebra is a branch of mathematics that deals with properties of operations and the structures these operations are defined on

One use of the word "algebra" is the abstract study of number systems and operations within them, including such advanced topics as groups, rings, invariant theory, and cohomology. This is the meaning mathematicians associate with the word "algebra." When there is the possibility of confusion, this field of mathematics is often referred to as abstract algebra.

Abstrair � condi��o necess�ria para se generalizar? Se for ent�o a �lgebra � o resultado da abstra��o de sistemas constitu�dos por s�mbolos e suas cole��es, pelas inter-rela��es desses s�mbolos e pelas regras que as regem.
\end{comment}
Em algum momento ao longo do desenvolvimento de sua vital capacidade para a comunica��o e tamb�m para a compreens�o do mundo, o ser humano precisou transmitir, numa linguagem menos subjetiva que objetiva, as impress�es sensoriais percebidas a partir dos objetos que o cercavam, de tal forma que essa transmiss�o ocorresse \emph{sem dubiedades}. As duas primeiras a��es humanas objeto dessa necessidade foram sucessivamente o contar e o medir. Nesses tempos primitivos, a Matem�tica surge como resultado do esfor�o para se codificar quantidades e tamanhos numa sem�ntica pr�pria, onde as interpreta��es subjetivas fossem minimizadas ou simplesmente suprimidas.

\section{Generaliza��o}\index{generaliza��o}

A crescente sofistica��o das demandas trouxe complexidade �s ques\-t�es afeitas � Matem�tica e a busca pela generaliza��o surgiu naturalmente, como uma estrat�gia para expandir sua aplicabilidade, para simplificar suas descri��es e m�todos de solu��o. Se h� estruturas que conceitualmente se repetem no tratamento espec�fico de diversos problemas, a generaliza��o dessas estruturas torn�-las-�o aplic�veis a novos problemas, � constru��o de novas estruturas ou � simplifica��o da complexidade em outras j� existentes.

No presente contexto, generalizar envolve tr�s a��es distintas: a) abstrair, quando se extrai mentalmente da estrutura global uma parcela ou subestrutura de interesse, apartando-a das demais, tornando-a independente; b) analisar, quan\-do se decomp�e a abstra��o para melhor com\-pre\-en\-d�-la; c) conceituar, quando se criam novos conceitos a partir da abstra��o analisada. O produto ou o resultado do processo de generaliza��o � aquilo que se conhece por \textsb{abstrato}\index{abstrato}; e � a produ��o desses entes abstratos -- construtos mentais rigorosamente concebidos -- que permite � Matem�tica passar a tratar com menos estruturas o concreto antes descrito por diversas estruturas espec�ficas, descrever por meio de concep��es mais simples abordagens antes complexas. A infinita necessidade do homem por conhecimento e a pr�tica incessante de generaliza��es sobre generaliza��es, disso resultando um abstrato cada vez mais abrangente e est�vel, conferem � Matem�tica seu forte car�ter psicol�gico, quando a consideramos uma express�o observ�vel das manifesta��es mais profundas do psiquismo humano. Sob esse ponto de vista, a Matem�tica, tal qual a pintura e a m�sica, � parte integrante da natureza humana; ou seja, � \emph{arte}.

Nos termos da abordagem exposta, \textsb{�lgebra}\index{�lgebra} � o campo da Matem�tica que envolve \emph{generaliza��es de estruturas constitu�das por s�mbolos e suas cole��es, pelas rela��es que envolvem ambos e pelas restri��es que as regem}. Como exemplo, letras representando n�meros reais, os conjuntos formados por elas, as fun��es que as t�m como argumentos e as regras expressas nessas fun��es s�o respectivamente s�mbolos, cole��es, rela��es e restri��es que fazem parte do objeto de estudo da �lgebra. Trata-se portanto de um campo cujo alcance � bastante amplo, a tal ponto que a generalidade de suas estruturas permitiram que a �lgebra aproximasse �reas da Matem�tica aparentemente distantes. Por este aspecto agregador e que perpassa diferentes ramos de estudo, n�o seria incorreto falar da �lgebra como uma �rea fundamental da Matem�tica, tanto em termos de import�ncia te�rica quanto da constru��o do conhecimento matem�tico.

Como resultado dessa influ�ncia em outros ramos e tamb�m da particulariza��o did�tica de seus conceitos, a �lgebra possui diversas subdivis�es. Dentre elas, as mais importantes s�o a \textsb{�lgebra elementar}\index{�lgebra!elementar} e a \textsb{�lgebra abstrata}\index{�lgebra!abstrata}: a primeira diz respeito ao n�vel mais b�sico de generaliza��o da \textsb{Aritm�tica}\index{Aritm�tica} e a segunda alcan�a n�veis profundos desta generaliza��o. Embora consagrada, a inadequada classifica��o ``abstrata'' para essa �lgebra mais profunda � um pleonasmo que visa a diferencia��o did�tica; a �lgebra como um ramo matem�tico cuida de entes abstratos. Para os objetivos deste livro, tamb�m ser� estudada a chamada \textsb{�lgebra linear}\index{�lgebra!linear}, que trata de vetores (s�mbolos), espa�os vetoriais (cole��es) e fun��es lineares (rela��es e regras).



\section{Prim�rdios}

O primeiro registro conhecido que se aproxima do pensamento alg�brico atual foi escrito pelo matem�tico grego Diofanto de Alexandria no terceiro s�culo depois de Cristo. De sua obra denominada \emph{Arithmetica}, composta originalmente por diversos livros, restaram apenas 189 problemas, expressos numa nota��o pr�pria, muito similar �quela utilizada nos dias de hoje para descrever equa��es: as inc�gnitas representadas em s�mbolos n�o num�ricos com as opera��es separadas por uma igualdade. A equa��o que se expressa atualmente por
\begin{equation*}
x^3-2x^2+10x-1=5
\end{equation*}
Diofanto a escreveu da seguinte forma\footnote{Ver \aut{Derbyshire}\cite{derbyshire_2006_1}.}:
\begin{equation*}
K^{Y}\overline{\alpha}\varsigma\overline{\iota}\rotpsi\Delta^Y\overline{\beta}\textbf{M}\overline{\alpha}'\text{�}\sigma\textbf{M}\overline{\varepsilon}\,, 	
\end{equation*}
onde os s�mbolos com barras superiores s�o constantes num�ricas, $'\text{�}\sigma$ representa ``igual'', $\rotpsi$ � o s�mbolo para diferen�a e os demais dizem respeito � inc�gnita $\varsigma$. Por conta de todo essa simbologia literal -- at� ent�o in�dita segundo os registros conhecidos -- na proposi��o e solu��o dos problemas apresentados, muitos historiadores da Matem�tica consideram Diofanto o pai da �lgebra. Outros tantos argumentam que Diofanto n�o apresentou evolu��o metodol�gica alguma nas solu��es dos problemas que prop�s: cada uma delas � aplic�vel apenas individualmente, v�lida somente para cada caso particular. N�o se observa na obra um esfor�o de generaliza��o para criar procedimentos de solu��o extens�veis a diversos problemas de diferentes naturezas.

Seiscentos anos ap�s a \emph{Arithmetica} de Diofanto, por volta de 820 d.C., o pol�mata persa Ab\=u 'Abd Muhammad Ibn M\=us\=a al-Khw\=arizm\={\i}\footnote{Segundo \aut{Knuth}\cite{knuth_1997_1}, o nome significa ``\textit{Pai de Abdullah, Mohammad, filho de Mois�s, nativo de Khw\=arizm\={\i}}'', regi�o ao sul do Mar de Aral.} (780-850), membro da renomada Casa da Sabedoria em Bagd�, escreveu \emph{Al-kit\=ab al-mukhtasar f\=i his\=ab al-\v gabr wa'l-muq\=abala}, ou \emph{Manual de ``al-jabr'' e de ``al-muqabala''} numa tradu��o mais literal. N�o h� palavras em portugu�s que expressem de maneira precisa as duas �rabes transliteradas e colocadas entre aspas, cujo conceito pode ser apreendido segundo a metodologia proposta pelo autor. O livro � composto por tr�s partes, sendo que a primeira trata da solu��o de equa��es lineares e quadr�ticas redut�veis a um dos seis tipos apresentados a seguir.

\begin{itemize}
\setlength\itemsep{1pt}
\item[1.] Quadrados igualam ra�zes: $ax^2=bx$;
\item[2.] Quadrados igualam n�meros: $ax^2=c$;
\item[3.] Ra�zes igualam n�meros: $bx=c$;
\item[4.] Quadrados e ra�zes igualam n�meros: $ax^2+bx=c$;
\item[5.] Quadrados e n�meros igualam ra�zes: $ax^2+c=bx$;
\item[6.] Ra�zes e n�meros igualam quadrados: $bx+c=ax^2$.
\end{itemize}

Esses problemas gen�ricos, al-Khw\=arizm\={\i} n�o os descreveu seguindo uma nota��o simb�lica, como Diofanto, mas literal, na forma das descri��es que iniciam os seis itens apresentados. Para cada um deles, o autor criou m�todos de solu��o literais, aplic�veis a qualquer problema espec�fico, desde que fosse poss�vel reduzi-lo numa das seis formas. Para tal, dois procedimentos foram propostos conforme se segue.

\begin{itemize}
\setlength\itemsep{1pt}	
\item[a)] ``al-jabr'' corresponde a a��es de adicionar ao lado que cont�m subtra��o um valor que ``restaure'' o termo subtra�do e de balancear a equa��o acrescendo o outro lado dessa mesma quantidade. A palavra ``al-jabr'' � a origem etimol�gica da palavra ``�lgebra''\footnote{Em \aut{Cervantes}\cite{cervantes_2010}, h� uma interessante passagem � p. 476: \textit{En esto fueron razonando los dos,  hasta que llegaon a un pueblo donde fue ventura hallar un algebrista, con quien se cur�...}. Na cita��o, Dom Quixote e seu fiel escudeiro t�m a sorte de encontrar um ``algebrista'', ou um curandeiro especialista em restaurar ossos quebrados ou deslocados. Conv�m ressaltar que a l�ngua espanhola foi fortemente influenciada pelas invas�es dos mouros ao sul da Espanha. Nessa mesma obra, o autor afirma que toda a palavra espanhola iniciada por ``al'' tem origem �rabe.}, esta constru�da a partir da pron�ncia daquela. Em nota��o simb�lica, eis o exemplo apresentado por al-Khw\=arizm\={\i}:
\begin{eqnarray*}
x^2&=&40x-4x^2\\
5x^2&=&40x\,.
\end{eqnarray*}
\item[b)] ``al-muqabala'' consiste em subtrair ambos os lados por um valor que elimine algum dos termos. O exemplo apresentado pelo autor �:
\begin{eqnarray*}
	50+x^2&=&29+10x\\
	21+x^2&=&10x\,.
\end{eqnarray*}
\end{itemize}

A tradu��o do manual de al-Khw\=arizm\={\i} para o latim em 1145 muito contribuiu para a incorpora��o pelo Ocidente dos diversos elementos da matem�tica �ra\-be. A partir dessas duas obras citadas, que essencialmente estudam equa��es e seus m�todos de solu��o, inicia-se o desenvolvimento daquilo que hoje denominamos �lgebra. 
    \chapter{Collections and Relationships}


It is at least innocuous to study a fundamental object that is completely isolated, without getting it together with other objects. Resulting or not from some selection criteria, this gathering of objects that defines a scope of study, a comprehensiveness of analysis, we shall call it a collection\index{collection}. Thereby, between these collected objects, called  \textsb{elements}\index{elements}, it is possible to establish relationships; even when these elements belong to different collections. A \textsb{rule}\index{rule} expresses  generically how such relationship\index{relationship} must occur and, because of this generic character, we describe it, together with collections and elements, in an algebraic approach. Thus, before developing the main subject of this chapter, let's start with some basic remarks about Algebra.



\section{What is Algebra?}

At some moment during the development of our vital urge for communication and our capacity to apprehend the world around, we humans felt the need to transmit less subjectively, \emph{without dubieties}, some of the sense impressions captured from the physical environment. Chronologically speaking, the first two human actions that emerged out of this necessity were to count and then to measure. In these ancient times, the beautiful art of Mathematics arose as a result of the effort to codify quantities and sizes in a particular symbology, where subjective interpretations were minimized or simply suppressed.

The ever growing sophistication of life demands brought complexity to the problems of
Mathematics and the need for generalization emerged naturally, as a strategy to broaden its applicability, to simplify its description and solution methods. If there are conceptual structures that occur repeatedly on the treatment of different problems, then the \textsb{generalization}\index{generalization} of theses structures can make them applicable to new problems, to creating new structures or simplifying others. In this context, the act of generalizing encompasses three distinct actions: a) to abstract, when we mentally extract from the global structure some part or substructure of interest, in order to concentrate only on it; b) to analyze, when we decompose the abstraction to understand it better; c) to conceptualize, when we create new concepts from analyzing the abstraction. The product or result of the generalization process is called the \textsb{abstract}\index{abstract}, a noun; and it is precisely the abstract -- as a rigorously conceived mental construct -- that allows Mathematics to describe the concrete with fewer, less complex, structures. The human endless quest for knowledge and the unending practice of generalization over generalization, resulting in an increasingly stable and generic abstract, bestow upon Mathematics a strong psychological character, if we consider it as an observable expression of the deepest manifestations of the human mind. From this point of view, Mathematics, such as Painting or Music, is an inherent part of human nature; in other words, it is undoubtedly \emph{art}.

From the ideas already exposed, we state that \textsb{Algebra}\index{Algebra} is the branch if Mathematics that deals with \emph{generalizations of structures formed by symbols and their collections, by relationships between these symbols and by restrictions governing these relationships}. As an example, letters representing real numbers, sets of these letters, functions that have these letters as arguments and the rules expressing these functions are respectively the symbols, collections, relationships and restrictions that constitute the fundamental objects of Algebra. It is surely a wide field of study, so wide that Algebra, with its deep generic structures, approximated other branches of Mathematics seemingly distant from each other. This aggregative aspect that pervades different areas of study allows us to regard Algebra as a fundamental mathematical branch, one of its pillars, both because of its theoretical significance and also for enabling mathematical knowledge as a whole.


As a consequence of its influence in other branches and also of didactic particularizations, Algebra has many divisions. Among them, the most important are  \textsb{elementary algebra}\index{algebra!elementary} and \textsb{abstract algebra}\index{algebra!abstract}: the former deals with the lowest generalization level of the \textsb{Arithmetics}\index{Arithmetics} and the latter reaches deeper and wider generalizations. As Mathematics is the art of abstraction, then the adjective in ``abstract algebra'' is indeed a pleonasm, in order to emphasize its non-numeric, non-specific symbolic character. In this work, we shall mostly study \textsb{linear algebra}\index{algebra!linear}, a subdivision of abstract algebra that deals with vectors (symbols), vector spaces (collections) and linear functions (relationships and rules).

Now, let's talk a little about historical matters. The first known record closest to the current algebraic thought was written by the greek mathematician Diophantus of Alexandria on the third century A.D. From this work, entitled \emph{Arithmetica}, composed originally of many books, only 189 problems remain, all expressed in a specific notation, very similar to the current practice of writing equations: the unknowns represented by non numeric symbols with an equality separating the operations. The equation currently expressed by
\begin{equation*}
x^3-2x^2+10x-1=5
\end{equation*}
Diophantus wrote it the following way\footnote{See \aut{Derbyshire}\cite{derbyshire_2006_1}.}:
\begin{equation*}
K^{Y}\overline{\alpha}\varsigma\overline{\iota}\rotpsi\Delta^Y\overline{\beta}\textbf{M}\overline{\alpha}'\text{�}\sigma\textbf{M}\overline{\varepsilon}\,,
\end{equation*}
where the symbols with overbars are numerical constants, $'\text{�}\sigma$ means ``equals to'', $\rotpsi$ represents difference an the other symbols are related to the unknown $\varsigma$. Because of this symbolic approach -- unprecedented until that time, according to known records -- in handling problems, some historians consider Diophantus the father of Algebra. Many others argue that the work of Diophantus did not bring methodological evolution to solving the purposed problems: every solution is applicable specifically, valid only for each particular case. There is no effort for generalization, for creating solution procedures extensible to different problems.


Six hundred years after the \emph{Arithmetica} of Diophantus, around 820 A.D., the persian polymath Ab\=u 'Abd Muhammad Ibn M\=us\=a al-Khw\=arizm\={\i}\footnote{According to  \aut{Knuth}\cite{knuth_1997_1}, the name means ``\textit{Father of Abdullah, Mohammad, son of Moses, native from Khw\=arizm\={\i}}'', southern region of the Sea of Aral.} (780-850), who was member of the famous House of Wisdom in Baghdad, wrote \emph{Al-kit\=ab al-mukhtasar f\=i his\=ab al-\v gabr wa'l-muq\=abala}, or \emph{Handbook of ``al-jabr'' and of ``al-mu\-qa\-ba\-la''} in a literal translation. There are no words in english language that express accurately the two transliterated arab words in quotation marks, whose meaning can be understood from the methodology proposed by the author. The book has three parts, with the first one devoted to solving quadratic and linear equations, reducible to one of the six following types.

\begin{itemize}
	\setlength\itemsep{1pt}
	\item[1.] Squares equal roots: $ax^2=bx$;
	\item[2.] Squares equal numbers: $ax^2=c$;
	\item[3.] Roots equal numbers: $bx=c$;
	\item[4.] Squares and roots equal numbers: $ax^2+bx=c$;
	\item[5.] Squares and numbers equal roots: $ax^2+c=bx$;
	\item[6.] Roots and numbers equal squares: $bx+c=ax^2$.
\end{itemize}

These generic problems, al-Khw\=arizm\={\i} did not solve them using a symbolic notation, as Diophantus did, but in literal terms, just like the descriptions in the six items. To each of these problems, the author created literal solution methods, applicable to any specific problem reducible to one of the six types. In order to do this, he proposed two procedures, presented as follows.

\begin{itemize}
	\setlength\itemsep{1pt}	
	\item[a)] ``al-jabr'' involves the acts of adding to the side where there is a subtraction a value that ``restores'' the subtracted term and of balancing the equation by adding this same value to the other side. The word ``al-jabr'' is the etymological ancestor of the word ``algebra''\footnote{In \aut{Cervantes}\cite{cervantes_2010}, there's an interesting passage at p. 476: \textit{En esto fueron razonando los dos,  hasta que llegaon a un pueblo donde fue ventura hallar un algebrista, con quien se cur�...}. The quote says that Don Quixote and his faithful esquire are lucky to find an algebraist, who was a healer that restored displaced bones. It should be said that the spanish language was strongly influenced by the successive arab invasions coming from the south. In this same novel, Cervantes also states that every spanish word started by ``al'' has an arabian source.}, the latter being constructed from pronouncing the former. Using a symbolic notation, this is the example presented by al-Khw\=arizm\={\i} in his handbook:
	\begin{eqnarray*}
		x^2&=&40x-4x^2\\
		5x^2&=&40x\,.
	\end{eqnarray*}
	\item[b)] ``al-muqabala'' means subtracting both sides by a value that eliminates one of the terms. Here is an example of this procedure from the handbook:
	\begin{eqnarray*}
		50+x^2&=&29+10x\\
		21+x^2&=&10x\,.
	\end{eqnarray*}
\end{itemize}

Translations to the latin language of the al-Khw\=arizm\={\i} work in 1145 helped to incorporate the arab mathematics in the western thought. The works of Diophantus and al-Khw\=arizm\={\i}, dealing essentially with the solution of equations, are the two most relevant origins of what today we call Algebra.


\section{Sets}

In conceptual terms, the least restricted algebraic collection is called set\index{set}, which can have a finite or infinite number of distinct elements, or none\footnote{This is not the approach of the Axiomatic Set Theory. See \aut{Cameron}\cite{cameron_1999_1}.}. Therefore, from the finite collection $a,b,b,a,c$, we can define a finite set $\lch \gloref{eleA},b,c \rch$ of three distinct\footnote{A set does not admit repeated elements. See \aut{Shen \& Vereshchagin}\cite{shen_2002_1}.} elements. In concrete examples, this distinction -- that can be done abstractly by labeling objects with letters -- depends on the element characteristics elected to distinguish each other, as in figure \ref{fg:conjunto}. It is important to say that the descriptive sequence of its elements does not alter the definition of a set: for example, definitions $\gloref{conjA}\gloref{defPor}\lch \ele{a},\ele{b}\rch$ and $\con{A}:=\lch \ele{b},\ele{a}\rch$ are identical. Speaking of comparisons, the sets $\con{A}$ and $\con{B}$ are said to be equal, $\con{A}=\con{B}$, if they have the same elements; otherwise, they are different: $\con{A}\neq\con{B}$.

\begin{figure}[!ht]
	\centering
	\begin{center}
		\scalebox{.85}{\input{partes/parte1/figs/c_algabst/conjunto.pstex_t}}
	\end{center}
	\titfigura{If the selection criteria was ``\texttt{triangular plates with heights between 5cm and 15cm}'', then $\varDelta$ is a mathematical collection if equality is based only on the feature ``number of sides''; additionally, if it is based on ``height'', $\varDelta$ is still a collection; but, if it is based also on ``fill color'', then $\varDelta$ is a mathematical set.}\label{fg:conjunto}
\end{figure}

The \textsb{empty set}\index{set!empty} $\gloref{emptyset}$ has no elements and it enforces the idea of set as a restricted collection that can be build from some selection criteria: the empty set may be the result of a criteria that no object obeyed. For example, if the selection criteria is ``\texttt{prime even numbers different from two}'', the result will be an empty set. In set theory, this ``selection criteria'' is called \textsb{specification}\index{set!specification of}, whose mathematical syntax is the following:
\begin{equation}
\texttt{\emph{set} }:=\,\,\,\lch  \texttt{ \emph{selection} } : \texttt{ \emph{criteria} }  \rch\,.
\end{equation}
From this syntax pattern and considering $\ele{x}$ a representation for an arbitrary integer,
\begin{equation}
\con{E}:=\lch x\gloref{pertence} \gloref{inte}: x \bmod 2 = 0 \rch
\end{equation}
is the set of even numbers. This specification reads ``the set $\con{E}$ defined by every element of the set of integers whose division by two has a zero remainder��.


By the intuitive sense of belonging, the most basic relationship between an object and a set determines whether the former is element of the latter or not. This idea has a fundamental importance in the so called Naive Set Theory, which we adopt here, following \aut{Halmos}\cite{halmos_19742_1}. In mathematical terms, if $\ele{a}$ is element of the set $\con{A}$, we say that it \textsb{belongs}\index{belongs} to the set or that $\ele{a}\in\con{A}\,$; otherwise, it doesn't belong to the set: $\ele{a}\gloref{notin}\con{A}\,$.
When all the elements of a set $\con{A}_1$ belong to the set $\con{A}$, we say that $\con{A}_1$ is a \textsb{subset}\index{subset} of $\con{A}$. If this is the case, when the equality $\con{A}_1=\con{A}$ is not admissible, $\con{A}_1$ is called a \textsb{proper subset}\index{subset!proper} of $\con{A}$, written also as $\con{A}_1\gloref{subset}\con{A}$; when it is admissible, $\con{A}_1$ is an \textsb{improper subset}\index{subset!improper} of $\con{A}$, or $\con{A}_1\gloref{subseteq}\con{A}$, from which we can state that every set is an improper subset of itself. The set $\emptyset$ would not be a subset of an arbitrary set $\con{A}$ if it had some element not belonging to $\con{A}$; but this is impossible because $\emptyset$ has no elements, and then we can state that every set has an empty subset.

The concept of belonging just presented can also be used to create sets. A \textsb{union}\index{set!union} of the sets  $\con{A}_1$ and $\con{A}_2$ is the set $\con{A}_1\gloref{cup}\con{A}_2$ to which all the elements of  $\con{A}_1$ and $\con{A}_2$ belong. The compact representation $\gloref{bigcup}\con{A}_i$ is the union of the $n$ sets $\con{A}_i$. A set of elements that belong both to $\con{A}_1$ and to $\con{A}_2$ is the \textsb{intersection}\index{set!intersection} $\con{A}_1\gloref{cap}\con{A}_2$. In other words, if the element $x \in \con{A}_1\cap\con{A}_2$ then $\lpa x \in \con{A}_1\rpa \gloref{wedge} \lpa x \in \con{A}_2\rpa$, where $\wedge$ means ``AND'' in english. Similarly to union, $\gloref{bigcap}\con{A}_i$ is how we write the intersection of $n$ sets $\con{A}_i$. By using the so called Venn diagrams (figure \ref{fg:interDistrib}), it can be verified that intersection is distributive in union, that is,
\begin{equation}
A_1\cap\lpa A_2\cup A_3\rpa=\lpa A_1\cap A_2\rpa\cup\lpa A_1\cap A_3\rpa.
\end{equation}
When a set $\con{A}_1\cap\con{A}_2$ is empty, we say that $\con{A}_1$ and $\con{A}_2$ are \textsb{disjoint}\index{sets!disjoint}. In this case, if $x\in\con{A}_1\cup\con{A}_2$ then $(x\in\con{A}_1)\gloref{vee} (x\in\con{A}_2)$, where the symbol $\vee$ means ``OR'' literally.
\begin{figure}[!ht]
\centering
\begin{center}
\scalebox{.72}{\input{partes/parte1/figs/c_algabst/interseccao.pstex_t}}
\end{center}
\titfigura{Intersection is distributive in union.}\label{fg:interDistrib}
\end{figure}

The \textsb{difference set}\index{set!difference} $\con{A}_1\gloref{setminus}\con{A}_2$ is the set of elements of $\con{A}_1$ not belonging to $\con{A}_2$. From this definition, if $\con{A}_1$ is subset of $\con{A}$, the set $\gloref{complementar}_1:=\con{A}\setminus\con{A}_1$ is called the \textsb{complement}\index{set!complement} of $\con{A}_1$ in $\con{A}$. Thereby, if $A_1$ and $A_2$ are subsets of $A$, it is possible to verify, also by Veen diagrams, that the union complement $\lpa A_1\cup A_2\rpa' = A_1'\cap A_2'$, that the intersection complement $\lpa A_1\cap A_2\rpa' = A_1'\cup A_2'$ and that the difference $A_1\setminus A_2= A_1\cap A_2'$. From these three equalities, we can say that in union the difference is distributed according to
\begin{equation}\label{eq:distribUniao}
A_1\setminus\lpa A_2\cup A_3\rpa=\lpa A_1\setminus A_2\rpa\cap \lpa A_1\setminus A_3\rpa\,,
\end{equation}
and in intersection according to
\begin{equation}\label{eq:distriInters}
A_1\setminus\lpa A_2\cap A_3\rpa=\lpa A_1\setminus A_2\rpa\cup \lpa A_1\setminus A_3\rpa\,.
\end{equation}
{\footnotesize
\begin{proof}
Let the sets $A_1,A_2,A_3$ be propers subsets of $A$. Therefore, we have $A_1\setminus\lpa A_2\cup A_3\rpa=A_1\cap\lpa A_2\cup A_3\rpa'=A_1\cap A_2'\cap A_3'$. We also have $\lpa A_1\setminus A_2\rpa\cap \lpa A_1\setminus A_3\rpa=A_1\cap A_2'\cap A_1\cap A_3'=A_1\cap A_2'\cap A_3'$. The equality \eqref{eq:distribUniao} is thus verified. In order to prove \eqref{eq:distriInters}, here is the following:
\begin{align}
A_1\setminus\lpa A_2\cap A_3\rpa&= A_1\cap\lpa A_2\cap A_3\rpa'\nonumber\\
&= A_1\cap\lpa A_2'\cup A_3'\rpa\nonumber\\
&= \lpa A_1\cap A_2' \rpa \cup \lpa A_1\cap A_3'\rpa\nonumber\\
&= \lpa A_1\setminus A_2\rpa\cup \lpa A_1\setminus A_3\rpa.\nonumber
\end{align}
\end{proof}}


\section{Sequences}\index{sequence}


In Algebra, a collection is called sequence when its elements need to be ordered. For this ordering, each element of the sequence has a position identified by an ordinal number\footnote{Ordinals are integer numbers, elements of $\gloref{naturais}$, used to label positions sequentially.}, called \textsb{index}\index{element!index of}, which grows from left to right on the sequence notation $\lpa a,b,c\rpa$. Thereby, every element of a sequence has a unique position, labeled by an index; and from this we conclude that two sequences are equal if and only if they have the same elements equally indexed. For example, the sequence $\lpa a,b,c\rpa\neq\lpa a,c,b\rpa$ because the elements $b$ and $c$ have different indexes in each of the sequences. Differently from sets, the positional restriction of sequences does not forbid indistinct elements: the collection $a,b,c,a$ is valid as a sequence $\lpa a,b,c,a\rpa$ since the two $a$ elements have different indexes. When a sequence is finite, it is called a \textsb{tuple}\index{tuple}. The tuple that has one element is a \textsb{monad}\index{monad}; two elements, a \textsb{double}\index{double}; three elements, a \textsb{triple}\index{triple}; four elements, a \textsb{quadruple}\index{quadruple}; $n$ elements, a $n$\textsb{-tuple}\index{$n$-tuple}, where $n\in\mathbb{N}$. There is also an \textsb{empty sequence}\index{sequence!empty}, called $0$-tuple. When two arbitrary elements in a $n$-tuple, $n>1$, interchange positions, we called it a \textsb{transposition}\index{transposition}.


Elements of sets can be used to build sequences. For instance, we can build doubles of the type $(x,x/2)$, where $x\in \mathbb{Z}$ and $x/2\in\gloref{racionais}$. If there is a collection of sets $\con{A}_1,\con{A}_2,\cdots,\con{A}_n$, not necessarily distinct, $n$-tuples of the type $(a_1,\cdots,a_n)$, where each element $a_i\in A_i$, can also be built. Thereby, the set of all these constructed $n$-tuples is called the \textsb{cartesian product}\index{product!cartesian} of the sets $A_i$. In mathematical terms, if a collection $\con{A}_1,\con{A}_2,\cdots,\con{A}_n$ of sets is given, the set
\begin{equation}
\con{A}_1 \gloref{times} \con{A}_2 \times \cdots \times \con{A}_n := \lch
\lpa \ele{a}_1,\ele{a}_2,\cdots,\ele{a}_n \rpa : \ele{a}_i \in
\con{A}_i\, , \,i=1,\cdots,n \rch \,,\, n > 1\,,
\end{equation}
is their cartesian product. In order to simplify notation, $\con{A}_1 \times \con{A}_2 \times \cdots \times \con{A}_n$ is compacted to $\gloref{crt}$. When all the $n$ sets are equal to $A$, we adopt the format $\gloref{crtEq}$, called \textsb{cartesian power}\index{cartesian!power}. If one of the terms in a cartesian product is the empty set, the result is also the empty set: $A_1\times \emptyset=\emptyset\times A_1=\emptyset$. Since element ordering distinguishes sequences, the cartesian product $A_1\times A_2$ is commutative only when one of the sets is empty or when they are equal; in other words, if a set $A_1\neq A_2\neq \emptyset$, then $A_1\times A_2\neq A_2\times A_1$. Element ordering in sequences also makes the cartesian product non-associative when sets involved are not empty. Thereby, the set $\lpa A \times B\rpa \times C\neq A \times \lpa B \times C\rpa$ because double of double and element differs from double of element and double; that is, the double $\lpa\lpa a, b\rpa, c\rpa\neq \lpa a, \lpa  b, c\rpa\rpa$, where $a,b,c$ are arbitrary elements of $A,B,C$ respectively, even when such sets are not disjoint. The cartesian product is distributive in union, intersection and difference of sets. Therefore, given arbitrary sets $A,B,C$, we can write the following:
\begin{align}
A \times \lpa B \cup C\rpa  &= \lpa A \times B\rpa \cup \lpa A \times C\rpa\,;\label{eq:distUniao}\\
A \times \lpa B \cap C\rpa  &= \lpa A \times B\rpa \cap \lpa A \times C\rpa\,;\label{eq:distInter}\\
A \times \lpa B \setminus C\rpa  &= \lpa A \times B\rpa \setminus \lpa A \times C\rpa\,;
\end{align}
{\footnotesize
\begin{proof}
Let the sets $A=\lch a_1,a_2,\cdots \rch$, $B=\lch b_1,b_2,\cdots \rch$ and $C=\lch c_1,c_2,\cdots \rch$. In order to verify \eqref{eq:distUniao}, here's the following development:
\begin{align}
\lch a_1,a_2,\cdots \rch\times\lpa\lch b_1,b_2,\cdots \rch\cup\lch c_1,c_2,\cdots \rch\rpa=\lch a_1,a_2,\cdots \rch\times\lpa\lch b_1,b_2,\cdots,c_1,c_2,\cdots \rch\rpa= \nonumber\\
=\lch \lpa a_1,b_1 \rpa,\lpa a_1,b_2 \rpa, \lpa a_2,b_1 \rpa,\lpa a_2,b_2 \rpa,\cdots,\lpa a_1,c_1 \rpa,\lpa a_1,c_2 \rpa, \lpa a_2,c_1 \rpa,\lpa a_2,c_2 \rpa,\cdots \rch=\nonumber\\
=\lpa\lch a_1,a_2,\cdots \rch\times\lch b_1,b_2,\cdots \rch\rpa\cup\lpa\lch a_1,a_2,\cdots \rch\times\lch c_1,c_2,\cdots \rch\rpa\,.\nonumber
\end{align}
Equality \eqref{eq:distInter} can be demonstrated through the following reasoning: if $\lpa a_1,x\rpa\in A\times\lpa B\cap C\rpa$ then, by the concepts of intersection and cartesian product,
\begin{equation*}
\lpa a_1\in A\rpa \wedge \lpa x\in\lpa B\cap C\rpa\rpa=\lpa a_1\in A\rpa\wedge\lpa x\in B\rpa \wedge \lpa x\in C\rpa\,.
\end{equation*}
And then the double $\lpa a_1,x\rpa\in \lpa A\times B\rpa\wedge\lpa a_1,x\rpa\in \lpa A\times C\rpa$. This same strategy can be used to prove the last equality: if  $\lpa a_1,x\rpa\in A\times\lpa B\setminus C\rpa$ then
\begin{equation*}
\lpa a_1\in A\rpa \wedge \lpa x\in\lpa B\setminus C\rpa\rpa=\lpa a_1\in A\rpa\wedge\lpa x\in B\rpa \wedge \lpa x\notin C\rpa\,.
\end{equation*}
Therefore, double $\lpa a_1,x\rpa\in \lpa A\times B\rpa\wedge\lpa a_1,x\rpa\notin \lpa A\times C\rpa$.
\end{proof}}



\section{Functions}\index{function}

The act of thinking is fundamentally based on the capacity of making relationships between entities in order to clarify something obscure or, more pretentiously, to disclose the unknown. To correlate entities in algebra thinking means to describe or establish a mathematical link between them. Thereby, among the many types of interactions that justify pairing these entities or objects, there can be links of cause and effect, transformations, dependencies, attributions, associations and so on. These relationships are described by rules that specify, through mathematical expressions, how a certain pairing of objects is done.

The fundamental concept that establishes algebraic relationships we call it func\-tion, here understood as \emph{a systematic assignment of one and only one object to each element of a given set}. More precisely, we define function as a double $\lpa \con{D},f \rpa$, where $\con{D}$ is this given set, called the \textsb{domain}\index{domain} of the function, and $f$ is the rule that implements the so called systematic assignment. In order to avoid confusions, we adopt here the usual notation that considers the rule also a function, that is, the context will define if $\gloref{funcao}$ is a rule or a function $\lpa \con{D},f \rpa$. The object related to an element $d\in D$ is represented by $\gloref{valor}$, called the \text{value of function}\index{function!value of} $f$ in $d$, which allows us to write the fundamental characteristic of functions, namely,
\begin{equation}
\fua{f}{\ele{d}_1}\neq\fua{f}{\ele{d}_2}\implies \ele{d}_1\neq \ele{d}_2\,,\,\,\forall\, d_1,d_2\in D.
\end{equation}
When we want to emphasize the domain $D$ of function $f$, we use the combined notation $\gloref{dominio}$. There is also an alternative notation for the function $f$ that makes its domain explicit: $\ele{d}\gloref{mapsto}\fua{f}{\ele{d}}$, where each element $d\in D$ is related to a value $\fua{f}{d}$ by $f$ in terms already described.

The description of $f$, as a rule, is done through an algebraic expression, where the symbol that represents an arbitrary element of the domain is called a \textsb{variable}\index{variable}. For example, let  $\lpa \gloref{real},f \rpa$ be a function and
\begin{equation}
 \fua{f}{x} = x^2 + 2\, ,
\end{equation}
where variable $x$ represents an arbitrary real value. This sentence says that the value of function $f$, on the left side, equals the value of the algebraic expression on the right. We also say that variable $x$ is the \textsb{argument}\index{argument} of $f$.

Besides domain, there are at least two additional special sets when we study functions. The first one, represented by $\gloref{imagem}$, is the \textsb{image}\index{image} of the function $f$, defined by all the values of $f$; in other words,
\begin{equation}
\con{R}_f :=  \lch \fua{f}{\ele{d}} : d \in D_f\rch \,.
\end{equation}
The second one arises when we want to study the part of the function domain, called \textsb{preimage}\index{preimage}, which is related to a certain subset of the image. In other words, given a subset $\con{B}\subseteq\con{R}_f$, the preimage of $\con{B}$ is the set $\gloref{preimagem}\subseteq\con{D}_f$ such that
\begin{equation}
\con{R}^{-1}_{\con{B}} :=  \lch \ele{d}\in\con{D}_f:
\fua{f}{\ele{d}}\in\con{B} \rch  \,.
\end{equation}

A function $f$ is said to be \textsb{invertible}\index{function!invertible} if it assigns distinct values to its domain elements, resulting in an element-value correlation of one-to-one, called \textsb{biunivocal correlation}\index{biunivocal!correlation}. In more rigorous terms, $f$ is invertible when
\begin{equation}\label{eq:funcaoInversivel}
\ele{d}_1\neq\ele{d}_2 \gloref{implicabid} \fua{f}{\ele{d}_1}\neq\fua{f}{\ele{d}_2}
,\,\forall\,  \ele{d}_1,\ele{d}_2 \in
\con{D}_f.
\end{equation}
When all the values of the invertible function $f$ define an image $\con{R}_f$, the function $\gloref{inversa}$ is called the \textsb{inverse}\index{function!inverse} of $f$ if
\begin{eqnarray}\label{eq:Inversa}
\con{D}_\fun{f^{-1}}=\con{R}_f&\text{and}&\fua{f^{-1}}{\fua{f}{d}}=d\,,\,\,\forall d\in D_f\,.
\end{eqnarray}
As a consequence, we can state that $f^{-1}$ is also invertible. Therefore, considering function $g:=f^{-1}$ and its image $R_g$, there exists a function $g^{-1}$ where, according to the previous definition,
\begin{equation}
\fua{f}{d}=\fua{g^{-1}}{\fua{g}{\fua{f}{d}}}=\fua{g^{-1}}{\fua{f^{-1}}{\fua{f}{d}}}=\fua{g^{-1}}{d}\,,
\end{equation}
for an arbitrary element $d\in D_f$. Thereby, we state that $f$ is the inverse function of $f^{-1}$ and then both are the inverse of each other. The superposition of a function and its inverse results in a special function, whose values equals the arguments, called the \textsb{identity}\index{function!identity} function, represented by $i$. Thereby, since  $\fua{i}{\ele{d}}=\ele{d}, \forall\, \ele{d}\in\con{D}_i$, image $R_i=D_i$, and then we conclude that $i=i^{-1}$ because $i$ is obviously invertible.


\section{Mappings}\index{mapping}\label{sec:mapping}

There is another algebraic relationship whose main purpose is to relate sets by using functions. In order to do this, it is considered that any function value is also a set element. Thereby, we have a source-set $U$, on which a function $f$ ``acts'' and a target-set $V$, to which all the values of $f$ belong. This relationship is called mapping when the domain $D_f=U$ and the image $R_f\subseteq V$, where $V$ is called the \textsb{codomain}\index{codomain} of $f$. In more rigorous terms, a mapping is a triple $\lpa U, V, f\rpa$ where $f$ maps $U$ to $V$, that is, $\ele{u} \mapsto \fua{f}{\ele{u}}\in \con{V}$ for every $\ele{u}\in\ele{U}$. Instead of representing a mapping by a tuple, we prefer to notate it as $\map{f}{\con{U}}{\con{V}}$, where the arrow makes source-target relationship explicit.

If the image $R_f$ equals codomain $V$, the mapping $\map{f}{\con{U}}{\con{V}}$ is called \textsb{surjective}\index{mapping!surjective} and the function $f$ is a \textsb{surjection}\index{surjection}. Thereby, we can say that in a surjective mapping notation, the function image is always explicit. Now, when $f$ is an invertible function, the mapping is said to be  \textsb{injective}\index{mapping!injective} and its function to be an \textsb{injection}\index{injection}. In this context, considering the image of the injection $f$, it is possible to define a mapping  $\map{f^{-1}}{\con{R}_f}{\con{U}}$, where $\con{R}_f$, domain of $f^{-1}$, is an improper subset of $V$. The function $f$ can cumulatively be an injection and a surjection, when it is called a \textsb{bijection}\index{bijection} and its respective mapping a \textsb{bijective mapping}\index{mapping!bijective}. This bijection $f$ invariably implies the existence of the bijective mapping $\map{f^{-1}}{\con{V}}{\con{U}}$.


\begin{figure}[!ht]
\centering
\begin{center}
\scalebox{.70}{\input{partes/parte1/figs/c_algabst/mapeamentos.pstex_t}}
\end{center}
\titfigura{Functions $f$ and $g$ map $U$ to $V$, where $f$ is an injection and $g$ a surjection. If image $R_f$ were equal to the codomain $V$, $f$ would be a bijection.}\label{fg:mapeamentos}
\end{figure}

The mapping $\map{f}{\con{U}}{\con{V}}$ is said to be an \textsb{operation}\index{operation} and its function an \textsb{operator}\index{operator} if the domain $U=V^n$. In this case, when $n$ is 1, 2, 3 or 4, operation and operator are classified as \textsb{unary}\index{operation!unary}, \textsb{binary}\index{operation!binary}, \textsb{ternary}\index{operation!ternary} and \textsb{quaternary}\index{operation!quaternary} respectively; when $n>4$, they are called $n$\textsb{-ary}\index{operation!$n$-ary}. The arguments of an operator are called \textsb{operands}\index{operand} and integer $n$ defines their quantities. It is interesting to note that the unary injective operation $\map{f}{\con{V}}{\con{V}}$ is always surjective since the condition of invertibility \eqref{eq:funcaoInversivel} assures that any pair of distinct elements of $V$ is related to a pair of distinct elements of $V$ through $f$; thereby, image $R_f=V$ and then we can state that any injective unary operator is a bijection.


At the end of the last section, we talked superficially about ``superposition'' of functions, namely, when a function has a function value as argument. This important concept, in more precise terms, can be presented as follows. Let's say that $\map{\fun{g}}{\con{U}}{\con{V}}$, $\map{f}{\con{V}}{\con{W}}$ and $\map{\fun{h}}{\con{U}}{\con{W}}$ are mappings where $\fua{h}{\ele{u}}=\fua{f}{\fua{g}{\ele{u}}}$, for all $\ele{u}\in U$. In this context, it is said that $h$ is a  \textsb{composite function}\index{function!composite} of $f$ and $g$, usually represented by $\fun{f \circ\fun{g}}$. Note that the composition of functions is not generally commutative, except in particular mappings whose functions and domains permit. Moreover, the following properties are valid:
\begin{itemize}\label{prop:Composicao}
	\setlength\itemsep{.1em}
	\item[i.] Given $\map{\fun{k}}{\con{W}}{\con{L}}$, we have
	$\fun{k}\circ\lpa f\circ\fun{g}\rpa=\lpa\fun{k}\circ f\rpa\circ\fun{g}$ ;
	\item[ii.] If $f$ and $\fun{g}$ are bijections, $f\circ\fun{g}$ is also a bijection and
	\begin{equation}
	\begin{array}{rcl}
	\lpa f\circ\fun{g}\rpa^{-1}& = & \fun{g}^{-1}\circ f^{-1}, \nonumber \\
	f\circ f^{-1} & = & \fun{i}_\con{W}, \nonumber \\
	f^{-1}\circ f & = & \fun{i}_\con{V}\,; \nonumber \\
	\end{array}
	\end{equation}
	\item[iii.]
	$f\circ\fun{i}_\con{V}=i_\con{W}\circ f=f$.
\end{itemize}


{\footnotesize
\begin{proof}
For the second item, if the relationship between $v$ and $u$ is biunivocal in $v=\fua{g}{u}$ for all $v\in V$, then the relationship between $u$ and $w$ is also biunivocal in $\fua{f^{-1}}{w}=\fua{g}{u}$ or $w=\fua{f}{\fua{g}{u}}$ for all $w\in W$; from where $f\circ g$ results a bijection. Now, considering $u$, $v$ and $w$ arbitrary elements of $U$, $V$ and $W$ respectively, the first equality in ii is verified as follows:
\begin{equation*}
\fua{\lpa f\circ\fun{g}\rpa^{-1}}{w}= u = \fua{g^{-1}}{v} = \fua{g^{-1}}{\fua{f^{-1}}{w}}=\fua{g^{-1}\circ f^{-1}}{w}.
\end{equation*}
The other equalities on the list can be easily proved from the definition of composite functions.
\end{proof}}


\section{Groups}\index{group}

The most fundamental algebraic entity that gathers the concepts of collection and relationship is called group, defined by a set and a mapping: a pair of set elements is related to an element of the same set by a mapping, namely, a binary operation that must obey certain restrictions. In other words, when a set defines a group, a double of set elements is functionally related to an element of the same set.

Addition and multiplication of real numbers, composition of invertible functions, subtraction of integers are all examples of mathematical combinations that the concept of group generalizes: they are all associative, they admit identity and inverse elements. Thereby, we can now define in more rigorous terms these intuitive concept. Let $\gloref{grupo}$ be a non-empty set and $\map{\ast}{\con{G}^2}{\con{G}}$ a binary operation from which the notation $\fua{\ast}{\ele{g}_1,\ele{g}_2}$ is shortened to $\ele{g}_1\ast\ele{g}_2$,
where $\ele{g}_1,\ele{g}_2\in\con{G}$. The double $\lpa \con{G},\ast \rpa$ is called a group when the following axioms are valid:
\begin{itemize}\label{ax:grupo}
	\setlength\itemsep{.1em}
	\item[i.] Associativity, where $\ele{g}_1\ast\lpa \ele{g}_2 \ast \ele{g}_3 \rpa =
	\lpa \ele{g}_1 \ast \ele{g}_2 \rpa \ast \ele{g}_3\, , \forall \, \ele{g}_1,\ele{g}_2,\ele{g}_3 \in
	\con{G}$;
	\item[ii.] Identity element, if $\gloref{umAum} \, \ele{e}\in\con{G}$ such that $ \ele{g}_1\ast\ele{e}=\ele{e}\ast\ele{g}_1=
	\ele{g}_1\,,\forall \, \ele{g}_1 \in \con{G}$;
	\item[iii.] Inverse element, if $\exists! \, \ele{b}\in\con{G}$ such that
	$\ele{g}_1 \ast \ele{b}= \ele{b} \ast \ele{g}_1 = \ele{e}$, $\forall \ele{g}_1\neq\ele{e}$.
\end{itemize}
As an example, the set $P$ of all unary invertible operators whose (co)domain is $V$, defines a group $\lpa \con{P},\circ\rpa$, according to the properties of function composition.

When it is convenient to make the operator explicit, we shall represent a group by $\lpa \con{G},\ast \rpa$; otherwise, we'll refer to a group $G$, indistinct from a set, in order to avoid abuse of notation. Thereby, considering this group $G$, there are operations $\ast$, like addition and multiplication of real numbers, that also obey the axiom of
\begin{itemize}
	\item[i.] Commutativity: $\ele{g}_1\ast \ele{g}_2= \ele{g}_2\ast \ele{g}_1\, ,
	\forall \, \ele{g}_1,\ele{g}_2 \in \con{G}$,
\end{itemize}
from which the group $G$ becomes \textsb{commutative}\index{group!commutative} or \textsb{abelian}\index{group!abelian}. In contrast, groups in which the order of operands affects the operator value are called \textsb{non-abelian}\index{group!non-abelian} or \textsb{non-commutative}\index{group!non-commutative}. The abelian group that implements the generalized concept of addition is called \textsb{additive}\index{group!additive} and of multiplication, \textsb{multiplicative}\index{group!multiplicative}, both represented respectively by $\lpa \con{G},+ \rpa$ and $\lpa \con{G},\cdot\rpa$. We adopt the notations $\ele{g}_1^{-1}$ and $-\ele{g}_1$ as the inverse elements of ${\ele{g}_1}\in G$ in multiplication and addition respectively.

Sets defining groups can also be used to define mappings. When this happens, the function must admit as argument any value of the operation involved, since the set of all these values is the domain itself. In other words, given a mapping $\map{h}{\con{G}}{\con{W}}$, where the sets involved define groups $\lpa G, \ast\rpa$ and $\lpa W, \rtimes\rpa$, we have each element $g\in G$ as a value of some $g_1\ast g_2$, where $g_1,g_2\in G$. Therefore, it is evident that the value $\fua{h}{g}=\fua{h}{g_1\ast g_2}$.

The operation in group $W$ can take the elements $\fua{h}{g_1}$ and $\fua{h}{g_2}$ of $W$ as operands, that is, $\fua{h}{g_1}\rtimes\fua{h}{g_2}$, and have a resulting value of $\fua{h}{g_1\ast g_2}\in W$. Moreover, if $h$ maps the identity element of $G$ to the identity element of $W$, we say that these two groups, in an operational context, are structurally similar or \textsb{homomorphic}\index{group!homomorphic} in $h$. In mathematical terms, the function in $\map{h}{\con{G}}{\con{W}}$ is said to be a \textsb{group homomorphism}\index{group!homomorphism} if it makes $G$ and $W$ homomorphic, that is, if
\begin{itemize}
	\setlength\itemsep{.1em}
	\item[i.] $\fua{h}{\ele{g}_1\ast\ele{g}_2} =
	\fua{h}{\ele{g}_1}{\rtimes}\,\fua{h}{\ele{g}_2}\, ,
	\,\forall\,\ele{g}_1,\ele{g}_2\in\con{G}$ and
	\item[ii.] $\fua{h}{\ele{e}_\con{G}} = \ele{e}_\con{W}$, where
	$\ele{e}_\con{G}\in\con{G}$ and $\ele{e}_\con{W}\in\con{W}$ are
	identity elements.
\end{itemize}
A bijection-homomorphism is named \textsb{isomorphism}\index{group!isomorphism} and the groups involved are \textsb{isomorphic}\index{group!isomorphic} in $h$. If the function in mapping $\map{f}{\con{G}}{\con{G}}$ is an isomorphism, then $f$ is called \textsb{automorphism}\index{group!automorphism}.


Now, let's consider the mapping $\map{\fun{k}}{\crt{G}{n}}{\con{W}}$ whose domain is the cartesian product of $n$ sets, each of them defining a group. In this case, the function $k$ can be called a group homomorphism if, for a group $G_i$ and arbitrary elements $g_{i_1},g_{i_2}\in G_i$,
\begin{align}
\lefteqn{\fua{k}{\ele{g}_1,\cdots,g_{i_1}\ast g_{i_2},\cdots,\ele{g}_n}=} & & \nonumber\\
& &\fua{k}{\ele{g}_1,\cdots,g_{i_1},\cdots,\ele{g}_n}\rtimes\fua{k}{\ele{g}_1,\cdots,g_{i_2},\cdots,\ele{g}_n}
\end{align}
and also
\begin{equation}
\fua{k}{e_{G_1},\cdots,e_{G_n}} = e_W\,.
\end{equation}
Similarly, $\con{k}$ is considered an isomorphism if it is a bijection-homomorphism.


In our study, we shall need to establish a relation between a group, which is a set of algebraic character, whose elements can be operated, and a set of geometric character, where sizes, forms and positions can be observed. A strategy to consistently accomplish this al\-ge\-braic-geometric relation is to use a function called \textsb{group action}\index{group!action}. Thereby, let $\map{\varphi}{\con{G}\times\con{B}}{\con{B}}$ be a mapping where $G$ is a group and $B$ is any non empty set. Function $\varphi$ is said to be a group action\footnote{In more precise terms, $\varphi$ in this case is called a \textsb{left group action}\index{group!action!left} in contrast to a \textsb{right group action}\index{group!action!right} $\tilde{\varphi}$ where $\map{\tilde{\varphi}}{\con{B}\times\con{G}}{\con{B}}$. Here, we'll always use the left action.} of the set $G$ on $B$ if the following axioms are valid:
\begin{itemize}
\setlength\itemsep{.1em}
    \item[i.]  Identity element, $ \fua{\varphi}{\ele{e},\ele{b}}=\ele{b}\,,
\forall \, \ele{b} \in \con{B}$, and
    \item[ii.]  Associativity, $ \fua{\varphi}{\ele{g}_1,\fua{\varphi}{\ele{g}_2,\ele{b}}}=
\fua{\varphi}{\ele{g}_1\ast\ele{g}_2,\ele{b}}, \forall \, \ele{b}
\in \con{B},\,\forall\, \ele{g}_1,\ele{g}_2\in\con{G}$.
\end{itemize}
When $\varphi$ observe these axioms, the structure of $B$ is preserved, now called a \textsb{$G$-set}\index{set!G-}, since its fundamental mathematical attributes remain unaltered. Thereby, given arbitrary elements $b_1,b_2\in B$, the group action $\varphi$ is classified as \textsb{simply transitive}\index{group!action!simply transitive} if
\begin{equation}
\exists !\,\, \ele{g}\in{\con{G}}\textrm{ such that } \fua{\varphi}{g,\ele{b_1}}=\ele{b_2}.
\end{equation}
From these definition, we conclude that \emph{in a simply transitive group action of $G$ on $B$, when we fix an element of $B$ in the domain, there results a biunivocal correspondence between the elements of $G$ and $B$.}


There can be an abelian group $F$ that is simultaneously additive and multiplicative, on which the multiplication of additions results the addition of multiplications, or in other words, on which the following distributivity is valid:
\begin{equation}
\alpha\cdot\gloref{sum}\alpha_i
=\sum_{i=1}^{n}\alpha\cdot \alpha_i\,\, , \forall \,
\alpha,\alpha_i \in
    \con{F}\,\text{ and } n\in\gloref{inteirosPosNNulo}.
\end{equation}
In these circumstances, the triple $\lpa F, +, \cdot \rpa$ is called a \textsb{field}\index{field}, abbreviated by $\gloref{campo}$. When it is convenient, in order to simplify notation, $\cam{F}$ will represent the definer set $F$. An element of $\cam{F}$ is named a \textsb{scalar}\index{scalar}, of which $\beta\in\real$ and $\gamma\in\gloref{complexo}$ are examples. Henceforth, for the purposes of our study, an arbitrary field $\cam{F}$ will always refer to either a complex field $\complexo$ or a real field $\real$, that is, $\cam{F}\in\{\real,\complexo\}$. In this sense, where the real field will not be considered  a subset of the complex field, when $\cam{F}$ is $\real$, we must define that the conjugate $\overline{\alpha}=\alpha$, the real part $\gloref{parteReal}(\alpha)=\alpha$ and imaginary part $\gloref{parteImag}(\alpha)=0$, for all $\alpha\in \cam{F}$. Moreover, considering $f$ an arbitrary scalar valued function and $\fua{\overline{f}}{x}:=\overline{\fua{f}{x}}$, we have $\overline{f}=f$ in the case of $\cam{F}=\cam{R}$.   


\section{Arrays}\index{array}


We already learned that sequences are collections of ordered elements; or, more precisely, elements arranged in a queue, where each one has a position labeled by an index. Generalizing this idea, we shall build collections whose elements are arranged in a higher number of perspectives, like a table arrangement, for instance. Thereby, let $H$ be a collection of scalars, arranged by the mapping $\map{\fun{h}}{\crt{N}{q}}{\cam{F}}$, where the function $h$ is called \textsb{addressing}\index{array!addressing of} and each $N_i = \lch 1,2,\cdots,n_i \rch$ is a finite subset of $\mathbb{N}^*$ whose elements are ordinals. In this context, we call the collection $H$ an array, represented by $\gloref{array}$, whose scalars $\fua{\fun{h}}{( i_1,\cdots,i_q)}$ are notated by $\gloref{elArray1}$, where the subscript show explicitly the element position. The description of perspectives or \textsb{dimension}\index{array!dimension of} of an array is given by the $n$-tuple $(n_1 , \cdots , n_q)$ or, more usually, $n_1\times\cdots\times n_q\,$, where the number of perspectives $q$ expresses the \textsb{order}\index{array!order of} of the array and $n_i$ the \textsb{size}\index{array!size of} of each perspective.

Arrays can be added and multiplied. If an array  $\mat{A+B}$ is the sum of $\mat{A}$ and $\mat{B}$, these three arrays have equal dimension $n_1\times\cdots\times n_q\,$ and each element
\begin{equation}\label{eq:adicaoArray}
\mat{(A+B)}_{i_1\cdots i_q}=\mat{A}_{i_1\cdots i_q}+\mat{B}_{i_1\cdots i_q}\,.
\end{equation}
On the other hand, the multiplication \gloref{prodArr} requires that the last $q$ elements of the dimension of $\mat{A}$ and the first $q$ elements of the dimension of $\mat{B}$ are equal; in other words, if array $\mat{A}$ has dimension $m_1\times\cdots\times m_p\times n_1\times\cdots\times n_q $, then $\mat{B}$ must have dimension $n_1\times\cdots\times n_q\times l_1\times\cdots\times l_s$. Thereby, $m_1\times\cdots\times m_p\times l_1\times\cdots\times l_s$ is the resulting dimension of array $\mat{A*_qB}$ and each element
\begin{equation}
\lpa\mat{A*_qB}\rpa_{i_1\cdots i_pj_1\cdots j_s} =
\sum_{k_1=1}^{n_1}\cdots\sum_{k_q=1}^{n_q} \mat{A}_{i_1\cdots
	i_pk_1\cdots k_q}\,\mat{B}_{k_1\cdots k_qj_1\cdots j_s}\,.
\end{equation}
From now on, we shall adopt that $\mat{AB}:=\mat{A*_1B}$. It is also very important for our study to present the so called the \textsb{scalar product}\index{matrix!scalar product of} or \textsb{Frobenius product} of arrays: given two arrays $\mat{A}$ and $\mat{B}$ with the same dimension $m_1\times\cdots\times m_p$, the scalar product 
\begin{equation}
\gloref{scalP}:=\sum_{i_1}^{m_1}\cdots\sum_{i_p}^{m_p}\mat{A}_{i_1\cdots i_p}\overline{\mat{B}_{i_1\cdots i_p}}\,.
\end{equation}
Speaking of scalars, it is also possible to multiply a scalar $\alpha$ and an array $\mat{C}$ with dimension $n_1\times\cdots\times n_q$ according to
\begin{equation}
\lpa\alpha\mat{C}\rpa_{i_1\cdots i_q} := \alpha \cdot \mat{C}_{i_1\cdots
	i_q}\,.
\end{equation}
This definition of multiplication by scalars in the case of $\alpha=-1$, from which we can establish the additive inverse $-\mat{C}$, together with the addition described
in \eqref{eq:adicaoArray}, allow us to state that the set $Y$ of all $n_1\times\cdots\times n_q$ arrays defines an additive group considering the existence of the null array\index{array!null} $\mat{0}\in Y$, whose elements are all zero. Conversely, since it is not possible to obtain a multiplicative inverse for every element of $Y$, this set can not define a group. Additionally, it is of fundamental importance for upcoming concepts to define $l_1\times\cdots \times l_q\times n_1\times\cdots\times n_q$ arrays $\mat{A}^\text{T}$ and $\mat{A}^\dagger$, which we call here the \textsb{transpose}\index{array!transpose} and the \textsb{conjugate transpose}\index{array!conjugate transpose} of array $A$ respectively, whose elements
\begin{alignat} {3}
\mat{A}^\text{T}_{i_1\cdots i_q j_1\cdots j_q}:=\mat{A}_{j_1\cdots j_q i_1\cdots i_q}&\qquad \text{ and } \qquad&\mat{A}^\dagger_{i_1\cdots i_q j_1\cdots j_q}:=\overline{\mat{A}_{j_1\cdots j_q i_1\cdots i_q}}\,.
\end{alignat}

An example of array that is widely used to accomplish indicial notation of complex quantities and operations is called \textsb{Levi-Civita Symbol}\index{Levi-Civita!Symbol} or \textsb{Permutation Symbol}\index{Permutation!Symbol}, notated by the letter $\epsilon$, whose scalars are defined the following way:
\begin{equation}
    \epsilon_{i_1\cdots i_n}=
\begin{dcases}
    \lpa-1\rpa^{\fua{\alpha_p}{ 1,\cdots,n }}  & \text{if } \exists\,\,\fua{p}{1,\cdots,n }=\lpa
i_1,\cdots,i_n\rpa\\\
0 & \text{if }  \nexists\,\,
\fua{p}{1,\cdots,n }=\lpa i_1,\cdots,i_n\rpa
\end{dcases}\,,
\end{equation}
where function $p$ permutes the $n$-tuple elements $(1,\cdots,n)$. In this array, the addressing domain is $N^n$, where the set of ordinals $N = \lch 1,2,\cdots,n \rch$. The term $\fua{\alpha_p}{ 1,\cdots,n }$ means the number of transpositions made on $\lpa 1,\cdots,n \rpa$, after which the resulting $n$-tuple is $\fua{p}{1,\cdots,n }$.
Because of its definition, the Levi-Civita Symbol is an array of order $n$ whose dimension is $n\times\cdots\times n$. Arrays having the same size $n$ in each perspective, like the Levi-Civita Symbol, are usually called \textsb{hypercubic}\index{array!hypercubic} of size $n$. Therefore, $\epsilon$ is a hypercubic array of order $n$ and size $n$. For instance, when the order and size are $2$, $\epsilon$ can be represented in a  tabular arrangement; more precisely,
\begin{equation}
\epsilon=
\begin{bmatrix}
    0 & 1  \\
    -1 & 0
\end{bmatrix}\,.
\end{equation}
The Levi-Civita symbol enables us to define a notable mapping $\map{\text{Det}}{\bar{A}}{\cam{F}}$, where $\bar{A}$ is the set of all $q$-th order hypercubic arrays of size $n$, $\cam{F}$ is a real or complex field and function $\text{Det}$ is called \textsb{hyperdeterminant}\index{array!hyperdeterminant of}\footnote{See \aut{Luque \& Thibon}\cite{luque_2006_1}.}, whose rule is
\begin{equation}\label{eq:Hiperdeterminante}
\hdet{\mat{X}} = \frac{1}{n!}\sum_{i_1^{(1)}=1}^n\cdots\sum_{i_n^{(1)}=1}^n\cdots
\sum_{i_1^{(q)}=1}^n\cdots\sum_{i_n^{(q)}=1}^n\epsilon_{i_1^{(1)}\cdots
	i_n^{(1)}}\cdots\epsilon_{i_1^{(q)}\cdots i_n^{(q)}}
\prod_{k=1}^{n}\mat{X}_{i_k^{(1)}\cdots i_k^{(q)}}
\end{equation}
when the order $q$ is even and $\hdet{\mat{X}}=0$ when it is odd. Although this definition lacks an adequate justification at this point, it will support future concepts that will have their own intuitive meaning. The term $1/n!$ is justified by the property    
\begin{equation}\label{eq:productProp}
\sum_{i_1^{(1)}=1}^n\cdots\sum_{i_n^{(1)}=1}^n\cdots
\sum_{i_1^{(q)}=1}^n\cdots\sum_{i_n^{(q)}=1}^n\epsilon_{i_1^{(1)}\cdots
	i_n^{(1)}}\cdots\epsilon_{i_1^{(q)}\cdots i_n^{(q)}}=n!\,,
\end{equation}
since $n!$ must be eliminated from the calculation of the hyperdeterminant.  


There is another special array $\delta$, also widely used in indicial notations, called \textsb{Kronecker Delta}\index{Kronecker Delta}, whose elements
\begin{equation}
    \delta_{i_1\cdots i_{q}j_1\cdots j_{q}}:=
\begin{dcases}
    \lpa-1\rpa^{\fua{\alpha_p}{ j_1,\cdots,j_q }}  & \text{if } \exists\,\,\fua{p}{j_1,\cdots,j_{q} }=\lpa
i_1,\cdots,i_q\rpa\\
0 & \text{se }  \nexists\,\,
\fua{p}{j_1,\cdots,j_{q} }=\lpa
i_1,\cdots,i_q\rpa
\end{dcases}\,,
\end{equation}
where the addressing domain is $N^{2q}$ and the set $N = \lch 1,2,\cdots,n \rch$. Differently from the Levi-Civita Symbol, in order to build $\delta$ it is necessary to specify an order of $2q$ and all dimensions equal to $n$. In the particular case of $q=1$, the Kronecker Delta results a $n\times n$ array, that is, a tabular array, as can be observed in 
\begin{equation}\label{eq:matrizIdentidade} 
\delta=
\begin{bmatrix}
    1      & 0       & \cdots & 0\\
    0      & 1       & \ddots & \vdots\\
	\vdots & \ddots  & \ddots & 0\\
	0      & \cdots  & 0      & 1
\end{bmatrix}\,.
\end{equation}
Along this text, the following important property of the Kronecker Delta will be used to handle indices in various proofs and expression developments:
\begin{equation}
\sum_{i_q=1}^{n}\mat{A}_{i_1\cdots i_q}\delta_{i_qj}=\mat{A}_{i_1\cdots i_{q-1}j}\,,
\end{equation}
where $\mat{A}$ is a $n\times\cdots\times n$ array, with $n$ repeated $q$ times. In this equality, the index $i_q$ of the array, on which the sum is made,  results changed by the index $j$ of the Kronecker Delta. Similarly,
\begin{equation}
\sum_{i_1=1}^{n}\delta_{ji_1}\mat{A}_{i_1\cdots i_q}=\mat{A}_{ji_2\cdots i_q}\,.
\end{equation}

At this point, it is convenient to say that every $n_1\times n_2$ array, whose elements are arranged in a tabular format, is called a \textsb{matrix}\index{matrix}. The elements of a matrix are addressed by lines and columns: the escalar with $ij$ position is located on the line $i$ and column $j$. If $n_1=n_2=n$, we have a hypercubic matrix, called \textsb{square matrix}\index{matrix!square}, whose size is $n$. Henceforth, recalling the additive group $Y$ of arrays, we will consider
$\bar{Y}\subset Y$ the set of hypercubic arrays of size $n$, $M\subset Y$ the set of matrices and  $\bar{M}\subset \bar{Y}$ the set of square matrices with size $n$. For example, the set $\bar{M}$ has the $n\times n$ Kronecker Delta, represented in \eqref{eq:matrizIdentidade}, as an element, which we usually call \textsb{identity matrix}\index{matrix!identity} $\mat{I}$ because $\mat{A}\mat{I} = \mat{I}\mat{A} = \mat{A}$, for an arbitrary $\mat{A}\in\bar{M}$.

If there exists a matrix $\mat{B}\in\bar{M}$ such that $\mat{A}\mat{B} = \mat{I}$, we call it the \textsb{inverse}\index{matrix!inverse} of $\mat{A}$, notated by $\mat{A}^{-1}$. The condition of invertibility of a matrix depends on the value of its hyperdeterminant. In the context of matrices, the hyperdeterminant function is called \textsb{determinant}\index{determinant}, which defines a mapping $\map{\gloref{determ}}{\bar{M}}{\cam{F}}$ where $\text{det}$ is described by the rule called \textsb{Leibniz formula}\index{Leibniz!formula}, namely
\begin{equation}\label{eq:Determinante}
\dete{\mat{X}} = \frac{1}{n!}\sum_{i_1=1}^n\cdots\sum_{i_n=1}^n
\sum_{j_1=1}^n\cdots\sum_{j_n=1}^n\epsilon_{i_1\cdots
	i_n}\epsilon_{j_1\cdots j_n}
\prod_{k=1}^{n}\mat{X}_{i_kj_k}\,,
\end{equation}
which is actually equality \eqref{eq:Hiperdeterminante} applied to square matrices of size $n$. This equality can be simplified by considering $j_k=k$, from which it can be obtained that
\begin{equation}\label{eq:DeterminanteSimplificado}
\dete{\mat{X}} = \sum_{i_1=1}^n\cdots\sum_{i_n=1}^n
\epsilon_{i_1\cdots i_n}\prod_{k=1}^{n}\mat{X}_{i_kk}\,,
\end{equation}
where $1/n!$ is no longer necessary because equality \eqref{eq:productProp} does not occur. When $\dete{\mat{X}}=0$, matrix $\mat{X}$ is said to be \textsb{singular}\index{matrix!singular} or \textsb{non-invertible}\index{matrix!non-invertible}; otherwise, it is called \textsb{non-singular}\index{matrix!non-singular} or \textsb{invertible}\index{matrix!invertible}. From the above definition, we can obtain a zero value determinant when its argument is the zero matrix and a unitary value for the identity matrix. Moreover, the main property of determinants of square matrices is to preserve multiplications, that is, the determinant of the product is the product of determinants; in other words $\dete{\mat{AB}}=\dete{\mat{A}}\cdot\dete{\mat{B}}$. Recalling our classification of groups, we can say that a subset of $\bar{M}$ whose matrices are all invertible defines, together with the operation of multiplication, a non-abelian group.



{\footnotesize
\begin{proof}
We want to verify the above multiplicative property of the determinant; and for that, we need a preliminar definition, presented as following. An invertible matrix is called \textsb{elementary}\index{matrix!elementary} when it differs from the identity matrix by one of the following three line actions: exchange, scalar multiplication and addition with the multiple of another line. The multiplication of an elementary matrix $\mat{E}$ by an arbitrary $\mat{B}\in\bar{M}$ means to accomplish in $\mat{B}$ the same line action accomplished on $\mat{I}$ to arrive at $\mat{E}$. On an invertible matrix $\mat{A}$, it is possible to accomplish successive line actions resulting the identity matrix. It means that
\begin{equation*}
\mat{E}_r\mat{E}_{r-1}\cdots\mat{E}_{2}\mat{E}_{1}\mat{A}=\mat{I}\,,
\end{equation*}
where matrices $\mat{E}_i$ are elementary. Thus, the equality
\begin{equation*}
\mat{A}=\mat{E}_{1}^{-1}\mat{E}_{2}^{-1}\cdots\mat{E}_{r-1}^{-1}\mat{E}_r^{-1}
\end{equation*}
is a decomposition of $\mat{A}$ in elementary matrices, since the inverse of an elementary matrix is also elementary. The determinant of a matrix $\mat{A}'$ that results from a line action on the matrix $\mat{A}$ is given by $\dete{\mat{A}'}=\alpha\dete{\mat{A}}$, onde $\alpha\in\real$. Therefore, we can state that $\dete{\mat{E}}=\beta\dete{\mat{I}}=\beta$, where $\beta\in\real$. Considering what has been defined so far, we can say that
\begin{equation*}
\dete{\mat{AB}}=\dete{\mat{E}_{1}^{-1}\mat{E}_{2}^{-1}\cdots\mat{E}_{r-1}^{-1}\mat{E}_r^{-1}\mat{B}}=\kappa\dete{\mat{B}}\,,
\end{equation*}
where $\kappa\in\real$ is obtained from the $r$ line actions on $\mat{B}$. We can also state that
\begin{equation*}
\dete{\mat{E}_{1}^{-1}\mat{E}_{2}^{-1}\cdots\mat{E}_{r-1}^{-1}\mat{E}_r^{-1}\mat{I}}=\kappa\dete{\mat{I}}=\kappa\,.
\end{equation*}
Therefore, we conclude that $\dete{\mat{AB}}=\dete{\mat{A}}\dete{\mat{B}}$. Now, considering $\mat{A}$ singular, it is clear that $\dete{\mat{A}}\dete{\mat{B}}=0$. If the matrix $\mat{AB}$ is invertible, then there exists a square matrix $C$ where $\mat{ABC}=\mat{I}$. But, saying this means saying also that matrix $\mat{BC}$ is an inverse of $\mat{A}$, which is inconsistent since $\mat{A}$ is singular. Therefore, $\mat{AB}$ is also singular, and then $\dete{\mat{AB}}=0$. In this case, we have also $\dete{\mat{AB}}=\dete{\mat{A}}\dete{\mat{B}}$.
\end{proof}}

If $\mat{A},\mat{B}\in M$ are $n_1\times n_2$ and $n_2\times n_1$ matrices respectively in such a way that $\mat{A}_{ij}=\mat{B}_{ji}$, we call them \textsb{transpose} or that one is the transpose\index{matrix!transpose} of the other, from which we adopt the representation $\gloref{mTransp}$ to $\mat{B}$ and $\mat{B}^\text{T}$ to $\mat{A}$. In particular, the matrix $\mat{S}\in\bar{M}$ is said to be \textsb{symmetric}\index{matrix!symmetric} when it is identical to its transpose and \textsb{antisymmetric}\index{matrix!antisymmetric} if $\mat{S}=-\mat{S}^\text{T}$. From the definition of transpose, we have the following:
\begin{itemize}
\setlength\itemsep{.1em}
	\item[i.] $(\mat{A}^\text{T})^\text{T} =
	\mat{A}$\,;
	\item[ii.] $\lpa\mat{A}+\mat{B}\rpa^\text{T} =
	\mat{A}^\text{T}+\mat{B}^\text{T}$\,;
	\item[iii.] $\lpa\mat{A}\mat{B}\rpa^\text{T} =
	\mat{B}^\text{T}\mat{A}^\text{T}$\,;
	\item[iv.] $\lpa\mat{A}^{-1}\rpa^\text{T} =
\lpa\mat{A}^\text{T}\rpa^{-1}$, if $\mat{A}$ is invertible\,.
\end{itemize}
Additionally, if $\mat{A}$ is invertible and $\mat{A}^{-1}=\mat{A}^\text{T}$, it is classified as \textsb{orthogonal}\index{matrix!orthogonal}. In this context, the property $\dete{\mat{A}}^\text{T}=\dete{\mat{A}}$ enables us to write 
\begin{equation}\label{eq:detMatOrtog}
1=\dete{\mat{AA}^{-1}}=\dete{\mat{AA}^\text{T}}= \dete{\mat{A}}\dete{\mat{A}} = (\dete{\mat{A}}) ^2\,,
\end{equation}
which means that $\dete{\mat{A}}=\pm 1$. Considering this result, matrix $\mat{A}$ is called a \textsb{proper orthogonal matrix}\index{matrix!proper orthogonal} when it has a positive determinant or an \textsb{improper orthogonal matrix}\index{matrix!improper orthogonal} when its determinant is negative. 

{\footnotesize
\begin{proof}
The verification of the first item is trivial. For the second, considering the main property of Kronecker Delta and the representation $\lpa\mat{A}+\mat{B}\rpa_{ji}$ as an element of $\lpa\mat{A}+\mat{B}\rpa^\text{T}$, we can state that
\begin{align*}
\lpa \mat{A} + \mat{B}\rpa_{ji} & = \sum_{i=1}^{n_1}\sum_{j=1}^{n_2}\delta_{ji}\lpa \mat{A} + \mat{B}\rpa_{ij}\delta_{ji}\nonumber\\
& = \sum_{i=1}^{n_1}\sum_{j=1}^{n_2}\delta_{ji}\lpa\mat{A}_{ij} + \mat{B}_{ij}\rpa\delta_{ji}\nonumber\\
& = \sum_{i=1}^{n_1}\sum_{j=1}^{n_2}\delta_{ji}\mat{A}_{ij}\delta_{ji} + \delta_{ji}\mat{B}_{ij}\delta_{ji}\nonumber\\
& = \sum_{i=1}^{n_1}\sum_{j=1}^{n_2}\mat{A}_{ji} + \mat{B}_{ji}= \mat{A}_{ji} + \mat{B}_{ji}\,.
\end{align*}
The third item can be verified similarly:
\begin{align*}
\lpa \mat{A}\mat{B}\rpa_{ji}& = \sum_{i=1}^{n_1}\sum_{j=1}^{n_2}\delta_{ji}\lpa \mat{A}\mat{B}\rpa_{ij}\delta_{ji}\nonumber\\
& = \sum_{i=1}^{n_1}\sum_{j=1}^{n_2}\delta_{ji}\lpa\sum_{k=1}^{n_2}\mat{A}_{ik} \mat{B}_{kj}\rpa\delta_{ji}\nonumber\\
& = \sum_{k=1}^{n_2}\sum_{i=1}^{n_1}\sum_{j=1}^{n_2}\delta_{ji}\mat{A}_{ik} \mat{B}_{kj}\delta_{ji}\nonumber\\
& = \sum_{k=1}^{n_2}\mat{A}_{jk} \mat{B}_{ki}= \sum_{k=1}^{n_2}\mat{B}^\text{T}_{ik} \mat{A}^\text{T}_{kj}\,.\nonumber
\end{align*}
The fourth property we verify from $\mat{A}^\text{T}(\mat{A}^\text{T})^{-1}=\mat{I}$ and transposing both sides of the equality $\mat{A}^{-1}\mat{A}=\mat{I}$, when
we equal the left sides of these two expressions, arriving at the equality $\mat{A}^\text{T}(\mat{A}^\text{T})^{-1}=\mat{A}^\text{T}(\mat{A}^{-1})^\text{T}$.
\end{proof}}


A more generic case of matrix transposition involves complex scalars and their conjugates. Thus, let's consider $\mat{H}$ and $\mat{F}$ matrices with dimensions $n_1\times n_2$ and $n_2\times n_1$ respectively, whose elements are complex numbers. We say that these matrices are \textsb{conjugate transpose}, or that one is the conjugate transpose
\index{matrix!conjugate transpose} of the other if $\mat{H}_{ij}=\gloref{compConj}$. In other words, $\mat{F}$ is the conjugate transpose \gloref{transpConj} of matrix  $\mat{H}$ if it is the transpose of the complex conjugates of the elements of $\mat{H}$. For example, if matrix
\begin{alignat*} {3}
\mat{H}=
\begin{bmatrix}
    1+3\mathrm{\gloref{imag}}      & -1 + \mathrm{i} & 2\mathrm{i}\\
    2+\mathrm{i}      & 4 & -\mathrm{i}\\
\end{bmatrix} & \qquad\text{then} \qquad & \mat{H}^\dagger=
\begin{bmatrix}
    1-3\mathrm{i}      & 2-\mathrm{i}\\
    -1 - \mathrm{i}      & 4\\
    -2\mathrm{i}      & \mathrm{i}\\
\end{bmatrix} \,.
\end{alignat*}
When the elements involved are all real scalars, we have the equality $\mat{H}^\dagger=\mat{H}^\text{T}$. Moreover, it is convenient to say that the four properties presented above for transpose matrices are also valid for conjugate transposes, namely, $(\mat{H}_1^\dagger)^\dagger =\mat{H}_1$,  $(\mat{H}_1+\mat{H}_2)^\dagger = \mat{H}_1^\dagger+\mat{H}_2^\dagger$, $(\mat{H}_1\mat{H}_2)^\dagger =\mat{H}_2^\dagger\mat{H}_1^\dagger$ and $(\mat{H}_1^\dagger)^{-1}=(\mat{H}_1^{-1})^\dagger$, if $\mat{H}_1$ is invertible, where matrices $\mat{H}_1,\mat{H}_2\in M$.


In the context of invertible matrices, a matrix whose inverse equals its conjugate transpose, that is, when  $\mat{U}^{-1}=\mat{U}^\dagger$ where $\mat{U}\in\bar{M}$, is called  \textsb{unitary}\index{matrix!unitary}. Considering the property $\dete{\mat{H}}^\dagger=\overline{\dete{\mat{H}}}$, we have
\begin{equation}\label{eq:determUnita}
1=\dete{\mat{UU}^{-1}}=\dete{\mat{UU}^\dagger}= \dete{\mat{U}}\overline{\dete{\mat{U}}} = | \dete{\mat{U}} |^2\,,
\end{equation}
from where we can conclude that $\dete{\mat{U}}=\pm 1$. When this determinant is positive, $\mat{U}$ is called a \textsb{proper unitary}\index{matrix!proper unitary} matrix. Given a matrix $\mat{A}\in\bar{M}$, we have $\dete{\mat{U}\mat{A}}=\pm\dete{\mat{A}}$; which reveals, except for an eventual sign, the \emph{neutrality} of the unitary matrix in the determinant of matrix products. If the unitary matrix elements are reals, it results that $\mat{U}^\dagger=\mat{U}^\text{T}=\mat{U}^{-1}$, or that it is orthogonal. Moreover, for an arbitrary invertible matrix $\mat{A}$, the determinant of the inverse is the inverse of the determinant, according to the following expression:
\begin{equation}
1=\dete{\mat{A}\mat{A}^{-1}}=\dete{\mat{A}}\cdot\dete{\mat{A}^{-1}}\implies \dete{\mat{A}^{-1}}= \lpa\dete{\mat{A}}\rpa^{-1}\,.
\end{equation}

A given square matrix $\mat{A}$ that equals its conjugate transpose $\mat{A}^\dagger$ is called \textsb{Hermitian}\index{matrix!Hermitian}. Real symmetric matrices are examples of Hermitian matrices: if the elements of the square matrix  $\mat{S}$ are real, then $\mat{S}^\dagger=\mat{S}^\text{T}$, and since $\mat{S}$ is symmetric, $\mat{S}^\dagger=\mat{S}$. Similarly to antisymmetric matrices, an \textsb{anti-Hermitian}\index{matrix!anti-Hermitian} matrix $\mat{A}$ equals the negative of its conjugate transpose, that is, $\mat{A}=-\mat{A}^\dagger$. Thus, let $\mat{B}\in\bar{M}$ be a matrix from which we write the following development:
\begin{equation}
\mat{B} = \dfrac{1}{2} \lpa \mat{B} + \mat{B}\rpa = \dfrac{1}{2} \lpa \mat{B} + \mat{B} + \mat{B}^\dagger - \mat{B}^\dagger\rpa = \underbrace{\dfrac{1}{2} \lpa \mat{B} + \mat{B}^\dagger\rpa}_{\mat{B}_1} + \underbrace{\dfrac{1}{2} \lpa \mat{B} - \mat{B}^\dagger\rpa}_{\mat{B}_2}\,.
\end{equation}
Considering the properties of conjugate transposes, we can obtain that the matrix $\mat{B}_1$ equals its conjugate transpose and matrix $\mat{B}_2$ equals the negative of its conjugate transpose; thereby, we say that they are respectively the \textsb{Hermitian} and \textsb{anti-Hermitian}\index{matrix!Hermitian part of}\index{matrix!anti-Hermitian part of} parts of $\mat{B}$. This result is generalized by saying that every square matrix can be decomposed additively in a Hermitian and an anti-Hermitian parts.


Any square matrix $\mat{N}$ is called \textsb{normal}\index{matrix!normal}\label{nm:Normal} when it commutes with its conjugate transpose, that is, when $\mat{N}^\dagger\mat{N}=\mat{N}\mat{N}^\dagger$. These matrices, which make the product $\mat{N}^\dagger\mat{N}$ an Hermitian matrix, \emph{are always susceptible of diagonalization by a unitary matrix}. In order to understand what this means, let's consider firstly the matrices $\mat{A},\mat{B}\in\bar{M}$ and say that they are called \textsb{similar}\index{matrix!similar} when an invertible matrix $\mat{Q}$ exists such that
\begin{equation}
\mat{A} = \mat{Q}^{-1}\mat{B}\mat{Q}\,.
\end{equation}
It is important to note that $\mat{A}$ and $\mat{B}$ positions in the equality do not affect the definition, since by adopting $\mat{P}:=\mat{Q}^{-1}$, we arrive at $\mat{B} = \mat{P}^{-1}\mat{A}\mat{P}$, where the concept of similarity is maintained. Moreover, similar matrices have the same determinant value as can be verified in the following equalities:
\begin{equation}
\dete {\mat{A}} = \dete {\mat{Q}^{-1}}\dete{\mat{B}}\dete{\mat{Q}}=(\dete{\mat{Q}})^{-1}\dete{\mat{Q}}\dete{\mat{B}}=\dete{\mat{B}}.
\end{equation}
A mapping $\map{q}{\bar{M}}{\bar{N}}$ is a \textsb{similarity transformation}\index{similarity!transformation} if
\begin{equation}
\fua{q}{\mat{X}} = \mat{Q}^{-1}\mat{X}\mat{Q}.
\end{equation}
An important example of this transformation is called \textsb{diagonalization}\index{diagonalization}, defined from the concept of \textsb{diagonal matrix}\index{matrix!diagonal}, which is a square matrix whose elements in positions $i\neq j$ are null. Thereby, if there exists a similarity transformation in a square matrix $\mat{A}$ which $\fua{q}{A}$ results diagonal, this transformation is called a diagonalization of $\mat{A}$ or matrix $\mat{A}$ is said to be \textsb{diagonalizable}\index{matrix!diagonalizable}. Finally, it is now possible to understand the diagonalization of normal matrices by unitary.

\begin{mteo}{Spectral Diagonalization}{decompSpec}
For any normal matrix $\mat{N}$, there is always an unitary matrix $\mat{U}$ such that
\begin{equation}
\widetilde{\mat{N}} = \mat{U}^\dagger\mat{N}\mat{U}\,,
\end{equation}
where $\widetilde{\mat{N}}$ is a diagonal matrix whose elements constitute the spectrum of $\mat{N}$.
\end{mteo}

{\footnotesize
\begin{proof}
Firstly we must say that, considering any square matrix $\mat{A}$ and a unitary matrix $\mat{U}$, it is always possible to obtain an \textsb{upper triangular matrix}\index{matrix!upper triangular} $\mat{T} = \mat{U}^\dagger\mat{A}\mat{U}$, whose elements in positions $i>j$ are null. This statement is known as \textsb{Schur's Lemma}\index{Schur!Lemma}, whose tedious proof can be seen in \aut{Strang}\cite{strang_2006_4}. Now we need to show that if $\mat{A}$ is normal, $\mat{T}$ is diagonal: if $\mat{A}$ is normal,
\begin{align}
\mat{A}\mat{A}^\dagger & = \mat{A}^\dagger\mat{A}\nonumber\\
\mat{UTU}^\dagger\lpa\mat{UTU}^\dagger\rpa^\dagger & = \lpa\mat{UTU}^\dagger\rpa^\dagger\mat{UTU}^\dagger\nonumber\\
\mat{U}\mat{TT}^\dagger\mat{U}^\dagger & = \mat{U}\mat{T}^\dagger\mat{T}\mat{U}^\dagger\nonumber\\
\mat{TT}^\dagger & = \mat{T}^\dagger\mat{T}\nonumber\,,
\end{align}
from where we conclude that $\mat{T}$ is also normal. From the last equality, we have to each position $ij$
\begin{align}
\sum_{k=1}^n\mat{T}_{ik}\mat{T}_{kj}^\dagger & = \sum_{k=1}^n\mat{T}_{ik}^\dagger\mat{T}_{kj}\nonumber\\
\sum_{k=1}^n\mat{T}_{ik}\overline{\mat{T}_{jk}} & = \sum_{k=1}^n\overline{\mat{T}_{ki}}\mat{T}_{kj}\,.\nonumber
\end{align}
For the element in position $i=j=1$, this last equality results $\sum_{k=1}^n|\mat{T}_{1k}|^2=|\mat{T}_{11}|^2$, from where we can say that $\sum_{k=2}^n|\mat{T}_{1k}|^2=0$.
Since the terms in this sum are nonnegative, they can only be null, that is, $|\mat{T}_{1k}|^2=0$ when $k>1$. By induction, when we run through all the positions $i=j$, we verify that not only the elements of $\mat{T}$ in $i>j$ are null but also in $i<j$; which proves $\mat{T}$ diagonal.
\end{proof}}

Now let's clarify the term ``spectrum'' cited by the theorem above; but before that, we need some important definitions. The function in $\map{\gloref{traco}}{\bar{M}}{\cam{F}}$ is called \textsb{trace}\index{matrix!trace of} when its rule is defined by
\begin{equation}
\trc{\mat{X}} = \sum_{i=1}^{n}\mat{X}_{ii}\,,
\end{equation}
that is, the trace of a matrix is the sum of its diagonal elements. Considering the matrices $\mat{A,B}\in\bar{M}$, it is clear that $\trc{\mat{A+B}}=\trc{\mat{A}}+\trc{\mat{B}}$, where sum is preserved; thereby, the trace function results a group homomorphism on $\bar{M}\subset Y$ because $Y$ is an additive group. Moreover, since the respective diagonal elements of $\mat{AB}$ and $\mat{BA}$ are identical, it is true that $\trc{\mat{AB}}=\trc{\mat{BA}}$. Thence, if $\mat{A}$ and $\mat{B}$ are similar,
\begin{equation}
\trc{\mat{A}}=\trc{\mat{Q}^{-1}\mat{B}\mat{Q}}=\trc{\mat{Q}\mat{Q}^{-1}\mat{B}}=\trc{\mat{B}}\,,
\end{equation}
and then we can state generically that all similar matrices have the same trace.

Among other benefits, the trace function is a convenient tool to develop what is known by \textsb{characteristic polynomial}\index{matrix!characteristic polynomial of}\label{pg:PolinomioCarac} of a matrix: it is the function in operation $\map{g}{\cam{F}}{\cam{F}}$, whose rule is
\begin{equation}
\fua{g}{x}=\dete {\mat{H} - x\mat{I}}  \,,
\end{equation}
where $\mat{H}$ is a square matrix. By developing the right term, we arrive at
\begin{equation}\label{eq:poliCaracDesen}
\fua{g}{x}=(-1)^nx^n+a_1x^{n-1}+a_2x^{n-2}+\cdots+a_{n-1}x+a_n\,,
\end{equation}
which is a polynomial of order $n$, named characteristic polynomial of $\mat{H}$, whose coefficients are
\begin{align}
a_1 & = \lpa-1\rpa^{n+1}\trc{\mat{H}}\label{eq:coefAUm}\,;\\
a_2 & = -\dfrac{1}{2}\lco a_{1}\trc{\mat{H}}+\lpa-1\rpa^n\trc{\mat{H}^2}\rco\,;\\
&\cdots\nonumber\\
a_n & = -\dfrac{1}{n}\lco a_{n-1}\trc{\mat{H}}+a_{n-2}\trc{\mat{H}^2}+\cdots+a_1\trc{\mat{H}^{n-1}}+\lpa-1\rpa^n\trc{\mat{H}^n}\rco\,.
\end{align}
Any scalar $\lambda\in\cam{F}$ that is root of this polynomial is called a \textsb{characteristic root}\index{matrix!characteristic root of} of $\mat{H}$. The set
$\lch\lambda_1,\cdots,\lambda_n\rch$ of characteristic roots of $\mat{H}$ is called the \textsb{spectrum}\index{matrix!spectrum of} of $\mat{H}$. In the case of a normal matrix $\mat{N}$ submitted to spectral diagonalization, its spectrum is the set of scalars that constitute the diagonal matrix $\widetilde{\mat{N}}$, which results from the diagonalization of $\mat{N}$ by unitary matrix. At this point, it is convenient to say that matrices involved in similarity transformations, like $\widetilde{\mat{N}}$ and $\mat{N}$, always have the same spectrum and that is why they are called \textsb{isospectral}\index{matrices!isospectral}.


{\footnotesize
\begin{proof}
Let's prove this last statement. We already know that the determinant of the inverse is the inverse of the determinant. Let $\mat{A}$ an $\mat{B}$ be similar matrices through invertible matrix $\mat{Q}$. Then, it can be said that the characteristic polynomial $\dete{\mat{A} - x\mat{I}}=\dete {\mat{Q}^{-1}\mat{B}\mat{Q} - x\mat{I}}$. From the following equalities
\begin{equation*}
\mat{Q}^{-1}x\mat{I}\mat{Q}=x\mat{Q}^{-1}\mat{I}\mat{Q}=x\mat{Q}^{-1}\mat{Q}=x\mat{I}\,,
\end{equation*}
it is possible to say that
\begin{align*}
\dete{\mat{A} - x\mat{I}}&=\dete{\mat{Q}^{-1}\mat{B}\mat{Q} - \mat{Q}^{-1}x\mat{I}\mat{Q}}\\
&=\dete {\mat{Q}^{-1}\lpa\mat{B} - x\mat{I}\rpa\mat{Q}}\\
&=\lpa\dete{\mat{Q}}\rpa^{-1} \dete{\mat{B} - x\mat{I}} \dete{\mat{Q}}\\
&=\dete{\mat{B} - x\mat{I}}\,,
\end{align*}
from where we conclude $\mat{A}$ and $\mat{B}$ isospectral.
\end{proof}}

The spectrum of a matrix has a close relation with determinant and trace functions, which \emph{respectively preserve the operations of multiplication and addition}. If
$\lch\lambda_1,\cdots,\lambda_n\rch$ is the spectrum of $\mat{A}$, this close relation is expressed through the following equalities:
\begin{alignat}{3}
\dete{\mat{A}}  = \prod_{i=1}^n  \lambda_i & \qquad\text{and} \qquad & \trc{\mat{A}} & = \sum_{i=1}^n  \lambda_i \,.
\end{alignat}
When two matrices have the same pair of determinant and trace, we can say that these matrices have a kind of equivalence if we interpret the two functions as scalar measures whose absolute values express a quantitative aspect and whose signs express a qualitative aspect of the matrix in question. Therefore, isospectral matrices are said to be equivalent in this sense; \emph{something that, in a certain way, confers a metric feature on the spectrum of matrices}.

{\footnotesize
\begin{proof}
Let's verify the two equalities above. For the first one, let the rule of the characteristic polynomial of $\mat{A}$ be described in its factorized form by
\begin{equation*}
\fua{g}{x}=(-1)^n\lpa x-\lambda_n\rpa\lpa x-\lambda_{n-1}\rpa\cdots\lpa x-\lambda_2\rpa\lpa x-\lambda_1\rpa\,,
\end{equation*}
from where the following equality results:
\begin{equation*}
\dete{(\mat{A}-x\mat{I})}=\lpa \lambda_n-x\rpa\lpa \lambda_{n-1}-x\rpa\cdots\lpa \lambda_2-x\rpa\lpa \lambda_1-x\rpa\,,
\end{equation*}
valid for all $x\in\cam{F}$. Thus, if $x=0$, we have
\begin{equation*}
\dete{\mat{A}}=\lambda_n\lambda_{n-1}\cdots\lambda_2\lambda_1\,.
\end{equation*}
In order to prove the second equality, one of the so called Vi�te Formulae, according to \aut{Vinberg}\cite{vinberg_2003_1}, states that
\begin{equation*}
-\dfrac{a_1}{(-1)^n}=\lambda_n+\lambda_{n-1}+\cdots+\lambda_2+\lambda_1\,,
\end{equation*}
valid for polynomials having the same format of \eqref{eq:poliCaracDesen}. Thence we can substitute the coefficient $a_1$ by the right term of \eqref{eq:coefAUm}, resulting
\begin{equation*}
\trc{\mat{A}}=\lambda_n+\lambda_{n-1}+\cdots+\lambda_2+\lambda_1\,.
\end{equation*}
\end{proof}}

A matrix $\mat{A}\in\bar{M}$ is said to be \textsb{nonnegative}\index{matrix!nonnegative} or \textsb{positive-semidefinite}\index{matrix!positive-semidefinite} if
\begin{equation}
\Re\lpa\mat{X}^\dagger\mat{A}\mat{X}\rpa_{11}\geqslant0\,,
\end{equation}
where $\mat{X}$ is a non zero $n\times 1$ matrix. When the inequality imposes the left side to be always positive, $\mat{A}$ is called  \textsb{positive-definite}\index{matrix!positive-definite}. A necessary and sufficient condition that ensures positivity of a matrix is that the spectrum of its hermitian part be constituted of nonnegative elements\footnote{This condition will be verified after the definition of eigenvalue on the next chapter %� p.\pageref{sec:autoPares}
.}, when this matrix results nonnegative; similarly, if these same elements are all positive, the matrix is positive-definite. Thereby, from the two equalities of the previous paragraph, it can be concluded that the determinant and trace of a nonnegative Hermitian matrix are always nonnegative, while for a positive-definite Hermitian matrix both are positive. Therefore, in the context of Hermitian matrices, the determinant value indicates whether a matrix is nonnegative or not. From this conclusion, we can state that \emph{every positive-definite hermitian matrix is invertible}. Moreover, when a positive-definite hermitian matrix $\mat{H}$ pre or post multiplies an arbitrary matrix
$\mat{B}\in\bar{M}$, we have
\begin{equation}
\sgn{\dete{\mat{BH}}}=\sgn{\dete{\mat{HB}}}=\sgn{\dete{\mat{B}}} \sgn{\dete{\mat{H}}}=\sgn{\dete{\mat{B}}}\,,
\end{equation}
where the sign ``\gloref{signum}'' of the determinant of $\mat{B}$ defines the sign of the product of the determinants, making the concept of matrix positivity similar to  scalar positivity, in which the sign of a product is not defined by an eventual positive number.

We shall finish this chapter by presenting the following fundamental equality, a property of any square matrix that will make further developments feasible.

\begin{mteo}{Cayley-Hamilton}{cayleyhamilton}
Let $\bar{M}$ be the set of all square matrices and $\map{\hat{g}}{\bar{M}}{\bar{M}}$ a mapping whose function rule is described by
\begin{equation}
\fua{\hat{g}}{\mat{X}}=(-1)^n\mat{X}^n+a_1\mat{X}^{n-1}+a_2\mat{X}^{n-2}+\cdots+a_{n-1}\mat{X}+a_n\mat{I}\,.
\end{equation}
If the coefficients on this rule equal the coefficients of the characteristic polynomial of a square matrix $\mat{H}$, then the matrix $\fua{\hat{g}}{\mat{H}}$ is null.
\end{mteo}


{\footnotesize
\begin{proof}
In order to verify this theorem, we need to present some preliminary definitions. There is an algorithm \footnote{After \aut{Knuth}\cite{knuth_1997_1}, the word ``algorithm'' has the same etymological source of ``algebra''.}, called \textsb{Laplace Expansion}\index{Expansion!Laplace} or \textsb{Cofactorial Expansion}\index{Expansion!Cofactorial}, that is used in certain cases to find determinants. Here it is: given a square matrix $\mat{A}$, we can obtain for an arbitrary line $i$ that
\begin{equation*}
\dete{\mat{A}} = \sum_{j=1}^{n}\mat{A}_{ij} \underbrace{\lpa-1\rpa^{i+j}\dete{\mat{M}_{(ij)}}}_{\mat{C}_{ij}}\,,
\end{equation*}
where $\mat{C}$ is called \textsb{cofactor matrix}\index{matrix!cofactor} of $\mat{A}$ and $\mat{M}_{(ij)}$ is a square matrix of dimension $n-1$ which results from removing the line $i$ and the column $j$ from $\mat{A}$. The matrix $\mat{C}^{T}$ is called the \textsb{adjugate matrix}\index{matrix!adjugate} of $\mat{A}$, represented by $\gloref{adju}{\mat{A}}$. Now, considering $\mat{A}=\mat{H}-x\mat{I}$ and the property $\adj{\mat{X}}\mat{X}=(\dete\mat{X})\mat{I}$, we have that
\begin{equation*}
\adj{\mat{A}}(\mat{H}-x\mat{I})=\fua{g}{x}\mat{I}=\mat{I}(-1)^nx^n+\mat{I}a_1x^{n-1}+\cdots+\mat{I}a_{n-1}x+\mat{I}a_n\,.
\end{equation*}
Through a tedious development, it can be obtained that the adjugate of $\mat{A}$ results a polynomial of order $q$, with matrix coefficients $\mat{H}_i$, described in
\begin{equation*}
\adj{\mat{A}} = \mat{H}_1x^q+\mat{H}_2x^{q-1}+\cdots+\mat{H}_{n-1}x+\mat{H}_{n}\,.
\end{equation*}
The product
\begin{equation*}
\adj{\mat{A}}(\mat{H}-x\mat{I})=-\mat{H}_1x^{q+1}+(\mat{H}_1\mat{H}-\mat{H}_2)x^q+\cdots+
(\mat{H}_{n-2}\mat{H}-\mat{H}_{n-1})x+\mat{H}_n\mat{H}\,.
\end{equation*}
Comparing the two expressions on the right that equal $\adj{\mat{A}}(\mat{H}-x\mat{I})$, we can conclude that integer $n=q+1$ and the following equalities:
\begin{align*}
\mat{I}(-1)^n&=-\mat{H}_1\\
\mat{I}a_1&=\mat{H}_1\mat{H}-\mat{H}_2\\
\vdots\\
\mat{I}a_{n-1}&=\mat{H}_{n-2}\mat{H}-\mat{H}_{n-1}\\
\mat{I}a_{n}&=\mat{H}_n\mat{H}\,.
\end{align*}
If we post-multiply the sequence of equalities successively by $\mat{H}^n,\mat{H}^{n-1},\cdots,\mat{H},\mat{H}^0$ and adding all of them, we arrive at
\begin{equation*}
(-1)^n\mat{H}^n+a_1\mat{H}^{n-1}+a_2\mat{H}^{n-2}+\cdots+a_{n-1}\mat{H}+a_n\mat{I}=0\,,
\end{equation*}
that is, $\fua{\hat{g}}{\mat{H}}=0$.
\end{proof}}











%%% Local Variables:
%%% mode: latex
%%% TeX-master: "../../msav.tex"
%%% End:

    
\chapter{A Primer on Linear Algebra}

A set is called \textsb{space}\index{space} when it is structured by another set, by an operation or by some relevant property to which all of its elements are subjected.
In the previous chapter, we created a space with an additive structure, called an additive group, and cumulatively assigned to this space a multiplicative structure, when it became a field. In this chapter, the cumulative structuring of these specific spaces is developed, now using fields, norms, metrics and inner products as structural entities. Firstly, we shall gather the concepts of additive group and field in such a way that, from this interaction, scalars end up assigning certain multiplicative properties to group elements, namely, abbreviation of repetitive additions, positivity and negativity. Regarding the relationships between these spaces, Linear Algebra deals mainly with specific homomorphic functions in which scalars are considered and structures preserved.


\section{Structuring by Field}\label{sec:espacoVet}

The group-field space is the fundamental object of Linear Algebra and the interaction between these two sets is subjected to restrictions. In order to present them, let's mathematically describe and complement what we have said so far. Let $V$ be an additive group structured by a field $\mathcal{R}$ through the function $p$ in mapping $\map{p}{\mathcal{R}\times\con{V}}{\con{V}}$. This function, whose values $\fua{p}{\alpha,\gloref{vetor}}$ are represented by $\alpha\vto{x}$ or $\vto{x}\alpha$, must obey the following axioms:
\begin{itemize}
\setlength\itemsep{.1em}
    \item[i.] $\alpha \lpa \vto{x} + \vto{y} \rpa = \alpha\vto{x}+\alpha\vto{y}$;
    \item[ii.] $\lpa \alpha + \beta\rpa  \vto{x} = \alpha\vto{x}+\beta\vto{x}$;
    \item[iii.] $\lpa \alpha\beta\rpa  \vto{x} = \alpha\lpa\beta\vto{x}\rpa$;
    \item[iv.] $1\vto{x}= \vto{x}$, where 1 is the multiplicative identity of $\mathcal{R}$;
    \item[v.] $ 0\vto{x} = \vto{0}$, where $\vto{0}$ is the null element of $\con{V}$;
\end{itemize}
where $\alpha,\beta\in \mathcal{R}$ and $\vto{x}, \vto{y}\in V$ are any elements of their respective sets. Under these conditions, an element of $V$ is named  \textsb{vector}\index{vector} and the triple $(V,\mathcal{R},p)$ is a \textsb{vector space}\index{space!vector} of $V$ in $\mathcal{R}$, whose representation is abbreviated by the symbol $\gloref{espacVet}$, which will be treated from now on as a set, in order to simplify notation. If the field $\mathcal{R}$ is complex, the vector space $V_\complexo$ is said to be \textsb{complex}\index{space!complex vector}, for any group $V$; similarly, $V_\real$ is called a \textsb{real vector space}\index{space!real vector}. From the previous definitions, we can conclude that if a field is group, then $\mathcal{R}_{\mathcal{R}}$ or $\mathcal{R}$ is also a vector space.

As a set admits a subset under the conditions already presented, spaces admit subspaces. A \textsb{vector subspace}\index{vector subspace}, structured by the field $\mathcal{R}$, is actually a vector space in $\mathcal{R}$ whose elements also belong to a set that defines a vector space in $\mathcal{R}$. In more precise terms, we say that the vector space $(S,\mathcal{R},\tilde{p})$, where $\map{\tilde{p}}{\mathcal{R}\times S}{S}$, is a vector subspace of $V_\mathcal{R}$ if the set $S\subseteq V$ or, in a detailed notation, if the space $S_\mathcal{R}\subseteq V_\mathcal{R}$. It is important to say that as all vector spaces are defined to have a null element $\vto{0}$, then $\vto{0}$ must belong to any vector subspace.


The possibility of multiplication by scalars, according to the mapping defined by $p$, enables us to combine the vectors of  $\con{\tilde{U}}=\lch \vto{v}_1,\vto{v}_2,\cdots,\vto{v}_n \rch\subset V_\mathcal{R}$ as in
\begin{equation}
\alpha_1\vto{v}_1+\alpha_2\vto{v}_2+\cdots+\alpha_n\vto{v}_n\,,
\end{equation}
where $\alpha_i$ are any scalars of $\mathcal{R}$. Thereby, this expression is called \emph{the} \textsb{linear combination}\index{vector!linear combination of} of $\con{\tilde{U}}$ in $\mathcal{R}$ and, when the scalars are given, the vector $\sum_{i=1}^n \alpha_i\vto{v}_i$ is said to be \emph{a} linear combination of $\con{\tilde{U}}$ in $\mathcal{R}$. Considering $n>1$, if the zero vector is a linear combination of $\con{\tilde{U}}$ when at least one of the scalars $\alpha_1,\cdots,\alpha_n$ is not null, then we classify $\con{\tilde{U}}$ as \textsb{linearly dependent}\index{linear dependence}. In this case, admitting that $\alpha _1\neq 0$, from the equality $\sum_{i=1}^n \alpha_i\vto{v}_i=\vto{0}$, we can write that $\vto{v}_1=\sum_{i=2}^n (\alpha_i/a_1)\vto{v}_i$, where $\vto{v}_1$ is said to be a linear combination of the other vectors. However, this linear combination of vectors can not be written when a sequence of null scalars is the only possible sequence to make any linear combination of $\con{\tilde{U}}$ equals the zero vector. In this context, if the vectors of $\con{\tilde{U}}$ are not zero, this set is called \textsb{linearly independent}\index{linear independence}.



Recalling our definition of vector space, it is important to observe that the multiplication by scalar defined in mapping $\map{p}{\mathcal{R}\times V}{V}$ together with the operation $\map{+}{V^2}{V}$, typical of additive groups, assure that every linear combination of any vectors of $V_\mathcal{R}$ is also a vector of $V_\mathcal{R}$; that is, if $n$ vectors $\vto{v}_i\in V_\mathcal{R}$, then the vector $\sum_{i=1}^n\alpha_i\vto{v}_i\in V_\mathcal{R}$. In this context, let $U$ be a non empty subset of $V_\mathcal{R}$, described the following way:
\begin{equation}
U=\bigcup_{i=1}^\infty \tilde{U}_i\,,
\end{equation}
where each set $\tilde{U}_i\subset U$ is finite. Thereby, the subset of $V_\mathcal{R}$ constituted by all linear combinations of the subsets $\tilde{U}_i$ is called a \textsb{span}\index{set!span of} of $U$, whose representation is  $\gloref{sconjGer}$. In other words,
\begin{equation}
\spn (U) := \lch \sum_{i=1}^n \alpha_i\vto{v}_i \,:\, \forall n\in \mathbb{N}\,,\,\,\forall \alpha_i \in \mathcal{R},\,\,\forall \vto{v}_i \in U \rch\,.
\end{equation}
If $\spn (U)$ is spanned or generated by $U$, then we can also say that $U$ spans or generates $\spn (U)$. Now, let's take any two elements of the subset spanned by $U$, namely the vectors $\vto{x}=\sum_{i=1}^n \varphi_i\vto{v}_i$ and $\vto{y}=\sum_{i=1}^n \beta_i\vto{v}_i$, where $\varphi_i,\beta_i\in\mathcal{R}$. Adding these two vectors results the vector $\vto{x}+\vto{y}=\sum_{i=1}^n (\varphi_i+\ele{\beta}_i)\vto{v}_i$, which is also an element of $\spn (U)$, since $\varphi_i+\beta_i\in\mathcal{R}$ and $\vto{v}_i\in U$; that is, the operation of addition can be defined by the mapping  $\map{+}{\spn (U)^2}{\spn (U)}$. Moreover, the product of any scalar $\alpha\in\mathcal{R}$ and $\vto{x}$ results $\alpha\vto{x}=\sum_{i=1}^n \alpha\varphi_i\vto{v}_i\in\spn{(U)}$, since
$\alpha\varphi_i\in\mathcal{R}$; which proves the multiplication $\map{p}{\mathcal{R}\times\spn (U)}{\spn (U)}$. From these facts, it is easily verified that $\spn (U)$ observes the five axioms of vectors spaces presented above; which permits us to conclude that the subset spanned by $U$ defines a vector space $\spn{(U)}_\mathcal{R}\subseteq V_\mathcal{R}$. Therefore, we can state generically that every spanned subset defines a \textsb{spanned subspace}\index{subspace!spanned}.


Considering the previous conditions in the case where $U$ spans the space $V_\mathcal{R}$ as a whole, we define the following: a) if $U$ is finite, $V_\mathcal{R}$ is said to be a  \textsb{finite-dimensional}\index{vector space!finite-dimensional} vector space; b) if $U$ is linearly independent, it is called a \textsb{basis}\index{vector space!basis of} of $V_\mathcal{R}$. Gathering these two definitions, when $U$ is a basis with $n$ elements that spans a finite-dimensional  $V_\mathcal{R}$, any vector $\vto{w}\in V_\mathcal{R}$ is generated by one and only one linear combination $\sum_{i=1}^n\alpha_i\vto{v}_i$. Therefore, in the context of the basis $U$, there is a biunivocal relationship between the vector $\vto{w}$ and the $n$-tuple $(\alpha_i,\cdots,\alpha_n)$, whose ordering follows the sequence of the basis vectors. This $n$-tuple of scalars that defines vector $\vto{w}$ on the basis $U$ is named the \textsb{coordinates}\index{vector!coordinates of} of $\vto{w}$ on $U$.

{\footnotesize
\begin{proof}
Let's verify if it is true that $\sum_{i=1}^n\alpha_i\vto{v}_i$ is the only linear combination that defines $\vto{w}$ on $U$. If there were another linear combination $\sum_{i=1}^n\beta_i\vto{v}_i$ defining $\vto{w}$, then the difference between them would be $\sum_{i=1}^n(\alpha_i-\beta_i)\vto{v}_i=\vto{0}$. As $U$ does not have a zero element, from the previous equality we have $\alpha_i-\beta_i=0$, or $\alpha_i=\beta_i$.
\end{proof}}


Now, let's consider $\con{U}_1=\lch \vto{v}_1,\cdots,\vto{v}_n \rch$ a basis of $V_\mathcal{R}$ and $\con{U}_2=\lch \vto{w}_1,\cdots,\vto{w}_m \rch$ a linear independent set such that $m \geqslant n$. If $U_1$ spans $V_\mathcal{R}$, then the linear dependent set $\{\vto{w}_1\}\cup U_1=\lch \vto{w}_1,\vto{v}_1,\cdots,\vto{v}_{n} \rch$ also spans $V_\mathcal{R}$. When an element $\vto{v}_k$ is removed from $U_1$, the resulting set $(\{\vto{w}_1\}\cup U_1)\setminus \{\vto{v}_k\}$ also spans $V_\mathcal{R}$ because $\vto{w}_1$ is a linear combination of $U_1$. If we proceed including elements of $U_2$ and removing elements of $U_1$, we shall obtain the set $\lch \vto{w}_1,\cdots,\vto{w}_{n} \rch$, which is a basis of $V_\mathcal{R}$. Thereby, linearly independent sets which span the same finite-dimensional vector space have the same number of vectors. From this general statement, we can say that every basis of $V_\mathcal{R}$ has $n$ elements, or that the \textsb{dimension}\index{vector space!dimension of} of $V_\mathcal{R}$ is $n$, written $\gloref{dimen}=n$. Therefore, we can also state that \emph{every subset of $V_\mathcal{R}$ having $n$ linearly independent vectors is a basis of $V_\mathcal{R}$}, from which results the following: if $W_\mathcal{R}\subset V_\mathcal{R}$ then $\dim (W_\mathcal{R}) <  \dim (V_\mathcal{R})$ .


There is an important type of vector space whose group is additionally structured by what is called a \textsb{norm}\index{norm}, which assigns to each one of the group elements a non negative real number that enables the concept of vector size or vector intensity. Like the case of structuring by field, structuring by norm also occurs according to some restrictions. Thereby, we say that a \textsb{normed space}\index{space!normed} is defined by the double $(V_\mathcal{R},\eta)$, where $V_\mathcal{R}$ is a vector space and the function in $\map{\eta}{V_\mathcal{R}}{\gloref{realNNeg}}$, called norm, observes the axioms
\begin{itemize}
	\setlength\itemsep{.1em}
	\item[i.] Of definition: $\fua{\eta}{\vto{v}}=0 \Leftrightarrow
\vto{v}=\vto{0}$;
	\item[ii.] Of homogeneity:
$\fua{\eta}{\alpha\vto{x}}=|\alpha|\fua{\eta}{\vto{x}}\text{ }$ and
	\item[iii.] Of triangular inequality\index{inequality!triangular}: $\fua{\eta}{\vto{x}+\vto{y}}\leq
\fua{\eta}{\vto{x}}+\fua{\eta}{\vto{y}}$;
\end{itemize}
where $\alpha\in\mathcal{R}$ and $\vto{x},\vto{y}\in V_\mathcal{R}$ are any elements of their respective sets. On the last item, the triangular inequality axiom imposes that the size of a vector sum is never greater than the sum of vector sizes. In notational terms, as the use of $\eta$ is not very common, $\gloref{norma}$ is also written to represent the value $\fua{\eta}{\vto{x}}$.

We define that two vectors project on each other or have a projective interrelationship when it is possible to describe one in terms of the other. In more precise terms, given the non null vectors $\vto{u},\vto{v}\in U_\mathcal{R}$, it is said that $\vto{u}$ projects on $\vto{v}$ if there is a vector multiple of $\vto{v}$ and function of $\vto{u}$, that is, if there is a mapping $\map{f}{U_\mathcal{R}}{U_\mathcal{R}}$ where the vetor $\fua{f}{\vto{u}}=\alpha\vto{v}$, $\alpha\in\mathcal{R}$. As a unary operator, $f$ is a bijection and then a projection results commutative: if $\vto{u}$ projects on $\vto{v}$, $\vto{v}$ projects on $\vto{u}$. The projective interrelationship of two vectors is usually expressed by scalar values, where zero value means that there is no projection between these vectors or that they are \textsb{orthogonal}\index{vectors!orthogonal}.  Let the function in $\map{\xi}{U_\mathcal{R}\times U_\mathcal{R}}{\mathcal{R}}$ express a projective relationship between any pair of vectors of $U_\mathcal{R}$, observing the axioms
\begin{itemize}\label{prop:produtoInterno}
	\setlength\itemsep{.1em}
	\item[i.] Of positivity: $\fua{\xi}{\vto{u},\vto{u}}\in \real^+$;
	\item[ii.] Of definition: $\fua{\xi}{\vto{u},\vto{u}}=0 \Leftrightarrow
	\vto{u}=\vto{0}$;
	\item[iii.] Of conjugate symmetry: $\fua{\xi}{\vto{u}_1,\vto{u}_2}=\overline{\fua{\xi}{\vto{u}_2,\vto{u}_1}}\,\,$;
	\item[iv.] Of linearity\footnote{See definition at p. \pageref{def:linear}.} in the first argument:\begin{equation*}
	\fua{\xi}{\alpha_1\vto{u}_1+\alpha_2\vto{u}_2,
		\vto{u}_3} =
	\alpha_1\fua{\xi}{\vto{u}_1,\vto{u}_3} + \alpha_2\fua{\xi}{\vto{u}_2,\vto{u}_3}\text{ and }
	\end{equation*}
	\item[v.] Of conjugate linearity in the second argument:\begin{equation*}
\fua{\xi}{\vto{u}_1,\alpha_2\vto{u}_2+
	\alpha_3\vto{u}_3} =
\overline{\alpha_2}\fua{\xi}{\vto{u}_1,\vto{u}_2} + \overline{\alpha_3}\fua{\xi}{\vto{u}_1,\vto{u}_3}\,;
	\end{equation*}
\end{itemize}
where $\vto{u}_1,\vto{u}_2,\vto{u}_3\in U_\mathcal{R}$ and  $\alpha_1,\alpha_2,\alpha_3\in\mathcal{R}$ are any elements of their respective sets. In this context, the function $\xi$ is called a \textsb{positive-definite inner product} because the projection of a vector on itself is a non negative real number, as described by the first axiom. In our study, $\xi$ is simply called an inner product\index{inner product}, and the double $(U_\mathcal{R},\xi)$ an \textsb{inner product space}\index{space!inner product}. From this double, we conclude that $\xi$ structures the group $U$ in such a way that a projective interrelationship of any pair of its elements can be obtained.
Henceforth, in order to shorten notation, $\gloref{prdint}$ will also be used to represent the inner product $\fua{\xi}{\vto{x},\vto{y}}$.

Considering the inner product space $(U_\mathcal{R},\xi)$, it is now possible to present in more mathematical terms the definition of orthogonality: any vectors $\vto{u}_1,\vto{u}_2\in U_\mathcal{R}$ are said to be orthogonal, or $\vto{u}_1\gloref{perpend}\vto{u}_2$, when $\vto{u}_1\cdot\vto{u}_2=0$. Given the subsets $U_1\subset U_\mathcal{R}$ and $U_2\subset U_\mathcal{R}$, if $\vto{u}\in U_\mathcal{R}$ is orthogonal to any vector of $U_1$, we write $\vto{u}\perp\con{U}_1$, and if any vetor of $\con{U}_1$ is orthogonal to any vector of $\con{U}_2$, we write $\con{U}_1\perp\con{U}_2$. A set $U_3=\{\vto{u}_1,\cdots,\vto{u}_n\}\subset U_\mathcal{R}$ is called orthogonal if $\vto{u}_i\perp\vto{u}_j$, $i\neq j$. Thereby, when the vectors of $U_3$ are non null, the inner product of each side of $\alpha\vto{u}_j=\vto{u}_i$ and $\vto{u}_j$, where $\alpha\in\mathcal{R}$ and $i\neq j$, the result is $\alpha\,(\vto{u}_j\cdot\vto{u}_j)=0$, from where we conclude that $\alpha=0$ or that every orthogonal set is linearly independent. This conclusion permits us to state that in a \emph{$n$-dimensional inner product space, every orthogonal subset of $n$ elements is a basis}.

From a cumulative structuring of a set $V_\mathcal{R}$ by norm and inner product, it is possible to define a triple $(V_\mathcal{R},\eta,\xi)$, called a \textsb{normed inner product space}\index{space!normed inner product}. In these spaces, the projective interrelationship of vectors, expressed by the inner product, can be used to define a norm  according the generic rule
\begin{equation}
 \fua{\eta}{\vto{x}} = \fua{g\circ\xi}{\vto{x},\vto{x}},
\end{equation}
where the function in $\map{g}{\real^+}{\real^+}$ enables us to say that \emph{the norm is induced by the inner product}\footnote{When the inner product induces the norm, some authors consider the inner product space involved as implicitly being a normed space.}. A very important property, called \textsb{Cauchy-Schwarz Inequality}\index{inequality!Cauchy-Schwarz}, valid for every normed inner product space whose inner product induces the norm through $\fua{g}{x}=\sqrt{x}$, assures that the value of a projection is never greater than the product of the sizes of the vectors involved, that is,
\begin{equation}
|\, \vto{v}_1\cdot\vto{v}_2| \leqslant
\|\vto{v}_1\|\,\,\|\vto{v}_2\|\,,\,\forall\,\vto{v}_1,\vto{v}_2\in\con{V}_\mathcal{R}\,.
\end{equation}
{\footnotesize
\begin{proof} If one of the vectors is null, the equality is straightforward. Now, let $\vto{v}=\vto{v}_1-\lambda\vto{v}_2$ be a vector where $\vto{v}_2\neq \vto{0}$ and $\lambda=(\vto{v}_1\cdot\vto{v}_2)/\|\vto{v}_2\|^2$. From the conjugate symmetry property of the inner product and knowing that the conjugate of the product is the product of the conjugates,
\begin{align*}
	0&\leqslant\vto{v}\cdot\vto{v}\\
	0&\leqslant\vto{v}_1\cdot\vto{v}_1-\vto{v}_1\cdot\lambda\vto{v}_2-\lambda\vto{v}_2\cdot\vto{v}_1+\lambda\vto{v}_2\cdot\lambda\vto{v}_2\\
	0&\leqslant\vto{v}_1\cdot\vto{v}_1-\overline{\lambda}\vto{v}_1\cdot\vto{v}_2-\lambda\overline{\vto{v}_1\cdot\vto{v}_2}+\lambda\overline{\lambda}\vto{v}_2\cdot\vto{v}_2\\
	0&\leqslant\|\vto{v}_1\|^2-\dfrac{|\vto{v}_1\cdot\vto{v}_2|^2}{\|\vto{v}_2\|^2}-\dfrac{|\vto{v}_1\cdot\vto{v}_2|^2}{\|\vto{v}_2\|^2}+\dfrac{|\vto{v}_1\cdot\vto{v}_2|^2}{\|\vto{v}_2\|^4}\|\vto{v}_2\|^2\\
	\|\vto{v}_1\|^2&\geqslant\dfrac{|\vto{v}_1\cdot\vto{v}_2|^2}{\|\vto{v}_2\|^2}\\
	\|\vto{v}_1\|\,\,\|\vto{v}_2\|&\geqslant |\vto{v}_1\cdot\vto{v}_2|\,\,.
\end{align*}
\end{proof}
}

Any two vectors $\gloref{unita}$ and $\vun{u}_2$ of $(V_\mathcal{R},\eta,\xi)$ are said to be orthonormal if they are orthogonal and each one is \textsb{unitary}\index{vector!unitary}, where $\|\vun{u}_i\|=1$. Thereby, if the vector space $V_\mathcal{R}$ is $n$-dimensional, a subset $\hat{U}=\{\vun{u}_1,\vun{u}_2,\cdots,\vun{u}_n\}$ of orthonormal vectors is called an \textsb{orthonormal basis}\index{basis!orthonormal} of $V_\mathcal{R}$. It is important to note that to every orthogonal basis $\{\vto{u}_1,\vto{u}_2,\cdots,\vto{u}_n\}$, there is always an orthonormal basis $\{\vun{u}_1,\vun{u}_2,\cdots,\vun{u}_n\}$ where each $\vun{u}_i:=\vto{u}_i/\|\vto{u}_i\|$, when we say that the orthonormal basis results from the \textsb{normalization}\index{basis!normalization of} of the orthogonal basis.


\section{Structuring by Metrics}


If the group-field interaction assigns to the group certain multiplicative features, a set that is structured by metrics carries with it the concept of distance. In other words, in a set-metrics space or a \textsb{metric space}\index{space!metric}, there is always a distance between two elements, measured in scalar values. This idea of distance is fundamental in Mathematics, making, for example, the usual notion of derivative viable and consequently of elementary Differential Calculus as a whole.


Like structuring by field, the structure of metrics in a set is also subjected to restrictions, described as follows. Let
$\con{A}$ be a set and $\map{\varrho}{\con{A}\times\con{A}}{\real}$ a mapping. Given any elements $a_1,a_2,a_3\in A$, the double $(A,\varrho)$ is said to be a metric space and the function $\varrho$ a \textsb{metric}\index{metric} or a \textsb{distance function}\index{function!distance} if it observes the axioms
\begin{itemize}
	\setlength\itemsep{.1em}
	\item[i.] Of positivity: $\fua{\varrho}{\ele{a}_1,\ele{a}_2}\geqslant 0$\,;
	\item[ii.] Of definition: $\fua{\varrho}{\ele{a}_1,\ele{a}_2}=0 \Leftrightarrow \ele{a}_1=\ele{a}_2$\,;
	\item[iii.] Of commutativity: $\fua{\varrho}{\ele{a}_1,\ele{a}_2}=\fua{\varrho}{\ele{a}_2,\ele{a}_1}$\,\,\,and
	\item[iv.] Of triangular inequality: $\fua{\varrho}{\ele{a}_1,\ele{a}_2}\leq\fua{\varrho}{\ele{a}_1,\ele{a}_3}+\fua{\varrho}{\ele{a}_3,\ele{a}_2}$\,.
\end{itemize}
If distances are intuitively seen as paths, from the last axiom we can state that the distance from $a_1$ to $a_2$ always establishes the shortest path between these two elements. Moreover, the existence of metric spaces enables us to call the function in a bijective mapping $\map{f}{A}{B}$, where the sets define $(A,\varrho_A)$ and $(B,\varrho_B)$, an \textsb{isometry}\index{isometry} when $\fua{\varrho_A}{\ele{a}_1,\ele{a}_2}=\fua{\varrho_B}{\fua{f}{\ele{a}_1},\fua{f}{\ele{a}_2}}$. In other words, an isometry preserves distances between the elements of its domain.


From the above definitions, many new concepts arise concerning the study of spaces structured by metrics. Among these concepts, we shall present hereafter those involved in the definition of ``continuum'', a space of fundamental relevance in our study. Let's start by considering a metric space $(A,\varrho)$, an element $a\in A$ and a scalar $r\in \real$, from which the set
\begin{equation}
\overline{B}_{\ele{a},\ele{r}}:=\lch \ele{x}\in\con{A} \,:\,
\fua{\varrho}{\ele{a},\ele{x}}\leqslant r \rch
\end{equation}
is said to be a \textsb{closed ball}\index{ball!closed} with center $a$ and radius $r$. It is then a subset of $A$ delimited by a ``spheric" set whose elements belong to this subset. When such sphere is not included in the subset, as is the case with
\begin{equation}
B_{\ele{a},\ele{r}}:=\lch \ele{x}\in\con{A} \,:\,
\fua{\varrho}{\ele{a},\ele{x}}< r \rch\,,
\end{equation}
we call $B_{\ele{a},\ele{r}}$ an \textsb{open ball}\index{ball!open} with center $a$ and radius $r$. Thereby, the sphere itself, also with center $a$ and radius $r$, can be defined as follows:
\begin{equation}
\partial B_{\ele{a},\ele{r}}:=\lch \ele{x}\in\con{A} \,:\,
\fua{\varrho}{\ele{a},\ele{x}}= r \rch\,.
\end{equation}

\begin{figure}[!ht]
	\centering
	\begin{center}
		\scalebox{.72}{\input{partes/parte1/figs/c_algabst/bolas.pstex_t}}
	\end{center}
	\titfigura{Sphere, closed and open balls.}\label{fg:bolas}
\end{figure}


A subset $A_1$ of $A$ is said to be \textsb{open}\index{set!open} in $A$ if any of its elements is the center of an open ball subset of $A_1$, that is, for every $a\in A_1$, there is always a scalar $\ele{r}\in\real$ such that  $\ele{B}_{\ele{a},\ele{r}}\subset A_1$. A set $A_2\subset A$ is
\textsb{closed}\index{set!closed} if its complement is open in $A$. Thereby, we can say that the complement of the open set $A_1$ is closed in $A$. In general terms, open sets, being a generalization of open intervals, are devoid of elements in borders, which refers to the idea of boundaries and interiors. Subsets that results from the union of a boundary and a interior are closed because their complements are open. In mathematical terms, considering a set $A_3\subset A$, there is an \textsb{interior}\index{set!interior of} $\widehat{A}_3$ of $A_3$ defined by
\begin{equation}
\widehat{A}_3=\lch\ele{x}\in A_3\,:\,\exists\,\ele{r}\in\real\text{ where }\ele{B}_{\ele{x},\ele{r}}\subset A_3\rch
\end{equation}
and a \textsb{closure}\index{set!closure of} $\overline{A}_3$ of $A_3$ defined by
\begin{equation}
\overline{A}_3=\lch\ele{x}\in\con{A}\,:\, A_3\cap\ele{B}_{\ele{x},\ele{r}}\neq\emptyset\,,\,\forall\,\ele{r}\in\real\rch\,,
\end{equation}
such that $\partial A_3:=\overline{A}_3\setminus \widehat{A}_3$ is the
\textsb{boundary}\index{set!boundary of} of $A_3$. From these definitions, we can conclude that $A_3$ is open when $A_3=\widehat{A}_3$ and closed when $A_3=\overline{A}_3$. An open subset is called closed-open or \textsb{clopen}\index{set!clopen} when its complement is also open. As an example, the sets $W_1={1,\cdots,2}$ e $W_2={3,\cdots,4}$, defined by intervals of real values, are clopen subsets of $W_1\cup W_2$.
\begin{figure}[!h]
	\centering
	\begin{center}
		\scalebox{.72}{\input{partes/parte1/figs/c_algabst/contorno.pstex_t}}
	\end{center}
	\titfigura{Boundary, closure and interior of $A_3$.}
\end{figure}
It is important to say that the open set and the set $A$, to which the elements of the subsets $A_i$ belong, are defined to be clopen.

{\footnotesize
\begin{proof}
Let's prove that $W_1$ and $W_2$ are clopen in $W:=W_1\cup W_2$. Let $(w-1/2,w+1/2)$ be an open interval in $\real$ where $w\in W$. This interval centered in $w=2$ results $(3/2,2)$, which is also open since there are no elements greater than 2 e less than $5/2$. Through this same process, it is always possible to find an open interval centered in any $w\in W_1$; when we conclude that $W_1$ is open in $W$. By this same reasoning, $W_2$ is also open. But $W_1$ and $W_2$ are also closed because they are each other's open complement in $W$.
\end{proof}}

Still considering the conditions above, the space $(A,\varrho)$ is called \textsb{connected}\index{space!connected} when there is no proper non empty subset that is clopen in $A$; otherwise, the space is said to be \textsb{disconnected}\index{space!disconnected}, as is the case of a metric space defined by $W_1\cup W_2$. In other words, a disconnected space results from the union of disjoint open non empty subsets. Intuitively, we can say that this space is fragmented, constituted by scattered collections of elements.

A metric space $(U,\varrho)$ is called \textsb{bounded}\index{space!bounded} if there are an element $u\in U$ and a scalar $r \in \real$ such that $U\subset B_{{u},r}$. Now we shall restrict this condition a little more, but firstly let $C=\{U_1, U_2,\cdots\}$ be an infinite set constituted by subsets of $U$. We say that $C$ covers $U$ or that $C$ is a \textsb{cover}\index{set!cover of} of $U$ when $U\subseteq \bigcup_{i=1}^\infty U_i$. If there is a finite set $\{ B_{{u_1},r}\,,B_{{u_2},r}\,,\cdots,B_{{u_n}\,,r}\}$, $u_i\in U$ and $r \in \real$, that covers $U$, we say that $(U,\varrho)$ is a \textsb{totally bounded}\index{space!totally bounded} space. Boundedness and, more strongly, total boundedness are restrictions that impose on the space in question a feature of being delimited, from which it is possible to attain the concept of size.

Considering a sequence where distances between its elements decrease as it progresses, there are metric spaces in which every such sequence is convergent. In simple terms, we may say that these spaces result devoid of ``voids'' or completely ``filled''. Mathematically, given a metric space $(V,\varrho)$, in a sequence of elements $v_1,v_2,\cdots,v_n\in V$ where
\begin{equation}
\lim_{\min{\lpa i,j\rpa}\to\infty}{\fua{\varrho}{\ele{v}_i,\ele{v}_j}}=0\,,
\end{equation}
called \textsb{Cauchy Sequence}\index{Cauchy!Sequence}, the infinite decrease of distances is assured. If any Cauchy Sequence in $V$ is convergent, that is, in addition to the limit above, if there is a $v\in V$ where
\begin{equation}
\lim_{i\to\infty}{\fua{\varrho}{\ele{v}_i,\ele{v}}}=0\,,
\end{equation}
the metric space in question is said to be \textsb{complete}\index{space!complete metric}. When a complete metric space is also connected and totally bounded, it is called a        
\textsb{continuum}\index{continuum}. Thereby, for the purposes of our study, \emph{every continuum is a metric space defined by a delimited set that is devoid of ``voids'' and not ``fragmented''.}

\begin{mteo}{Isometry Preserves Completeness}{isoComp}
If an isometry has a complete domain then its image is also complete.
\end{mteo}

{\footnotesize
\begin{proof}
Considering the isometric mapping  $\map{f}{V}{W}$, where the domain $V$ is complete, and $\ele{v}_1,\ele{v}_2,\cdots,\ele{v}_n$ any Cauchy Sequence in $V$, from the definition of isometry, the equalities
\begin{equation*}
\lim_{\min{\lpa i,j\rpa}\to\infty}{\fua{\varrho_V}{\ele{v}_i,\ele{v}_j}}=\lim_{\min{\lpa i,j\rpa}\to\infty}{\fua{\varrho_W}{\fua{f}{\ele{v}_i},\fua{f}{\ele{v}_j}}}=0
\end{equation*}
show that $\fua{f}{\ele{v}_1},\fua{f}{\ele{v}_2},\cdots,\fua{f}{\ele{v}_n}\in \con{R}_{f}$ is also a Cauchy Sequence. Moreover, if $\ele{v}_1,\ele{v}_2,\cdots,\ele{v}_n$ converges to $v$, the equalities
\begin{equation*}
\lim_{i\to\infty}{\fua{\varrho_V}{\ele{v}_i,\ele{v}}}=\lim_{i\to\infty}{\fua{\varrho_W}{\fua{f}{\ele{v}_i},\fua{f}{\ele{v}}}}=0
\end{equation*}
show that every Cauchy Sequence in $\con{R}_{f}$ is convergent.
\end{proof}}


Now, let's bring the concept of distance to the subject of vector spaces, which are the most fundamental constructs of Linear Algebra. A vector space $V_\mathcal{R}$ that is structured by a metric $\varrho$ is defined to be a \textsb{metric vetor space}\index{vector space!metric} $(V_\mathcal{R},\varrho)$. From this definition, specific types of metric and vector spaces already presented can be combined, and then three important spaces arise: a normed complete space or, more briefly, a
\textsb{Banach space}\index{space!Banach}\index{Banach space}, represented by  $(V_\mathcal{R},\varrho,\eta)$, where
\begin{equation}
\fua{\varrho}{\vto{v}_1,\vto{v}_2} := \fua{\eta}{\vto{v}_1-\vto{v}_2}
,\,\forall
\,\vto{v}_1,\vto{v}_2\in\con{V}_\mathcal{R}\,;
\end{equation}
a Banach space with an inner product $(V_\mathcal{R},\varrho,\eta,\xi)$, called a \textsb{Hilbert space}\index{space!Hilbert}\index{Hilbert space}, whose inner product induces the norm through $\fua{\eta}{\vto{x}}=\sqrt{\fua{\xi}{\vto{x},\vto{x}}}$; and a real $n$-dimensional Hilbert space $(V_\real,\varrho,\eta,\xi)$, called \textsb{Euclidean space}\index{space!Euclidean}. In order to avoid notational abuse, all metric vector spaces will henceforth be identified only by the definer vector space: for example, the quadruple $(V_\mathcal{R},\varrho,\eta,\xi)$ will be described by ``the Hilbert space $V_\mathcal{R}$'', where the functions are implied.

\begin{figure}[ht]
\centering
{\small
\begin{forest}
	for tree={align=center,parent anchor=south, child anchor=north}
	[Vector\\$U_\mathcal{R}$
	[Inner Product\\$(U_\mathcal{R}{,}\xi)$ [Normed Inner Product\\$(U_\mathcal{R}{,}\eta{,}\xi)$,name=normProdInt ] ]
	[Normed\\$(U_\mathcal{R}{,}\eta)$,name=normd]
	[Complete Vector\\$(U_\mathcal{R}{,}\varrho)$
	[Banach\\$(U_\mathcal{R}{,}\varrho{,}\eta)$,name=bana [Hilbert\\$(U_\mathcal{R}{,}\varrho{,}\eta{,}\xi)$,name=hilb [Euclidean\\$(U_\real{,}\varrho{,}\eta{,}\xi)$]]]] ]
	]
	\draw (normProdInt)--(normd);
	\draw (bana)--(normd);
	\draw (hilb)--(normProdInt);
\end{forest}
}
\newline
\titfigura{Relevant combinations of vector spaces.}
\end{figure}

\begin{mteo}{Orthogonal Basis in Hilbert Spaces}{temOrtonormal}
	Every Hilbert space has orthogonal basis.
\end{mteo}


{\footnotesize
\begin{proof}
Through a long and tedious proof, starting from the so called \textsb{Zorn's Lemma}\index{Zorn's Lemma}, it is possible to obtain that every Hilbert space has a basis\footnote{See \aut{Kreyszig}\cite{kreyszig_1978_1}.}. Once the existence of a basis is assured, the \textsb{Gram-Schmidt Algorithm}\index{Gram-Schmidt!Algorithm} is able to find an orthogonal set from any other set as follows. Let $U=\{\vto{u}_1,\cdots,\vto{u}_n\}$ be a basis and $X=\{\vto{x}_1,\cdots,\vto{x}_n\}$ a set where $\vto{x}_1=\vto{u}_1$. If  $n=2$, the goal is to find a $\vto{x}_2\perp\vto{x}_1$ that makes $X$ orthogonal. The algorithm proposes that $\vto{x}_2=p_{21}\vto{x}_1+\vto{u}_2$, where $p_{21}:=-(\vto{u}_2\cdot\vto{x}_1)/\|\vto{x}_1\|^2$. It is evident that any vector $\vto{u}=\alpha_1\vto{u}_1+\alpha_2\vto{u}_2$, and then $\vto{u}=(\alpha_1-p_{21})\vto{x}_1+\alpha_2\vto{x}_2$; therefore, $\spn (U)=\spn(X)$. When $n=3$, vector $\vto{x}_3\perp\{\vto{x}_1,\vto{x}_2\}$ is found from $\vto{x}_3=p_{31}\vto{x}_1+p_{32}\vto{x}_2+\vto{u}_3$, where scalar $p_{31}:=-(\vto{u}_3\cdot\vto{x}_1)/\|\vto{x}_1\|^2$ and scalar $p_{32}:=-(\vto{u}_3\cdot\vto{x}_2)/\|\vto{x}_2\|^2$. A vector $\vto{u}=\alpha_1\vto{u}_1+\alpha_2\vto{u}_2+\alpha_3\vto{u}_3$ can be rewritten as the vector $\vto{u}=(\alpha_1-\alpha_2 p_{21}-\alpha_3 p_{31})\vto{x}_1+(\alpha_2-\alpha_3 p_{32})\vto{x}_2+\alpha_3\vto{x}_3$, from which results $\spn (U)=\spn(X)$. This same process can be done for any $n>3$.             
\end{proof}
}

The theorem above also assures the existence of orthonormal bases because they can be obtained from normalization of orthogonal bases. Thereby, let  $\vto{x}$ and $\vto{y}$ be any vectors of Euclidean space $E_\real$, of which $\hat{B}=\{\vun{v}_1,\cdots,\vun{v}_n\}$ is an orthonormal basis. Then, we can say that
\begin{equation}
	\vto{x}\cdot\vto{y}=\sum_{i=1}^{n}\sum_{j=1}^{n}\alpha_i\beta_j\vun{v}_i\cdot\vun{v}_j=\sum_{i=1}^{n}\sum_{j=1}^{n}\alpha_i\beta_j\delta_{ij}=\sum_{i=1}^{n}\alpha_i\beta_i\,,
\end{equation}
where $(\alpha_1,\cdots,\alpha_n)$ and $(\beta_1,\cdots,\beta_n)$ are the coordinates of $\vto{x}$ and $\vto{y}$ respectively. As a consequence of this equality, where inner products of basis vectors do not contribute numerically, a standard orthonormal basis $O=\,$\gloref{baseNatural}, called \textsb{natural basis}\index{basis!natural}, is defined in Euclidean spaces. From this basis, it is possible to say that the scalars  $x_i:=\vto{x}\cdot\vun{e}_i$ constitute the \textsb{natural coordinates}\index{coordinates!natural} \gloref{coordNat} of $\vto{x}$.   


The presence of \textsb{reciprocal sets}\index{sets!reciprocal} is another consequence of the existence of orthogonal sets in Hilbert spaces. We say that $\con{U}=\lch \vto{u}_1,\cdots,\vto{u}_n
\rch$ and $\con{W}=\lch \vto{w}_1,\cdots,\vto{w}_n \rch$, subsets of the Hilbert space $V_\mathcal{R}$, are reciprocal or \textsb{biorthogonal}\index{sets!biorthogonal} if their vectors are non zero and $\vto{u}_i\cdot\vto{w}_j = \delta_{ij}$. As the pair of reciprocal sets is unique, notations relative to one of these sets are usually defined: for instance, a set $U^\perp:=W$ and vectors $\vto{u}^i:=\vto{w}_i$. It is interesting to note that if the subset $U$ is orthonormal, its reciprocal set $U^\perp=U$. Now, considering $B$ a basis of $V_\mathcal{R}$ and $\con{B}^\perp$ its reciprocal set, let $\vto{u}=\sum_{i=1}^n\gamma_i\vto{u}^i$. If this vector $\vto{u}$ is zero, then     
\begin{equation}
(\sum_{j=1}^n\gamma_j\vto{u}^j)\cdot\vto{u}_i\,=\,\sum_{j=1}^n\gamma_j\delta_{ij}\,=\,\gamma_i\,=\,0\,.
\end{equation}
This result shows that $B^\perp$ is linearly independent since the scalars $\gamma_i$ are zero when $\vto{u}=0$. Moreover, as both reciprocal sets have the same number of elements, we can conclude that if one of them is a basis of $V_\mathcal{R}$, so is the other. Thereby, if $(\alpha_1,\cdots,\alpha_n)$ are the coordinates of a vector on basis $B$, we usually use $(\alpha^1,\cdots,\alpha^n)$ to represent the coordinates of this same vector on basis $B^\perp$. 



{\footnotesize
\begin{proof}
Let's verify the uniqueness and existence of reciprocal sets on the context above. From theorem \ref{teo:temOrtonormal}, we can admit an orthogonal subset $Z=\{\vto{z}_1,\cdots,\vto{z}_n\}$. Thereby, let $\{\vto{\tilde{z}}_1,\cdots,\vto{\tilde{z}}_n\}$ be a subset where $\vto{\tilde{z}}_i:=\vto{z}_i/\|\vto{z}_i\|^2$. Therefore, $\vto{z}_i\cdot\vto{\tilde{z}}_j=(\vto{z}_i\cdot\vto{z}_j)/\|\vto{z}_j\|^2=\delta_{ij}$, which proves the existence. Now, supposing that there exists another subset $\{\vto{x}_1,\cdots,\vto{x}_n\}$ reciprocal to $Z$, we can say that $\vto{z}_i\cdot(\vto{\tilde{z}}_j-\vto{x}_j)=0$. As vectors $\vto{z}_i$, $\vto{\tilde{z}}_j$ and $\vto{x}_j$ can not be zero, then $\vto{\tilde{z}}_j=\vto{x}_j$, which proves the uniqueness.
\end{proof}
}




\section{Linear Functions}\label{sec:FuncLin}


The most fundamental relationships studied in Linear Algebra have the feature of preserving the group structures involved, including those defined by fields. Such relationships are expressed by homomorphisms whose main property is to keep vector spaces structures unaltered. Moreover, if these vector spaces are metric, it is required that this additional structure to remain unchanged as well. In practical terms, this means that if the homomorphism domain is a metric vector space, so must be its image. Selected through a criteria of same mapping definition, we study these functions by gathering them in a vector space, where there are additional restrictions concerning the relations to their arguments.       


Let's start the study of linear functions considering first an additive group $V^{U}$ constituted of generic functions that define mappings of the type $U\mapsto V$, where $U$ and $V$ are complete spaces with an additive structure. This group is said to define a vector space \gloref{espacFunc}, usually called a \textsb{function space}\index{space!function}, if for any $\vtf{f},\vtf{g}\in V^U_\mathcal{R}$ and $\alpha\in\mathcal{R}$ the following restrictions are observed:
\begin{itemize}
	\setlength\itemsep{.1em}
	\item[i.] $\fua{\vtf{0}}{x}=0$;
	\item[ii.] $\fua{\lco\alpha\vtf{f}\rco}{x}=\alpha\,\fua{\vtf{f}}{x}$;
	\item[iii.]$\fua{\lco\vtf{f}+\vtf{g}\rco}{x}=\fua{\vtf{f}}{x}+\fua{\vtf{g}}{x}$.
\end{itemize}   
The domain $U$ may eventually be a cartesian product $W^{\times q}$, where any function $\vtf{f}$ of the function space has a $q$-tuple of vectors or $q$ vectors as arguments, and its value is represented by $\fua{\vtf{f}}{w_1,\cdots,w_q}$, where the tuple $(w_1,\cdots,w_q)\in W^{\times q}$ or the vectors $w_i\in W_i$.

An important example of function space is the space constituted by continuous functions. In order to define these type of functions, we need to say firstly that a set $S\subset U$ is called a \textsb{neighborhood}\index{neighborhood} of an element $u\in S$, represented by $\viz{u}$, when there is a real number $r>0$ that defines an open ball $B_{u,r}\subset S$. In this context, the function in $\map{\vtf{g}}{U}{V}$ is said to be 
\textsb{continuous}\index{function!continuous} on an element $u\in S$ if for any neighborhood $\viz{\fua{\vtf{g}}{u}}$ in the codomain there is a neighborhood $\viz{u}$ in the domain where every element $x\in\viz{u}$ is related to a value $\fua{\vtf{g}}{x}\in \viz{\fua{\vtf{g}}{u}}$. In more direct terms, $\vtf{g}$ is continuous on $u$ when      
\begin{equation}
\lim_{x\to u}\fua{\vtf{g}}{x}=\fua{\vtf{g}}{u},\,\, \forall\, x\in U\,,
\end{equation}
that is, when $x\to u$ implies $\fua{\vtf{g}}{x}\to\fua{\vtf{g}}{u}$. In the case of a function that is continuous on any element of the domain, it is called continuous on the domain or simply continuous. Moreover, if a bijection and its inverse function are continuous on their respective domains, each one is called a  \textsb{homeomorphism}\index{homeomorphism}\footnote{Not to be confused with homomorphism, without ``e''.}.


There is also a particular type of function continuity that has a stronger restriction than that presented above: a function $\vtf{g}$ is said to be \textsb{Lipschitz continuous}\index{function!Lipschitz continuous} on $u$ if there exists a non zero number $\vartheta\in\real^+$, called \textsb{Lipschitz constant}\index{Lipschitz!constant}, where
\begin{equation}
\vartheta \geqslant\dfrac{ \fua{\varrho}{\fua{\vtf{g}}{x},\fua{\vtf{g}}{u}}}{\fua{\varrho}{x,u}}\,,\forall\, x\in \{U\setminus \{u\}\}\,.
\end{equation}
From this definition we can conclude that every Lipschitz continuous function is also continuous, with the property of presenting upper limited distance ratios relative to every element $u$ of its domain. 


Now, let a function $\vto{h}\in V^U_\mathcal{R}$ be a homomorphism through which the additive structure of $U$ is preserved. 


 Seja ent�o a fun��o $\vto{h}\in V^U_\mathcal{R}$ um homomorfismo pelo qual a estrutura aditiva do grupo $U$ fica preservada. O homomorfismo precisa preservar tamb�m a estrutura criada pelo campo $\mathcal{R}$ de tal forma que
\begin{equation}
\fua{\vtf{h}}{\alpha\vto{u}_1+\beta\vto{u}_2}=\alpha\fua{\vtf{h}}{\vto{u}_1}+\beta\fua{\vtf{h}}{\vto{u}_2}
\end{equation}
e $\fua{\vtf{h}}{\vto{0}}=\vto{0}$, para quaisquer $\alpha,\beta\in\mathcal{R},\,\vto{u}_1,\vto{u}_2\in U$. Nessas condi��es, dizemos que $\vtf{h}$ � uma \textsb{fun��o linear}\index{fun��o!linear}\label{def:linear} e o mapeamento por ela definido uma \textsb{transforma��o linear}\index{transforma��o!linear}. Se o espa�o de fun��es $V^U_\mathcal{R}$ for constitu�do apenas por fun��es lineares, costuma-se represent�-lo pela nota��o $\gloref{evl}$. Agora, para o caso de $U_\mathcal{R}=W^{\times q}_\mathcal{R}$, uma fun��o $\vtf{k}$ � dita \textsb{multilinear}\index{fun��o!multilinear}, ou  \textsb{bilinear}\index{fun��o!bilinear} se $q=2$,  quando
\begin{align}
\lefteqn{\fua{\vtf{k}}{\vto{w}_1,\cdots,\alpha\vto{w}_i+ \beta\vto{w},\cdots,\vto{w}_q}=} & & \nonumber\\
& &\alpha\fua{\vtf{k}}{\vto{w}_1,\cdots,\vto{w}_i,\cdots,\vto{w}_q}+\beta\fua{\vtf{k}}{\vto{w}_1,\cdots,\vto{w},\cdots,\vto{w}_q}
\end{align}
e $\fua{\vtf{k}}{\vto{0},\cdots,\vto{0}}=\vto{0}$, para quaisquer $\alpha,\beta\in\mathcal{R}$ e $\vto{w},\vto{w}_i\in W_i$. Nesses termos, sendo o dom�nio   $U_\mathcal{R}=V_\mathcal{R}^q$, os vetores de $V^{V^q}_\mathcal{R}$ resultam \textsb{operadores multilineares}\index{operador!multilinear} e os mapeamentos que definem s�o \textsb{opera��es multilineares}\index{opera��o!multilinear}. Todas essas fun��es lineares podem ser cont�nuas e constituir um espa�o normado de fun��es se for definida uma norma. Nesse sentido, considerando $Z_\mathcal{R}$ e $Y_\mathcal{R}$ espa�os de Banach, uma condi��o necess�ria e suficiente para que uma fun��o linear $\vtf{h}$ seja cont�nua em $Z_\mathcal{R}$ imp�e que ela seja \textsb{limitada}\index{fun��o!limitada}, isto �, que exista um $\nu\in\real^+$ onde
\begin{equation}\label{eq:funcaoLimitada}
\nu\geqslant \|\fua{\vtf{h}}{\vto{z}}\|/\|\vto{z}\| \,,\forall\, \vto{z}\in \{Z_\mathcal{R}\setminus\{\vto{0}\}\}\,.
\end{equation}
A partir da�, $Y_\mathcal{R}^Z$ torna-se um espa�o normado de fun��es lineares cont�nuas quando a norma � definida como sendo o menor dos valores de $\nu$, ou pela regra
\begin{equation}\label{eq:normaFuncao}
\fua{\eta}{\vtf{x}}=\sup\lch \|\fua{\vtf{x}}{\vto{z}}\|/\|\vto{z}\| \,,\forall\, \vto{z}\in \{Z_\mathcal{R}\setminus\{\vto{0}\}\} \rch\,.
\end{equation}
Para o nosso estudo, \emph{conv�m que o espa�o vetorial de fun��es lineares cont�nuas $Y_\mathcal{R}^Z$, al�m de normado, seja tamb�m m�trico produto interno, onde $\fua{\varrho}{\vtf{h}_1,\vtf{h}_2} := \|\vtf{h}_1-\vtf{h}_2\|$ e o produto interno\label{txt:prodInt} induza a norma segundo  $\|\vtf{h}\|:=\sqrt{\vtf{h}\cdot\vtf{h}}\,\,$}.

Considerando as condi��es anteriores, se o espa�o destino $V_\mathcal{R}=\mathcal{R}_\mathcal{R}$, um elemento $\vtf{f}\in \mathcal{R}^{U_\mathcal{R}}$ que define o mapeamento $\map{\vtf{f}}{U_\mathcal{R}}{\mathcal{R}_\mathcal{R}}$ � denominado \textsb{funcional}\index{funcional}. Em termos menos rigorosos, podemos afirmar que \emph{o funcional mapeia um espa�o vetorial para seu campo estruturante}. A partir dos conceitos de funcional e de fun��o linear, as coordenadas de um vetor qualquer numa determinada base podem ser definidas como uma sequ�ncia de valores de funcionais lineares que t�m esse vetor como argumento. Assim, dada uma base $B=\{\vto{u}_1,\cdots,\vto{u}_n\}$ do espa�o vetorial $U_\mathcal{R}$, diz-se que uma �nupla de funcionais lineares $(\vtf{f}_1^{B},\cdots,\vtf{f}_n^{B})$ cont�m os \textsb{funcionais coordenados}\index{funcionais coordenados} da base $B$ se cada \gloref{funcCoord} pertencer a $\mathcal{R}^{U_\mathcal{R}}$ e
\begin{equation}
\vto{u}=\sum_{i=1}^{n}\fua{\vtf{f}^\con{B}_{i}}{\vto{u}}\vto{u}_i\,,\,\forall\,\, \vto{u}\in U_\mathcal{R}\,,
\end{equation}
onde $(\fua{\vtf{f}^\con{B}_{1}}{\vto{u}},\cdots,\fua{\vtf{f}^\con{B}_{n}}{\vto{u}})$ s�o as coordenadas de $\vto{u}$ na base $B$. Diante disso, como o vetor
\begin{equation}
\vto{u}_i=\sum_{j=1}^{n}\fua{\vtf{f}^\con{B}_{j}}{\vto{u}_i}\vto{u}_j\,,
\end{equation}
resulta que $\fua{\vtf{f}^{B}_j}{\vto{u}_i}=\delta_{ij}$. Ademais, considerando $U_\mathcal{R}$ um espa�o de Hilbert e o conjunto $\hat{B}=\{\vun{u}_1,\cdots,\vun{u}_n\}$ uma base ortonormal desse espa�o, podemos afirmar que
\begin{equation}
	\vto{x}\cdot\vun{u}_i=\sum_{j=1}^{n}\fua{\vtf{f}^\con{\hat{B}}_{j}}{\vto{x}}\vun{u}_j\cdot\vun{u}_i=\sum_{j=1}^{n}\fua{\vtf{f}^\con{\hat{B}}_{j}}{\vto{x}}\delta_{ji}=\fua{\vtf{f}^\con{\hat{B}}_{i}}{\vto{x}}\,,\,\forall\, \vto{x}\in U_\mathcal{R}\,,
\end{equation}
de onde a seguinte regra garante a exist�ncia de funcionais coordenados:
\begin{equation}\label{eq:regraCoord}
\fua{\vtf{f}^{\hat{B}}_i}{\vto{x}}=\vto{x}\cdot\vun{u}_i\,.
\end{equation}

Agora, considerando $U_\mathcal{R}$ e $V_\mathcal{R}$ espa�os de Hilbert, dizemos que $\vtf{g}^\dagger\in U^V_\mathcal{R}$ � a \textsb{fun��o adjunta}\index{fun��o!adjunta} de $\vtf{g}\in V^U_\mathcal{R}$ quando, dados os vetores quaisquer $\vto{u}\in U_\mathcal{R}$ e $\vto{v}\in V_\mathcal{R}$,
\begin{equation}\label{eq:funcaoTransposta}
\fua{\vtf{g}}{\vto{u}}\cdot\vto{v}=\vto{u}\cdot\fua{\vtf{g}^\dagger}{\vto{v}}\,.
\end{equation}
 Em particular, se o campo $\mathcal{R}=\real$, dizemos que a fun��o $\vtf{g}^\dagger$ � a \textsb{transposta}\index{fun��o!transposta} de $\vtf{g}$. Pela igualdade anterior, ficam v�lidas as propriedades a seguir para quaisquer $\alpha\in\mathcal{R}$ e $\vtf{k}\in U^V_\mathcal{R}$.
\begin{itemize}
	\setlength\itemsep{.1em}
	\item[i.] $\lpa\alpha\vtf{g}\rpa^\dagger=\overline{\alpha}\vtf{g}^\dagger$;
	\item[ii.] $\lpa\vtf{g}\circ\vtf{k}\rpa^\dagger=\vtf{k}^\dagger\circ\vtf{g}^\dagger$;
	\item[iii.] Se $\vtf{g}$ for linear, $\vtf{g}^\dagger$ tamb�m � linear;
	\item[iv.] Se $\vtf{g}$ for uma bije��o, h� uma fun��o $\vtf{g}^{-\dagger}:=\lpa\vtf{g}^{-1}\rpa^\dagger=\lpa\vtf{g}^\dagger\rpa^{-1}$.
\end{itemize}

{\footnotesize
\begin{proof}
Primeiramente, precisamos demonstrar a exist�ncia e a unicidade das fun��es transpostas. Na igualdade apresentada, sejam $\vto{u}=\vun{u}_k$ e $\vto{v}=\vun{v}_k$, onde os vetores � direita pertencem �s bases ortonormais $B_1$ e $B_2$ respectivamente, ambas dos espa�os $n$-dimensionais $U_\mathcal{R}$ e $V_\mathcal{R}$. Assim, pode-se realizar o seguinte desenvolvimento:
\begin{align*}
\vun{u}_k\cdot\fua{\vtf{g}^\dagger}{\vun{v}_k}&=\fua{\vtf{g}}{\vun{u}_k}\cdot\vun{v}_k\\
\vun{u}_k\cdot\sum_{i=1}^n{\vtf{f}_i^{B_2}}[\fua{\vtf{g}^\dagger}{\vun{v}_k}]\vun{u}_i&=\sum_{i=1}^n{\vtf{f}_i^{B_1}}\lco\fua{\vtf{g}}{\vun{u}_k}\rco\vun{v}_i\cdot\vun{v}_k\\
\overline{{\vtf{f}_k^{B_2}}[\fua{\vtf{g}^\dagger}{\vun{v}_k}]}&={\vtf{f}_k^{B_1}}\lco\fua{\vtf{g}}{\vun{u}_k}\rco\,.
\end{align*}
Nessas condi��es, podemos dizer que se $\vtf{g}$ existe, $\vtf{g}^\dagger$ tamb�m existe. Agora, supondo que existam duas fun��es $\vtf{g}_1^\dagger$ e $\vtf{g}_2^\dagger$ transpostas de $\vtf{g}$, tem-se duas igualdades conforme \eqref{eq:funcaoTransposta}. Ao subtra�-las, obt�m-se $\vto{u}\cdot(\fua{\vtf{g}_1^\dagger}{\vto{v}}-\fua{\vtf{g}_2^\dagger}{\vto{v}})=0$,  que � v�lida para quaisquer $\vto{u}$ e $\vto{v}$; assim $\vtf{g}_1^\dagger=\vtf{g}_2^\dagger$. No caso das propriedades, a primeira pode ser comprovada a partir da igualdade $\fua{\lpa\alpha\vtf{g}\rpa^\dagger}{\vto{u}}\cdot\vto{v}=\vto{u}\cdot\alpha\fua{\vtf{g}}{\vto{v}}$ e da igualdade  $\overline{\alpha}\fua{\vtf{g}^\dagger}{\vto{u}}\cdot\vto{v}=\vto{u}\cdot\alpha\fua{\vtf{g}}{\vto{v}}$. A segunda propriedade podemos constat�-la atrav�s das seguintes igualdades: $\fua{\vtf{g}\circ\vtf{k}}{\vto{u}}\cdot\vto{v}=\fua{\vtf{k}}{\vto{u}}\cdot\fua{\vtf{g}^\dagger}{\vto{v}}=\vto{u}\cdot\fua{\vtf{k}^\dagger\circ\vtf{g}^\dagger}{\vto{u}}$. Demonstremos agora a afirma��o de que a transposta de uma fun��o linear tamb�m � linear. Dado o escalar $\alpha\in\mathcal{R}$,
\begin{equation*}
\vto{v}\cdot\fua{\vtf{g}^\dagger}{\ele{a}\vto{u}}=\overline{\alpha}\lco\fua{\vtf{g}}{\vto{v}}\cdot\vto{u}\rco=\overline{\alpha}[\vto{v}\cdot\fua{\vtf{g}^\dagger}{\vto{u}}]=\vto{v}\cdot\alpha\fua{\vtf{g}^\dagger}{\vto{u}}.
\end{equation*}
Considerando $\vto{u}_1,\vto{u}_2\in U^V_\mathcal{R}$, temos que
\begin{equation*}
\vto{v}\cdot\fua{\vtf{g}^\dagger}{\vto{u}_1+\vto{u}_2}=\fua{\vtf{g}}{\vto{v}}\cdot\lpa\vto{u}_1+\vto{u}_2\rpa=\vto{v}\cdot[\fua{\vtf{g}^\dagger}{\vto{u}_1}+\fua{\vtf{g}^\dagger}{\vto{u}_2}].
\end{equation*}
Para provar a quarta propriedade, precisamos saber que uma fun��o identidade sempre � igual � sua transposta; algo que n�o � dif�cil de verificar. Sendo assim, transpondo ambos os lados da igualdade $\vtf{i}_V=\vtf{g}\circ\vtf{g}^{-1}$ obt�m-se $\vtf{i}_V=(\vtf{g}^{-1})^\dagger\circ\vtf{g}^\dagger$, devido � segunda propriedade. Sabemos tamb�m que $\vtf{i}_V=(\vtf{g}^\dagger)^{-1}\circ\vtf{g}^\dagger$, o que comprova $(\vtf{g}^{-1})^\dagger=(\vtf{g}^\dagger)^{-1}$.
\end{proof}
}

Dado um espa�o vetorial $V_\mathcal{R}$, o espa�o vetorial $\evl{\mathcal{R}}{V}{\mathcal{R}}$ � chamado \textsb{espa�o dual}\index{espa�o!dual} de $V_\mathcal{R}$, representado $V^*_\mathcal{R}$, cujos elementos s�o ditos \textsb{vetores duais}\index{vetor!dual}. Em termos gen�ricos, vetores duais s�o medidas escalares dos vetores de um determinado espa�o vetorial cuja estrutura fica preservada; algo que a norma, como medida escalar, n�o garante, por ser ela um funcional n�o linear. J� nos funcionais coordenados, o vetor $\vtf{f}^\con{B}_{i}$ � dual e, de alguma forma, ``mede'' seu argumento em rela��o ao vetor de �ndice $i$ da base $B$, quando chamamos de coordenada o valor da medida. Conforme apresentado, no contexto de bases ortonormais, a medi��o dos funcionais coordenados se expressa na rela��o de incid�ncia do argumento com um vetor da base, ou seja, no produto interno dos dois. Um subconjunto $\{\vtf{g}_1,\cdots,\vtf{g}_m\}$ de $V^*_\mathcal{R}$ � dito o \textsb{conjunto dual}\index{conjunto!dual} de $\{\vto{w}_1,\cdots,\vto{w}_m\}\subset V_\mathcal{R}$ se $\fua{\vtf{g}_i}{\vto{w}_j}=\delta_{ij}$. Quando tal subconjunto for dual de uma base $B=\{\vto{u}_1,\cdots,\vto{u}_n\}$ de $V_\mathcal{R}$, seus elementos ser�o os funcionais coordenados de $B$, conforme demonstram as seguintes igualdades:
\begin{equation*}
\fua{\vtf{g}_i}{\vto{x}}=\vtf{g}_i(\sum_{j=1}^{n}\fua{\vtf{f}^\con{B}_{j}}{\vto{x}}\vto{u}_j)=\sum_{j=1}^{n}\fua{\vtf{f}^\con{B}_{j}}{\vto{x}}\fua{\vtf{g}_i}{\vto{u}_j}=\sum_{j=1}^{n}\fua{\vtf{f}^\con{B}_{j}}{\vto{x}}\delta_{ij}=\fua{\vtf{f}^\con{B}_{i}}{\vto{x}}\,.
\end{equation*}
Ocorre que um conjunto formado por funcionais coordenados, no caso o subconjunto $B^*:=\{\vtf{f}^\con{B}_1,\cdots,\vtf{f}^\con{B}_n\}$, � uma base do espa�o dual, ou seja, o conjunto dual $B^*$ de uma base $B$ � ele pr�prio uma base do espa�o dual, quando o chamamos \textsb{base dual}\index{base!dual}. Diante disso, podemos afirmar que espa�os duais possuem a mesma dimens�o dos espa�os vetoriais aos quais est�o relacionados; nos termos descritos, $\dim (V_\mathcal{R})=\dim (V_\mathcal{R}^*)$. Assim, dado um vetor dual qualquer $\vtf{h}\in V^*_\mathcal{R}$, as igualdades
\begin{equation*}
\fua{\vtf{h}}{\vto{x}}=\vtf{h}[\,{\sum_{i=1}^n\fua{\vtf{f}_i^B}{\vto{x}}\vto{u}_i}\,]=\sum_{i=1}^n\fua{\vtf{h}}{\vto{u}_i}\fua{\vtf{f}_i^B}{\vto{x}}=[\,\sum_{i=1}^n\fua{\vtf{h}}{\vto{u}_i}\vtf{f}_i^B\,] (\vto{x})
\end{equation*}
permitem concluir que os escalares $\fua{\vtf{h}}{\vto{u}_i}$ constituem as coordenadas de $\vtf{h}$ em $B^*$. Conv�m que essa estreita correspond�ncia entre espa�os vetoriais e duais fique ainda mais forte, de tal forma que vetores e vetores duais se relacionem de maneira un�voca, quando esses �ltimos s�o denominados \textsb{covetores}\index{covetor}. Aproveitando o formato da regra \eqref{eq:regraCoord}, o teorema descrito a seguir, de extrema relev�ncia para o nosso estudo, estabelece essa rela��o.

\begin{mteo}{Representa��o de Riesz-Fr�chet}{repRiesz}
Seja o mapeamento $\map{\Phi}{U_\mathcal{R}}{U^*_\mathcal{R}}$, onde $U_\mathcal{R}$ � um espa�o de Hilbert e $U^*_\mathcal{R}$ seu espa�o dual. Se para qualquer vetor $\vto{u}\in U_\mathcal{R}$ for definido um covetor $\fua{\Phi}{\vto{u}}$, representado \gloref{covetor}, com regra
\begin{equation}
\fua{\vtf{u}^*}{\vto{x}}=\vto{x}\cdot\vto{u}\,,
\end{equation}
a fun��o $\Phi$ resulta um isomorfismo e, sendo $\vtf{u}^*$ cont�nuo, a norma $\|\vtf{u}^*\|_{U^*_\mathcal{R}}=\|\vto{u}\|_{U_\mathcal{R}}$.
\end{mteo}

{\footnotesize
\begin{proof}
Primeiramente, verifiquemos se um subconjunto de funcionais coordenados � mesmo uma base do espa�o dual. Dados um vetor dual qualquer $\vtf{g}\in V^*_\mathcal{R}$, um vetor qualquer $\vto{x}\in V_\mathcal{R}$ e o subconjunto $B=\{\vto{v}_1,\cdots,\vto{v}_n\}$ uma base qualquer de $V_\mathcal{R}$, tem-se o seguinte desenvolvimento:
\begin{eqnarray}
\fua{\vtf{g}}{\vto{x}} & = &
\vtf{g}(\sum_{i=1}^{n}\fua{\vtf{f}^\con{B}_{i}}{\vto{x}}\vto{v}_i)\nonumber\\
\fua{\vtf{g}}{\vto{x}} & = &
\sum_{i=1}^{n}\fua{\vtf{g}}{\vto{v}_i}\fua{\vtf{f}^\con{B}_{i}}{\vto{x}}\nonumber\\
\fua{\vtf{g}}{\vto{x}} & = &
\sum_{i=1}^{n}[\fua{\vtf{g}}{\vto{v}_i}\vtf{f}^\con{B}_{i}]\lpa\vto{x}\rpa\nonumber\\
\fua{\vtf{g}}{\vto{x}} & = &
[\sum_{i=1}^{n}\fua{\vtf{g}}{\vto{v}_i}\vtf{f}^\con{B}_{i}]\lpa\vto{x}\rpa\nonumber\\
\vtf{g} & = &
\sum_{i=1}^{n}\fua{\vtf{g}}{\vto{v}_i}\vtf{f}^\con{B}_{i}\nonumber\,,
\end{eqnarray}
de onde se pode concluir que os funcionais coordenados de $B$ geram $U^*_\mathcal{R}$. Al�m disso, se $\vto{x}$ for nulo, ent�o $\sum_{i=1}^{n}\fua{\vtf{f}^\con{B}_{i}}{\vto{x}}\vto{v}_i=\vto{0}$, de onde resulta $\fua{\vtf{f}^\con{B}_{i}}{\vto{x}}=0$ v�lido para qualquer $\vto{x}$; logo, $\vtf{f}^\con{B}_{i}=\vto{0}$. Diante dessa independ�ncia linear, os funcionais coordenados de $B$ constituem uma base de $V^*_\mathcal{R}$. Nos termos do teorema, considerando $\vto{u}$ e $\vto{v}$ vetores quaisquer de $U_\mathcal{R}$, constatamos que  $\Phi$ � um homomorfismo pelas seguintes igualdades:
\begin{equation*}
\fua{\lco\fua{\Phi}{\vto{u}+\vto{v}}\rco}{\vto{x}}=\fua{\lpa\vto{u}+\vto{v}\rpa^*}{\vto{x}}=\vto{x}\cdot(\vto{u}+\vto{v})=\fua{\vto{u}^*}{\vto{x}}+\fua{\vto{v}^*}{\vto{x}}=\fua{\lco\fua{\Phi}{\vto{u}}+\fua{\Phi}{\vto{v}}\rco}{\vto{x}}\,.
\end{equation*}
Se $\Phi$ n�o fosse uma inje��o existiriam diferentes covetores $\vto{u}^*$ e $\vto{v}^*$ onde o escalar $\fua{\vto{u}^*}{\vto{x}}=\fua{\vto{v}^*}{\vto{x}}$ ou $\vto{x}\cdot\vto{u}=\vto{x}\cdot\vto{v}$. Dessa suposi��o resultam as igualdades $\vto{x}\cdot(\vto{u}-\vto{v})=\fua{(\vto{u}-\vto{v})^*}{\vto{x}}=0$, que n�o corroboram $\vto{u}^*\neq\vto{v}^*$. Para provar que $\Phi$ � uma sobreje��o, precisamos obter para qualquer funcional $\vtf{g}\in U_\mathcal{R}^*$ um vetor $\vto{u}\in U_\mathcal{R}$ tal que $\fua{\Phi}{\vto{u}}=\vtf{g}$. Considerando a regra \eqref{eq:regraCoord} e o conjunto $\hat{B}=\{\vun{u}_1,\cdots,\vun{u}_n\}$ uma base ortonormal de $U_\mathcal{R}$ cujos funcionais coordenados geram $U^*_\mathcal{R}$, podemos dizer que
\begin{equation*}
\fua{\vtf{g}}{\vto{x}}=\fua{\lco\sum_{i=1}^n\alpha_i\vtf{f}^{\hat{B}}_i\rco}{\vto{x}}=\sum_{i=1}^n\alpha_i\fua{\vtf{f}^{\hat{B}}_i}{\vto{x}}=\sum_{i=1}^n\alpha_i\lpa\vto{x}\cdot\vun{u}_i\rpa=\vto{x}	\cdot \lpa \sum_{i=1}^n\overline{\alpha_i} \vun{u}_i\rpa\,.
\end{equation*}
Como $\fua{\vtf{g}}{\vto{x}}=\fua{\lco\fua{\Phi}{\vto{u}}\rco}{\vto{x}}=\vto{x}\cdot\vto{u}$, conseguimos constatar  a exist�ncia de $\vto{u}=\sum_{i=1}^n\overline{\alpha_i} \vun{u}_i$. Finalmente, aplicando a defini��o \eqref{eq:normaFuncao} nas condi��es aqui colocadas e deixando impl�cita a representa��o dos espa�os nas normas, temos $\|\vto{u}^*\|=\sup\{ |\vto{x}\cdot\vto{u}|/\|\vto{x}\|\}$ para qualquer $\vto{x}$ n�o nulo. Se $\vto{u}$ for nulo, fica evidente que $\|\vto{u}^*\|=\|\vto{u}\|$; se n�o for, $\vto{u}^*$ � n�o nulo e podemos concluir que $\|\vto{u}^*\|\geqslant |\, \vto{x}\cdot\vto{u}|/\|\vto{x}\|$. A desigualdade de Cauchy-Schwarz preconiza que $|\, \vto{x}\cdot\vto{u}| \leqslant \|\vto{x}\|\,\,\|\vto{u}\|$. Subtraindo essas duas desigualdades obt�m-se $(\|\vto{u}^*\|-\|\vto{u}\|)\|\vto{x}\|\geqslant 0$, cujo lado esquerdo pode ser nulo para quaisquer $\vto{u}^*$, $\vto{u}$ e $\vto{x}$ n�o nulos; logo $\|\vto{u}^*\|=\|\vto{u}\|$.
\end{proof}
}

A Representa��o de Riesz-Fr�chet permite definir uma regra para funcionais coordenados de bases n�o necessariamente ortonormais, apresentada a seguir. Considerando as mesmas condi��es do teorema, seja $B=\{\vto{u}_1,\cdots,\vto{u}_n\}$ uma base de $U_\mathcal{R}$ e $B^*=\{\vtf{f}_1^B,\cdots,\vtf{f}_n^B\}$ sua base dual. Nesse contexto, os funcionais $\vtf{f}_i^B\in U_\mathcal{R}^*$ est�o relacionados univocamente a vetores $\vto{v}_i\in U_\mathcal{R}$, de tal forma que, dado um vetor qualquer $\vto{u}\in U_\mathcal{R}$, temos as igualdades
\begin{equation*}
\fua{\vtf{f}_i^B}{\vto{u}}=\vto{u}\cdot\vto{v_i}=\sum_{j=1}^n\fua{\vtf{f}_j^B}{\vto{u}}\vto{u}_j \cdot \vto{v}_i\,,
\end{equation*}
de onde resulta $\vto{u}_j \cdot \vto{v}_i=\delta_{ji}$, quando se chega � identidade $\fua{\vtf{f}_i^B}{\vto{u}}=\fua{\vtf{f}_i^B}{\vto{u}}$. Assim, conclu�mos que o subconjunto $\{\vto{v}_1,\cdots,\vto{v}_n\}$, univocamente relacionado a $B^*$, � a base rec�proca $B^\perp=\{\vto{u}^1,\cdots,\vto{u}^n\}$, ou seja, os covetores $(\vto{u}^i)^*=\vtf{f}_i^B$.
A regra de funcionais coordenados pode ser expressa por
\begin{equation}\label{eq:regraFuncCoord}
\fua{\vtf{f}_i^B}{\vto{x}}=\fua{(\vto{u}^i)^*}{\vto{x}}=\vto{x} \cdot \vto{u}^i\,.
\end{equation}
Se $B^\perp$ � a base rec�proca de $B$, o inverso tamb�m � verdadeiro; logo, a partir da regra anterior, podemos afirmar que
\begin{equation}
\fua{\vtf{f}_i^{B^\perp}}{\vto{x}}=\fua{(\vto{u}_i)^*}{\vto{x}}=\vto{x} \cdot \vto{u}_i\,.
\end{equation}
A figura \ref{fg:espacoDual} resume os relacionamentos entre a base $B$ com as bases rec�procas e duais que ela induz.
\begin{figure}[!ht]
\centering
\begin{center}
\scalebox{.70}{\input{partes/parte1/figs/c_algabst/espacoDual.pstex_t}}
\end{center}
\titfigura{Relacionamentos das bases induzidas por $B$ nos termos do teorema \ref{teo:repRiesz}.}\label{fg:espacoDual}
\end{figure}
No caso particular de uma base ortonormal $\hat{B}$, j� sabemos que ela � rec�proca � ela pr�pria, ou seja, os vetores  $\vun{u}_i=\vun{u}^i$, o que nos leva a concluir que os vetores de $\hat{B}$ e de $\hat{B}^*$ t�m uma rela��o un�voca nos termos do teorema anterior. Assim, podemos dizer que os escalares
\begin{equation}
\fua{\vtf{f}_i^{\hat{B}}}{\vto{x}}=\fua{(\vun{u}_i)^*}{\vto{x}}=\vto{x} \cdot \vun{u}_i\,.
\end{equation}
Nesse contexto, o produto interno entre dois vetores quaisquer $\vto{u}$ e $\vto{v}$ conduzem �s seguintes igualdades:
\begin{equation}\label{eq:prodIntGen}
\vto{u}\cdot\vto{v}=\sum_{i=1}^n\sum_{j=1}^n\fua{\vtf{f}_i^{B}}{\vto{u}}\overline{\fua{\vtf{f}_j^{B^\perp}}{\vto{v}}}\,\vto{u}_i\cdot\vto{u}^j=\sum_{i=1}^n\fua{\vtf{f}_i^{B}}{\vto{u}}\overline{\fua{\vtf{f}_i^{B^\perp}}{\vto{v}}}\,.
\end{equation}
Se $B$ for ortonormal, ent�o a base $B^\perp=B$. Al�m disso, se o campo $\mathcal{R}$ for real, tem-se que  $\vto{u}\cdot\vto{v}=\sum_{i=1}^n\fua{\vtf{f}_i^{B}}{\vto{u}}\fua{\vtf{f}_i^{B}}{\vto{v}}$.

Ainda considerando os termos do teorema anterior, a igualdade entre os valores das normas de vetores e de covetores cont�nuos permite dizer que $U^*_\mathcal{R}$ � tamb�m um espa�o de Hilbert. Como a propriedade � importante, vamos apresent�-la em termos mais formais, no corol�rio a seguir.

\begin{mcoro}{Espa�o Dual de Hilbert}{dualHilb}
Se os funcionais lineares de $U_\mathcal{R}^*$ forem cont�nuos e o espa�o $U_\mathcal{R}$ for de Hilbert, ent�o $U_\mathcal{R}^*$ tamb�m ser� espa�o de Hilbert.
\end{mcoro}
\hspace{1pt}
{\footnotesize
\begin{proof}
O espa�o $U_\mathcal{R}^*$ � normado pela defini��o \eqref{eq:normaFuncao} e m�trico produto interno tamb�m por  defini��o (Ver p. \pageref{txt:prodInt}). Resta agora mostrarmos que $U_\mathcal{R}^*$ � completo, condi��o que fica garantida se a bije��o $\Phi$ for isom�trica, nos termos do teorema \ref{teo:isoComp}. A partir da regra de covetores, definida na Representa��o de Riesz-Fr�chet, se $\vto{u}=\vto{v}-\vto{w}$ ent�o
\begin{equation*}
\fua{(\vto{v}-\vto{w})^*}{\vto{x}}=\vto{x}\cdot(\vto{v}-\vto{w})=\vto{x}\cdot\vto{v}-\vto{x}\cdot\vto{w}=\fua{\vto{v}^*}{\vto{x}}-\fua{\vto{w}^*}{\vto{x}}=\fua{(\vto{v}^*-\vto{w}^*)}{\vto{x}}\,,
\end{equation*}
de onde conclu�mos que $(\vto{v}-\vto{w})^*=(\vto{v}^*-\vto{w}^*)$, ou que $\fua{\Phi}{\vto{v}-\vto{w}}=\fua{\Phi}{\vto{v}}-\fua{\Phi}{\vto{w}}$. Assim, sabendo que $\|\fua{\Phi}{\vto{x}}\|_{U_\mathcal{R}^*}=\|\vto{x}\|_{U_\mathcal{R}}$, temos que a m�trica $\fua{\varrho_{U_\mathcal{R}^*}}{\fua{\Phi}{\vto{v}},\fua{\Phi}{\vto{w}}}$ � igual a
\begin{equation*}
\|\fua{\Phi}{\vto{v}}-\fua{\Phi}{\vto{w}}\|_{U_\mathcal{R}^*}= \|\fua{\Phi}{\vto{v}-\vto{w}}\|_{U_\mathcal{R}^*}= \|\vto{v}-\vto{w}\|_{U_\mathcal{R}}\,,
\end{equation*}
 de onde conclu�mos que $\Phi$ � uma isometria.
\end{proof}
}

Um vetor $\vtf{h}$ do espa�o de fun��es $V^V_\mathcal{R}$ � dito um operador  \textsb{Hermitiano}\index{operador!Hermitiano} ou \textsb{auto-adjunto}\index{operador!auto-adjunto} quando $\vtf{h}=\vtf{h}^\dagger$; mas, se $\vtf{h}=-\vtf{h}^\dagger$, ele � chamado \textsb{anti-Hermitiano}\index{operador!anti-Hermitiano}. Quando o campo $\mathcal{R}$ for real, o operador Hermitiano recebe o nome de \textsb{sim�trico}\index{operador!sim�trico} enquanto o anti-Hermitiano � denominado \textsb{antissim�trico}\index{operador!antissim�trico}.
Uma fun��o de $U^V_\mathcal{R}$ � dita \textsb{unit�ria}\index{fun��o!unit�ria} quando for uma bije��o linear cuja adjunta � igual � inversa. Dessa defini��o, dizemos que um operador unit�rio $\vtf{q}\in V^V_\mathcal{R}$ preserva o produto interno porque, para quaisquer $\vto{u},\vto{v}\in V_\mathcal{R}$, s�o v�lidas as igualdades
\begin{equation}
\fua{\vtf{q}}{\vto{u}}\cdot\fua{\vtf{q}}{\vto{v}}=\vto{u}\cdot\fua{\vtf{q}^\dagger\circ\vtf{q}}{\vto{v}}=\vto{u}\cdot\fua{\vtf{q}^{-1}\circ\vtf{q}}{\vto{v}}=\vto{u}\cdot\vto{v}\,.
\end{equation}
Diante disso, no contexto de espa�os de Hilbert, dizemos que \emph{operadores unit�rios preservam a norma}, pois $\|\fua{\vtf{q}}{\vto{u}}\|^2=\fua{\vtf{q}}{\vto{u}}\cdot\fua{\vtf{q}}{\vto{u}}=\vto{u}\cdot\vto{u}=\|\vto{u}\|^2$.

No cap�tulo anterior, dissemos que o conjunto de todos os operadores un�rios invers�veis constitui um grupo na opera��o de composi��o. Vejamos agora se o conjunto $O\subset V^V_\mathcal{R}$ de todos os operadores unit�rios, por serem operadores un�rios e invers�veis, define o grupo $(\gloref{grOrto},\circ)$, denominado \textsb{grupo unit�rio}\index{grupo!unit�rio}. Isso pode ser verificado pelas igualdades
\begin{equation}
\lpa \vtf{q}_1\circ\vtf{q}_2 \rpa^{-1} =
\vtf{q}_2^{-1}\circ\vtf{q}_1^{-1} = \vtf{q}_2^\dagger\circ\vtf{q}_1^\dagger =
\lpa \vtf{q}_1\circ\vtf{q}_2 \rpa^{T}
\end{equation}
que evidenciam, para quaisquer $\vtf{q}_1,\vtf{q}_2\in O$, a bije��o\footnote{Ver propriedades na p�gina \pageref{prop:Composicao}.} $\vtf{q}_1\circ\vtf{q}_2$ como elemento do conjunto $O$. Grupos e operadores unit�rios no contexto de campos reais s�o denominados ortogonais. Al�m desse tipo de operador linear, h� os que preservam a m�trica ou a dist�ncia, quando s�o chamados \textsb{operadores isom�tricos}\index{operador!isom�trico}. Em termos mais rigorosos, $\vtf{k}\in V^V_\mathcal{R}$ � um operador isom�trico quando for uma bije��o linear onde  $\varrho\,[\fua{\vtf{k}}{\vto{u}},\fua{\vtf{k}}{\vto{v}}]=\fua{\varrho}{\vto{u},\vto{v}}$, para quaisquer $\vto{u},\vto{v}\in V_\mathcal{R}$. De maneira similar aos operadores unit�rios, o conjunto de todas os operadores isom�tricos define um grupo $(I,\circ)$, denominado \textsb{grupo isom�trico}\index{grupo!isom�trico}, porque a composi��o de operadores isom�tricos � tamb�m uma operador isom�trico, ou seja, considerando quaisquer $\vtf{k},\vtf{g}\in I$, tem-se
\begin{equation}
\varrho\,\lco\fua{\vtf{g}\circ\vtf{k}}{\vto{u}},\fua{\vtf{g}\circ\vtf{k}}{\vto{v}}\rco = \varrho\,\lco{\fua{\vtf{g}}{\vto{u}},\fua{\vtf{g}}{\vto{v}}}\rco=\fua{\varrho}{\vto{u},\vto{v}}
\end{equation}
para quaisquer $\vto{u},\vto{v}\in V_\mathcal{R}$. \emph{No contexto de espa�os Euclidianos, um grupo ortogonal � sempre isom�trico porque todo operador que preserva produto interno � uma isom�trico. Al�m disso, um grupo isom�trico de operadores lineares sempre preserva produto interno.}

{\footnotesize
\begin{proof}
Conv�m que verifiquemos essas t�o categ�ricas afirma��es. Se $V_\real$ for um espa�o Euclidiano e se $\vtf{k}\in V^V_\real$ preservar o produto interno, ent�o
\begin{align*}
	\varrho\lco\fua{\vtf{k}}{\vto{u}},\fua{\vtf{k}}{\vto{v}}\rco^2&=\|\fua{\vtf{k}}{\vto{u}}-\fua{\vtf{k}}{\vto{v}}\|^2\\
	&=\fua{\vtf{k}}{\vto{u}}\cdot\fua{\vtf{k}}{\vto{u}}-2\fua{\vtf{k}}{\vto{u}}\cdot\fua{\vtf{k}}{\vto{v}}+\fua{\vtf{k}}{\vto{v}}\cdot\fua{\vtf{k}}{\vto{v}}\\
	&={\vto{u}}\cdot{\vto{u}}-2{\vto{u}}\cdot{\vto{v}}+{\vto{v}}\cdot{\vto{v}}\\
	&=\|\vto{u}-\vto{v}\|^2 \\
	&= \fua{\varrho}{\vto{u},\vto{v}}^2\,,
\end{align*}
de onde se constata $\vtf{k}$ uma isometria. Agora, seja $\vtf{g}\in V^V_\real$ uma isometria. Elevando os dois lados da igualdade  $\varrho[\fua{\vtf{g}}{\vto{u}},\fua{\vtf{g}}{\vto{v}}]=\fua{\varrho}{\vto{u},\vto{v}}$ ao quadrado, obtemos
\begin{align*}
\fua{\vtf{g}}{\vto{u}}\cdot\fua{\vtf{g}}{\vto{u}}-2\fua{\vtf{g}}{\vto{u}}\cdot\fua{\vtf{g}}{\vto{v}}+\fua{\vtf{g}}{\vto{v}}\cdot\fua{\vtf{g}}{\vto{v}}&={\vto{u}}\cdot{\vto{u}}-2{\vto{u}}\cdot{\vto{v}}+{\vto{v}}\cdot{\vto{v}}\\
\varrho[\fua{\vtf{g}}{\vto{u}},\vto{0}]^2-2\fua{\vtf{g}}{\vto{u}}\cdot\fua{\vtf{g}}{\vto{v}}+\varrho[\fua{\vtf{g}}{\vto{v}},\vto{0}]^2&=\fua{\varrho}{\vto{u},\vto{0}}^2-2{\vto{u}}\cdot{\vto{v}}+\fua{\varrho}{\vto{v},\vto{0}}^2\\
\varrho[\fua{\vtf{g}}{\vto{u}},\fua{\vtf{g}}{\vto{0}}]^2-2\fua{\vtf{g}}{\vto{u}}\cdot\fua{\vtf{g}}{\vto{v}}+\varrho[\fua{\vtf{g}}{\vto{v}},\fua{\vtf{g}}{\vto{0}}]^2&=\fua{\varrho}{\vto{u},\vto{0}}^2-2{\vto{u}}\cdot{\vto{v}}+\fua{\varrho}{\vto{v},\vto{0}}^2\\
\fua{\vtf{g}}{\vto{u}}\cdot\fua{\vtf{g}}{\vto{v}}&= {\vto{u}}\cdot{\vto{v}}\,\,.
\end{align*}
\end{proof}
}




\section{Representa��es Matriciais}

Sabemos que um vetor qualquer pode ser identificado por suas coordenadas, representadas por uma �nupla, numa determinada base, de tal sorte que �nuplas distintas nunca implicam vetores iguais: a rela��o � portanto un�voca. Assim, express�es que envolvem vetores podem ser descritas por elementos coordenados, coligidos convenientemente em matrizes, quando fica dispon�vel todo o arcabou�o aritm�tico-funcional aplic�vel a esse tipo de cole��o, apresentado no cap�tulo anterior. A rela��o que se estabelece entre vetores e matrizes tamb�m � un�voca e, como ocorre com as �nuplas, dependente de uma base. Na pr�tica, a representa��o matricial de um vetor ocorre da seguinte forma: seja $\vto{u}$ um elemento qualquer do espa�o vetorial  $U_\mathcal{R}$, do qual o subconjunto $B=\{\vto{u}_1,\cdots,\vto{u}_n\}$ � uma base qualquer. Assim, definimos \gloref{repVet} como sendo a \textsb{matriz representativa}\index{matriz!representativa} de $\vto{u}$ na base $B$, cuja dimens�o � $n\times 1$ e os elementos $\mav{\vto{u}}{B}_{i1}:=\fua{\vtf{f}^\con{B}_{i}}{\vto{u}}$. Diante disso, fica evidente que a matriz representativa do vetor nulo � sempre nula. Al�m disso, da linearidade de funcionais coordenados podemos desenvolver
\begin{equation*}
\alpha\sum_{i=1}^{n}\fua{\vtf{f}^\con{B}_{i}}{\vto{x}}\vto{u}_i+\beta\sum_{i=1}^{n}\fua{\vtf{f}^\con{B}_{i}}{\vto{y}}\vto{u}_i= \sum_{i=1}^{n}\lco\alpha\fua{\vtf{f}^\con{B}_{i}}{\vto{x}}+\beta\fua{\vtf{f}^\con{B}_{i}}{\vto{y}}\rco\vto{u}_i=\sum_{i=1}^{n}\fua{\vtf{f}^\con{B}_{i}}{\alpha\vto{x}+\beta\vto{y}}\vto{u}_i,
\end{equation*}
de onde conclu�mos que
\begin{equation}
\alpha\mav{\vto{x}}{B}+\beta\mav{\vto{y}}{B}=\mav{\alpha\vto{x}+\beta\vto{y}}{B}\,,
\end{equation}
onde $\vto{x},\vto{y}\in U_\mathcal{R}$ s�o vetores quaisquer e $\alpha,\beta\in\mathcal{R}$ escalares quaisquer. No que diz respeito aos vetores da base $B$, se os representamos relativos � pr�pria base $B$, temos $\mav{\vto{u}_j}{B}_{i1}=\fua{\vtf{f}^\con{B}_{i}}{\vto{u}_j}=\delta_{ij}$. Costuma-se adotar tal estrat�gia para a base natural $O=\{\vun{e}_1,\cdots,\vun{e}_n\}$ de espa�os Euclidianos, onde $\mav{\vun{e}_j}{O}_{i1}=\delta_{ij}\,$.

Considerando $V_\mathcal{R}$ um espa�o vetorial $m$-dimensional, seja $\vtf{g}$ um elemento qualquer de um espa�o de fun��es $\evl{\mathcal{R}}{U}{V}$. No ato de mapear vetores de $U$ para vetores de $V$, a fun��o $\vtf{g}$ termina por relacionar indiretamente duas bases distintas, de tal forma que a representa��o matricial desse relacionamento precisa evidenciar ambas as bases. Diante disso, se $C=\{\vto{v}_1,\cdots,\vto{v}_m\}$ for uma base de $V_\mathcal{R}$ e $\vto{u}\in U_\mathcal{R}$ um vetor qualquer, temos
\begin{align*}
\fua{\vtf{g}}{\vto{u}}&= \sum_{i=1}^{m}\fua{\vtf{f}^\con{C}_{i}}{\fua{\vtf{g}}{\vto{u}}}\vto{v}_i\nonumber\\
\fua{\vtf{g}}{\vto{u}}&=\sum_{i=1}^{m}\vtf{f}^\con{C}_{i}  \lco \sum_{j=1}^{n} \fua{\vtf{f}^\con{B}_{j}}{\vto{u}}\fua{\vtf{g}}{\vto{u}_j}\rco\vto{v}_i\nonumber\\
\fua{\vtf{g}}{\vto{u}}&=\sum_{i=1}^{m}\sum_{j=1}^{n} \vtf{f}^\con{C}_{i}\lco\fua{\vtf{g}}{\vto{u}_j}  \rco\fua{\vtf{f}^\con{B}_{j}}{\vto{u}}\vto{v}_i\,,\nonumber
\end{align*}
quando podemos afirmar que
\begin{equation}\label{eq:repMatMape}
\mav{\fua{\vtf{g}}{\vto{u}}}{C}=\maf{\vtf{g}}{B}{C}\mav{\vto{u}}{B}\,,
\end{equation}
onde \gloref{repFun} � a matriz $m\times n$ de elementos $\maf{\vtf{g}}{B}{C}_{ij}:=\vtf{f}^\con{C}_{i}\lco\fua{\vtf{g}}{\vto{u}_j}\rco$, representativa nas bases $B$ e $C$ da fun��o linear $\vtf{g}\in\evl{\mathcal{R}}{U}{V}$. Em outras palavras, para bases quaisquer $B$ e $C$ de dimens�o $n$ e $m$ respectivamente, a matriz $m\times 1$ representativa em $C$ do valor $\fua{\vtf{g}}{\vto{u}}$ resulta do produto da matriz $m\times n$, representativa em $B$ e $C$ da fun��o $\vtf{g}$, com a matriz $n\times 1$ representativa em $B$ do vetor $\vto{u}$. Do procedimento  que conduziu � esse resultado, dados $\vtf{h}\in\evl{\mathcal{R}}{U}{V}$ e $\alpha,\beta\in\mathcal{R}$, pode-se obter tamb�m a igualdade
\begin{equation}
\mav{[\fua{\alpha\vtf{g}+\beta\vtf{h}]}{\vto{u}}}{C}=\lpa\alpha\maf{\vtf{g}}{B}{C}+\beta\maf{\vtf{h}}{B}{C}\rpa\mav{\vto{u}}{B}\,.
\end{equation}
Pode ocorrer que $U$ seja igual a $V$, quando as fun��es envolvidas resultam operadores lineares. Diante disso, as bases $B$ e $C$ tamb�m podem ser iguais, o que nos permite escrever, por exemplo,  $\mav{\fua{\vtf{g}}{\vto{u}}}{B}=\maf{\vtf{g}}{B}{B}\mav{\vto{u}}{B}$, de onde se diz que $\maf{\vtf{g}}{B}{B}$ � a matriz representativa em $B$ do operador linear $\vtf{g}$.


As fun��es lineares compostas tamb�m podem ser representadas em termos matriciais. Vejamos como. Considerando $\vtf{l}$ uma fun��o do espa�o $\evl{\mathcal{R}}{V}{W}$, o conjunto $Z$ uma base do espa�o vetorial $q$-dimensional $W_\mathcal{R}$ e $\vto{v}=\fua{\vtf{g}}{\vto{u}}$ um vetor de $W_\mathcal{R}$, podemos escrever
$\mav{\fua{\vtf{l}}{\vto{v}}}{Z}=\maf{\vtf{l}}{C}{Z}\mav{\vto{v}}{C}$. Dessa igualdade, resulta que a matriz
\begin{equation}
\mav{\fua{\vtf{l}\circ\vtf{g}}{\vto{u}}}{Z}=\maf{\vtf{l}}{C}{Z}\mav{\fua{\vtf{g}}{\vto{u}}}{C}=\maf{\vtf{l}}{C}{Z}\maf{\vtf{g}}{B}{C}\mav{\vto{u}}{B}\,.
\end{equation}
Aproveitando esse interessante resultado, se a fun��o linear $\vtf{g}$ for uma bije��o e a dimens�o $m=n$, temos ent�o que
\begin{equation}\label{eq:matRepInv}
\mav{\vto{u}}{B}=\mav{\fua{\vtf{g}^{-1}\circ\vtf{g}}{\vto{u}}}{B}=\maf{\vtf{g}^{-1}_C}{}{B}\maf{\vtf{g}}{B}{C}\mav{\vto{u}}{B} \implies \maf{\vtf{g}^{-1}_C}{}{B} = {\maf{\vtf{g}}{B}{C}}^{-1}\,.
\end{equation}
Em termos matriciais, o produto interno dos vetores $\vto{x},\vto{y}\in X_\mathcal{R}$, onde $X_\mathcal{R}$ � um espa�o de Hilbert, pode ser descrito por uma das seguintes igualdades:
\begin{equation}
\vto{x}\cdot\vto{y} = \lpa\mav{\vto{x}}{B}{\mav{\vto{y}}{B^\perp}}^\text{T}\rpa_{11} = \lpa{\mav{\vto{x}}{B}}^\text{T}{\mav{\vto{y}}{B^\perp}}\rpa_{11}\,,
\end{equation}
conforme a igualdade \eqref{eq:prodIntGen}. Se a fun��o $\vtf{g}\in\evl{\mathcal{R}}{X}{Y}$, onde $Y_\mathcal{R}$ tamb�m � um espa�o de Hilbert, possui um transposta adjunta, a partir dos vetores das bases $B=\{\vto{x}_i,\cdots,\vto{x}_n\}$ de $U_\mathcal{R}$ e $C=\{\vto{y}_i,\cdots,\vto{y}_m\}$ de $Y_\mathcal{R}$, podemos realizar o desenvolvimento a seguir:
\begin{align}\label{eq:matRepTransp}
\fua{\vtf{g}}{\vto{x}_i}\cdot \vto{y}^j &= \overline{\vtf{g}^\dagger(\vto{y}^j)\cdot\vto{x}_i}\nonumber\\
\vtf{f}^C_j\lco\fua{\vtf{g}}{\vto{x}_i}\rco &=\overline{\vtf{f}^{B^\perp}_i[\vtf{g}^\dagger(\vto{y}^j)]}\nonumber\\
{\maf{\vtf{g}}{B}{C}}^\dagger&=[\vtf{g}^\dagger_{C^\perp}]^{B^\perp}\,,
\end{align}
onde as matrizes � direita e � esquerda t�m dimens�o $n\times m$.

H� conceitos que surgem a partir destas representa��es matriciais de vetores e fun��es lineares. Um deles, de fundamental import�ncia, denominamos \textsb{mudan�a de coordenadas}\index{mudan�a!de coordenadas}. Neste nosso estudo, \emph{mudar as coordenadas de um vetor ou operador linear de uma base $B$ para uma base $C$ significa relacionar univocamente a matriz representativa desse vetor ou operador linear em $B$ com sua respectiva matriz representativa em $C$}. Para fundamentar matematicamente essa ideia, apresentamos o teorema a seguir.

\begin{mteo}{Mudan�a de Coordenadas em Vetores}{mudCoordVec}
Se $U(B)_\mathcal{R}$ e $U(C)_\mathcal{R}$ s�o espa�os vetoriais constitu�dos pelas matrizes representativas dos vetores de um espa�o vetorial $U_\mathcal{R}$ nas suas  bases $B$ e $C$ respectivamente, h� uma �nica transforma��o linear bijetora $\map{\Gamma}{U(B)_\mathcal{R}}{U(C)_\mathcal{R}}$, denominada mudan�a de coordenadas de $B$ para $C$, onde ${\Gamma}({\mav{\vto{x}}{B}})=\mav{\vto{x}}{C}$ para qualquer $\vto{x}\in U_\mathcal{R}$.
\end{mteo}

{\footnotesize
\begin{proof}
Se $B=\{\vto{x}_1,\vto{x}_2\}$ e $C=\{\alpha_1\vto{x}_1,\alpha_2\vto{x}_2\}$, a exist�ncia de $\Gamma$ fica garantida pela regra
\begin{equation*}
\fua{\Gamma}{\mat{X}}=\begin{bmatrix}
1/\alpha_1 & 0\\
0&1/\alpha_2
\end{bmatrix} \mat{X}\,.
\end{equation*}
A unicidade de $\Gamma$ � resultado trivial da suposi��o ${\Gamma_1}({\mav{\vto{x}}{B}})=\mav{\vto{x}}{C}$ e ${\Gamma_2}({\mav{\vto{x}}{B}})=\mav{\vto{x}}{C}$. Nas condi��es do teorema, podemos afirmar tamb�m que $\mav{\alpha\vto{u}+\beta\vto{v}}{B}=\alpha\mav{\vto{u}}{B}+\beta\mav{\vto{v}}{B}$ para quaisquer $\vto{u},\vto{v}\in U_\mathcal{R}$ e $\alpha,\beta\in\mathcal{R}$, de onde se conclui a linearidade de $\Gamma$, ou seja,
\begin{align*}
\Gamma(\alpha\mav{\vto{u}}{B}+\beta\mav{\vto{v}}{B})&=\Gamma(\mav{\alpha\vto{u}+\beta\vto{v}}{B})\\
&=\mav{\alpha\vto{u}+\beta\vto{v}}{C}\\
&=\alpha\mav{\vto{u}}{C}+\beta\mav{\vto{v}}{C}\\
&=\alpha\Gamma(\mav{\vto{u}}{B})+\beta\Gamma(\mav{\vto{v}}{B}).
\end{align*}
Como as matrizes que descrevem $\vto{x}\in U_\mathcal{R}$ nas bases $B$ e $C$ s�o �nicas, fica evidente que $\Gamma$ � uma inje��o. Aliado a isso, $\Gamma$ resulta uma bije��o porque inexiste matriz em $U(C)_\mathcal{R}$ que n�o tenha uma correspondente em $U(B)_\mathcal{R}$, pois qualquer vetor de $U_\mathcal{R}$ pode ser descrito em $B$ e $C$.
\end{proof}
}

Considerando agora $B=\{\vto{u}_1,\cdots,\vto{u}_n\}$ e $C=\{\vto{v}_1,\cdots,\vto{v}_n\}$ bases distintas do espa�o vetorial $U_\mathcal{R}$, ambas viabilizam representa��es matriciais distintas para qualquer vetor $\vto{u}\in U_\mathcal{R}$. Sendo assim, as igualdades
\begin{equation*}
\fua{\vtf{f}_i^C}{\vto{u}}=\vtf{f}_i^C[\,{\sum_{j=1}^n\fua{\vtf{f}_j^B}{\vto{u}}\vto{u}_j}\,]=\sum_{j=1}^n\fua{\vtf{f}_i^C}{\vto{u}_j}\fua{\vtf{f}_j^B}{\vto{u}}
\end{equation*}
em conjunto com a express�o \eqref{eq:repMatMape} permitem afirmar que
\begin{equation}\label{eq:matrizMudBase}
\mav{\vto{u}}{C}=\maf{\vtf{i}}{B}{C}\mav{\vto{u}}{B}\,,
\end{equation}
onde $\vtf{i}$ � a fun��o identidade em $U_\mathcal{R}$. Assim sendo, nos termos do teorema anterior, podemos dizer de uma regra
\begin{equation}\label{eq:regraMudCoord}
\fua{\Gamma}{\mat{X}}=\maf{\vtf{i}}{B}{C}\mat{X}
\end{equation}
relativa � mudan�a de coordenadas $\map{\Gamma}{U(B)_\mathcal{R}}{U(C)_\mathcal{R}}$, de onde resulta que a matriz representativa $\fua{\Gamma^{-1}}{\mat{X}}=\maf{\vtf{i}}{C}{B}\mat{X}$. No caso espec�fico de espa�os de Hilbert, h� uma regra para funcionais coordenados descrita pela igualdade \eqref{eq:regraFuncCoord}, que permite especificar o elemento matricial $\maf{\vtf{i}}{B}{C}_{ij}=\vto{u}_j\cdot\vto{v}^i$.


Nas condi��es da igualdade \eqref{eq:matrizMudBase}, como a matriz $\maf{\vtf{i}}{B}{C}$ � quadrada de ordem $n$ e uma fun��o identidade � igual � sua inversa e � sua transposta adjunta, podemos afirmar, a partir de \eqref{eq:matRepInv} e \eqref{eq:matRepTransp}, que
\begin{equation}\label{eq:mudaBaseTransp}
\maf{\vtf{i}}{B}{C}={\maf{\vtf{i}}{C}{B}}^{-1}
\end{equation}
e, no caso espec�fico de espa�os de Hilbert,
\begin{equation}\label{eq:mudaBaseTranspHilbert}
\maf{\vtf{i}}{B}{C}={\maf{\vtf{i}}{C}{B}}^{-1}={\maf{\vtf{i}}{C^\perp}{B^\perp}}^\dagger\,.
\end{equation}
Em outras palavras, a matriz que viabiliza a mudan�a de coordenadas de $C$ para $B$ tem como inversa a matriz que viabiliza a mudan�a de coordenadas de $B$ para $C$, cuja transposta conjugada viabiliza a mudan�a de $C^\perp$ para $B^\perp$.

Sendo $U_\mathcal{R}$ um espa�o vetorial qualquer, os vetores da base $C$ podem ser descritos na base $B$ de acordo com as seguintes express�es:
\begin{equation}\label{eq:mudaBase}
\vto{v}_j=\sum_{i=1}^n\vto{u}_i\fua{\vtf{f}_i^B}{\vto{v}_j}=\sum_{i=1}^n\vto{u}_i\maf{\vtf{i}}{C}{B}_{ij}\,,
\end{equation}
a partir das quais, dada uma base qualquer $Z$ de $U_\mathcal{R}$, podemos dizer que
\begin{equation*}
\fua{\vtf{f}_k^Z}{\vto{v}_j}=\sum_{i=1}^n\fua{\vtf{f}_k^Z}{\vto{u}_i}\maf{\vtf{i}}{C}{B}_{ij}=\sum_{i=1}^n\maf{\vtf{i}}{B}{Z}_{ki}\maf{\vtf{i}}{C}{B}_{ij}\,.
\end{equation*}
Assim, reunindo todos os vetores de ambas as bases, temos
\begin{equation}
\begin{bmatrix}
\mav{\vto{v}_1}{Z}_{11} & \mav{\vto{v}_2}{Z}_{11} & \cdots & \mav{\vto{v}_n}{Z}_{11}  \\
\vdots & \vdots & \vdots & \vdots \\
\vdots & \vdots & \vdots & \vdots \\
\mav{\vto{v}_1}{Z}_{n1} & \mav{\vto{v}_2}{Z}_{n1} & \cdots & \mav{\vto{v}_n}{Z}_{n1}
\end{bmatrix} = \begin{bmatrix}
\mav{\vto{u}_1}{Z}_{11} & \mav{\vto{u}_2}{Z}_{11} & \cdots & \mav{\vto{u}_n}{Z}_{11}  \\
\vdots & \vdots & \vdots & \vdots \\
\vdots & \vdots & \vdots & \vdots \\
\mav{\vto{u}_1}{Z}_{n1} & \mav{\vto{u}_2}{Z}_{n1} & \cdots & \mav{\vto{u}_n}{Z}_{n1}
\end{bmatrix} \maf{\vtf{i}}{C}{B}\,,
\end{equation}
de onde se conclui que $\maf{\vtf{i}}{C}{B}$ muda a base $B$ para $C$, ou seja, ela viabiliza uma \textsb{mudan�a de base}\index{mudan�a!de base}. Nessa mudan�a, quando a matriz que muda as coordenadas � inversa �quela que muda a base, dizemos que as coordenadas dos vetores de $U_\mathcal{R}$ s�o \textsb{contravariantes}\index{coordenadas!contravariantes}, pois sofrem uma transforma��o contr�ria � da mudan�a de base, como � o caso dos valores de $\Gamma$ em \eqref{eq:regraMudCoord}. Agora, vejamos o que acontece com mudan�as de coordenadas em vetores duais. Se $\vtf{h}\in U^*_\mathcal{R}$ � um vetor dual qualquer, j� sabemos que $\mav{\vtf{h}}{B^*}$ � sua matriz representativa na base dual $B^*$. Assim, a partir de \eqref{eq:mudaBase}, temos as igualdades
\begin{equation}
\mav{\vtf{h}}{C^*}=\fua{\vtf{h}}{\vto{v}_j}=\sum_{i=1}^n\fua{\vtf{h}}{\vto{u}_i}\maf{\vtf{i}}{C}{B}_{ij}=\sum_{i=1}^n\mav{\vtf{h}}{B^*}_i\maf{\vtf{i}}{C}{B}_{ij}\,,
\end{equation}
de onde conclu�mos que a mudan�a de coordenadas em $\vtf{h}$ de $B^*$ para $C^*$ � viabilizada pela mesma matriz $\maf{\vtf{i}}{C}{B}$ respons�vel pela mudan�a de base, quando dizemos que as coordenadas de vetores duais s�o \textsb{covariantes}\index{coordenadas!covariantes}. Pode-se definir ent�o uma mudan�a de coordenadas $\map{\Gamma^*}{U^*(B^*)_\mathcal{R}}{U^*(C^*)_\mathcal{R}}$ onde
\begin{equation}
\fua{\Gamma^*}{\mat{X}}=\mat{X}\,\maf{\vtf{i}}{C}{B}\,.
\end{equation}
Diante do exposto, se o espa�o $U_\mathcal{R}$ for de Hilbert, os escalares $\fua{\vtf{f}_i^B}{\vto{u}}=\vto{u}\cdot\vto{u}^i$ s�o as coordenadas contravariantes de um vetor $\vto{u}\in U_\mathcal{R}$, cujo covetor
\begin{equation*}
\vto{u}^*=\sum_{i=1}^n\fua{\vto{u}^*}{\vto{u}_i} \vtf{f}_i^{B^*}= \sum_{i=1}^n (\vto{u}_i\cdot\vto{u}) (\vto{u}^i)^*\,.
\end{equation*}
Embora impreciso, costuma-se considerar os escalares $\vto{u}_i\cdot\vto{u}$ como sendo elementos das coordenadas ``covariantes'' do vetor $\vto{u}$, no contexto de um espa�o de Hilbert. Se tal espa�o for real, considerando $\hat{B}=\{\vun{u}_1,\cdots,\vun{u}_n\}$ uma base ortonormal de $U_\real$, podemos dizer que as coordenadas contravariantes e covariantes de um vetor $\vto{u}$ qualquer s�o sempre iguais porque as igualdades $\vun{u}_i=\vun{u}^i$ e a comutatividade do produto interno implicam $\fua{\vto{u}^*}{\vto{u}_i}=\fua{\vtf{f}_i^{\hat{B}}}{\vto{u}}$. Isso tamb�m ocorre num espa�o Euclidiano tridimensional, quando a base $\{\vun{e}_1,\vun{e}_2,\vun{e}_3\}$  � denominada \textsb{base cartesiana}\index{base!cartesiana} e as combina��es lineares de seus elementos s�o \textsb{vetores cartesianos}\index{vetor!cartesiano}. Agora, \emph{uma consequ�ncia importante das igualdades \eqref{eq:mudaBaseTransp} � que a matriz ${\maf{\vtf{i}}{C}{B}}$ resulta unit�ria se as bases envolvidas forem ortonormais}, pois a rec�proca de uma base ortonormal � ela pr�pria. A partir dessa constata��o, seja o corol�rio a seguir, que descreve a rela��o entre matrizes representativas de operadores lineares.

\begin{mcoro}{Mudan�a de Coordenadas em Operadores Lineares}{mudaBase}
Se $Y(B)_\mathcal{R}$ e $Y(C)_\mathcal{R}$ s�o espa�os vetoriais constitu�dos pelas matrizes representativas dos operadores de $\evl{\mathcal{R}}{Y}{Y}$ descritas nas bases $B$ e $C$ do espa�o vetorial $Y_\mathcal{R}$ respectivamente, a mudan�a de coordenadas $\map{\Theta}{Y(B)_\mathcal{R}}{Y(C)_\mathcal{R}}$ � sempre uma transforma��o de similaridade onde $\fua{\Theta}{\mat{X}}=\maf{\vtf{i}}{B}{C}\mat{X}\,\maf{\vtf{i}}{C}{B}$.
\end{mcoro}

{\footnotesize
\begin{proof}
Considerando uma fun��o qualquer $\vtf{g}\in\evl{\mathcal{R}}{Y}{Y}$ e um vetor qualquer $\vto{u}\in Y_\mathcal{R}$, a �ltima igualdade do desenvolvimento
\begin{align}
\mav{\fua{\vtf{g}}{\vto{u}}}{B}&=\maf{\vtf{g}}{B}{B}\mav{\vto{u}}{B}\nonumber\\
\maf{\vtf{i}}{C}{B}\mav{\fua{\vtf{g}}{\vto{u}}}{C}&=\maf{\vtf{g}}{B}{B}\maf{\vtf{i}}{C}{B}\mav{\vto{u}}{C}\nonumber\\
\mav{\fua{\vtf{g}}{\vto{u}}}{C}&=\maf{\vtf{i}}{B}{C}\maf{\vtf{g}}{B}{B}\maf{\vtf{i}}{C}{B}\mav{\vto{u}}{C}\nonumber
\end{align}
e a igualdade $\mav{\fua{\vtf{g}}{\vto{u}}}{C}=\maf{\vtf{g}}{C}{C}\mav{\vto{u}}{C}$ nos permitem afirmar que
\begin{equation*}
\maf{\vtf{g}}{C}{C} = \maf{\vtf{i}}{B}{C}\maf{\vtf{g}}{B}{B}\maf{\vtf{i}}{C}{B}\,.
\end{equation*}
Como $\maf{\vtf{i}}{B}{C}={\maf{\vtf{i}}{C}{B}}^{-1}$, conclu�mos que as matrizes $\maf{\vtf{g}}{C}{C}$ e $\maf{\vtf{g}}{B}{B}$ s�o similares.
\end{proof}
}

Neste cap�tulo, seguiremos ent�o trabalhando com fun��es lineares que admitem mudan�a de coordenadas, ou seja, com operadores lineares. Nesse contexto, alguns conceitos t�picos de matrizes podem ser aplicados a tais operadores. Vejamos quais. Nas condi��es do corol�rio anterior, as matrizes $\fua{\Theta}{\mat{X}}$ e $\mat{X}$ s�o similares, de onde se pode concluir que \emph{o determinante e o tra�o de matrizes representativas de operadores lineares s�o imunes ou indiferentes a mudan�as de base}. Diante dessa indiferen�a, consideramos os escalares $\det\vtf{g}:=\det [\vtf{g}]$ e $\trc\vtf{g}:= \trc{\maf{\vtf{g}}{}{}}$ como sendo respectivamente o \textsb{determinante}\index{determinante!de operador linear} e o \textsb{tra�o}\index{tra�o!de operador linear} do operador $\vtf{g}\in\evl{\mathcal{R}}{Y}{Y}$, onde $\maf{\vtf{g}}{}{}$ � a matriz representativa desse operador numa base qualquer de $Y_\mathcal{R}$. Um outro conceito que operadores lineares herdam de matrizes � o de positividade. Considerando $B$ uma base qualquer de $Y_\mathcal{R}$, se o escalar
\begin{equation}
\Re\lpa{\mav{\vto{y}}{B}}^\dagger\maf{\vtf{g}}{B}{B}{\mav{\vto{y}}{B}}\rpa_{11}\geqslant0\,,\,\,\forall\,\vto{y}\in Y_\real\,,
\end{equation}
a matriz $\maf{\vtf{g}}{B}{B}$ � dita n�o-negativa, segundo defini��o apresentada no cap�tulo anterior. Essa desigualdade continua v�lida se $B$ for ortonormal, quando se obt�m
\begin{equation}
\underbrace{\Re\lpa{\mav{\vto{y}}{B}}^\dagger\mav{\fua{\vtf{g}}{\vto{y}}}{B}\rpa_{11}   }_{\Re(\vto{y}\cdot\fua{\vtf{g}}{\vto{y}})}\geqslant0\,,\,\,\forall\,\vto{y}\in Y_\real\,,
\end{equation}
onde $\vtf{g}$ � dito um \textsb{operador linear n�o-negativo}\footnote{Ou positivo-semidefinido\index{operador linear!positivo-semidefinido}.}\index{operador linear!n�o-negativo} ou um \textsb{operador linear positivo-definido}\index{operador linear!positivo-definido} se o produto interno � esquerda for sempre positivo.

Neste ponto, vamos interromper brevemente a evolu��o do conte�do te�rico para tratarmos de um exemplo: o assunto mudan�a de coordenadas � demasiado importante para o nosso estudo e conv�m discorrermos sobre algo menos abstrato.

\begin{example}
Eis uma historinha pueril. A professora Bruna, residente � margem de um rio de largura extensa, entrega a um barqueiro, em frente � sua casa, um presente que deve ser transportado at� um ponto da outra margem, onde mora o engenheiro Carlos, estimado destinat�rio da encomenda. Num determinado ponto da travessia, o barqueiro � obrigado a mudar sua velocidade de sorte que tal manobra o impedir� de chegar na hora marcada e no local exato onde Carlos aguarda ansiosamente o presente. Ciente da import�ncia de sua incumb�ncia, o barqueiro, a partir das medi��es de seus instrumentos, envia uma mensagem de texto para o telefone de Carlos informando o seguinte: \textsl{Carlos, agora s�o 14:00 e ap�s percorridos 30km rio adentro (perpendicular � margem) e 5km rio acima (contra o sentido da correnteza) em rela��o � Bruna, eu estava a 49km/h rio adentro e 13,1km/h rio acima quando avistei uma parte muito rasa do leito, sendo obrigado a reduzir em 20\% minha velocidade norte e em 40\% a velocidade leste. Como n�o me ser� poss�vel corrigir a rota, pe�o que me encontre nos novos local e hora em que chegarei � sua margem. N�o se esque�a que Bruna est� localizada em rela��o � voc� 30km rio abaixo e 55km rio adentro.} Embora bastante frustrado, Carlos respirou fundo, manteve a calma e lembrou-se de suas saudosas aulas de �lgebra Linear. Voltou para casa, pegou l�pis, papel e calculadora; sentou-se � mesa e raciocinou: ``Em primeiro lugar, vou admitir um espa�o Euclidiano bidimensional com a base natural $O=\{\vun{e}_1,\vun{e}_2\}$, onde $\vun{e}_1$ representa o leste e $\vun{e}_2$ o norte. Assim, em rela��o � essa base, j� sei que $\mav{\vto{c}_1}{O}=[-1\;\;0]^\text{T}$ e $\mav{\vto{c}_2}{O}=[-0,42\;\;0,91]^\text{T}$ s�o as matrizes representativas dos vetores de meu ponto de vista $C=\{\vto{c}_1,\vto{c}_2\}$, onde $\vto{c}_1$ relaciona-se com o sentido rio adentro e $\vto{c}_2$ com rio acima. Por essa mesma l�gica, no caso de Bruna, sei que as matrizes $\mav{\vto{b}_1}{O}=[0,87\;\;0,5]^\text{T}$ e $\mav{\vto{b}_2}{O}=[0\;\;1]^\text{T}$ representam os vetores de seu ponto de vista $B=\{\vto{b}_1,\vto{b}_2\}$.  O barqueiro disse que sua velocidade $\vto{v}$ era descrita por $\mav{\vto{v}}{B}=[49\;\;13.1]^\text{T}$ quando precisou fazer uma mudan�a $\vtf{f}$ representada por
\begin{equation*}
\mav{\vtf{f}_O}{O}=\begin{bmatrix}
0,6      & 0 \\
0      & 0,8 \\
\end{bmatrix}\,.
\end{equation*}
Para obter o que preciso, vou descrever $\vto{v}$ e $\vtf{f}$ no meu ponto de vista. Ainda me lembro das igualdades $\mav{\vto{v}}{C}=\maf{\vtf{i}}{B}{C}\mav{\vto{v}}{B}$ e $\mav{\vtf{f}_C}{C}=\maf{\vtf{i}}{O}{C}\mav{\vtf{f}_O}{O}\maf{\vtf{i}}{C}{O}$, onde
\begin{alignat*}{3}
\maf{\vtf{i}}{B}{C}_{ij}=\vto{b}_j\cdot\vto{c}^i & \qquad\text{e} \qquad & \maf{\vtf{i}}{O}{C}_{ij}=\vun{e}_j\cdot\vto{c}^i\,.
\end{alignat*}
Diante disso, preciso descobrir minha base rec�proca $C^\perp=\{\vto{c}^1,\vto{c}^2\}$, onde o produto interno $\vto{c}_i\cdot\vto{c}^j=\delta_{ij}$. Dessa �ltima igualdade, posso escrever os sistemas
\begin{alignat*}{3}
\begin{cases}
-x=1\\-0,42x+0,91y=0	
\end{cases}
& \qquad\text{e} \qquad &
\begin{cases}
	-x=0\\-0,42x+0,91y=1	
\end{cases}\,,
\end{alignat*}
cujas solu��es conduzem � $[\vto{c}^1]^{O}=[-1\;\;-0,46]^\text{T}$ e $[\vto{c}^2]^{O}=[0\;\;1,1]^\text{T}$, que viabilizam as matrizes
\begin{alignat*}{3}
\maf{\vtf{i}}{B}{C}=
\begin{bmatrix}
	-1,1      & -0,46 \\
	0,55      & 1,1 \\
\end{bmatrix}
	& \qquad\text{e} \qquad &
\maf{\vtf{i}}{O}{C}=
\begin{bmatrix}
	-1      & 0 \\
	-0,42     & 0,91 \\
\end{bmatrix}\,.
\end{alignat*}
Da�, chego �s representa��es sob minha perspectiva:
\begin{alignat*}{3}
	\mav{\vto{v}}{C}=
	\begin{bmatrix}
		-59,93       \\
		41,36       \\
	\end{bmatrix}
	& \qquad\text{e} \qquad &
	\mav{\vtf{f}_C}{C}=
	\begin{bmatrix}
		0,6      & 0,25 \\
		0,25     & 0,77 \\
	\end{bmatrix}\,.
\end{alignat*}
Ap�s o desvio, a velocidade $\fua{\vtf{f}}{\vto{v}}$ do barco fica  $\mav{\fua{\vtf{f}}{\vto{v}}}{C}=[-25,62\;\;16,87]^\text{T}$. Bem, no momento do desvio, o deslocamento $\vto{u}$ do barco era $\mav{\vto{u}}{B}=[30\;\;5]^\text{T}$ ou, do meu ponto de vista, $\mav{\vto{u}}{C}=\maf{\vtf{i}}{B}{C}\mav{\vto{u}}{B}=[-35,3\;\;22]^\text{T}$. Como ele informou a posi��o de Bruna em rela��o a mim, o barco est� percorrendo o deslocamento que lhe resta $\mav{\vto{z}}{C}=[-24,7\;\;8]^\text{T}$ numa velocidade rio adentro de -25,62km/h, o que demandar� 0,96h. Nesse tempo, com velocidade de 16,87km/h rio acima, ele vai percorrer 16,2km nesse sentido, que significa 8,2km rio acima de onde estou. Ele chegar� nesse ponto por volta das 15:00. Como agora s�o 14:30, de carro chegarei a tempo.''
\end{example}


\section{Autovalores e Autovetores}\label{sec:autoPares}

Considerando um espa�o de Hilbert $U_\mathcal{R}$ $n$-dimensional, denominamos $\alpha\vto{u}\in U_\mathcal{R}$, onde $\alpha\in\mathcal{R}$, um \textsb{m�ltiplo escalar}\index{m�ltiplo escalar} do vetor $\vto{u}$. O escalar $\alpha$, que indica essa multiplicidade, no contexto da norma $\|\alpha\vto{u}\|=|\alpha|\|\vto{u}\|$, promove uma esp�cie de redimensionamento de $\vto{u}$, ou seja, se $|\alpha|<1$, h� uma diminui��o de seu tamanho ou intensidade; se $|\alpha|>1$, h� um aumento. O vetor $\vto{u}$ � chamado \textsb{autovetor}\index{autovetor} de um operador qualquer $\vtf{l}\in\evl{\mathcal{R}}{U}{U}$ se for n�o nulo e o valor $\fua{\vtf{l}}{\vto{u}}$ for seu m�ltiplo escalar, ou seja, se $\fua{\vtf{l}}{\vto{u}}=\alpha\vto{u}$, onde a multiplicidade $\alpha$ � dita o \textsb{autovalor}\index{autovalor} de $\vtf{l}$. Em outras palavras, \emph{um vetor que $\vtf{l}$ redimensione � seu autovetor e o sinal-magnitude desse redimensionamento seu autovalor}.

Vamos supor agora que, conhecido o operador $\vtf{l}$, queremos descobrir todos os seus autovalores e autovetores a partir das inc�gnitas da equa��o $\fua{\vtf{l}}{\vto{x}}=\lambda\vto{x}$ ou, melhor dizendo, a partir de $\lambda$ e $\vto{x}$ em
\begin{equation}\label{eq:probAutoValor}
\fua{(\vtf{l}-\lambda\vtf{i})}{\vto{x}}=\vto{0},
\end{equation}
onde $\vtf{i}$ � a fun��o identidade em $U_\mathcal{R}$. Numa base qualquer desse espa�o, a matriz $[\vtf{i}]=\mat{I}$ e a representa��o matricial da equa��o anterior resulta $([\vtf{l}]-\lambda\mat{I})[\vto{x}]=0$. Se a matriz $[\vtf{l}]-\lambda\mat{I}$ fosse invers�vel, a pr� multiplica��o de sua inversa por ambos os lados da equa��o matricial anterior resultaria em $[\vto{x}]=0$ ou num autovetor nulo. Para que isso n�o ocorra, tal matriz precisa ser singular, ou seja,
\begin{equation}\label{eq:autoMatricial}
\det{([\vtf{l}]-\lambda\mat{I})}=0\,.
\end{equation}
Sabemos que o lado esquerdo dessa equa��o � o polin�mio caracter�stico\footnote{Ver defini��o � p. \pageref{pg:PolinomioCarac}.} de $[\vtf{l}]$ na vari�vel $\lambda$, cujas $n$ ra�zes caracter�sticas de $[\vtf{l}]$ solucionam \eqref{eq:autoMatricial}, ou seja, h� $n$ autovalores de $\vtf{l}$, n�o necessariamente distintos. De posse dos autovalores, podemos determinar cada um dos autovetores correspondentes a partir da equa��o $([\vtf{l}]-\lambda\mat{I})[\vto{x}]=0$. Como todo esse desenvolvimento independe de base, diz-se que $\det(\vtf{l}-\lambda\vtf{i})$ � o polin�mio caracter�stico de $\vtf{l}$.

Para subsidiar o importante Teorema da Decomposi��o Polar, sobre o qual discorreremos mais adiante, conv�m apresentar agora tr�s propriedades do operador hermitiano: a) \emph{os autovalores de um operador hermitiano s�o sempre reais}; b) \emph{dois autovetores distintos de um operador hermitiano s�o sempre ortogonais}; c) \emph{o operador hermitiano resultante da composi��o de um operador com o seu adjunto � sempre n�o-negativo}. Essa �ltima propriedade implica dizer que os $n$ autovetores do operador hermitiano s�o distintos entre si e o conjunto por eles formado � uma base do espa�o $U_\mathcal{R}$, pois conjuntos ortogonais s�o sempre linearmente independentes.

{\footnotesize
\begin{proof}
Vamos demonstrar as tr�s propriedades descritas acima. Seja $\vtf{h}\in\evl{\mathcal{R}}{U}{U}$ um operador hermitiano com autovalores $\lambda_i$ e autovetores $\vto{x}_i$, de onde podemos escrever a express�o $\fua{\vtf{h}}{\vto{x}_i}=\lambda_i\vto{x}_i$. Fazendo o produto interno de um autovalor $\vto{x}_j$ pelos dois lados dessa igualdade, obt�m-se $\vto{x}_j\cdot\fua{\vtf{h}}{\vto{x}_i}=\vto{x}_j\cdot\lambda_i\vto{x}_i$ (a). Agora, tomando a igualdade $\fua{\vtf{h}}{\vto{x}_j}=\lambda_i\vto{x}_j$ e fazendo o produto interno de ambos os lados por $\vto{x}_i$, o resultado � $\fua{\vtf{h}}{\vto{x}_j}\cdot\vto{x}_i=\lambda_j\vto{x}_j\cdot\vto{x}_i$ (b). Porque $\vtf{h}$ � hermitiano, subtraindo-se (b) de (a), pode-se realizar o seguinte desenvolvimento:
\begin{align*}
\vto{x}_j\cdot\fua{\vtf{h}}{\vto{x}_i}-\fua{\vtf{h}}{\vto{x}_j}\cdot\vto{x}_i&=\vto{x}_j\cdot\lambda_i\vto{x}_i-\lambda_j\vto{x}_j\cdot\vto{x}_i\\
0&=(\overline{\lambda_i}-\lambda_j)\vto{x}_j\cdot\vto{x}_i\,.
\end{align*}
Como a �ltima igualdade � v�lida para qualquer par $(i,j)$, quando $i=j$, o escalar $\overline{\lambda_i}=\lambda_i$, o que comprova a propriedade (a). A partir dela e se $\vto{x}_i\neq\vto{x}_j$, o escalar real $\lambda_i\neq\lambda_j$. Diante disso e da �ltima igualdade, a conclus�o que $\vto{x}_i\perp\vto{x}_j$ demonstra a propriedade (b). Para a terceira propriedade, considerando o operador hermitiano $\vtf{g}\circ\vtf{g}^\dagger$, onde $\vtf{g}\in\evl{\mathcal{R}}{U}{U}$, precisamos mostrar que � n�o-negativo o n�mero real $\Re([\vtf{y}]^\dagger[\vtf{g}][\vtf{g}]^\dagger[\vtf{y}])_{11}$, cujas matrizes s�o descritas numa base ortonormal qualquer. Se a matriz $\mat{A}=[\vtf{g}]^\dagger[\vtf{y}]$, o n�mero real anterior fica $\Re(\mat{A}^\dagger \mat{A})_{11}$, que � sempre n�o-negativo, pois $(\mat{A}^\dagger \mat{A})_{11}=\sum_{i=1}^{n}\overline{\mat{A}_{i1}}\mat{A}_{i1}=\sum_{i=1}^{n}|\mat{A}_{i1}|^2$.
\end{proof}
}

Al�m das propriedades citadas, todo operador hermitiano, no contexto de espa�os de Hilbert, possui matriz representativa hermitiana, pois para uma base ortonormal qualquer $\hat{B}$, s�o v�lidas as seguintes igualdades:
\begin{equation}\label{eq:matOpeHermit}
{\maf{\vtf{h}}{\hat{B}}{\hat{B}}}^\dagger=[\vtf{h}^\dagger_{\hat{B}}]^{\hat{B}}=[\vtf{h}_{\hat{B}}]^{\hat{B}}\,,
\end{equation}
a partir de \eqref{eq:matRepTransp}. Por serem hermitianas, tais matrizes representativas s�o normais, ou seja, s�o pass�veis de diagonaliza��o espectral\footnote{Ver defini��o de matriz normal � p. \pageref{nm:Normal}.}. Assim, dado um operador hermitiano qualquer $\vtf{h}\in\evl{\mathcal{R}}{U}{U}$,
seja $\vto{x}_i$ cada um de seus $n$ autovetores mutuamente ortogonais. Se $\widetilde{\mat{H}}$ for a matriz diagonal resultante da diagonaliza��o espectral de $[\vtf{h}_{\hat{B}}]^{\hat{B}}$ e  $X=\{\vto{x}_1,\cdots,\vto{x}_n\}$ a base ortogonal de seus autovetores, tem-se
\begin{equation}
\widetilde{\mat{H}}_{ij}=\lambda_{j}\delta_{ij}=\lambda_{j}\vto{x}_j\cdot\vto{x}^i=\lambda_{j}\fua{\vtf{f}_i^X}{\vto{x}_j}=\fua{\vtf{f}_i^X}{\lambda_{j}\vto{x}_j}=\fua{\vtf{f}_i^X}{\fua{\vtf{h}}{\vto{x}_j}}=[\vtf{h}_X]^X_{ij}\,.
\end{equation}
� essa matriz representativa de $\vtf{h}$, descrita pela base de seus autovetores, que corresponde � matriz diagonal espectral de $[\vtf{h}_{\hat{B}}]^{\hat{B}}$, damos o nome de \textsb{representa��o espectral} do operador $\vtf{h}$. Ainda nas condi��es colocadas, queremos agora mudar a base $X$, que descreve a matriz representativa de $\vtf{h}$, para uma base qualquer $C$ de $U_\mathcal{R}$. Podemos escrever ent�o que
\begin{equation}
[\vtf{h}_C]^C=\maf{\vtf{i}}{X}{C}\mav{\vtf{h}_X}{X}{\maf{\vtf{i}}{C}{X}}=\maf{\vtf{i}}{X}{C}\mav{\vtf{h}_X}{X}{\maf{\vtf{i}}{X}{C}}^\dagger\,,
\end{equation}
pois $\maf{\vtf{i}}{X}{C}$ � uma matriz unit�ria. Al�m disso, como as matrizes representativas, em bases ortonormais, do operador hermitiano $\vtf{h}$ s�o normais, a mudan�a de base de $X$ para $C$ resulta uma decomposi��o espectral\footnote{Nos termos do teorema \ref{teo:decompSpec}, diz-se que as igualdades $\widetilde{\mat{N}} = \mat{U}^\dagger\mat{N}\mat{U}$ e $\mat{N}= \mat{U}\widetilde{\mat{N}}\mat{U}^\dagger$ s�o a diagonaliza��o e a \textsb{decomposi��o}\index{decomposi��o espectral} espectrais de $\mat{N}$ respectivamente.}. Agora, consideremos o operador $\vtf{h}$ n�o-negativo. Da defini��o apresentada ao final da se��o anterior, podemos escrever que o escalar $\Re(\vto{x}_i\cdot\fua{\vtf{h}}{\vto{x}_i})\geqslant 0$. A partir dessa desigualdade, como os autovalores de $\vtf{h}$ s�o reais, temos $\lambda_i(\vto{x}_i\cdot\vto{x}_i)\geqslant 0$, de onde resultam autovalores $\lambda_i$ n�o negativos, uma vez que o produto interno $\vto{x}_i\cdot\vto{x}_i$ � positivo.

Antes de tratarmos do teorema que vai finalizar este cap�tulo, precisamos apresentar uma defini��o adicional no �mbito dos operadores lineares. Em nosso estudo, um operador hermitiano n�o-negativo $\vtf{h}$ pode ser decomposto segundo a igualdade $\vtf{h}=\vtf{h}^{\nicefrac{1}{2}}\circ\vtf{h}^{\nicefrac{1}{2}}$, onde o operador $\vtf{h}^{\nicefrac{1}{2}}\in\evl{\mathcal{R}}{U}{U}$, �nico e hermitiano n�o-negativo, � denominado \index{operador!raiz quadrada de} \textsb{raiz quadrada} de $\vtf{h}$. J� sabemos  que a composi��o de fun��es se expressa, em termos matriciais, como o produto das matrizes representativas dessas fun��es. Assim, a denomina��o ``raiz quadrada'' se deve � igualdade $[\vtf{h}]=[\vtf{h}^{\nicefrac{1}{2}}][\vtf{h}^{\nicefrac{1}{2}}]$, que remete ao mesmo conceito aplicado a escalares.

{\footnotesize
\begin{proof}\footnote{Demonstra��o adaptada de \aut{Gurtin}\cite{gurtin_1981}, pp. 13-14.}
Precisamos mostrar que a raiz quadrada existe e � �nica. Para a �ltima igualdade apresentada, v�lida para uma base qualquer, vamos escolher a base ortonormal $\hat{X}=\{\vun{x}_1,\cdots,\vun{x}_n\}$ formada a partir dos autovetores de $\vtf{h}$. Assim,
\begin{equation*}
\maf{\vtf{h}}{\hat{X}}{\hat{X}}=\maf{\vtf{h}^{\nicefrac{1}{2}}}{\hat{X}}{\hat{X}}\maf{\vtf{h}^{\nicefrac{1}{2}}}{\hat{X}}{\hat{X}},
\end{equation*}
e j� sabemos que $\maf{\vtf{h}}{\hat{X}}{\hat{X}}_{ij}=\lambda_i\delta_{ij}$, sendo $\lambda_i\geq 0$. Se admitirmos $\maf{\vtf{h}^{\nicefrac{1}{2}}}{\hat{X}}{\hat{X}}_{ij}=\delta_{ij}\sqrt{\lambda_{i}}$, essa matriz resulta n�o-negativa e hermitiana, de onde se conclui $\vtf{h}^{\nicefrac{1}{2}}$ hermitiano n�o-negativo por conta das igualdades \eqref{eq:matOpeHermit}. Diante disso, fica constatada a exist�ncia de uma raiz quadrada de $\vtf{h}$. Para demonstrar a unicidade de $\vtf{h}^{\nicefrac{1}{2}}$, por hip�tese, seja $\vtf{c}^{\nicefrac{1}{2}}\circ\vtf{c}^{\nicefrac{1}{2}}=\vtf{h}$. Adotando uma base qualquer $\con{B}$, um vetor $\vto{u}\in\con{V}$ e a igualdade \eqref{eq:probAutoValor}, pode-se fazer o seguinte
desenvolvimento:
\begin{eqnarray}
0&=&\lpa \mad{\vtf{h}^{\nicefrac{1}{2}}_{\con{B}}}{B}\mad{\vtf{h}^{\nicefrac{1}{2}}_{\con{B}}}{B}-\lambda\mat{I}\rpa \lco \vto{x}_i \rco^{B} \nonumber\\
&=&\lpa\mad{\vtf{h}^{\nicefrac{1}{2}}_{\con{B}}}{B}+\sqrt{\lambda_i}\,\,\mat{I}\rpa\underbrace{\lpa\mad{\vtf{h}^{\nicefrac{1}{2}}_{\con{B}}}{B}-\sqrt{\lambda_i}\,\,\mat{I}\rpa\lco\vto{x}_i \rco^{B}}_{\lco \vto{u} \rco^{B}} \nonumber\,,
\end{eqnarray}
de onde se conclui que
\begin{equation}
-\sqrt{\lambda_i}\mav{\vto{u}}{B} =
\mad{\vtf{h}^{\nicefrac{1}{2}}_{\con{B}}}{B}\mav{\vto{u}}{B}\,.\nonumber
\end{equation}
A matriz $\lco \vto{u} \rco^{B}$, que abrevia o termo destacado, �
nula; caso contr�rio, ocorreria a situa��o imposs�vel de um
autovalor negativo associado ao operador hermitiano
n�o-negativo $\vtf{h}^{\nicefrac{1}{2}}$. No caso de $\lambda_i=0$,
n�o h� restri��o para a matriz $\lco \vto{u} \rco^{B}$, podendo
ser nula, por exemplo. Ent�o, o termo destacado fica assim:
\begin{equation}
\sqrt{\lambda_i}\lco \vto{x}_i \rco^{B} =
\mad{\vtf{h}^{\nicefrac{1}{2}}_{\con{B}}}{B}\lco \vto{x}_i
\rco^{B}\,.\nonumber
\end{equation}
Este mesmo procedimento pode ser aplicado ao operador
$\vtf{c}^{\nicefrac{1}{2}}$, de onde se conclui que
\begin{equation}
\mad{\vtf{h}^{\nicefrac{1}{2}}_{\con{B}}}{B}\lco \vto{x}_i
\rco^{B}=\mad{\vtf{c}^{\nicefrac{1}{2}}_{\con{B}}}{B}\lco \vto{x}_i
\rco^{B}\nonumber
\end{equation}
para qualquer um dos autovetores de $\vtf{h}$. J� que eles s�o n�o
nulos, $\vtf{h}^{\nicefrac{1}{2}}$ � �nico.
\end{proof}
}


Um operador do grupo unit�rio $(O,\circ)$, cujos elementos t�m o espa�o de Hilbert $U_\mathcal{R}$ como dom�nio, pode ser representado por uma matriz unit�ria, se a base utilizada for ortonormal. Em outras palavras, se $\vtf{q}\in O$ e $\hat{B}$ � base ortonormal de $U_\mathcal{R}$, pelas igualdades \eqref{eq:matRepInv} e \eqref{eq:matRepTransp}, a matriz
\begin{equation}
{\maf{\vtf{q}}{\hat{B}}{\hat{B}}}^{-1}={\maf{\vtf{q}}{\hat{B}}{\hat{B}}}^\dagger.
\end{equation}
Conv�m recordar que um operador unit�rio preserva a norma, ou seja, ele n�o altera o ``tamanho'' ou a ``intensidade'' dos vetores de seu dom�nio. Para o nosso estudo, seria muito interessante poder discriminar essa caracter�stica em operadores lineares quaisquer por meio de uma decomposi��o, de tal sorte que haja uma parcela unit�ria e uma n�o-unit�ria, respons�vel exclusivamente por altera��es da norma. O teorema a seguir viabiliza essa demanda\footnote{O termo ``polar'' que d� nome ao teorema diz respeito a uma caracter�stica similar � forma polar de um n�mero complexo, onde h� uma parcela real n�o-negativa que descreve magnitude e outra de magnitude sempre unit�ria}.

\begin{mteo}{Decomposi��o Polar}{decoPolar}\label{teo:decompPolar}\index{decomposi��o polar}
Uma bije��o qualquer $\vtf{g}\in\evl{\mathcal{R}}{U}{U}$, onde $U_\mathcal{R}$ � um espa�o de Hilbert, possui uma �nica decomposi��o $\vtf{g}=\vtf{q}\circ\vtf{h}^{\nicefrac{1}{2}}\,$, onde $\vtf{q}\in\evl{\mathcal{R}}{U}{U}$ � unit�rio e $\vtf{h}=\vtf{g}\circ\vtf{g}^\dagger$ hermitiano n�o-negativo.
\end{mteo}


{\footnotesize
\begin{proof}
Sejam a bije��o $\vtf{g}\in\evl{\mathcal{R}}{U}{U}$ e o operador hermitiano n�o-negativo $\vtf{h}=\vtf{g}\circ\vtf{g}^{\dagger}$. Sabemos que a decomposi��o 	$\vtf{h}=\vtf{h}^{\nicefrac{1}{2}}\circ\vtf{h}^{\nicefrac{1}{2}}$ existe, e portanto
\begin{align*}
\vtf{h}^{\nicefrac{1}{2}}\circ\vtf{h}^{\nicefrac{1}{2}}&=\vtf{g}\circ\vtf{g}^{\dagger}\\
\vtf{i}&=\underbrace{\vtf{h}^{-\nicefrac{1}{2}}\circ\vtf{g}}_{\vtf{q}}\circ\underbrace{\vtf{g}^{\dagger}\circ\vtf{h}^{-\nicefrac{1}{2}}}_{\vtf{q}^{\dagger}}\,,
\end{align*}	
de onde conclu�mos que o termo destacado $\vtf{q}$ � unit�rio. Assim, podemos afirmar que a decomposi��o polar existe pois a bije��o $\vtf{g}=\vtf{q}\circ\vtf{h}^{\nicefrac{1}{2}}$. Como a raiz quadrada $\vtf{h}^{\nicefrac{1}{2}}$ � �nica para $\vtf{g}\circ\vtf{g}^\dagger$, ent�o o operador unit�rio $\vtf{q}=\vtf{g}\circ\vtf{h}^{-\nicefrac{1}{2}}$ tamb�m � �nico; o que resulta numa decomposi��o polar �nica para $\vtf{g}$.
\end{proof}
}

\begin{mcoro}{Decomposi��es Polares � Direita e � Esquerda}{decompPolarEsquerda}\label{teo:decompPolarEsquerda}
Dada a decomposi��o polar $\vtf{g}=\vtf{q}\circ\vtf{h}_1^{\nicefrac{1}{2}}$, a igualdade $\vtf{g}=\vtf{h}_2^{\nicefrac{1}{2}}\circ\vtf{q}$, onde $\vtf{h}_2=\vtf{g}^\dagger\circ\vtf{g}$, � �nica. Por isso, a primeira decomposi��o denominamos \textsb{decomposi��o polar � direita}\index{decomposi��o polar!� direita} e a segunda \textsb{decomposi��o polar � esquerda}\index{decomposi��o polar!� esquerda}.
\end{mcoro}

{\footnotesize
\begin{proof}
A demonstra��o da decomposi��o polar � esquerda segue o mesmo procedimento da demonstra��o do teorema anterior. Comprovemos agora que os operadores unit�rios em ambas as decomposi��es s�o iguais. A partir da decomposi��o polar � esquerda cujo operador unit�rio � $\vtf{q}_1$, se a parcela $\vtf{c}=\vtf{q}_1^{-1}\circ\vtf{h}_2^{\nicefrac{1}{2}}\circ\vtf{q}_1$ da igualdade $\vtf{g}=\vtf{q}_1\circ\vtf{q}_1^{-1}\circ\vtf{h}_2^{\nicefrac{1}{2}}\circ\vtf{q}_1$ for hermitiana n�o-negativa,  constata-se uma decomposi��o polar � direita e portanto $\vtf{q}_1=\vtf{q}$. A comprova��o de que $\vtf{c}=\vtf{c}^\dagger$ � trivial. Vamos verificar agora se, dada uma base ortonormal qualquer e um vetor $\vto{x}\in U_\mathcal{R}$, o escalar $\Re([\vto{x}]^\dagger[\vtf{c}][\vto{x}])_{11}$ � n�o-negativo. Da igualdade $[\vto{x}]^\dagger[\vtf{c}][\vto{x}]=[\vto{x}]^\dagger[\vtf{q}_1]^{-1}[\vtf{h}_2^{\nicefrac{1}{2}}][\vtf{q}_1][\vto{x}]$, podemos concluir que $\Re(\mat{A}^\dagger[\vtf{h}_2^{\nicefrac{1}{2}}]\mat{A})_{11}\geq 0$, onde $\mat{A}=[\vtf{q}_1][\vto{x}]$, pois $\vtf{h}_2^{\nicefrac{1}{2}}$ � n�o-negativo.
\end{proof}

% Quando for falar sobre tensores, dizer que um tensor � elemento do espa�o dual de um espa�o vetorial constitu�do por enuplas de vetores. Diz-se que o elemento desse espa�o vetorial � um vetor de ordem n e o elemento do espa�o dual um vetor dual de ordem n ou tensor.

%Estudar esta afirma��o com cuidado: "quando a regra de um tensor f for f(x,y)=u*(x)v*(y), ent�o a rela��o entre f e (u,v) � un�voca, quando chamados f de produto tensorial de u com v, representando u\otimes v".

}



%%% Local Variables:
%%% mode: latex
%%% TeX-master: "../../msav.tex"
%%% End: 
    \chapter{Tensor Algebra}


Continuing our study of Linear Algebra, whose foundations we presented in the previous chapter, let's deal now with something a bit more sophisticated, although extremely useful for describing physical phenomena concerning deformable bodies. The form and content that we shall present about this topic differs from the usual approach available in Continuum Mechanics literature, where the concept of tensor arises suddenly, through simplistic definitions: formally, we shall proceed the same way as we have been doing so far, in a continuous flow of successively dependent concepts, and present tensors as ordinary elements of a vector space; on the content side, we shall not restrict ourselves to the strictly necessary, typical of pragmatic approaches, since our objective, in this and the next chapter, is to provide the reader with a generous framework of concepts related to tensors, simply because of their mathematical beauty and wide application. 


\section{A Very Short History of Tensors}

The word ``tensor'' was used for the first time in Mathematics in 1846 by \aut{Hamilton}\cite{hamilton_1846_1} in order to designate the modulus of numbers that he called quaternions, which are a generalization of complex numbers. In 1881, \aut{Gibbs}\cite{gibbs_1881} introduced the notion of dyad as being an ``undefined product'' of two vectors, starting, in a rather obscure way, the concept that nowadays we understand as tensor product. Since the term ``tensor'' has the latin word \emph{tendere}, which means ``to stretch'', as its etymological origin, \aut{Voigt}\cite{voigt_1898_1} adopted a meaning closer to the current mechanical approach, when he used tensor to designate the combination of a vector measure, representing the forces and stretches involved, with another vector measure, normal to the surface in question.        

At the end of XIX century, after ten years conceiving his ``Absolute Differential Calculus'', the italian mathematician Gregorio Ricci-Curbastro(1853-1925) published a seminal paper entitled \emph{M�thodes de Calcul Diff�rentiel Absolu et Leurs Applications}\footnote{See \aut{Ricci \& Levi-Civita}\cite{ricci_1900_1} and \aut{Hermann}\cite{hermann_1975}.}, together with his pupil  Tullio Levi-Civita(1873-1941), in which they established the foundations of what is known today as Tensor Analysis. In this work, a tensor or ``system'', as they called, is a measure that arises from the coefficients of polynomials like     
\begin{equation*}
\sum_{i_1=1}^n\cdots\sum_{i_m=1}^n\underbrace{{(\alpha_1)}_{i_1}\cdots{(\alpha_m)}_{i_m}}_{\mat{A}_{i_1\cdots i_m}} {(x_1)}_{i_1}\cdots {(x_m)}_{i_m}\,,
\end{equation*}
where  $n\times\cdots \times n$ ($n$ times) array $\mat{A}$ represents this measure numerically. Moreover, the values of such polynomials must be immune to changes of variables, that is,  
\begin{equation*}
\sum_{i_1=1}^n\cdots\sum_{i_m=1}^n\mat{A}_{i_1\cdots i_m} {(x_1)}_{i_1}\cdots {(x_m)}_{i_m}=
\sum_{i_1=1}^n\cdots\sum_{i_m=1}^n\mat{B}_{i_1\cdots i_m} {(y_1)}_{i_1}\cdots {(y_m)}_{i_m}\,,
\end{equation*}
where the following is valid:
\begin{equation*}
{(y_k)}_{i_k}=\sum_{j_k=1}^n\mat{C}_{i_kj_k} {(x_k)}_{j_k}\,,
\end{equation*}
from which the transformation law
\begin{equation*}
\mat{A}_{i_1\cdots i_m}=\sum_{j_1=1}^n\cdots\sum_{j_m=1}^n\mat{C}_{j_1i_1}\cdots\mat{C}_{j_mi_m}\mat{B}_{j_1\cdots j_m}
\end{equation*}
can be obtained. Similarly to vector measures, a tensor must have such an abstraction level that its definition results independent of the choice of variables, that is, in expressions above, the tensor represented by $\mat{A}$ is the same tensor represented by $\mat{B}$. Thereby, in rather generic terms, a tensor was initially seen as an abstract measure that arises from the interaction of different sets of variables, arranged in an scalar-valued expression of linear character. 

Ricci's Tensor Calculus was presented by the hungarian mathematician Marcel Grossman(1878-1936) to his old school fellow Albert Einstein(1879-1955). When Einstein was developing his General Theory of Relativity, Grossman used to update him with the most recent mathematical tools available; in this case, non Euclidean geometry. During his work, Einstein noticed that there is not a preferred reference frame, inertial or not, that describes a certain physical phenomenon more adequately, that is, changes of frame do not promote qualitative modification on the description of phenomena. Since a tensor is immune to changes of variables, its use became a rather convenient mathematical tool for enabling the theory, in a concise and elegant development. About this tool and its mathematical foundations, \aut{Einstein}\cite{einstein_1915_1} wrote the following compliment: \emph{The charm of this theory will hardly scape from anyone who really understands it; it means the real triumph of the methodology of Absolute Differential Calculus, established by GAUSS, RIEMANN, CHRISTOFFEL, RICCI and LEVI-CIVITA}\footnote{Translated by me from the original \emph{Dem Zauber dieser Theorie wird sich kaum jemand entziehen k�nnen, der sie wirklich erfa�t hat; sie bedeutet einen wahren Triumph der durch GAUSS, RIEMANN, CHRISTOFFEL, RICCI und LEVI-CIVITER begr�ndeten Methode des allgemeinen Differentialkalkuls}.}. From then on, Tensor Analysis raised to prominence: its development intensified and its application spread out to other branches of Mathematical Physics, of which Continuum Mechanics is the most relevant example.  


\section{Tensor Spaces}


Let's recall vector spaces of functions that have a tuple of vectors as argument, that is, functions whose domain is a cartesian product of many sets, each one defining a vector space. Among these function spaces, we have a particular interest in those constituted by multilinear functions, on the conditions presented in section \ref{sec:FuncLin}. In order to proceed with the theory, based on what was already studied, we chose to adopt an approach that considers tensor spaces as vector spaces of multilinear functionals, called tensors. In more mathematical terms, given a cartesian product $U^{\times m}$ where each set $U_i$ defines a vector space $(U_i)_\mathcal{R}$, a function space $\evl{\mathcal{R}}{U^{\times m}}{\mathcal{R}}$, constituted by multilinear functionals, is said to be a \textsb{tensor space}\index{space!tensor} of \textsb{order}\index{order!of tensor space} $m$, whose representation is here shortened by \gloref{espacoTens}, and each functional $\vtf{t}\in\ete{\mathcal{R}}{U^{\times m}}$ is said to be a \textsb{tensor}\index{tensor} of \textsb{order}\index{order!tensor} $m$, represented by \gloref{tensor}. In the particular case of $m=1$, a tensor space is considered to be a dual space, that is, $\ete{\mathcal{R}}{U}=U^*_\mathcal{R}$, where a tensor is a linear functional. Moreover, it is possible to build particular tensor spaces from catersian powers together with dual spaces: a tensor space $\ete{\mathcal{R}}{V^{p}\times {V^*}^{q}}$, represented by \gloref{espacoTensTipo}, where $V^*_\mathcal{R}$ is the dual space of $V_\mathcal{R}$, is said to have a \textsb{contravariant order}\index{order!contravariant} $p$ and a \textsb{covariant order}\index{order!covariant} $q$ or we can call it a type $(p,q)$ tensor space.   

A tensor can be defined from the dual spaces of its constituent vector spaces, that is, a linear functional in each vector space $(U_i^*)_\mathcal{R}$ defines a multilinear functional (tensor) in $\ete{\mathcal{R}}{U^{\times m}}$. Let's see how this happens. The tuple $(\vtf{g}_1,\cdots,\vtf{g}_m)$, which is an element of cartesian product $(U^*)^{\times m}=U^*_1\times\cdots\times U^*_m$, specifies a tensor in $\ete{\mathcal{R}}{U^{\times m}}$ when the following rule is defined: 
\begin{equation}
\fua{\tnr{G}}{\vto{x}_1,\cdots,\vto{x}_m}=\prod_{i=1}^m\fua{\vtf{g}_i}{\vto{x}_i}\,,
\end{equation}
where tensor $\tnr{G}\in\ete{\mathcal{R}}{U^{\times m}}$ is classified dyadic\index{tensor!dyadic} if $m=2$, \textsb{triadic}\index{tensor!triadic} if $m=3$ and \textsb{polyadic}\index{tensor!polyadic} if $m>3$. Under these conditions, each linear functional $\vtf{g}_i$ is called \textsb{dyad}\index{dyad}, \textsb{triad}\index{triad} and \textsb{polyad}\index{polyad} respectively. In order to handle and make its polyads explict, the polyadic tensor that $(\vtf{g}_1,\cdots,\vtf{g}_m)$ defines we represent it by $\vtf{g}_1\otimes\cdots\otimes\vtf{g}_m$, where symbol $\otimes$ is a function of two arguments, definer of a mapping called tensor product. In more formal and generic terms, considering tensor spaces $\ete{\mathcal{R}}{V^{\times p}}$ and $\ete{\mathcal{R}}{W^{\times q}}$, a mapping $\map{\otimes}{\ete{\mathcal{R}}{V^{\times p}}\times\ete{\mathcal{R}}{W^{\times q}}}{\ete{\mathcal{R}}{V^{\times p}\times W^{\times q}}}$ is called a \textsb{tensor product}\index{product!tensor} if, for all $\tnr{V}\in\ete{\mathcal{R}}{V^{\times p}}$ and $\tnr{W}\in\ete{\mathcal{R}}{W^{\times q}}$,
\begin{equation}\label{eq:prodTens}
\fua{\tnr{V}\otimes\tnr{W}}{\vto{x}_1,\cdots,\vto{x}_p,\vto{y}_1,\cdots,\vto{y}_q}=\fua{\tnr{V}}{\vto{x}_1,\cdots,\vto{x}_p}\fua{\tnr{W}}{\vto{y}_1,\cdots,\vto{y}_q}\,,
\end{equation}
where tensor $\tnr{V}\otimes\tnr{W}:=\fua{\otimes}{\tnr{V},\tnr{W}}$ belongs to $\ete{\mathcal{R}}{V^{\times p}\times W^{\times q}}$. From the following items, the representation of a polyadic tensor by a succession of tensor products, as is the case of $\vtf{g}_1\otimes\cdots\otimes\vtf{g}_m$ above, results from the associativity property. Items
\begin{itemize}
\setlength\itemsep{.1em}
\item[i.] Zero tensor: $\tnr{U}\otimes\tnr{V}=\tnr{0} \implies\tnr{U}=\tnr{0}$ or $\tnr{V}=\tnr{0}$,
\item[ii.] Multiplication by scalar: $\alpha\tnr{U}\otimes\beta\tnr{V}=\alpha\beta(\tnr{U}\otimes\tnr{V})=\beta\tnr{U}\otimes\alpha\tnr{V}$,
\item[iii.] Associativity:
$\lpa\tnr{U}\otimes\tnr{V}\rpa\otimes\tnr{W}=\tnr{U}\otimes\lpa\tnr{V}\otimes\tnr{W}\rpa$,
\item[iv.] Left distributivity:
$\lpa\tnr{U}+\tnr{V}\rpa\otimes\tnr{W}=\tnr{U}\otimes\tnr{W}+\tnr{V}\otimes\tnr{W}\,\,$ and
	\item[v.] Right distributivity:
	$\tnr{U}\otimes\lpa\tnr{V}+\tnr{W}\rpa=\tnr{U}\otimes\tnr{V}+\tnr{U}\otimes\tnr{W}$
\end{itemize}
are valid properties for all $\alpha,\beta\in\mathcal{R}$ and  $\tnr{U},\tnr{V},\tnr{W}\in\ete{\mathcal{R}}{U^{\times m}}$. Moreover, the following theorem states that a set of specific polyadic tensors, defined by tuples of coordinate functionals, is able to span a tensor space. 

\begin{mteo}{Basis of Polyadic Tensors}{geraEte}
Given a tensor space $\ete{\mathcal{R}}{U^{\times m}}$ and a basis $B_i=\{\vto{u}_1^{(i)},\cdots,\vto{u}_{n_i}^{(i)}\}$ of each constituent vector space $(U_i)_\mathcal{R}$, $i=1,\cdots,m\,$, the set $B$ of all polyadic tensors $\vtf{f}_{j_1}^{B_1}\otimes\cdots\otimes\vtf{f}_{j_m}^{B_m}$, $j_i=1,\cdots,n_i\,$, is a basis of $\ete{\mathcal{R}}{U^{\times m}}$. 
\end{mteo}

{\footnotesize
\begin{proof}
Firstly, concerning the above properties of tensor products, each one can be verified from the definition \eqref{eq:prodTens} and from trivial multiplicative properties with scalars. About the theorem, we need to prove previously that if equalities $\tnr{T}_1({\vto{u}_{j_1}^{(1)},\cdots,\vto{u}_{j_m}^{(m)}})={\tnr{T}_2}(\vto{u}_{j_1}^{(1)},\cdots,\vto{u}_{j_m}^{(m)})$, where $\tnr{T}_1,\tnr{T}_2\in\ete{\mathcal{R}}{U^{\times m}}$, are valid, it results that $\tnr{T}_1=\tnr{T}_2$. In order to do that, we can say that 
\begin{align*}
\tnr{T}_1(\vto{x}_1,\cdots,\vto{x}_m)&=\sum_{{j_1}=1}^{n_1}\cdots\sum_{{j_m}=1}^{n_m}\fua{\vtf{f}^\con{B_1}_{j_1}}{\vto{x}_1}\cdots\fua{\vtf{f}^\con{B_m}_{j_m}}{\vto{x}_m}\tnr{T}_1({\vto{u}_{j_1}^{(1)},\cdots,\vto{u}_{j_m}^{(m)}})\\
&=\sum_{{j_1}=1}^{n_1}\cdots\sum_{{j_m}=1}^{n_m}\fua{\vtf{f}^\con{B_1}_{j_1}}{\vto{x}_1}\cdots\fua{\vtf{f}^\con{B_m}_{j_m}}{\vto{x}_m}\tnr{T}_2({\vto{u}_{j_1}^{(1)},\cdots,\vto{u}_{j_m}^{(m)}})\\
&=\tnr{T}_2(\vto{x}_1,\cdots,\vto{x}_m)\,,
\end{align*}
from which we conclude $\tnr{T}_1=\tnr{T}_2$. Now, let $\tnr{T}\in\ete{\mathcal{R}}{U^{\times m}}$ be a tensor where scalars $\beta_{i_1\cdots i_m}=\tnr{T}({\vto{u}_{i_1}^{(1)},\cdots,\vto{u}_{i_m}^{(m)}})$ define a tensor 
\begin{equation*}
\tnr{X}=\sum_{{i_1}=1}^{n_1}\cdots\sum_{{i_m}=1}^{n_m}\beta_{i_1\cdots i_m}\vtf{f}_{i_1}^{B_1}\otimes\cdots\otimes\vtf{f}_{i_m}^{B_m}\,.
\end{equation*}
Thus we can write the following:
\begin{equation*}
\tnr{X}({\vto{u}_{j_1}^{(1)},\cdots,\vto{u}_{j_m}^{(m)}}) = \sum_{{i_1}=1}^{n_1}\cdots\sum_{{i_m}=1}^{n_m}\beta_{i_1\cdots i_m}\prod_{k=1}^m\delta_{i_kj_k}=\beta_{j_1\cdots j_m}=\tnr{T}({\vto{u}_{j_1}^{(1)},\cdots,\vto{u}_{j_m}^{(m)}})\,.
\end{equation*}
Such equalities enable us to conclude that $\tnr{X}=\tnr{T}$ and then $\spn{(B)}=\ete{\mathcal{R}}{U^{\times m}}$, since $\tnr{T}$ is an arbitrary tensor of $\ete{\mathcal{R}}{U^{\times m}}$. Now we need to verify that if $B$ is linearly independent. This is true when coefficients $\alpha_{i_1\cdots i_m}$ of the linear combination 
\begin{equation*}
\sum_{{i_1}=1}^{n_1}\cdots\sum_{{i_m}=1}^{n_m}\alpha_{i_1\cdots i_m}\vtf{f}_{i_1}^{B_1}\otimes\cdots\otimes\vtf{f}_{i_m}^{B_m}=\tnr{0}
\end{equation*}
are zero, which can be verified through the following development:  
\begin{align*}
\sum_{{i_1}=1}^{n_1}\cdots\sum_{{i_m}=1}^{n_m}\alpha_{i_1\cdots i_m}\vtf{f}_{i_1}^{B_1}\otimes\cdots\otimes\vtf{f}_{i_m}^{B_m}({\vto{u}_{j_1}^{(1)},\cdots,\vto{u}_{j_m}^{(m)}})&=\tnr{0}({\vto{u}_{j_1}^{(1)},\cdots,\vto{u}_{j_m}^{(m)}})\\
\sum_{{i_1}=1}^{n_1}\cdots\sum_{{i_m}=1}^{n_m}\alpha_{i_1\cdots i_m}\prod_{k=1}^m\delta_{i_kj_k}&=0\\
\alpha_{j_1\cdots j_m}&=0
\end{align*}
\end{proof}
}

In the context of the above theorem, it can be noted that the number of polyadic tensors $\vtf{f}_{j_1}^{B_1}\otimes\cdots\otimes\vtf{f}_{j_m}^{B_m}$ of basis $B$ is given by 
$n_1\cdots n_m$. Moreover, from the previous chapter, we already know that a vector space and its dual have the same dimension, that is, $\dim ((U_i)_\mathcal{R})=\dim ((U_i^*)_\mathcal{R})$. Thereby, we can state that the dimension of tensor space $\ete{\mathcal{R}}{U^{\times m}}$ is the product of dimensions of each of vector spaces $(U_i)_\mathcal{R}$. In other words,
\begin{equation}
\dim (\ete{\mathcal{R}}{U^{\times m}}) = \prod_{i=1}^m \dim ((U_i)_\mathcal{R})\,.
\end{equation}
Like any other basis of a vector space, each element of basis $B$ of tensor space $\ete{\mathcal{R}}{U^{\times m}}$ also have a related coordinate functional, here represented by $\vtf{f}_{j_1\cdots j_m}^{B}$. Thereby, given an arbitrary tensor $\tnr{T}\in\ete{\mathcal{R}}{U^{\times m}}$, the tuple constituted by scalars $\vtf{f}_{j_1\cdots j_m}^{B}(\tnr{T})$, $j_i=1,\cdots,n_i\,$ is said to be the coordinates of $\tnr{T}$ on $B$. Using this definition, tensor $\tnr{T}$ can be described by expression 
\begin{equation}
\tnr{T} = \sum_{k_1=1}^{n_1}\cdots\sum_{k_m=1}^{n_m} \vtf{f}_{k_1\cdots k_m}^{B}(\tnr{T})\,\vtf{f}_{k_1}^{B_1}\otimes\cdots\otimes\vtf{f}_{k_m}^{B_m}\,,
\end{equation}
from which the equalities   
\begin{equation*}
\fua{\tnr{T}}{\vto{u}_{j_1}^{(1)},\cdots,\vto{u}_{j_m}^{(m)}} = \sum_{k_1=1}^{n_1}\cdots\sum_{k_m=1}^{n_m}\vtf{f}_{k_1\cdots k_m}^{B}(\tnr{T})\,\prod_{i=1}^m \fua{\vtf{f}_{k_i}^{B_i}}{\vto{u}_{j_i}^{(i)}}=\vtf{f}_{j_1\cdots j_m}^{B}(\tnr{T})
\end{equation*}
enable to define the following rule for coordinate functionals on basis $B$: 
\begin{equation}\label{eq:regraCoordTens}
\vtf{f}_{j_1\cdots j_m}^{B}(\tnr{X})= \fua{\tnr{X}}{\vto{u}_{j_1}^{(1)},\cdots,\vto{u}_{j_m}^{(m)}}\,.
\end{equation}
In type $(p,q)$ tensor spaces, the coordinates of a tensor $\tnr{V}\in\ete{\mathcal{R}}{V^{(p,q)}}$ are defined by scalars usually represented by 
\begin{equation}
(\vtf{f}^{Z})^{j_1\cdots j_p}_{k_1\cdots k_q}(\tnr{V}),\,\, j_i,k_l=1,\cdots,n\,,
\end{equation}
where $Z$ is a basis of $\ete{\mathcal{R}}{V^{(p,q)}}$, $i=1,\cdots,p$ and $l=1,\cdots,q\,\,$. The superscript indexes $j_i$ are related to space $V_\mathcal{R}$ of vectors whose coordinates are contravariant and the subscript indexes $k_l$ related to ${V}_\mathcal{R}^*$ of vectors whose coordinates are covariant. Tensors are called  \textsb{covariant}\index{tensor!covariant} or \textsb{contravariant}\index{tensor!contravariant} when they exclusively have covariant or contravariant indexes respectively and called \textsb{mixed}\index{tensor!mixed} when they have both. For $(p,q)$ tensor spaces, given a basis $W=\{\vto{w}_{1},\cdots,\vto{w}_{n}\}$ of $V_\mathcal{R}$, rule \eqref{eq:regraCoordTens} for coordinate functionals results   
\begin{equation}
(\vtf{f}^{Z})^{j_1\cdots j_p}_{k_1\cdots k_q}(\tnr{X})=\fua{\tnr{X}}{\vto{w}_{j_1},\cdots,\vto{w}_{j_p},\vtf{f}_{k_1}^{W},\cdots,\vtf{f}_{k_q}^{W}}\,,
\end{equation}
where bases  $W^*=\{\vtf{f}_{1}^{W},\cdots,\vtf{f}_{n}^{W}\}$ of $V_\mathcal{R}^*$ and $W^{**}=\{\vtf{f}_{1}^{W^*},\cdots,\vtf{f}_{n}^{W^*}\}$ of $V_\mathcal{R}^{**}$ define basis $Z$, in such a way that its elements have the format
\begin{equation*}
\vtf{f}_{j_1}^{W}\otimes\cdots\otimes\vtf{f}_{j_p}^{W}\otimes\vtf{f}_{k_1}^{W^*}\otimes\cdots\otimes\vtf{f}_{k_q}^{W^*}\,,
\end{equation*} 
according to theorem \ref{teo:geraEte}.

Since we just learned the rule to obtain the coordinates of a tensor on a certain basis, let's see how these coordinates behave when there is a change of basis. In the context of theorem \ref{teo:geraEte}, let $C_i=\{\vto{v}_1^{(i)},\cdots,\vto{v}_{n_i}^{(i)}\}$ be a basis of each vector space $(U_i)_\mathcal{R}$ and $\maf{\vtf{i}}{C_i}{B_i}$ a matrix that changes basis $B_i$ to basis $C_i$. From the rule \eqref{eq:regraCoordTens}, given a tensor 
$\tnr{T}\in\ete{\mathcal{R}}{U^{\times m}}$, we can develop the following:
\begin{align}
\fua{\vtf{f}_{j_1\cdots j_m}^{C}}{\tnr{T}}&=\fua{\tnr{T}}{\vto{v}_{j_1}^{(1)},\cdots,\vto{v}_{j_m}^{(m)}}  \nonumber\\
&=\sum_{k_1=1}^{n_1}\cdots\sum_{k_m=1}^{n_m} \vtf{f}_{k_1\cdots k_m}^{B}(\tnr{T})\,\fua{\vtf{f}_{k_1}^{B_1}\otimes\cdots\otimes\vtf{f}_{k_m}^{B_m}}{\vto{v}_{j_1}^{(1)},\cdots,\vto{v}_{j_m}^{(m)}}\nonumber\\
&=\sum_{k_1=1}^{n_1}\cdots\sum_{k_m=1}^{n_m} \vtf{f}_{k_1\cdots k_m}^{B}(\tnr{T})\,\prod_{i=1}^m\fua{\vtf{f}_{k_i}^{B_i}}{\vto{v}_{j_i}^{(i)}}\nonumber\\
&=\sum_{k_1=1}^{n_1}\cdots\sum_{k_m=1}^{n_m} \vtf{f}_{k_1\cdots k_m}^{B}(\tnr{T})\,\prod_{i=1}^m\maf{\vtf{i}}{C_i}{B_i}_{k_ij_i}\,,
\end{align}
from which the last equality describes a change of coordinates of tensor $\tnr{T}$ from basis $B$ to basis $C$. Through this same process, it is possible to change the coordinates of a type $(p,q)$ tensor.  


Considering a tensor space $\ete{\mathcal{R}}{U^{\times m}}$, if its definer vector spaces $(U_i)_\mathcal{R}$ are Hilbert spaces, we can lay hands on the biunivocal relationship between elements $\vtf{f}_{j_i}^{B_i}$ of each dual basis $B_i^*$ and elements $(\vto{u}^{(i)})^{j_i}$ of each reciprocal basis $B^\perp_{i}$, as a consequence of Riesz-Fr�chet Representation. Thereby, the polyadic tensors of basis $B$ are represented by 
\begin{equation}\label{eq:poliadicoHilbert}
\vto{u}^{(1)}_{j_1}\otimes\cdots\otimes\vto{u}^{(m)}_{j_m}:=((\vto{u}^{(1)})^{j_1})^*\otimes\cdots\otimes((\vto{u}^{(m)})^{j_m})^*
\end{equation}
in order to simplify notation. If the constituent bases $B_i$ are orthonormal, then $B_i=B_i^\perp$ and  $((\vto{u}^{(i)})^{j_i})^*=(\vto{u}^{(i)}_{j_i})^*$ in the above definition. 

Still considering Hilbert spaces, if a tensor space has a unitary order, its tensors are covectors, that is, given an arbitrary vector $\vto{u}\in U_\mathcal{R}$, its covector $\vto{u}^*$ is a tensor of order one (linear functional), or $\vto{u}^*\in \ete{\mathcal{R}}{U}$. In other words, \emph{concerning tensor spaces defined by Hilbert spaces, covectors are tensors of unitary order}. Moreover, given a tensor space $\ete{\real}{E^m}$, where $E_\real$ is a three-dimensional Euclidean space, let $\hat{B}$ be a basis whose $3^m$ elements have the format $\vun{e}_{j_1}\otimes\cdots\otimes\vun{e}_{j_m}$, $j_i=1,2,3$, according to definition \eqref{eq:poliadicoHilbert}. In these conditions, we call \textsb{cartesian tensor}\index{tensor!cartesian} any linear combination of the elements of basis $\hat{B}$. 

The tensors of space $\ete{\mathcal{R}}{V^{m}}$, $m>1$, can have the positions of their arguments interchanged, that is, given a tensor of $\ete{\mathcal{R}}{V^{2}}$, for example, arguments in $(\vto{x},\vto{y})$ can be changed to $(\vto{y},\vto{x})$, where $\vto{x},\vto{y}\in V_\mathcal{R}$. In this context, we say that a tensor $\tnr{T}_1$ is a permutation of tensor $\tnr{T}_2$ if $\fua{\tnr{T}_1}{\vto{x},\vto{y}}=\fua{\tnr{T}_2}{\vto{y},\vto{x}}$. In general terms, given a bijective mapping $\map{\pi}{\{ 1,\cdots,n \}}{\{1,\cdots,n \}}$, where the sets involved are constituted by ordinals, and a function $\pi$ called permutation of order $m$, in definition
\begin{equation}\label{eq:permutacaoTensor}
\fua{\tnr{T}_\pi}{\vto{v}_1,\cdots,\vto{v}_m}:=\tnr{T}\,(\vto{v}_{\fua{\pi}{1}},\cdots,\vto{v}_{\fua{\pi}{m}})\,,\forall\,
\vto{v}_i\in {V}_\mathcal{R}\,,
\end{equation}
the tensor $\tnr{T}_\pi\in\ete{\mathcal{R}}{V^{m}}$ is called the $\pi$ \textsb{permutation}\index{tensor!permutation of} of $\tnr{T}\in\tnr{V}^{m}_\mathcal{R}$. There are two important properties: a) the permutation of sum is the sum of permutations, that is, if $\tnr{T}=\tnr{A}+\tnr{B}$ then $(\tnr{A}+\tnr{B})_\pi=\tnr{A}_\pi+\tnr{B}_\pi$; b) tensor $(\alpha\tnr{T})_\pi=\alpha\tnr{T}_\pi$, $\forall\,\alpha\in\mathcal{R}$. Now, considering the specific rule $\fua{\pi}{x}=a\delta_{xb}+b\delta_{xa}+x(1-\delta_{xa}-\delta_{xb})$ that interchanges indexes $a$ and $b$, we call $\tnr{T}_\pi$ the $(a,b)$ \textsb{transposition}\index{tensor!transposition of} of $\tnr{T}$, represented by $\tnr{T}_{(a,b)}$.  



{\footnotesize
\begin{proof}
Let's verify the two properties above. As any other vector, a tensor can always be the result of a sum of tensors; thereby, from definition \eqref{eq:permutacaoTensor} and considering $\tnr{T}=\tnr{A}+\tnr{B}$, we have the following development:
\begin{align*}
\fua{(\tnr{A}+\tnr{B})_\pi}{\vto{u}_1,\cdots,\vto{u}_m}&=(\tnr{A}+\tnr{B})\,(\vto{u}_{\fua{\pi}{1}},\cdots,\vto{u}_{\fua{\pi}{m}})\nonumber\\
&=\tnr{A}\,(\vto{u}_{\fua{\pi}{1}},\cdots,\vto{u}_{\fua{\pi}{m}})+\tnr{B}\,(\vto{u}_{\fua{\pi}{1}},\cdots,\vto{u}_{\fua{\pi}{m}})\nonumber\\
&=\fua{\tnr{A}_\pi}{\vto{u}_1,\cdots,\vto{u}_m}+\fua{\tnr{B}_\pi}{\vto{u}_1,\cdots,\vto{u}_m}\nonumber\\
&=\fua{(\tnr{A}_\pi+\tnr{B}_\pi)}{\vto{u}_1,\cdots,\vto{u}_m}\,,
\end{align*}
from which we conclude that the permutation of sum is the sum of permutations. The proof of the second property follows this same process.
\end{proof}
}

The concept of permutation enables the concept of symmetry as follows: a tensor $\tnr{S}\in\ete{\mathcal{R}}{V^{m}}$ is said to be \textsb{symmetric}\index{tensor!symmetric} if equality $\tnr{S}_\pi=\tnr{S}$ is valid for any permutation $\pi$. Conversely, a tensor $\tnr{P}\in\tnr{V}^{m}_\mathcal{R}$ is called \textsb{antisymmetric}\index{tensor!antisymmetric} or \textsb{alternating}\index{tensor!alternating} if tensor $\tnr{P}_\pi=\epsilon_{\fua{\pi}{1}\cdots \fua{\pi}{m}}\,\tnr{P}$ for any $\pi$. Since $\tnr{P}$ is antisymmetric, we can conclude that scalar $\fua{\tnr{P}}{\vto{x}_1,\cdots,\vto{x}_a,\cdots,\vto{x}_b,\cdots,\vto{x}_m}=0$ if $\vto{x}_a=\vto{x}_b$, where $\vto{x}_i\in V_\mathcal{R}$ and $a,b\in\{1,\cdots,m\}$. Thereby, the consequence is  
\begin{equation}\label{eq:igAntis}
\fua{\tnr{P}}{\vto{x}_{i_1},\cdots,\vto{x}_{i_m}}=\epsilon_{i_1\cdots i_m}\fua{\tnr{P}}{\vto{x}_1,\cdots,\vto{x}_m}\,.
\end{equation}
Moreover, since the sum of two (anti)symmetric tensors is an (anti)symmetric tensor and multiplication by scalar also preserves (anti)symmetry, the set of all (anti)symmetric tensors defines a tensor subspace of $\ete{\mathcal{R}}{V^{m}}$. Then, it is possible, and intriguing, to obtain that \emph{every tensor subspace defined by antisymmetric tensors is one-dimensional.}

{\footnotesize
\begin{proof}
First, let's prove \eqref{eq:igAntis}. If $\vto{x}_a=\vto{x}_b$, then 
\begin{align*}
\fua{\tnr{P}_\pi}{\vto{x}_1,\cdots,\vto{x}_a,\cdots,\vto{x}_b,\cdots,\vto{x}_m}&=\fua{\tnr{P}_\pi}{\vto{x}_1,\cdots,\vto{x}_b,\cdots,\vto{x}_a,\cdots,\vto{x}_m}\\
\epsilon_{i_1\cdots i_a\cdots i_b\cdots i_m}\fua{\tnr{P}}{\vto{x}_1,\cdots,\vto{x}_a,\cdots,\vto{x}_b,\cdots,\vto{x}_m}&=\epsilon_{i_1\cdots i_b\cdots i_a\cdots i_m}\fua{\tnr{P}}{\vto{x}_1,\cdots,\vto{x}_b,\cdots,\vto{x}_a,\cdots,\vto{x}_m}\\
\fua{\tnr{P}}{\vto{x}_1,\cdots,\vto{x}_a,\cdots,\vto{x}_b,\cdots,\vto{x}_m}&=-\fua{\tnr{P}}{\vto{x}_1,\cdots,\vto{x}_b,\cdots,\vto{x}_a,\cdots,\vto{x}_m}\,,
\end{align*}
from which we conclude that $\fua{\tnr{P}}{\vto{x}_1,\cdots,\vto{x}_a,\cdots,\vto{x}_b,\cdots,\vto{x}_m}=0$. From this result, if there is some index ${i_a}={i_b}$ in \eqref{eq:igAntis}, we obtain identity $0=0$. If all indexes are different, we can define the rule $\fua{\pi}{x}=i_x$, when equality results in the definition of antisymmetric tensor, that is, scalar
\begin{align*}
\tnr{P}({\vto{x}_{\fua{\pi}{1}},\cdots,\vto{x}_{\fua{\pi}{m}}})&=\epsilon_{\fua{\pi}{1}\cdots \fua{\pi}{m}}\fua{\tnr{P}}{\vto{x}_1,\cdots,\vto{x}_m}\\
\fua{\tnr{P}_\pi}{\vto{x}_1,\cdots,\vto{x}_m}&=\epsilon_{\fua{\pi}{1}\cdots \fua{\pi}{m}}\fua{\tnr{P}}{\vto{x}_1,\cdots,\vto{x}_m}\\
\tnr{P}_\pi&=\epsilon_{\fua{\pi}{1}\cdots \fua{\pi}{m}}\tnr{P}\,.
\end{align*}
Let's verify if (anti)symmetric tensors really define tensor subspaces. Let $\tnr{A}$ and $\tnr{B}$ be symmetric tensors of $\ete{\mathcal{R}}{V^{m}}$, where $\tnr{A}=\tnr{A}_\pi$ and $\tnr{B}=\tnr{B}_\pi$. Adding these equalities, we obtain $\tnr{A}+\tnr{B}=\tnr{A}_\pi+\tnr{B}_\pi=(\tnr{A}+\tnr{B})_\pi$, which shows that $\tnr{A}+\tnr{B}$ sim�trico. This same process can be used for multiplication by scalar. Thereby, it is thus trivial to verify that a set of symmetric tensors observes the five definer axioms of vector spaces, presented in section \ref{sec:espacoVet}. In the case of antisymmetric tensors $\tnr{C}$ and $\tnr{D}$, we have
\begin{equation*}
	\epsilon_{\fua{\pi}{1}\cdots \fua{\pi}{m}}(\tnr{C}+\tnr{D})=\tnr{C}_\pi+\tnr{D}_\pi=(\tnr{C}+\tnr{D})_\pi\,,
\end{equation*}
which shows $\tnr{C}+\tnr{D}$ antisymmetric. Thereby, we can state that a set of antisymmetric tensors also defines a tensor space. Now, we'll verify the curious one-dimensionality of antisymmetric tensors by adapting to our case the idea of \aut{Backus}\cite{backus_1997_1}, pp. 15-17. Let $\etn{\mathcal{R}}{V^m}\subset\ete{\mathcal{R}}{V^m}$ be a tensor subspace of antisymmetric tensors, $B=\{\vto{v_1},\cdots,\vto{v}_n\}$ a basis of vector space $V_\mathcal{R}$ and an antisymmetric tensor  $\tnr{H}\in\etn{\mathcal{R}}{V^m}$ such that 
\begin{equation*}
\tnr{H}(\vto{x_1},\cdots,\vto{x}_m)=\sum_{i_1=1}^n\cdots\sum_{i_m=1}^n\fua{\vtf{f}^\con{B}_{i_1}}{\vto{x}_1}\cdots\fua{\vtf{f}^\con{B}_{i_m}}{\vto{x}_m}\epsilon_{i_1\cdots i_m}\,,
\end{equation*}
where $\vto{x_i}\in V_\mathcal{R}$. This tensor $\tnr{H}$ is verified to be antisymmetric through the following development: 
\begin{align*}
\fua{\tnr{H}_{(a,b)}}{\cdots,\vto{x}_a,\cdots,\vto{x}_b,\cdots}&=\cdots
\sum_{i_a=1}^n\cdots\sum_{i_b=1}^n\cdots\,\fua{\vtf{f}^\con{B}_{i_a}}{\vto{x}_b}\cdots\,\fua{\vtf{f}^\con{B}_{i_b}}{\vto{x}_a}\cdots \,\epsilon_{\cdots i_a\cdots i_b\cdots}\\
&=\cdots
\sum_{i_b=1}^n\cdots\sum_{i_a=1}^n\cdots\,\fua{\vtf{f}^\con{B}_{i_b}}{\vto{x}_b}\cdots\,\fua{\vtf{f}^\con{B}_{i_a}}{\vto{x}_a}\cdots \,\epsilon_{\cdots i_b\cdots i_a\cdots}\\
&=-\,\,\cdots
\sum_{i_b=1}^n\cdots\sum_{i_a=1}^n\cdots\,\fua{\vtf{f}^\con{B}_{i_b}}{\vto{x}_b}\cdots\,\fua{\vtf{f}^\con{B}_{i_a}}{\vto{x}_a}\cdots \,\epsilon_{\cdots i_a\cdots i_b\cdots}\\
&=\epsilon_{1\cdots b\cdots a\cdots m}\,\fua{\tnr{H}}{\cdots,\vto{x}_a,\cdots,\vto{x}_b,\cdots}\,
\end{align*}
where terms with indexes $i_1$ and $i_m$ are omitted and $a,b$ are arbitrary ordinals between $1$ and $m$. Considering an arbitrary tensor $\tnr{P}\in\etn{\mathcal{R}}{V^m}$, we know that
\begin{align*}
\fua{\tnr{P}}{\vto{x}_1,\cdots,\vto{x}_m}&=\sum_{{i_1}=1}^{n}\cdots\sum_{{i_m}=1}^{n}\fua{\vtf{f}^\con{B}_{i_1}}{\vto{x}_1}\cdots\fua{\vtf{f}^\con{B}_{i_m}}{\vto{x}_m}\tnr{P}({\vto{v}_{i_1},\cdots,\vto{v}_{i_m}})\\	&=\underbrace{\sum_{{i_1}=1}^{n}\cdots\sum_{{i_m}=1}^{n}\fua{\vtf{f}^\con{B}_{i_1}}{\vto{x}_1}\cdots\fua{\vtf{f}^\con{B}_{i_m}}{\vto{x}_m}\epsilon_{i_1\cdots i_m}}_{\tnr{H}(\vto{x_1},\cdots,\vto{x}_m)}\underbrace{\fua{\tnr{P}}{\vto{v}_1,\cdots,\vto{v}_m}}_{\alpha}
\end{align*}
from which we conclude $\tnr{P}=\alpha\tnr{H}$, that is, $\{\tnr{H}\}$ is a basis of $\etn{\mathcal{R}}{V^m}$.
\end{proof}
}

Al�m dos tensores sim�tricos e antissim�tricos, h� um outro tensor not�vel que vai subsidiar conceitos muito relevantes para o nosso estudo, como os de isotropia e anti isotropia: refiro-me ao tensor identidade, definido a partir de espa�os tensoriais de ordem par. Dizemos que \gloref{tensorI}$\,\in\ete{\mathcal{R}}{V^{\times p}\times V^{\times p}}$ � um \textsb{tensor identidade}\index{tensor!identidade} de ordem $2p$ se, dado    $B_i=\{\vto{v}_1^{(i)},\cdots,\vto{v}_{n_i}^{(i)}\}$ uma base qualquer de $(V_\mathcal{R})_i$, os escalares
\begin{equation}
\fua{\tnr{I}}{\vto{v}_{i_1}^{(1)},\cdots,\vto{v}_{i_p}^{(p)},\vto{v}_{j_1}^{(1)},\cdots,\vto{v}_{j_p}^{(p)}}=\prod_{k=1}^p \delta_{i_kj_k}\,.
\end{equation}
Dessa defini��o, considerando $\vto{x}_i$ e $\vto{y}_i$ vetores quaisquer de $(V_\mathcal{R})_i$, podemos desenvolver o escalar $\fua{\tnr{I}}{\vto{x}_1,\cdots,\vto{x}_p,\vto{y}_1,\cdots,\vto{y}_p}$ na forma
\begin{equation*}
\sum_{{i_1,j_1}=1}^{n_1}\cdots\sum_{{i_p,j_p}=1}^{n_p}\fua{\vtf{f}^\con{B_1}_{i_1}}{\vto{x}_1}\fua{\vtf{f}^\con{B_1}_{j_1}}{\vto{y}_1}\cdots\fua{\vtf{f}^\con{B_p}_{i_p}}{\vto{x}_p}\fua{\vtf{f}^\con{B_p}_{j_p}}{\vto{y}_p}\prod_{k=1}^p \delta_{i_kj_k}\,,
\end{equation*}
de onde conclu�mos que 
\begin{equation}\label{eq:identidade}
\fua{\tnr{I}}{\vto{x}_1,\cdots,\vto{x}_p,\vto{y}_1,\cdots,\vto{y}_p}=\sum_{{i_1}=1}^{n_1}\cdots\sum_{{i_p}=1}^{n_p}\,\prod_{k=1}^p\fua{\vtf{f}^\con{B_k}_{i_k}}{\vto{x}_k}\fua{\vtf{f}^\con{B_k}_{i_k}}{\vto{y}_k}\,.
\end{equation}
Desse resultado, pode-se considerar 
\begin{equation}
\tnr{I}=\sum_{{i_1}=1}^{n_1}\cdots\sum_{{i_p}=1}^{n_p}\vtf{f}_{i_1}^{B_1}\otimes\cdots\otimes\vtf{f}_{i_p}^{B_p}\otimes\vtf{f}_{i_1}^{B_1}\otimes\cdots\otimes\vtf{f}_{i_p}^{B_p}
\end{equation}
como um exemplo de tensor identidade. Al�m disso, no �mbito de espa�os tensoriais $\ete{\mathcal{R}}{V^2}$, onde $p=1$, definidos a partir de espa�os Euclidianos, j� sabemos que, se a base $B_1$ for ortonormal, a igualdade $\sum_{{i_1}=1}^{n_1}\fua{\vtf{f}^\con{B_1}_{i_1}}{\vto{x}_1}\fua{\vtf{f}^\con{B_1}_{i_1}}{\vto{y}_1}=\vto{x}_1\cdot\vto{y}_1$ � v�lida, ou seja, a express�o \eqref{eq:identidade} resulta $\fua{\tnr{I}}{\vto{x}_1,\vto{y}_1}=\vto{x}_1\cdot\vto{y}_1$ e 
\begin{equation}
\tnr{I}=\sum_{{i_1}=1}^{n_1}\vun{v}_{i_1}\otimes\vun{v}_{i_1}=\sum_{{i_1}=1}^{n_1}\sum_{{j_1}=1}^{n_1}\delta_{i_1j_1}\vun{v}_{i_1}\otimes\vun{v}_{j_1}
\end{equation}
� um exemplo de tensor identidade de segunda ordem. Da �ltima express�o dessas igualdades, tem-se que as coordenadas de $\tnr{I}$ podem ser arranjadas numa matriz identidade.

\section{Fun��es Tensoriais}


\begin{comment}
Existe um grande n�mero de fen�menos f�sicos que podem ser
descritos por quantidades que se relacionam matematicamente de
maneira linear. Tal relacionamento pode ser representado por uma
fun��o multilinear cujo dom�nio inclui as quantidades f�sicas
envolvidas. Para alguns problemas estudados, � conveniente que a
imagem desta fun��o seja formada por escalares. Eles quantificam,
de uma forma simples, a linearidade entre os elementos do dom�nio.

A modelagem da caracter�stica linear entre vari�veis atrav�s de
funcionais multilineares � o principal objetivo do estudo da
teoria de tensores. Atualmente, existem duas abordagens para a
apresenta��o desta teoria, descritas a seguir.
\begin{itemize}
  \item[a)] Abordagem baseada em coordenadas: considera um tensor como um array
multidimensional. As coordenadas do tensor s�o os escalares deste
array, obtidos a partir da defini��o de uma base. Vetores s�o
tratados como arrays unidimensionais.
  \item[b)] Abordagem intr�nseca: um tensor � visto simplesmente como uma fun��o multilinear,
cuja defini��o independe do estabelecimento de uma base. As
coordenadas do tensor s�o os valores definidos por seus funcionais
coordenados. Neste caso, vetores s�o tensores que possuem apenas
um argumento (fun��o linear).
\end{itemize}

Ser� utilizada, ao longo do texto, a abordagem intr�nseca para o
estudo de tensores. Consideramos que tal abordagem facilita o
entendimento por serem mais mais simples seus conceitos e mais
agrad�vel sua manipula��o.

\section{Vetores e Funcionais Lineares}

\subsection{Espa�o Dual}\index{espa�o!dual}
Seja um espa�o de Hilbert $\ehr{V}{F}$. O espa�o vetorial de
funcionais lineares $\evl{V}{F}{F}$ � denominado espa�o dual de
$\ehr{V}{F}$, representado por $\edu{V}{F}$, onde
$V^*:=\cfl{V}{F}$. Na presen�a de um espa�o dual, os vetores dos
conjuntos $\con{V}$ e $\con{V^*}$ s�o chamados
\emph{contravariantes}\index{vetor!contravariante} e
\emph{covariantes}\index{vetor!covariante} respectivamente.

Considerando o espa�o de Hilbert dimensionalmente finito,  seja
$\con{U}=\lch \vto{v}_1,\cdots,\vto{v}_n\rch$ uma base de
$\ehr{V}{F}$. Seja um funcional qualquer $\vtf{g}$ de $V^*$ e um
vetor qualquer $\vto{v}\in\con{V}$. O desenvolvimento a seguir
mostra que o espa�o dual � tamb�m $n$-dimensional:
\begin{eqnarray}\label{eq:baseDual}
\fua{\vtf{g}}{\vto{v}} & = &
\fua{\vtf{g}}{\sum_{i=1}^{n}\fua{\vtf{f}^\con{U}_{i}}{\vto{v}}\vto{v}_i}\nonumber\\
\fua{\vtf{g}}{\vto{v}} & = &
\lco\sum_{i=1}^{n}\fua{\vtf{g}}{\vto{v}_i}\vtf{f}^\con{U}_{i}\rco\lpa\vto{v}\rpa\nonumber\\
\vtf{g} & = &
\sum_{i=1}^{n}\fua{\vtf{g}}{\vto{v}_i}\vtf{f}^\con{U}_{i}\,.
\end{eqnarray}
Observa-se que os funcionais coordenados $\vtf{f}^\con{U}_{i}$
constituem uma base em $\edu{V}{F}$.

\subsection{A Rela��o Vetor e Funcional Linear}
Seja um espa�o de Hilbert $\ehr{V}{F}$ e seu espa�o dual
$\edu{V}{F}$. Existe uma rela��o linear bijetora, descrita no teorema
apresentado a seguir, entre os vetores de $\con{V}$ e os
funcionais de $\con{V^*}$, de tal forma que, dado um vetor
qualquer $\vto{u}\in\con{V}$ e seu funcional linear correspondente
$\vtf{u}\in\con{V^*}$, � poss�vel definir
\begin{equation}\label{eq:VetorFuncional}
\fua{\vto{u}}{\vto{v}}=\fua{\vtf{u}}{\vto{v}}\,,\forall\,\vto{v}\in\con{V}.
\end{equation}

\begin{teo}[Representa��o de Riesz]\index{Representa��o de Riesz!Teorema da}\label{teo:RepresentacaoRiesz}
Seja um espa�o de Hilbert $\ehr{V}{F}$ e seu espa�o dual
$\edu{V}{F}$. Considerando um vetor qualquer $\vto{v}\in\con{V}$,
seja uma transforma��o linear
$\map{\vartheta_{\vto{v}}}{\con{V}}{F}$ onde o funcional
$\vartheta_{\vto{v}}\in\con{V^*}$ � descrito pela regra
\begin{equation}
\fua{\vartheta_{\vto{v}}}{\vto{x}}=\vto{v}\cdot\vto{x}\,.\nonumber
\end{equation}
Desta forma, o mapeamento $\map{\Theta}{\con{V}}{\con{V^*}}$ �
sempre uma transforma��o linear bijetora se
$\fua{\Theta}{\vto{v}}=\vartheta_{\vto{v}}$.
\end{teo}
\begin{prova}
Primeiramente, vamos mostrar que $\Theta$ � uma fun��o linear.
Dados os vetores quaisquer $\vto{v}$, $\vto{u}$, $\vto{x}$ de
$\con{V}$ e os escalares quaisquer $\alpha,\beta\in\con{F}$, a
transforma��o linear � constatada a partir do seguinte
desenvolvimento:
\begin{eqnarray}
\fua{\vartheta_{\alpha\vto{u}+\beta\vto{v}}}{\vto{x}}&=
&\lpa\alpha\vto{u}+\beta\vto{v}\rpa\cdot\vto{x}\nonumber\\
&=&\alpha\lpa\vto{u}\cdot\vto{x}\rpa+\beta\lpa\vto{v}\cdot\vto{x}\rpa\nonumber\\
&=&\alpha\fua{\vartheta_\vto{u}}{\vto{x}}+\beta\fua{\vartheta_\vto{v}}{\vto{x}}\nonumber\\
&=&\fua{\lco\alpha\vartheta_\vto{u}
+\beta\vartheta_\vto{v}\rco}{\vto{x}}\nonumber\\
\vartheta_{\alpha\vto{u}+\beta\vto{v}}&=&\alpha\vartheta_\vto{u}
+\beta\vartheta_\vto{v}\nonumber\\
\fua{\Theta}{\alpha\vto{u}+\beta\vto{v}}&=&\alpha\fua{\Theta}{\vto{u}}+\beta\fua{\Theta}{\vto{v}}\,.\nonumber
\end{eqnarray}
Agora, para mostrar que $\Theta$ define um mapeamento injetor, se
\begin{eqnarray}
\lco\fua{\Theta}{\vto{u}}\rco\lpa\vto{x}\rpa&=&\lco\fua{\Theta}{\vto{v}}\rco\lpa\vto{x}\rpa\nonumber\\
\vto{u}\cdot\vto{x}&=&\vto{v}\cdot\vto{x}\,,\nonumber
\end{eqnarray}
logo $\lpa\vto{u}-\vto{v}\rpa\cdot\vto{x}=0$. Como esta igualdade
deve ser v�lida para qualquer $\vto{x}$, se $\vto{x}\neq\vto{0}$,
ent�o $\vto{u}=\vto{v}$. Para provar que o mapeamento de $\Theta$
� sobrejetor, consideremos uma base
$\con{U}=\lch\vto{v}_1,\cdots,\vto{v}_n\rch$ de $\ehr{V}{F}$, sua
base rec�proca $\con{W}=\lch\vto{w}_1,\cdots,\vto{w}_n\rch$ e um
funcional qualquer $\vartheta\in\con{V^*}$. Dado um vetor
$\vto{v}\in\con{V}$ definido por
$\vto{v}=\sum_{i=1}^{n}\fua{\vartheta}{\vto{v}_i}\vto{w}_i$ e um
vetor qualquer $\vto{x}=\sum_{i=1}^{n}\alpha_i\vto{v}_i$, tem-se o
desenvolvimento a seguir:
\begin{eqnarray}
\fua{\vartheta_{\vto{v}}}{\vto{x}}&=&\vto{v}\cdot\vto{x}\nonumber\\
&=&\lpa\sum_{i=1}^{n}\fua{\vartheta}{\vto{v}_i}\vto{w}_i\rpa\cdot\lpa\sum_{j=1}^{n}\alpha_j\vto{v}_j\rpa\nonumber\\
&=&\sum_{i=1}^{n}\sum_{j=1}^{n}\fua{\vartheta}{\vto{v}_i}\alpha_j\delta_{ij}\nonumber\\
&=&\sum_{i=1}^{n}\alpha_i\fua{\vartheta}{\vto{v}_i}\nonumber\\
&=&\fua{\vartheta}{\sum_{i=1}^{n}\alpha_i\vto{v}_i}\nonumber\\
&=&\fua{\vartheta}{\vto{x}}\,.\nonumber
\end{eqnarray}
Desta forma, tem-se que $\vartheta_{\vto{v}}=\vartheta$, de onde
se conclui que o funcional $\vartheta$ � definido por
$\fua{\Theta}{\vto{v}}$.
\end{prova}



\subsection{Coordenadas de Vetores}\index{coordenadas! de vetores}\label{subsec:funcionaisVetores}
Seja um espa�o de Hilbert $\ehr{V}{F}$ e $\edu{V}{F}$ seu espa�o
dual. Seja $\con{U}=\lch\vto{v}_1,\cdots,\vto{v}_n\rch$ uma base
de $\ehr{V}{F}$ e
$\vtf{f}_1^\con{U},\cdots,\vtf{f}_n^\con{U}\in\con{V^*}$ os
funcionais coordenados por ela definidos. A partir do teorema
\ref{teo:RepresentacaoRiesz}, sejam os vetores
$\vtf{f}_1,\cdots,\vtf{f}_n\in\con{V}$ respectivamente associados
aos funcionais $\vtf{f}_1^\con{U},\cdots,\vtf{f}_n^\con{U}$. Desta
forma, dado um vetor qualquer $\vto{v}\in\con{V}$, tem-se que
\begin{equation}
\vto{v}=\sum_{i=1}^n \fua{\vtf{f}_i^\con{U}}{\vto{v}}\vto{v}_i =
\sum_{i=1}^n \fua{\vartheta_{\vtf{f}_i}}{\vto{v}}\vto{v}_i =
\sum_{i=1}^n \lpa \vtf{f}_i \cdot \vto{v} \rpa\vto{v}_i\,.
\end{equation}
Apesar da conclus�o desta igualdade, a regra para os funcionais
coordenados $\vtf{f}_i^\con{U}$ fica indeterminada j� que os
vetores $\vtf{f}_i$ s�o desconhecidos. Para resolver o problema de
uma forma simples, pode-se definir que
$\lch\vtf{f}_1,\cdots,\vtf{f}_n\rch$ e $\con{U}$ s�o conjuntos
rec�procos, ou seja, que
\begin{equation}
\vtf{f}_i\cdot\vto{v}_j = \delta_{ij}\,.
\end{equation}

Como o par de conjuntos rec�procos $\con{U}$ e
$\lch\vtf{f}_1,\cdots,\vtf{f}_n\rch$ � �nico\rodape{Ver se��o
\ref{sec:conjuntosReciprocos}.}, o conjunto ordenado
$\lch\vtf{f}_1,\cdots,\vtf{f}_n\rch$ pode ser representado por
$\con{U}_*=\lch \vto{v}_1^*,\cdots,\vto{v}_n^* \rch$ e a regra
para os funcionais coordenados na base $\con{U}$ resulta na
igualdade
\begin{equation}\label{eq:regraGeralVetor}
\fua{\vtf{f}_i^\con{U}}{\vto{x}}=\vto{v}_i^*\cdot\vto{x}\,.
\end{equation}
A unicidade no par de bases rec�procas torna
$\lch\vtf{f}_1,\cdots,\vtf{f}_n\rch$ uma base e permite dizer
tamb�m que $\lpa\con{U}_*\rpa_*=\con{U}$. Desta forma, os
funcionais coordenados na base $\con{U}_*$ possuem a regra
\begin{equation}\label{eq:regraGeralVetorCovariante}
\fua{\vtf{f}_i^\con{U_*}}{\vto{x}}=\vto{v}_i\cdot\vto{x}\,.
\end{equation}
Os valores $\fua{\vtf{f}_i^\con{U_*}}{\vto{x}}$ e $\fua{\vtf{f}_i^\con{U}}{\vto{x}}$ s�o denominados as
coordenadas covariantes e contravariantes do vetor $\vto{x}$
respectivamente. Para que estas coordenadas sejam determinadas
relativas somente � base $\con{U}$, considera-se que
\begin{equation}
\vtf{f}_i^{{}_*\con{U}}:=\vtf{f}_i^\con{U_*}\,.
\end{equation}

\subsubsection{Produto Interno}

Considerando as condi��es anteriores, dados dois vetores quaisquer
$\vto{v},\vto{w}\in\con{V}$, pode-se realizar o seguinte
desenvolvimento:
\begin{eqnarray}\label{eq:coordenadasProdutoInterno}
\vto{v}\cdot\vto{w}&=&\sum_{i=1}^{n}\fua{\vtf{f}_i^\con{U}}{\vto{v}}\vto{v}_i\cdot
\sum_{j=1}^{n}\fua{\vtf{f}_j^{{}_*\con{U}}}{\vto{w}}\vto{v}_j^*\nonumber\\
&=&\sum_{i=1}^{n}\sum_{j=1}^{n}\fua{\vtf{f}_i^\con{U}}{\vto{v}}
\fua{\vtf{f}_j^{{}_*\con{U}}}{\vto{w}}\delta_{ij}\nonumber\\
&=&\sum_{i=1}^{n}\fua{\vtf{f}_i^\con{U}}{\vto{v}}
\fua{\vtf{f}_i^{{}_*\con{U}}}{\vto{w}}\,.
\end{eqnarray}
� interessante observar que se o espa�o de Hilbert $\ehr{V}{F}$
for Euclidiano e $\con{U}$ base ortonormal\rodape{Toda base ortonormal � rec�proca a ela pr�pria, segundo a se��o \ref{sec:conjuntosReciprocos}.}, obt�m-se
(\ref{eq:funcionalProdutoInterno}).

\section{Tensores}

\subsection{Espa�o Tensorial}\index{espa�o!tensorial}
Sejam os espa�os de Hilbert $\ehr{V_1}{F},\cdots,\ehr{V_n}{F}$. O
espa�o vetorial de funcionais multilineares
$\evl{\crt{V}{n}}{F}{F}$ � denominado espa�o tensorial,
representado por $\ete{\crt{V}{n}}{F}$. Um funcional $\vtf{t}$ do
conjunto $\cfl{\crt{V}{n}}{F}$, agora representado\rodape{Alguns
autores costumam denominar o conjunto $\cfl{\crt{V}{n}}{\con{F}}$
de \emph{produto tensorial}\index{produto!tensorial} entre os
conjuntos $\con{V}_1,\cdots,\con{V}_n$ e represent�-lo por
$\con{V}_1\otimes\cdots\otimes\con{V}_n$.} por
$\cft{\crt{V}{n}}{F}$, � dito um \emph{tensor}\index{tensor} de
ordem $n$, cuja nota��o � alterada para $\tnr{T}$. Em particular,
dado o conjunto $\con{V}$, diz-se que
$\tnr{T}\in\cft{\con{V}^n}{F}$ � um tensor de ordem $n$ em
$\con{V}$.

\begin{prp}\label{prp:DualTensorial}
O espa�o tensorial $\ete{\con{V}}{F}$ � o espa�o dual
$\edu{V}{F}$.
\end{prp}
\begin{prova}
No teorema \ref{teo:RepresentacaoRiesz}, a transforma��o linear
bijetora provoca $\cft{\con{V}}{F}=\con{V}^*$.
\end{prova}


\subsubsection{Tipos}\index{espa�o!tensorial!tipos de} Seja o conjunto $\con{V}^{(p,q)}:=\con{V}^p\times\lpa\con{V^*}\rpa^q$.
 Todo espa�o tensorial de ordem $p+q$ na forma
$\ete{\con{V}^{(p,q)}}{F}$ � classificado como sendo do tipo $\lpa
p,q\rpa$ em $\con{V}$. Tais espa�os s�o ditos possuir \emph{ordem
contravariante}\index{ordem!contravariante} $p$ e \emph{ordem
covariante}\index{ordem!covariante} $q$.

Agora, seja um espa�o tensorial $\ete{\con{U}^{(p,q)}}{F}$. Se
este espa�o for do tipo $\lpa 0,0\rpa$, os tensores do conjunto
$\cft{\con{U}^{(0,0)}}{F}$ s�o, por defini��o, escalares. Caso o
espa�o tensorial for do tipo $\lpa 1,0\rpa$, segundo a igualdade
(\ref{eq:VetorFuncional}) e a proposi��o \ref{prp:DualTensorial},
pode-se considerar que os tensores em quest�o s�o os vetores de
$\con{U}$. A partir desta considera��o, um conjunto
$\con{V}^{(p,q)}$, cuja tupla ordenada contem um misto de $p$
vetores e $q$ funcionais lineares, pode ser convenientemente
interpretado como um conjunto $\con{V}^{p+q}$, dotado com tuplas
de $p$ vetores \emph{contravariantes} e $q$ vetores
\emph{covariantes}.

\subsection{Permuta��o de Tensor}\index{permuta��o!de tensor}
A bije��o que define $\map{\pi}{\{ 1,\cdots,n \}}{\{
1,\cdots,n \}}$ � denominada \emph{permuta��o de
ordem}\index{permuta��o!de tensor!ordem de} $n$. Seja o espa�o
tensorial $\ete{\con{V}^n}{F}$. Define-se que a permuta��o de um
tensor qualquer $\tnr{T}\in\cft{\con{V}^n}{F}$ � o tensor
$\tnr{T}_\pi\in\cft{\con{V}^n}{F}$, onde
\begin{equation}\label{eq:permutacaoTensor}
\fua{\tnr{T}_\pi}{\vto{v}_1,\cdots,\vto{v}_n}=
\fua{\tnr{T}}{\vto{v}_{\fua{\pi}{1}},\cdots,\vto{v}_{\fua{\pi}{n}}}\,,\forall\,
\vto{v}_i\in\con{V}\,.
\end{equation}

\subsubsection{Transposi��o de Tensor}\index{transposi��o!de tensor}
Um caso particular da permuta��o considerada anteriormente � a
transposi��o de tensor, onde, para qualquer $\vto{v}_i\in\con{V}$,
\begin{eqnarray}\label{eq:transposicaoTensor}
  \lefteqn{\fua{\tnr{T}_{(\alpha,\beta)}}{\vto{v}_1,\cdots,\vto{v}_{\alpha-1},\vto{v}_\alpha,\vto{v}_{\alpha+1}
\cdots,\vto{v}_{\beta-1},\vto{v}_\beta,\vto{v}_{\beta+1},\cdots,\vto{v}_n}
=} & & \nonumber\\
  &
&\fua{\tnr{T}}{\vto{v}_1,\cdots,\vto{v}_{\alpha-1},\vto{v}_\beta,\vto{v}_{\alpha+1}
\cdots,\vto{v}_{\beta-1},\vto{v}_\alpha,\vto{v}_{\beta+1},\cdots,\vto{v}_n}\,.
\end{eqnarray}
No caso de $p\leqslant n$ transposi��es sucessivas do tensor
$\tnr{T}$, utiliza-se
\begin{equation}\label{eq:transposicaoSucessiva}
\tnr{T}_{(\alpha_1,\beta_1)(\alpha_p,\beta_p)}:=\lco\cdots\lco\tnr{T}_{(\alpha_1,\beta_1)}\rco_{(\alpha_2,\beta_2)}\cdots\rco_{(\alpha_p,\beta_p)}\,.
\end{equation}


\subsubsection{Simetria de Tensor}\index{simetria!de tensor}
Seja $\con{R}_n$ o conjunto de todas as permuta��es de ordem $n$.
Considerando o espa�o tensorial $\ete{\con{V}^n}{F}$, um tensor
$\tnr{S}\in\cft{\con{V}^n}{F}$ �
\emph{sim�trico}\index{tensor!sim�trico} se
\begin{equation}
\tnr{S}_\pi= \tnr{S}\,,\forall\, \pi\in\con{R}_n\,.
\end{equation}
Um tensor $\tnr{P}\in\cft{\con{V}^n}{F}$ � \emph{alternante}\index{tensor!alternante} ou
\emph{anti-sim�trico}\index{tensor!anti-sim�trico} se
\begin{equation}\label{eq:tensorAlternante}
\tnr{P}_\pi=\epsilon_{\fua{\pi}{1}\cdots\fua{\pi}{n}}\tnr{P}\,,\forall\,
\pi\in\con{R}_n\,.
\end{equation}
Pode-se obter que os subconjuntos $\cfs{\con{V}^n}{F}$ e $\cfa{\con{V}^n}{F}$ de
$\cft{\con{V}^n}{F}$, formados por todos os tensores sim�tricos e todos os
anti-sim�tricos respectivamente, definem os subespa�os tensoriais $\ets{\con{V}^n}{F}$ e
$\etn{\con{V}^n}{F}$ de $\ete{\con{V}^n}{F}$.

\begin{prp}\label{teo:dimensaoEspacoAlternante}
Seja um espa�o tensorial $\ete{\con{V}^n}{F}$ e seu subespa�o de tensores anti-sim�tricos
$\etn{\con{V}^n}{F}$. Neste contexto, sempre � v�lido que
\begin{equation*}
\dmq{\etn{\con{V}^n}{F}} = 1\,.
\end{equation*}
\end{prp}
\begin{prova}\rodape{Adaptada de \aut{Backus}\cite{backus_1997_1}, pp. 15-17. }
Primeiramente, seja um tensor $\tnr{A}\in\cft{\con{V}^n}{F}$ e
$\con{U}=\lch\vto{u}_1,\cdots,\vto{u}_n\rch$ uma base qualquer de $\ehr{V}{F}$ tais que
\begin{equation*}
\fua{\tnr{A}}{\vto{x}_1,\cdots,\vto{x}_n}=\sum_{i_1=1}^n\cdots\sum_{i_n=1}^n\lpa\vto{x}_1\cdot\vto{u}_{i_1}^*\rpa\cdots\lpa\vto{x}_n\cdot\vto{u}_{i_n}^*\rpa\epsilon_{i_1\cdots
i_n}\,\,,\forall\, \vto{x}_i\in\con{V}\,.
\end{equation*}
Com base nesta regra, pode-se realizar o seguinte desenvolvimento:
\begin{eqnarray*}
\fua{\tnr{A}_{(\alpha,\beta)}}{\vto{x}_1,\cdots,\vto{x}_\alpha,\cdots,\vto{x}_\beta,\cdots,\vto{x}_n}=
\sum_{i_1=1}^n\cdots\sum_{i_n=1}^n\lpa\vto{x}_1\cdot\vto{u}_{i_1}^*\rpa\cdots
\lpa\vto{x}_\beta\cdot\vto{u}_{i_\alpha}^*\rpa\cdots &&\\
\cdots(\vto{x}_\alpha\cdot\vto{u}_{i_\beta}^*)
\cdots\lpa\vto{x}_n\cdot\vto{u}_{i_n}^*\rpa\epsilon_{i_1\cdots i_\alpha\cdots
i_\beta\cdots i_n}&&\\
=\sum_{i_1=1}^n\cdots\sum_{i_n=1}^n\lpa\vto{x}_1\cdot\vto{u}_{i_1}^*\rpa\cdots
(\vto{x}_\beta\cdot\vto{u}_{i_\beta}^*)\cdots && \\
\cdots(\vto{x}_\alpha\cdot\vto{u}_{i_\alpha}^*)
\cdots\lpa\vto{x}_n\cdot\vto{u}_{i_n}^*\rpa\epsilon_{i_1\cdots i_\beta\cdots
i_\alpha\cdots i_n}&&\\
=-\sum_{i_1=1}^n\cdots\sum_{i_n=1}^n\lpa\vto{x}_1\cdot\vto{u}_{i_1}^*\rpa\cdots
(\vto{x}_\beta\cdot\vto{u}_{i_\beta}^*)\cdots && \\
\cdots(\vto{x}_\alpha\cdot\vto{u}_{i_\alpha}^*)
\cdots\lpa\vto{x}_n\cdot\vto{u}_{i_n}^*\rpa\epsilon_{i_1\cdots i_\alpha\cdots
i_\beta\cdots i_n}\\
=-\fua{\tnr{A}}{\vto{x}_1,\cdots,\vto{x}_\alpha,\cdots,\vto{x}_\beta,\cdots,\vto{x}_n}\,,\,\forall\,
\alpha,\beta\in\lch1,\cdots,n\rch\,,&&
\end{eqnarray*}
de onde se conclui que $\tnr{A}\in\cfa{\con{V}^n}{F}\,$. Consideremos agora um outro
tensor anti-sim�trico $\tnr{P}\in\cfa{\con{V}^n}{F}$, a partir do qual o desenvolvimento
a seguir � realizado.
\begin{eqnarray*}
\fua{\tnr{P}}{\vto{x}_1,\cdots,\vto{x}_n}&=&\fua{\tnr{P}}{\sum_{i_1=1}^n\fua{\vtf{f}_{i_1}^\con{U_*}}{\vto{x}_1}\vto{u}_{i_1},\cdots,
\sum_{i_n=1}^n\fua{\vtf{f}_{i_n}^\con{U_*}}{\vto{x}_n}\vto{u}_{i_n}}\\
\fua{\tnr{P}}{\vto{x}_1,\cdots,\vto{x}_n}&=&\sum_{i_1=1}^n\cdots\sum_{i_n=1}^n\fua{\vtf{f}_{i_1}^\con{U_*}}{\vto{x}_1}\cdots
\fua{\vtf{f}_{i_n}^\con{U_*}}{\vto{x}_n}\fua{\tnr{P}}{\vto{u}_{i_1},\cdots,\vto{u}_{i_n}}\,.
\end{eqnarray*}
De acordo com (\ref{eq:transposicaoTensor}) e (\ref{eq:tensorAlternante}), podemos
continuar da seguinte forma:
\begin{eqnarray*}
\fua{\tnr{P}}{\vto{x}_1,\cdots,\vto{x}_n}&=&\sum_{i_1=1}^n\cdots\sum_{i_n=1}^n
\fua{\vtf{f}_{i_1}^\con{U_*}}{\vto{x}_1}\cdots\fua{\vtf{f}_{i_n}^\con{U_*}}{\vto{x}_n}
\fua{\tnr{P}_\pi}{\vto{u}_{1},\cdots,
\vto{u}_{n}}\\
&=&\sum_{i_1=1}^n\cdots\sum_{i_n=1}^n
\fua{\vtf{f}_{i_1}^\con{U_*}}{\vto{x}_1}\cdots\fua{\vtf{f}_{i_n}^\con{U_*}}{\vto{x}_n}
\epsilon_{i_{i_1}\cdots i_{i_n}}\fua{\tnr{P}}{\vto{u}_{1},\cdots, \vto{u}_{n}}\\
&=&\fua{\tnr{P}}{\vto{u}_{1},\cdots, \vto{u}_{n}}\sum_{i_1=1}^n\cdots\sum_{i_n=1}^n
\lpa\vto{x}_1\cdot\vto{u}_{i_1}^*\rpa\cdots\lpa\vto{x}_n\cdot\vto{u}_{i_n}^*\rpa\epsilon_{i_1\cdots
i_n}\\
&=&\underbrace{\fua{\tnr{P}}{\vto{u}_{1},\cdots, \vto{u}_{n}}}_{\in\,
\con{F}}\fua{\tnr{A}}{\vto{x}_1,\cdots,\vto{x}_n}\,,\,\forall\,
\tnr{P}\in\cfa{\con{V}^n}{F}.
\end{eqnarray*}
Podemos concluir da �ltima igualdade que o conjunto de tensores anti-sim�tricos
$\cfa{\con{V}^n}{F}=sp\lch\tnr{A}\rch$\,.
\end{prova}

\paragraph{Espa�o Orientado de Hilbert.}
Sejam um espa�o de Hilbert $\ehr{V}{F}$ $n$-dimensional e um espa�o tensorial
$\ete{\con{V}^n}{F}$. Sejam $\hat{\con{U}}=\lch\vun{u}_1,\cdots,\vun{u}_n\rch$ uma base
ortonormal qualquer de $\ehr{V}{F}$ e um tensor anti-sim�trico qualquer
$\hat{\tnr{P}}\in\cfa{\con{V}^n}{F}$ tal que
\begin{equation}\label{eq:HilbertOrientado}
\fua{\hat{\tnr{P}}}{\vun{u}_1,\cdots,\vun{u}_n}=\pm 1\,.
\end{equation}
Neste contexto, o par ordenado $(\ehr{V}{F},\hat{\tnr{P}})$ � dito um espa�o orientado de Hilbert. Se o resultado da igualdade anterior seja positivo, diz-se que a base
$\hat{\con{U}}$ est� \emph{positivamente orientada}\index{base!positivamente orientada},
caso contr�rio ela est� \emph{negativamente orientada}\index{base!negativamente
orientada}.

\subsection{Produto Tensorial}\index{produto!tensorial}

Sejam os espa�os tensoriais $\ete{\crt{V}{p}}{\con{F}}$ ,
$\ete{\crt{U}{q}}{\con{F}}$ e $\ete{\crt{V}{p}\times\crt{U}{q}}{\con{F}}$. Um dado mapeamento
$\map{\otimes}{\cft{\crt{V}{p}}{\con{F}}\times\cft{\crt{U}{q}}{\con{F}}}
{\cft{\crt{V}{p}\times\crt{U}{q}}{\con{F}}}$ � denominado produto
tensorial se, a partir dos tensores quaisquer
$\tnr{V}\in\cft{\crt{V}{p}}{\con{F}}$ e
$\tnr{U}\in\cft{\crt{U}{q}}{\con{F}}$, for v�lida, para qualquer
$\lpa\vto{v}_1,\cdots,\vto{v}_p,\vto{u}_1,\cdots,\vto{u}_q\rpa\in\crt{V}{p}\times\crt{U}{q}$,
a seguinte igualdade:
\begin{equation}\label{eq:produtoTensorial}
\fua{\tnr{V}\otimes\tnr{U}}{\vto{v}_1,\cdots,\vto{v}_p,\vto{u}_1,\cdots,\vto{u}_q}=
\fua{\tnr{V}}{\vto{v}_1,\cdots,\vto{v}_p}\fua{\tnr{U}}{\vto{u}_1,\cdots,\vto{u}_q}\,.
\end{equation}
Fica evidente que um produto tensorial gera um tensor cuja ordem �
a soma das ordens dos tensores que o definem. Caso $\tnr{V}$ e
$\tnr{U}$ sejam tensores do tipo $\lpa p,m \rpa$ e $\lpa q,n \rpa$
respectivamente, ent�o $\tnr{V}\otimes\tnr{U}$ � do tipo $\lpa
p+q,m+n \rpa$.

A partir da igualdade (\ref{eq:produtoTensorial}), as propriedades
a seguir podem ser facilmente obtidas para quaisquer
$\tnr{V}_1,\tnr{V}_2\in\cft{\crt{V}{p}}{\con{F}}$,
$\tnr{U}\in\cft{\crt{U}{q}}{\con{F}}$ e
$\tnr{W}\in\cft{\crt{W}{r}}{\con{F}}$:
\begin{itemize}
  \item[i.] $\tnr{U}\otimes\tnr{W}=\negmath{0} \implies\tnr{U}=\negmath{0}$ ou $\tnr{W}=\negmath{0}$ ;
  \item[ii.] Associatividade:
$\lpa\tnr{V}_1\otimes\tnr{U}\rpa\otimes\tnr{W}=\tnr{V}_1\otimes\lpa\tnr{U}\otimes\tnr{W}\rpa$;
\item[iii.] Distributividade � direita:
$\lpa\tnr{V}_1+\tnr{V}_2\rpa\otimes\tnr{U}=\tnr{V}_1\otimes\tnr{U}+\tnr{V}_2\otimes\tnr{U}$
; \item[iv.] Distributividade � esquerda:
$\tnr{U}\otimes\lpa\tnr{V}_1+\tnr{V}_2\rpa=\tnr{U}\otimes\tnr{V}_1+\tnr{U}\otimes\tnr{V}_2$
.
\end{itemize}

\subsection{Tensor Poli�dico}\index{tensor!poli�dico}

Considerando os espa�os tensoriais
$\ete{\con{V}_1}{\con{F}},\cdots,\ete{\con{V}_n}{\con{F}}$, um
tensor
$\vto{v}_1\otimes\cdots\otimes\vto{v}_n\in\cft{\crt{V}{n}}{\con{F}}$,
onde $\vto{v}_i$ � um vetor qualquer de $\con{V}_i$, � qualificado
como poli�dico de ordem $n$. Se $n=2$, o tensor �
\emph{di�dico}\index{tensor!di�dico}; para $n=3$, o tensor �
\emph{tri�dico}\index{tensor!tri�dico} e assim por diante. Cada
vetor que comp�e o tensor poli�dico � uma
\emph{pol�ade}\index{pol�ade}. Os vetores de um tensor di�dico s�o
\emph{d�ades}\index{d�ade}; os do tensor tri�dico s�o
\emph{tr�ades}\index{tr�ade}, etc... A partir do teorema
\ref{teo:RepresentacaoRiesz} e utilizando
(\ref{eq:produtoTensorial}), tem-se para qualquer $\lpa
\vto{x}_1,\cdots,\vto{x}_n\rpa\in\crt{V}{n}$ que
\begin{equation}\label{eq:produtorioPoliadico}
\fua{\vto{v}_1\otimes\cdots\otimes\vto{v}_n}{\vto{x}_1,\cdots,\vto{x}_n}=\prod_{i=1}^{n}
\vto{v}_i\cdot\vto{x}_i\,.
\end{equation}

\begin{prp}\label{prp:igualdadeTensorial}
Sejam os espa�os de Hilbert
$\ehr{V_1}{\con{F}},\cdots,\ehr{V_p}{\con{F}}$ dimensionalmente
finitos e o conjunto gen�rico $\con{U}_i=\{ {\vto{v}_1}^{(i)}
,\cdots, {\vto{v}_{n_i}}^{(i)}\}$, $1\leqslant i\leqslant p\,$, representando uma base ordenada
qualquer de um destes espa�os. Considerando tensores quaisquer
$\tnr{T}_1,\tnr{T}_2\in\cft{\crt{V}{p}}{\con{F}}$ e uma tupla
qualquer $\lpa \vto{x}_1,\cdots,\vto{x}_p\rpa\in\crt{V}{p}$,
tem-se o seguinte:
\begin{itemize}
  \item[i.]� sempre v�lido que
\begin{eqnarray}
  \lefteqn{\fua{\tnr{T}_1}{
\vto{x}_1,\cdots,\vto{x}_p}
=} & & \nonumber \\
  &
&\sum_{j_1=1}^{n_1}\cdots\sum_{j_p=1}^{n_p}
\fua{\vtf{f}_{j_1}^{\con{U}_1}}{\vto{x}_1}\cdots
\fua{\vtf{f}_{j_p}^{\con{U}_p}}{\vto{x}_p}\fua{\tnr{T}_1}{
{\vto{v}_{j_1}}^{(1)},\cdots,{\vto{v}_{j_p}}^{(p)}}\nonumber\,.
\end{eqnarray}
  \item[ii.]Se as igualdades
\begin{equation}
\fua{\tnr{T}_1}{
{\vto{v}_{j_1}}^{(1)},\cdots,{\vto{v}_{j_p}}^{(p)}}=\fua{\tnr{T}_2}{
{\vto{v}_{j_1}}^{(1)},\cdots,{\vto{v}_{j_p}}^{(p)}}\nonumber
\end{equation}
forem verificadas, ent�o $\tnr{T}_1=\tnr{T}_2$.
\end{itemize}
\end{prp}
\begin{prova}
A verifica��o da igualdade no �tem i. depende fundamentalmente das
propriedades de multilinearidade. Da�, seja o seguinte
desenvolvimento:
\begin{eqnarray}
 & \fua{\tnr{T}_1}{
\vto{x}_1,\cdots,\vto{x}_p}
  & \nonumber\\
  &
=\fua{\tnr{T}_1}{
\sum_{j_1=1}^{n_1}\fua{\vtf{f}_{j_1}^{\con{U}_1}}{\vto{x}_1}{\vto{v}_{j_1}}^{(1)},
\cdots,\sum_{j_p=1}^{n_p}\fua{\vtf{f}_{j_p}^{\con{U}_p}}{\vto{x}_p}{\vto{v}_{j_p}}^{(p)}}
& \nonumber \\
&=\sum_{j_1=1}^{n_1}\fua{\vtf{f}_{j_1}^{\con{U}_1}}{\vto{x}_1}\fua{\tnr{T}_1}{
{\vto{v}_{j_1}}^{(1)},
\cdots,\sum_{j_p=1}^{n_p}\fua{\vtf{f}_{j_p}^{\con{U}_p}}{\vto{x}_p}{\vto{v}_{j_p}}^{(p)}}
& \nonumber\\
&=\sum_{j_1=1}^{n_1}\fua{\vtf{f}_{j_1}^{\con{U}_1}}{\vto{x}_1}\sum_{j_2=1}^{n_2}\fua{\vtf{f}_{j_2}^{\con{U}_2}}{\vto{x}_2}\fua{\tnr{T}_1}{
{\vto{v}_{j_1}}^{(1)},{\vto{v}_{j_2}}^{(2)},
\cdots,\sum_{j_p=1}^{n_p}\fua{\vtf{f}_{j_p}^{\con{U}_p}}{\vto{x}_p}{\vto{v}_{j_p}}^{(p)}}\,,
& \nonumber
\end{eqnarray}
e assim sucessivamente. No caso do �tem ii. se
\begin{equation}
\fua{\tnr{T}_1}{
{\vto{v}_{j_1}}^{(1)},\cdots,{\vto{v}_{j_p}}^{(p)}}=\fua{\tnr{T}_2}{
{\vto{v}_{j_1}}^{(1)},\cdots,{\vto{v}_{j_p}}^{(p)}}, \nonumber
\end{equation} � �bvio que
\begin{eqnarray}
  & \fua{\tnr{T}_1}{
\vto{x}_1,\cdots,\vto{x}_p}
  & \nonumber\\
&=\sum_{j_1=1}^{n_1}\cdots\sum_{j_p=1}^{n_p}
\fua{\vtf{f}_{j_1}^{\con{U}_1}}{\vto{x}_1}\cdots
\fua{\vtf{f}_{j_p}^{\con{U}_p}}{\vto{x}_p}\fua{\tnr{T}_2}{
{\vto{v}_{j_1}}^{(1)},\cdots,{\vto{v}_{j_p}}^{(p)}} & \nonumber\\
&=\fua{\tnr{T}_2}{ \vto{x}_1,\cdots,\vto{x}_p}\,. & \nonumber
\end{eqnarray}
\end{prova}

\begin{teo}\label{teo:BasesPoliadicas}
Sejam os espa�os de Hilbert
$\ehr{V_1}{\con{F}},\cdots,\ehr{V_p}{\con{F}}$ dimensionalmente
finitos e o conjunto gen�rico $\{ {\vto{v}_1}^{(i)} ,\cdots,
{\vto{v}_{n_i}}^{(i)}\} \subset \con{V}_i$ representando uma base
ordenada qualquer de um destes espa�os. O conjunto
$\con{T}^{\otimes}_{\crt{V}{p}}\subset\cft{\crt{V}{p}}{\con{F}}$
formado por todos os tensores poli�dicos de ordem p constru�dos na
forma
${\vto{v}_{j_1}}^{(1)}\otimes{\vto{v}_{j_2}}^{(2)}\otimes\cdots\otimes{\vto{v}_{j_p}}^{(p)}$,
onde $j_i=1,\cdots,n_i$ , � uma base do espa�o tensorial
$\ete{\crt{V}{p}}{\con{F}}$.
\end{teo}
\begin{prova}
Em primeiro lugar, devemos verificar se os tensores
${\vto{v}_{j_1}}^{(1)}\otimes\cdots\otimes{\vto{v}_{j_p}}^{(p)}$
s�o linearmente independentes. Para tal, seja um tensor
\begin{equation}
\tnr{T}:=\sum_{j_1=1}^{n_1}\cdots\sum_{j_p=1}^{n_p}\alpha_{j_1\cdots
j_p}{\vto{v}_{j_1}}^{(1)}\otimes\cdots\otimes{\vto{v}_{j_p}}^{(p)},\nonumber
\end{equation}
onde $\alpha_{j_1\cdots j_p}\in\con{F}$. Para que os tensores
poli�dicos sejam linearmente independentes, se $\tnr{T}=\negmath{0}$,
ent�o $\alpha_{j_1\cdots j_p}=0$. Isto � comprovado da seguinte
forma: seja $(\vto{x}_1,\cdots,\vto{x}_p)$ uma tupla ordenada
qualquer de $\crt{V}{p}$. Para $\tnr{T}=\negmath{0}$, tem-se que cada
um dos termos
\begin{eqnarray}
\fua{\alpha_{j_1\cdots
j_p}{\vto{v}_{j_1}}^{(1)}\otimes\cdots\otimes{\vto{v}_{j_p}}^{(p)}}{\vto{x}_1,\cdots,\vto{x}_p}&=&0\nonumber\\
\alpha_{j_1\cdots j_p} \prod_{k=1}^{p}
{\vto{v}_{j_k}}^{(k)}\cdot\vto{x}_k&=&0\,.\nonumber
\end{eqnarray}
Em particular, caso os vetores de $(\vto{x}_1,\cdots,\vto{x}_p)$
sejam n�o nulos, ent�o
\begin{eqnarray}
\alpha_{j_1\cdots j_p} \prod_{k=1}^{p} \underbrace{\overbrace{{\vto{v}_{j_k}}^{(k)}}^{\neq\vto{0}}\cdot\overbrace{\vto{x}_k}^{\neq\vto{0}}}_{\neq 0}&=&0\nonumber\\
\alpha_{j_1\cdots j_p} &=&0\,.\nonumber
\end{eqnarray}
Em segundo lugar, devemos mostrar que os tensores poli�dicos geram
$\cft{\crt{V}{p}}{\con{F}}$. Para tal, seja um tensor qualquer
$\tnr{T}_1\in\cft{\crt{V}{p}}{\con{F}}$ onde
\begin{equation}
\beta_{j_1\cdots
j_p}:=\fua{\tnr{T}_1}{{\vto{v}_{j_1}}^{(1)},\cdots,{\vto{v}_{j_p}}^{(p)}}\nonumber
\end{equation}
e um tensor
\begin{equation}
\tnr{T}_2:=\sum_{j_1=1}^{n_1}\cdots\sum_{j_p=1}^{n_p}\beta_{j_1\cdots
j_p}{\vto{v}_{j_1}}^{(1)}\otimes\cdots\otimes{\vto{v}_{j_p}}^{(p)}\,.\nonumber
\end{equation}
Seja, agora, o seguinte desenvolvimento:
\begin{eqnarray}
\fua{\tnr{T}_2}{{\vto{v}_{j_1}}^{(1)},\cdots,{\vto{v}_{j_p}}^{(p)}}&=&\beta_{j_1\cdots
j_p}\underbrace{\prod_{k=1}^{p}
{\vto{v}_{j_k}}^{(k)}\cdot{\vto{v}_{j_k}}^{(k)}}_{\gamma_{j_1\cdots j_p}}\nonumber\\
&=&\fua{\tnr{T}_1}{{\vto{v}_{j_1}}^{(1)},\cdots,{\vto{v}_{j_p}}^{(p)}}\gamma_{j_1\cdots
j_p}\nonumber\,,
\end{eqnarray}
de onde, a partir do �tem ii. da proposi��o
\ref{prp:igualdadeTensorial}, tem-se
\begin{equation}
\tnr{T}_1=
\sum_{j_1=1}^{n_1}\cdots\sum_{j_p=1}^{n_p}\frac{\beta_{j_1\cdots
j_p}}{\gamma_{j_1\cdots
j_p}}{\vto{v}_{j_1}}^{(1)}\otimes\cdots\otimes{\vto{v}_{j_p}}^{(p)}\,.\nonumber
\end{equation}
Conclui-se que o tensor qualquer $\tnr{T}_1$ � uma combina��o
linear dos tensores poli�dicos em quest�o.
\end{prova}


\begin{prp}\label{prp:dimensaoTensorial}
Sejam os espa�os de Hilbert
$\ehr{V_1}{\con{F}},\cdots,\ehr{V_p}{\con{F}}$ dimensionalmente
finitos e um espa�o tensorial $\ete{\crt{V}{p}}{\con{F}}$. Sempre
� v�lido que
\begin{equation}
\dim\lpa\ete{\crt{V}{p}}{\con{F}}\rpa=\prod_{i=1}^{p}
\dim\lpa\ehr{V_i}{\con{F}}\rpa\,.\nonumber
\end{equation}
\end{prp}
\begin{prova}
A constru��o da base de tensores poli�dicos descrita no teorema
\ref{teo:BasesPoliadicas} permite constatar que existem
$n_1n_2\cdots n_p$ tensores na forma
${\vto{v}_{j_1}}^{(1)}\otimes\cdots\otimes{\vto{v}_{j_p}}^{(p)}$,
onde $j_i=1,\cdots,n_i$ e cujos vetores pertencem �
$\{{\vto{v}_1}^{(i)} ,\cdots, {\vto{v}_{n_i}}^{(i)}\}$, base de
$\ehr{V_i}{\con{F}}$.
\end{prova}



\subsection{Coordenadas de Tensores}\index{coordenadas!de tensor}

Seja o espa�o tensorial dimensionalmente finito
$\ete{\crt{V}{p}}{\con{F}}$, o conjunto
$\con{T}^{\otimes}_{\crt{V}{p}}$ sua base de tensores poli�dicos
constru�da a partir do subconjunto gen�rico $\con{U}_i=\{
{\vto{v}_1}^{(i)} ,\cdots, {\vto{v}_{n_i}}^{(i)}\}$, base de
$\ehr{V_i}{\con{F}}$. Segundo o teorema \ref{teo:BasesPoliadicas},
um tensor qualquer $\tnr{T}\in\cft{\crt{V}{p}}{\con{F}}$ pode ser
decomposto da seguinte forma:
\begin{equation}
\tnr{T}=\sum_{i_1=1}^{n_1}\cdots\sum_{i_p=1}^{n_p}\beta_{i_1\cdots
i_p}{\vto{v}_{i_1}}^{(1)}\otimes\cdots\otimes{\vto{v}_{i_p}}^{(p)}\,,
\end{equation}
onde $\beta_{i_1\cdots i_p}\in\con{F}$ s�o as coordenadas de
$\tnr{T}$ na base $\con{T}^{\otimes}_{\crt{V}{p}}$. Como ocorre no
caso de vetores, a partir de transforma��es lineares do tipo
\begin{equation}
\map{\vtf{f}_{i_1\cdots
i_p}^{\con{T}^{\otimes}_{\crt{V}{p}}}}{\cft{\crt{V}{p}}{\con{F}}}{\con{F}}\,,
\end{equation}
pode-se considerar que os escalares
\begin{equation}
\beta_{i_1\cdots i_p}:=\fua{\vtf{f}_{i_1\cdots
i_p}^{\con{T}^{\otimes}_{\crt{V}{p}}}}{\tnr{T}}\,,
\end{equation}
onde cada fun��o � o funcional coordenado de $\tnr{T}$ na base
$\con{T}^{\otimes}_{\crt{V}{p}}$. A regra de cada um destes
funcionais � definida por
\begin{equation}\label{eq:contravarianteTensor}
\fua{\vtf{f}_{i_1\cdots
i_p}^{\con{T}^{\otimes}_{\crt{V}{p}}}}{\tnr{X}}=\fua{\tnr{X}}
{{\vto{v}_{i_1}^*}^{(1)},\cdots,{\vto{v}_{i_p}^*}^{(p)}}\,,
\end{equation}
cujos valores s�o as coordenadas contravariantes de $\tnr{X}$ na
base $\con{T}^{\otimes}_{\crt{V}{p}}$. Nesta mesma base, suas
coordenadas covariantes s�o os valores
\begin{equation}\label{eq:covarianteTensor}
\fua{\vtf{f}_{i_1\cdots
i_p}^{{}_*\con{T}^{\otimes}_{\crt{V}{p}}}}{\tnr{X}}:=\fua{\tnr{X}}
{{\vto{v}_{i_1}}^{(1)},\cdots,{\vto{v}_{i_p}}^{(p)}}\,.
\end{equation}
Caso $p=1$, as regras destes funcionais abrangem o caso de
funcionais coordenados para vetores (\ref{eq:regraGeralVetor}) e
(\ref{eq:regraGeralVetorCovariante}).

Conclui-se, ent�o, que dada uma base de tensores poli�dicos, um
tensor qualquer $\tnr{X}$ de $\cft{\crt{V}{p}}{\con{F}}$ pode ser
caracterizado por dois arrays de dimens�o $n_1\times\cdots\times
n_p$, onde $n_i=\dim\lpa\ehr{V_i}{\con{F}}\rpa$, representados por
\begin{eqnarray}
\lco\tnr{X}\rco^{\con{T}^{\otimes}_{\crt{V}{p}}}&\mathrm{e}&
\lco\tnr{X}\rco^{{}_*\con{T}^{\otimes}_{\crt{V}{p}}}\,.
\end{eqnarray}


\subsubsection{Transforma��o de Coordenadas}\index{transforma��o!de coordenadas!de tensor}
Com base nas condi��es anteriores, seja o conjunto $W_i=\{
{\vto{w}_1}^{(i)} ,\cdots, {\vto{w}_{n_i}}^{(i)}\}$ uma outra base
qualquer de $\ehr{V_i}{\con{F}}$, a partir da qual constr�i-se a
base de tensores poli�dicos
$\tilde{\con{T}}^{\otimes}_{\crt{V}{p}}$. Nos termos do teorema
\ref{teo:temUnicaFuncaoBase}, seja o espa�o vetorial
${\evl{V_i}{W_i}{\con{F}}}$, $1\leqslant i\leqslant p\,$, e a bije��o linear
$\vtf{q}^{(i)}\in\cfl{V_i}{W_i}$ promotora da mudan�a de base
\begin{equation}
\fua{\vtf{q}^{(i)}}{\vto{v}_j^{(i)}}=\vto{w}_{j}^{(i)},
j=1,\cdots,n_i\,.
\end{equation}

Admitindo conhecidas as coordenadas do tensor $\tnr{T}$ na base
$\con{T}^{\otimes}_{\crt{V}{p}}$, utilizando
(\ref{eq:FuncaoLinearMatriz}) e (\ref{eq:regraGeralVetor}) , pode-se realizar o seguinte
desenvolvimento:
\begin{eqnarray}\label{eq:transformacaoTensor}
\fua{\vtf{f}_{i_1\cdots
i_p}^{\tilde{\con{T}}^{\otimes}_{\crt{V}{p}}}}{\tnr{T}}&=&\fua{\tnr{T}}
{{\vto{w}_{i_1}^*}^{(1)},\cdots,{\vto{w}_{i_p}^*}^{(p)}}\nonumber\\
&=&\fua{\sum_{j_1=1}^{n_1}\cdots\sum_{j_p=1}^{n_p}\fua{\vtf{f}_{j_1\cdots
j_p}^{\con{T}^{\otimes}_{\crt{V}{p}}}}{\tnr{T}}{\vto{v}_{j_1}}^{(1)}
\otimes\cdots\otimes{\vto{v}_{j_p}}^{(p)}}{{\vto{w}_{i_1}^*}^{(1)},\cdots,{\vto{w}_{i_p}^*}^{(p)}}\nonumber\\
&=&\sum_{j_1=1}^{n_1}\cdots\sum_{j_p=1}^{n_p}\fua{\vtf{f}_{j_1\cdots
j_p}^{\con{T}^{\otimes}_{\crt{V}{p}}}}{\tnr{T}}\prod_{k=1}^p
{\vto{v}_{j_k}}^{(k)}\cdot{\vto{w}_{i_k}^*}^{(k)}\nonumber\\
&=&\sum_{j_1=1}^{n_1}\cdots\sum_{j_p=1}^{n_p}\fua{\vtf{f}_{j_1\cdots
j_p}^{\con{T}^{\otimes}_{\crt{V}{p}}}}{\tnr{T}}\prod_{k=1}^p
\fua{\vtf{f}_{i_k}^\con{W_k}}{\vto{v}_{j_k}^{(k)}}\nonumber\\
&=&\sum_{j_1=1}^{n_1}\cdots\sum_{j_p=1}^{n_p}\fua{\vtf{f}_{j_1\cdots
j_p}^{\con{T}^{\otimes}_{\crt{V}{p}}}}{\tnr{T}}\prod_{k=1}^p
\mad{\vtf{i}_\con{U_k}}{W_k}_{i_kj_k}\,.
\end{eqnarray}
Pode-se observar que a �ltima igualdade revela uma transforma��o
nas coordenadas de $\tnr{T}$ da base
$\con{T}^{\otimes}_{\crt{V}{p}}$ para a base
$\tilde{\con{T}}^{\otimes}_{\crt{V}{p}}\,$. Em termos gerais,
conhecendo-se, a priori, as coordenadas de um dado tensor
descritas em uma determinada base, � poss�vel obt�-las descritas
em qualquer outra.

\begin{prp}\label{prp:somaPoliadicos}
Dado um espa�o tensorial $\ete{\crt{V}{p}}{\con{F}}$
dimensionalmente finito, um tensor qualquer
$\tnr{T}\in\cft{\crt{V}{p}}{\con{F}}$ pode sempre ser decomposto
numa soma de tensores poli�dicos, ou seja,
\begin{equation}
\tnr{T}=\sum_{k=1}^m\vto{v}_{1k}\otimes\cdots\otimes\vto{v}_{pk}\,,\,
m\geq 1,\nonumber
\end{equation}
onde $\vto{v}_{ik}\in\con{V}_i$ � a $i$-�sima pol�ade do $k$-�simo
tensor poli�dico.
\end{prp}
\begin{prova}
Segundo o teorema \ref{teo:BasesPoliadicas}, considerando uma base
$\{{\vto{v}_{1}}^{(i)},\cdots,{\vto{v}_{n_i}}^{(i)}\}$ de
$\ehr{V_i}{\con{F}}$, um tensor qualquer
\begin{eqnarray}
\tnr{T}&=&\sum_{i_1=1}^{n_1}\cdots\sum_{i_q=1}^{n_q}\cdots\sum_{i_p=1}^{n_p}\beta_{i_1\cdots
i_p}{\vto{v}_{i_1}}^{(1)}\otimes\cdots\otimes{\vto{v}_{i_q}}^{(q)}\otimes\cdots\otimes{\vto{v}_{i_p}}^{(p)}\nonumber\\
&=&\sum_{i_1=1}^{n_1}\cdots\sum_{i_q=1}^{n_q}\cdots\sum_{i_p=1}^{n_p}{\vto{v}_{i_1}}^{(1)}
\otimes\cdots\otimes\underbrace{\lpa\beta_{i_1\cdots
i_p}\rpa{\vto{v}_{i_q}}^{(q)}}_{\vto{u}_{i_1\cdots
i_p}}\otimes\cdots\otimes{\vto{v}_{i_p}}^{(p)}\nonumber\,,
\end{eqnarray}
onde o vetor destacado define $\tnr{T}$ como uma soma de
$n_1\cdots n_q \cdots n_p$ tensores poli�dicos, cada um com $p$
pol�ades.
\end{prova}


\begin{prp}\label{prp:produtorioFuncionais}
Seja um espa�o tensorial $\ete{\crt{V}{p}}{\con{F}}$ e $\con{U}_i$
uma base de $\ehr{V_i}{\con{F}}$. Dado que
$\con{T}^{\otimes}_{\crt{V}{p}}$ � constru�do a partir dos vetores
de $\con{U}_1,\cdots,\con{U}_p$, considerando os vetores quaisquer
$\vto{v}_i\in\con{V_i}$, as coordenadas
\begin{equation}
\fua{\vtf{f}_{j_1\cdots
j_p}^{\con{T}^{\otimes}_{\crt{V}{p}}}}{\vto{v}_1\otimes\cdots\otimes\vto{v}_p}=\prod_{k=1}^p
\fua{\vtf{f}_{j_k}^{\con{U}_k}}{\vto{v}_k}\nonumber
\end{equation}
e
\begin{equation}
\fua{\vtf{f}_{j_1\cdots
j_p}^{{}_*\con{T}^{\otimes}_{\crt{V}{p}}}}{\vto{v}_1\otimes\cdots\otimes\vto{v}_p}=\prod_{k=1}^p
\fua{\vtf{f}_{j_k}^{{}_*\con{U}_k}}{\vto{v}_k}\nonumber\,.
\end{equation}
\end{prp}
\begin{prova}
Tomando  (\ref{eq:contravarianteTensor}) e
(\ref{eq:covarianteTensor}), as igualdades s�o conseq��ncias de (\ref{eq:regraGeralVetor}),
(\ref{eq:regraGeralVetorCovariante}),
(\ref{eq:produtorioPoliadico}).
\end{prova}


\subsection{Tensor Identidade}\index{tensor!identidade}
Seja o espa�o tensorial
$\ete{\crt{V}{p}\times\crt{V}{p}}{\con{F}}$. Um tensor
$\tnr{I}\in\cft{\crt{V}{p}\times\crt{V}{p}}{\con{F}}$ � dito
identidade se
\begin{equation}\label{eq:tensorIdentidade}
\fua{\tnr{I}}{\vto{x}_1,\cdots,\vto{x}_p,\vto{y}_1,\cdots,\vto{y}_p}=
\prod_{i=1}^{p}\vto{x}_i\cdot\vto{y}_i\,,\,\forall\,
\vto{x}_i,\vto{y}_i\in\con{V}_i.
\end{equation}
Considerando $\con{U}_i=\{ {\vto{u}_1}^{(i)} ,\cdots,
{\vto{u}_{n_i}}^{(i)}\}$ uma base de $\ehr{V_i}{\con{F}}$, a
partir de (\ref{eq:coordenadasProdutoInterno}), a igualdade
(\ref{eq:tensorIdentidade}) pode ser reescrita na forma
\begin{equation}
\fua{\tnr{I}}{\vto{x}_1,\cdots,\vto{x}_p,\vto{y}_1,\cdots,\vto{y}_p}=
\prod_{i=1}^{p}\sum_{j_i}^{n_i}\fua{\vtf{f}_{j_i}^{\con{U}_i}}{\vto{x}_i}
\fua{\vtf{f}_{j_i}^{{}_*\con{U}_i}}{\vto{y}_i}
\end{equation}
ou na forma
\begin{equation}
\fua{\tnr{I}}{\vto{x}_1,\cdots,\vto{x}_p,\vto{y}_1,\cdots,\vto{y}_p}=
\prod_{i=1}^{p}\sum_{j_i}^{n_i}\fua{\vtf{f}_{j_i}^{{}_*\con{U}_i}}{\vto{x}_i}
\fua{\vtf{f}_{j_i}^{\con{U}_i}}{\vto{y}_i}\,.
\end{equation}
Nestas circunst�ncias, a fim de obter as igualdades anteriores respectivas, o
tensor
\begin{equation}\label{eq:TensorIdentidadeExplicito1}
\tnr{I} =
\sum_{j_1=1}^{n_1}\cdots\sum_{j_p=1}^{n_p}{\vto{u}_{j_1}^*}^{(1)}\otimes\cdots\otimes
{\vto{u}_{j_p}^*}^{(p)}\otimes{\vto{u}_{j_1}}^{(1)}\otimes\cdots\otimes{\vto{u}_{j_p}}^{(p)}
\end{equation}
ou
\begin{equation}\label{eq:TensorIdentidadeExplicito2}
\tnr{I} =
\sum_{j_1=1}^{n_1}\cdots\sum_{j_p=1}^{n_p}{\vto{u}_{j_1}}^{(1)}\otimes\cdots\otimes
{\vto{u}_{j_p}}^{(p)}\otimes{\vto{u}_{j_1}^*}^{(1)}\otimes\cdots\otimes{\vto{u}_{j_p}^*}^{(p)}\,.
\end{equation}


\section{Fun��es Tensoriais}\index{fun��o!tensorial}
A fun��o cujos elementos do dom�nio s�o tensores e os elementos da
imagem tamb�m s�o tensores � chamada fun��o tensorial. Dados os
espa�os tensoriais $\ete{\crt{V}{p}}{F}$ e $\ete{\crt{W}{q}}{F}$,
se um determinado mapeamento
$\map{\psi}{\cft{\crt{V}{p}}{F}}{\cft{\crt{W}{q}}{F}}$ for uma
transforma��o linear, ent�o $\psi$ � uma \emph{fun��o tensorial
linear}\index{fun��o!tensorial!linear}.

\subsection{Arrays Associados}\index{array!associado!a fun��o tensorial}
Sejam os espa�os tensoriais $\ete{\crt{V}{p}}{\con{F}}$,
$\ete{\crt{W}{q}}{\con{F}}$ e o espa�o vetorial de fun��es
tensoriais
\begin{equation}
\evl{\cft{\crt{V}{p}}{\con{F}}}{\cft{\crt{W}{q}}{\con{F}}}{\con{F}}\,.\nonumber
\end{equation}
 Sejam os tensores $\tnr{T}\in\cft{\crt{V}{p}}{\con{F}}$ e
$\tnr{Z}\in\cft{\crt{W}{q}}{\con{F}}$, tais que
$\tnr{Z}=\fua{\psi}{\tnr{T}}$, onde
\begin{equation}
\psi\in\cfl{\cft{\crt{V}{p}}{\con{F}}}{\cft{\crt{W}{q}}{\con{F}}}\,.\nonumber
\end{equation}
Sejam $\{ {\vto{v}_1}^{(i)} ,\cdots, {\vto{v}_{n_p}}^{(i)}\}$ e
$\{ {\vto{w}_1}^{(i)} ,\cdots, {\vto{w}_{n_q}}^{(i)}\}$ bases de
$\ehr{V_i}{\con{F}}$ e $\ehr{W_i}{\con{F}}$ respectivamente,
geradoras das bases de tensores poli�dicos poli�dicos
$\con{T}^{\otimes}_{\crt{V}{p}}$ e
$\con{T}^{\otimes}_{\crt{W}{q}}$. De maneira similar ao
desenvolvimento que gera (\ref{eq:FuncaoLinearMatriz}), tem-se
\begin{eqnarray}\label{eq:desenvarrayFuncaoTensorial}
\tnr{Z}&=&{\sum_{i_1}^{n_1}\cdots\sum_{i_q}^{n_q}\fua{\vtf{f}_{i_1\cdots
i_q}^{\con{T}^{\otimes}_{\crt{W}{q}}}}{\fua{\psi}{\tnr{T}}}{\vto{w}_{i_1}}^{(1)}
\otimes\cdots\otimes{\vto{w}_{i_q}}^{(q)}}\nonumber\\
&=&\sum_{i_1}^{n_1}\cdots\sum_{i_p}^{n_q}\fua{\vtf{f}_{i_1\cdots
i_q}^{\con{T}^{\otimes}_{\crt{W}{q}}}}{
\sum_{j_1}^{n_1}\cdots\sum_{j_p}^{n_p}\fua{\vtf{f}_{j_1\cdots
j_p}^{\con{T}^{\otimes}_{\crt{V}{p}}}}{\tnr{T}}\fua{\psi}{{\vto{v}_{j_1}}^{(1)}
\otimes\cdots\otimes{\vto{v}_{j_p}}^{(p)}}}\nonumber\\
& & {\vto{w}_{i_1}}^{(1)}
\otimes\cdots\otimes{\vto{w}_{i_q}}^{(q)}\nonumber\\
&=&\sum_{i_1}^{n_1}\cdots\sum_{i_p}^{n_q}
\sum_{j_1}^{n_1}\cdots\sum_{j_p}^{n_p}\underbrace{\fua{\vtf{f}_{j_1\cdots
j_p}^{\con{T}^{\otimes}_{\crt{V}{p}}}}{\tnr{T}}}_{\mat{B}_{j_1\cdots
j_p}}\underbrace{\fua{\vtf{f}_{i_1\cdots
i_q}^{\con{T}^{\otimes}_{\crt{W}{q}}}}{\fua{\psi}{{\vto{v}_{j_1}}^{(1)}
\otimes\cdots\otimes{\vto{v}_{j_p}}^{(p)}}}}_{\mat{A}_{i_1\cdots
i_qj_1\cdots
j_p}}\nonumber\\
& & {\vto{w}_{i_1}}^{(1)}
\otimes\cdots\otimes{\vto{w}_{i_q}}^{(q)}\,,
\end{eqnarray}
tal que
\begin{eqnarray}
\lco\psi_{\con{T}^{\otimes}_{\crt{V}{p}}}\rco^{\con{T}^{\otimes}_{\crt{W}{q}}}:=\mat{A}&\mathrm{e}&
\mat{B}=\lco\tnr{T}\rco^{\con{T}^{\otimes}_{\crt{V}{p}}}\,.\nonumber
\end{eqnarray}
Desta forma,
\begin{equation}\label{eq:arrayFuncaoTensorial}
\lco\tnr{Z}\rco^{\con{T}^{\otimes}_{\crt{W}{q}}} =
\lco\psi_{\con{T}^{\otimes}_{\crt{V}{p}}}\rco^{\con{T}^{\otimes}_{\crt{W}{q}}}
\lco\tnr{T}\rco^{\con{T}^{\otimes}_{\crt{V}{p}}}\,.
\end{equation}

\subsubsection{Arrays e Fun��es Tensoriais Compostas}\index{array!associado!a fun��o tensorial composta}
Considerando as condi��es anteriores e o espa�o vetorial
\begin{equation}
\evl{\cft{\crt{W}{q}}{\con{F}}}{\cft{\crt{V}{p}}{\con{F}}}{\con{F}}\,,\nonumber
\end{equation}
sejam as fun��es
\begin{equation}
\psi_1\in\cfl{\cft{\crt{V}{p}}{\con{F}}}{\cft{\crt{W}{q}}{\con{F}}}\nonumber
\end{equation}
e
\begin{equation}
\psi_2\in\cfl{\cft{\crt{W}{q}}{\con{F}}}{\cft{\crt{V}{p}}{\con{F}}}\,.\nonumber
\end{equation}
Dado o tensor $\tnr{G}\in\cft{\crt{V}{p}}{\con{F}}$, onde
$\tnr{G}=\fua{\psi_2\circ\psi_1}{\tnr{T}}$, pode-se obter, de
maneira semelhante a (\ref{eq:funcaoCompostaMatriz}), a igualdade
\begin{equation}
\lco\tnr{G}\rco^{\con{T}^{\otimes}_{\crt{V}{p}}} =
\lco{\lpa\psi_2\circ\psi_1\rpa}_{\con{T}^{\otimes}_{\crt{V}{p}}}\rco^{\con{T}^{\otimes}_{\crt{V}{q}}}
\lco\tnr{T}\rco^{\con{T}^{\otimes}_{\crt{V}{p}}}\,,
\end{equation}
onde
\begin{equation}\label{eq:arrayFuncaoTensorialComposta}
\lco{\lpa\psi_2\circ\psi_1\rpa}_{\con{T}^{\otimes}_{\crt{V}{p}}}\rco^{\con{T}^{\otimes}_{\crt{V}{q}}}=
\lco{\psi_2}_{\con{T}^{\otimes}_{\crt{W}{q}}}\rco^{\con{T}^{\otimes}_{\crt{V}{p}}}
\lco{\psi_1}_{\con{T}^{\otimes}_{\crt{V}{p}}}\rco^{\con{T}^{\otimes}_{\crt{W}{q}}}\,.
\end{equation}


\subsection{Eleva��o}\index{Eleva��o}
A uma fun��o multilinear qualquer $\vtf{g}$, cujos argumentos s�o
vetores, � sempre poss�vel associar uma fun��o linear
$\vtf{g}^\otimes$ que tem um tensor como argumento. Diz-se que
$\vtf{g}$ � ``elevada'' � $\vtf{g}^\otimes$. Esta
eleva��o\rodape{O termo em ingl�s � ``lifting''.} estabelece uma
conveniente e importante rela��o, descrita no teorema a seguir,
entre uma seq��ncia qualquer de vetores e um tensor. Fica
impl�cito que este teorema fundamenta quest�es de exist�ncia e
unicidade para os diversos tipos de fun��es tensoriais lineares a
serem apresentados nas se��es subseq�entes.

\begin{teo}[Eleva��o]\index{Eleva��o!Teorema da}\label{teo:Lifting}
Sejam os espa�os de Hilbert
$\ehr{V_1}{\con{F}},\cdots,\ehr{V_p}{\con{F}}$ e o espa�o vetorial
$\evt{W}{\con{F}}$. Seja o espa�o vetorial
$\evl{\crt{V}{p}}{W}{\con{F}}$ de fun��es multilineares que mapeiam uma tupla ordenada de vetores para um vetor de $\con{W}$ e
$\evl{\cft{\crt{V}{p}}{\con{F}}}{W}{\con{F}}$ o espa�o vetorial
das fun��es lineares que mapeiam um tensor para um vetor de
$\con{W}$. Dada uma fun��o qualquer
$\vtf{g}\in\cfl{\crt{V}{p}}{W}$ e uma tupla qualquer
$(\vto{v}_1,\cdots,\vto{v}_p)\in\crt{V}{p}$, existe uma �nica
fun��o $\vtf{g}^\otimes\in\cfl{\cft{\crt{V}{p}}{\con{F}}}{W}$,
chamada fun��o elevada de $\vtf{g}$, onde o vetor
\begin{equation}
\fua{\vtf{g}}{\vto{v}_1,\cdots,\vto{v}_p}=\fua{\vtf{g}^\otimes}{\vto{v}_1\otimes\cdots\otimes\vto{v}_p}.\nonumber
\end{equation}
\end{teo}
\begin{prova}\rodape{Adaptada de \aut{Backus}\cite{backus_1997_1}, pp. 43-45.}
Seja um tensor qualquer $\tnr{T}\in\cft{\crt{V}{p}}{\con{F}}$ e
sua decomposi��o
\begin{equation}
\tnr{T}=\sum_{i_1=1}^{n_1}\cdots\sum_{i_p=1}^{n_p}\fua{\vtf{f}_{i_1\cdots
i_p}^{\con{T}^{\otimes}_{\crt{V}{p}}}}{\tnr{T}}{\vto{u}_{i_1}}^{(1)}\otimes\cdots\otimes{\vto{u}_{i_p}}^{(p)}\,,
\nonumber
\end{equation}
onde se considera
$\con{U}_i=\{{\vto{u}_{1}}^{(i)},\cdots,{\vto{u}_{n_i}}^{(i)}\}$
uma base de $\ehr{V_i}{\con{F}}$. A partir da igualdade deste
teorema, pode-se dizer que
\begin{eqnarray}
\fua{\vtf{g}^\otimes}{\tnr{T}}&=&\sum_{i_1=1}^{n_1}\cdots\sum_{i_p=1}^{n_p}\fua{\vtf{f}_{i_1\cdots
i_p}^{\con{T}^{\otimes}_{\crt{V}{p}}}}{\tnr{T}}\fua{\vtf{g}^\otimes}{{\vto{u}_{i_1}}^{(1)}\otimes\cdots\otimes{\vto{u}_{i_p}}^{(p)}}\nonumber\\
&=&\sum_{i_1=1}^{n_1}\cdots\sum_{i_p=1}^{n_p}\underbrace{\fua{\vtf{f}_{i_1\cdots
i_p}^{\con{T}^{\otimes}_{\crt{V}{p}}}}{\tnr{T}}}\underbrace{\fua{\vtf{g}}{{\vto{u}_{i_1}}^{(1)},\cdots,{\vto{u}_{i_p}}^{(p)}}}\,.\nonumber
\end{eqnarray}
Os termos destacados s�o unicamente determinados para qualquer
tensor $\tnr{T}$, logo $\vtf{g}^\otimes$ existe e fica unicamente
determinado. A fim de constatar a igualdade do teorema, seja
\begin{equation}
\tnr{T}=\vto{v}_1\otimes\cdots\otimes\vto{v}_p\,.\nonumber
\end{equation}
Do resultado do desenvolvimento anterior, considerando as propriedades de fun��es multilineares e a
proposi��o \ref{prp:produtorioFuncionais}, tem-se que
\begin{eqnarray}
\fua{\vtf{g}^\otimes}{\vto{v}_1\otimes\cdots\otimes\vto{v}_p}&=&\sum_{i_1=1}^{n_1}\cdots\sum_{i_p=1}^{n_p}
\prod_{k=1}^p \fua{\vtf{f}_{i_k}^{\con{U}_k}}{\vto{v}_k}
\fua{\vtf{g}}{{\vto{u}_{i_1}}^{(1)},\cdots,{\vto{u}_{i_p}}^{(p)}}\nonumber\\
&=&
\fua{\vtf{g}}{\sum_{i_1=1}^{n_1}\fua{\vtf{f}_{i_1}^{\con{U}_1}}{\vto{v}_1}
{\vto{u}_{i_1}}^{(1)},\cdots,\sum_{i_p=1}^{n_p}\fua{\vtf{f}_{i_p}^{\con{U}_p}}{\vto{v}_p}{\vto{u}_{i_p}}^{(p)}}\nonumber\nonumber\\
&=& \fua{\vtf{g}}{\vto{v}_1,\cdots,\vto{v}_p}\,.\nonumber
\end{eqnarray}
\end{prova}



\subsection{Contra��o}\index{contra��o}
Seja um espa�o tensorial $\ete{\con{V}_{(p)}}{\con{F}}$, onde
$\con{V}_{(p)}:=\crt{V}{p}$, e o conjunto $\con{V}_{(p-n)}$,
formado a partir da retirada de $n$ conjuntos quaisquer de
$\con{V}_{(p)}$. Considerando o espa�o tensorial
$\ete{\con{V}_{(p-n)}}{\con{F}}$ e o espa�o vetorial
$\evl{\con{V}_{(p)}}{\cft{\con{V}_{(p-n)}}{\con{F}}}{\con{F}}$, a
fun��o elevada de
$\vtf{c}_n\in\cfl{\con{V}_{(p)}}{\cft{\con{V}_{(p-n)}}{\con{F}}}$,
representada $\vtf{c}_n^{\otimes}$, � uma fun��o tensorial linear
denominada contra��o de ordem $n$. A partir do teorema
\ref{teo:Lifting}, dado um tensor qualquer
$\tnr{T}\in\cft{\con{V}_{(p)}}{\con{F}}$ com decomposi��o
\begin{equation}\label{eq:decomposicaoTensor}
\tnr{T}=\sum_{k=1}^m\vto{v}_{1k}\otimes\cdots\otimes\vto{v}_{pk}\,,\,
m\geq 1,
\end{equation}
tem-se que o tensor
\begin{equation}\label{eq:elevadaTensor}
\fua{\vtf{c}_n^\otimes}{\tnr{T}}=\sum_{k=1}^m\fua{\vtf{c}_n}{\vto{v}_{1k},\cdots,\vto{v}_{pk}}
\end{equation}
e sua ordem � $p-n$.

\subsubsection{Tra�o de Tensor}\index{tra�o!de tensor}
Seja o conjunto $\con{V}_{(p)}:=\crt{V}{p}$, onde $p\geq 2$, cujo
produto cartesiano contem pelo menos um par de conjuntos iguais.
Dado que $\con{V}_r=\con{V}_s$, $1\leq r < s \leq p$, seja o
conjunto
\begin{equation}
\con{V}_{(p-2)}:=\con{V}_1\times\cdots\times\con{V}_{r-1}\times\con{V}_{r+1}
\times\cdots\times\con{V}_{s-1}\times\con{V}_{s+1}\times\cdots\times\con{V}_p\,.
\end{equation}
Com base nos espa�os $\ete{\con{V}_{(p)}}{\con{F}}$,
$\ete{\con{V}_{(p-2)}}{\con{F}}$ e
$\evl{\con{V}_{(p)}}{\cft{\con{V}_{(p-2)}}{\con{F}}}{\con{F}}$,
diz-se que a contra��o $\vtf{c}_2^\otimes$, cujo valor
\begin{eqnarray}
  \lefteqn{\fua{\vtf{c}_2}{\vto{u}_1,\cdots,\vto{u}_p}:=} & & \nonumber\\
  &
&\lpa
\vto{u}_r\cdot\vto{u}_s\rpa\vto{u}_1\otimes\cdots\otimes\vto{u}_{r-1}\otimes\vto{u}_{r+1}
\otimes\cdots\otimes\vto{u}_{s-1}\otimes\vto{u}_{s+1}\otimes\cdots\otimes\vto{u}_p\,,\,\forall\,\vto{u}_i\in\con{V}_i\,,\nonumber\\
\end{eqnarray}
� o tra�o em $r,s$, representada $\trt{r}{s}$. Nas condi��es da
igualdade (\ref{eq:elevadaTensor}), tem-se que
\begin{eqnarray}\label{eq:tracoTensor}
  \lefteqn{\fua{\trt{r}{s}}{\tnr{T}}=} & & \nonumber\\
  &
&\sum_{k=1}^m\lpa
\vto{v}_{rk}\cdot\vto{v}_{sk}\rpa\vto{v}_{1k}\otimes\cdots\otimes\vto{v}_{r-1k}\otimes\vto{v}_{r+1k}
\otimes\cdots\otimes\vto{v}_{s-1k}\otimes\vto{v}_{s+1k}\otimes\cdots\otimes\vto{v}_{pk}\,.\nonumber\\
\end{eqnarray}

\subsubsection{Produto Contrativo}\index{produto!contrativo}
Sejam os produtos cartesianos $\con{W}=\crt{W}{t}\times\crt{V}{s}$
e $\con{Z}=\crt{V}{s}\times\crt{Z}{m}$, a partir dos quais se
estabelece que $\con{V}_{(p)}:=W\times\con{Z}$, $p=t+2s+m$. Seja,
ent�o, o conjunto
\begin{eqnarray}
\lefteqn{\con{V}_{(p-2q)}:=} & & \nonumber\\
& &
\con{W}_1\times\cdots\con{W}_t\times\con{V}_{q+1}\times\cdots\times\con{V}_{s}\times\con{V}_{q+1}\times\cdots\times\con{V}_{s}\times\con{Z}_1\times\cdots\times\con{Z}_m\,,\nonumber\\
\end{eqnarray}
resultante da retirada dos primeiros $q$ pares de conjuntos iguais
em $\con{V}_{(p)}$. Seja a contra��o $\vtf{c}_{2q}^\otimes$ tal
que
\begin{eqnarray}
\lefteqn{\fua{\vtf{c}_{2q}}{\vto{w}_1,\cdots,\vto{w}_t,\vto{x}_1,\cdots,\vto{x}_s,\vto{y}_1,\cdots,\vto{y}_s,\vto{z}_1,\cdots,\vto{z}_m}:=} & & \nonumber\\
  &
&\lpa
\vto{x}_1\cdot\vto{y}_1\rpa\cdots\lpa\vto{x}_q\cdot\vto{y}_q\rpa
 \vto{w}_1\otimes\cdots\otimes\vto{w}_{t}\otimes\vto{x}_{q+1}\otimes\cdots\otimes\vto{x}_{s}\otimes\nonumber\\
& & \vto{y}_{q+1}\otimes\cdots\otimes\vto{y}_{s}
\otimes\vto{z}_1\otimes\cdots\otimes\vto{z}_m\,,
\end{eqnarray}
onde $\vto{w}_i\in\con{W}_i$, $\vto{x}_j,\vto{y}_j\in\con{V}_j$,
$\vto{z}_k\in\con{W}_k$. Considerando os tensores quaisquer
$\tnr{W}\in\cft{\con{W}}{\con{F}}$ e
$\tnr{Z}\in\cft{\con{Z}}{\con{F}}$ com decomposi��es
\begin{equation}
\tnr{W}=\sum_{k=1}^m\vto{w}_{1k}\otimes\cdots\otimes\vto{w}_{tk}\otimes
\vto{x}_{1k}\otimes\cdots\otimes\vto{x}_{qk},\,\vto{w}_{ik}\in\con{W}_i,\,\vto{x}_{jk}\in\con{V}_j,
\end{equation}
e
\begin{equation}
\tnr{Z}=\sum_{k=1}^n\vto{y}_{1k}\otimes\cdots\otimes\vto{y}_{tk}\otimes
\vto{z}_{1k}\otimes\cdots\otimes\vto{z}_{qk},\,\vto{y}_{ik}\in\con{V}_i,\,\vto{z}_{jk}\in\con{Z}_j,
\end{equation}
tem-se, segundo (\ref{eq:elevadaTensor}), o tensor de ordem $p-2q$
\begin{eqnarray}\label{eq:funcaoContrativoTensorial}
\lefteqn{\fua{\vtf{c}_{2q}^\otimes}{\tnr{W}\otimes\tnr{Z}}=} & & \nonumber\\
& &\sum_{i=1}^m\sum_{j=1}^n\lpa
\vto{x}_{1i}\cdot\vto{y}_{1j}\rpa\cdots\lpa\vto{x}_{qi}\cdot\vto{y}_{qj}\rpa
\vto{w}_{1i}\otimes\cdots\otimes\vto{w}_{ti}\otimes\vto{x}_{q+1i}\otimes\cdots\otimes\vto{x}_{si}\otimes\nonumber\\
& &
\vto{y}_{q+1j}\otimes\cdots\otimes\vto{y}_{sj}\otimes\vto{z}_{1j}\otimes\cdots\otimes\vto{z}_{mj}\,.
\end{eqnarray}
A fun��o no mapeamento
$\map{\odot_q}{\cft{\con{W}}{\con{F}}\times\cft{\con{Z}}{\con{F}}}{\cft{\con{V}_{(p-2q)}}{\con{F}}}$,
onde
\begin{equation}\label{eq:produtoContrativo}
\tnr{W}\odot_q\tnr{Z}:=\fua{\vtf{c}_{2q}^\otimes}{\tnr{W}\otimes\tnr{Z}},
\end{equation}
� denominada produto contrativo de ordem $q$. Para este produto,
s�o v�lidas as igualdades:
\begin{itemize}
\item[i.] $\tnr{W}\odot_0\tnr{Z}=\tnr{W}\otimes\tnr{Z}$;
\item[ii.] Bilinearidade: dados
$\tnr{Z}_1,\tnr{Z}_2\in\cft{\con{Z}}{\con{F}}$ e
$\ele{a},\ele{b}_1,\ele{b}_2\in\con{F}$ quaisquer,
\begin{equation}
\lpa\ele{a}\tnr{W}\rpa\odot_q\lpa\ele{b}_1\tnr{Z}_1+\ele{b}_2\tnr{Z}_2\rpa=\ele{a}\ele{b}_1\lpa
\tnr{W}\odot_q\tnr{Z}_1\rpa+\ele{a}\ele{b}_2\lpa
\tnr{W}\odot_q\tnr{Z}_2\rpa.\nonumber
\end{equation}
\end{itemize}
Al�m destas propriedades, o produto contrativo pode ser
associativo. Sejam os tensores $\tnr{Y}\in\cft{\con{Y}}{\con{F}}$
e $\tnr{K}\in\cft{\con{K}}{\con{F}}$, onde
\begin{equation}
\begin{array}{rcl}
\con{Y}&=&\con{V}_1\times\cdots\con{V}_s\times\con{Y}_1\times\cdots\times\con{Y}_u\times\con{W}_1\times\cdots\times\con{W}_m,\\
\con{K}&=&\con{W}_1\times\cdots\con{W}_m\times\con{K}_1\times\cdots\times\con{K}_n.
\end{array}
\end{equation}
Nestas condi��es, tem-se associatividade:
\begin{itemize}
\item[iii.]
$\tnr{W}\odot_q\lpa\tnr{Y}\odot_r\tnr{K}\rpa=\lpa\tnr{W}\odot_q\tnr{Y}\rpa\odot_r\tnr{K}$,
onde $q\leqslant s$ e $r\leqslant m$.
\end{itemize}

\paragraph{Produto Interno.} Considerando o produto cartesiano $\crt{V}{s}$,
o produto contrativo $\tnr{T}_1\odot_s\tnr{T}_2$, onde
$\tnr{T}_1,\tnr{T}_2\in\cft{\crt{V}{s}}{\con{F}}$, representado na
forma $\tnr{T}_1\odot\tnr{T}_2$, torna-se o produto interno
positivo-definido entre $\tnr{T}_1$ e $\tnr{T}_2$, pois a fun��o
$\odot$ define o mapeamento
$\map{\odot}{\crt{V}{s}\times\crt{V}{s}}{\con{\con{F}}}$,
obedecendo as propriedades apresentadas na se��o
\ref{subsec:EspacoProdutoInterno}. Na pr�tica, a nota��o �
alterada para $\tnr{T}_1\cdot\tnr{T}_2$ se $s=1$, para
$\tnr{T}_1:\tnr{T}_2$  se $s=2$, para $\tnr{T}_1\cdot:\tnr{T}_2$
se $s=3$, para $\tnr{T}_1::\tnr{T}_2$ se $s=4$, etc...

Diz-se ent�o que $\ete{\crt{V}{s}}{\con{F}}$ � um \emph{espa�o
tensorial produto interno}\index{espa�o!tensorial!produto
interno}. Caso sejam definidas uma m�trica e uma norma para este
espa�o, pode-se ter, de forma similar ao caso de espa�os
vetoriais, \emph{espa�os tensoriais de Banach e Hilbert}.


\subsection{Operador Tensorial Identidade}\index{operador!tensorial!identidade} Seja um espa�o tensorial
$\ete{\crt{V}{p}}{\con{F}}$ e o espa�o vetorial
$\evl{\cft{\crt{V}{p}}{\con{F}}}{\cft{\crt{V}{p}}{\con{F}}}{\con{F}}$.
O operador tensorial
\begin{equation}
\vtf{i}_{\crt{V}{p}}\in\cfl{\cft{\crt{V}{p}}{\con{F}}}{\cft{\crt{V}{p}}{\con{F}}}\nonumber
\end{equation}
� identidade se sua regra for descrita por
\begin{equation}
\fua{\vtf{i}_{\crt{V}{p}}}{\tnr{X}}=\tnr{X}\,.
\end{equation}
Desta forma, fica evidente que $\vtf{i}_{\crt{V}{p}}$ define um
mapeamento bijetor.

Seja uma base qualquer de tensores poli�dicos
$\con{T}^{\otimes}_{\crt{V}{p}}$  de $\ete{\crt{V}{p}}{\con{F}}$,
gerada pela base $\con{U}_i=\{ {\vto{v}_1}^{(i)} ,\cdots,
{\vto{v}_{n_p}}^{(i)}\}$ de $\ehr{V_i}{\con{F}}$. Tomando o array
associado a $\vtf{i}_{\crt{V}{p}}$ na base
$\con{T}^{\otimes}_{\crt{V}{p}}$ e utilizando a proposi��o
\ref{prp:produtorioFuncionais}, pode-se realizar o seguinte
desenvolvimento:
\begin{eqnarray}
\lco{\lpa\vtf{i}_{\crt{V}{p}}\rpa}_{\con{T}^{\otimes}_{\con{V}^{p}}}\rco^{\con{T}^{\otimes}_{\con{V}^{p}}}_{i_1\cdots
i_pj_1\cdots j_p}&=&\fua{\vtf{f}_{i_1\cdots
i_p}^{\con{T}^{\otimes}_{\crt{V}{p}}}}{\fua{\vtf{i}_{\crt{V}{p}}}{{\vto{v}_{j_1}}^{(1)}
\otimes\cdots\otimes {\vto{v}_{j_p}}^{(p)}}}\nonumber\\
&=&\prod_{k=1}^p
\fua{\vtf{f}_{i_k}^{\con{U}_k}}{{\vto{v}_{j_k}}^{(k)}}\nonumber\\
&=&\prod_{k=1}^p \lco\lpa\vtf{i}_{V_k}\rpa_{U_k}\rco^{U_k}_{i_k
j_k}\,.
\end{eqnarray}
� poss�vel obter que
\begin{equation}\label{eq:hiperdertIdentidade}
\hpd{\lco{\lpa\vtf{i}_{\crt{V}{p}}\rpa}_{\con{T}^{\otimes}_{\con{V}^{p}}}\rco^{\con{T}^{\otimes}_{\con{V}^{p}}}}=
\prod_{k=1}^p\det{\lco\lpa\vtf{i}_{V_k}\rpa_{U_k}\rco^{U_k}}=1\,.
\end{equation}


\subsection{Fun��o Tensorial Inversa}\index{fun��o!tensorial!inversa} Sejam os espa�os
tensoriais $\ete{\crt{V}{p}}{\con{F}}$,
$\ete{\crt{W}{q}}{\con{F}}$. Sejam os espa�os vetoriais de fun��es
tensoriais
$\evl{\cft{\crt{V}{p}}{\con{F}}}{\cft{\crt{W}{q}}{\con{F}}}{\con{F}}$,
$\evl{\cft{\crt{W}{q}}{\con{F}}}{\cft{\crt{V}{p}}{\con{F}}}{\con{F}}$
e
$\evl{\cft{\crt{V}{p}}{\con{F}}}{\cft{\crt{V}{p}}{\con{F}}}{\con{F}}$.
Dada a fun��o tensorial identidade
\begin{equation}
\vtf{i}_{\crt{V}{p}}\in\cfl{\cft{\crt{V}{p}}{\con{F}}}{\cft{\crt{V}{p}}{\con{F}}}\,,\nonumber
\end{equation}
denomina-se a fun��o
\begin{equation}
\psi^{-1}\in\cfl{\cft{\crt{W}{q}}{\con{F}}}{\cft{\crt{V}{p}}{\con{F}}}\nonumber
\end{equation}
de fun��o inversa de
\begin{equation}
\psi\in\cfl{\cft{\crt{V}{p}}{\con{F}}}{\cft{\crt{W}{q}}{\con{F}}}\nonumber
\end{equation}
se
\begin{equation}
\psi\circ\psi^{-1} = \vtf{i}_{\crt{V}{p}}\,.
\end{equation}

\subsection{Mudan�a de Base}\index{tensor!mudan�a de base} Considerando o espa�o tensorial
$\ete{\crt{V}{p}}{\con{F}}$, sejam duas bases quaisquer
$\con{U}_i=\{ {\vto{v}_1}^{(i)} ,\cdots, {\vto{v}_{n_p}}^{(i)}\}$
e $\con{Z}_i=\{ {\vto{w}_1}^{(i)} ,\cdots,
{\vto{w}_{n_q}}^{(i)}\}$ de $\ehr{V_i}{\con{F}}$, geradoras das
bases de tensores poli�dicos
$\hat{\con{T}}^{\otimes}_{\crt{V}{p}}$ e
$\tilde{\con{T}}^{\otimes}_{\crt{V}{p}}$ respectivamente. Dado o
espa�o vetorial de fun��es tensoriais lineares
$\evl{\cft{\crt{V}{p}}{\con{F}}}{\cft{\crt{V}{p}}{\con{F}}}{\con{F}}$,
diz-se que o operador tensorial
$\varsigma\in\cfl{\cft{\crt{V}{p}}{\con{F}}}{\cft{\crt{V}{p}}{\con{F}}}$,
onde
\begin{equation}
\fua{\varsigma}{{\vto{v}_{i_1}}^{(1)} \otimes\cdots\otimes
{\vto{v}_{i_p}}^{(p)}}={\vto{w}_{i_1}}^{(1)} \otimes\cdots\otimes
{\vto{w}_{i_p}}^{(p)}\,,
\end{equation}
� uma fun��o linear que promove uma mudan�a de base de
$\hat{\con{T}}^{\otimes}_{\crt{V}{p}}$ para
$\tilde{\con{T}}^{\otimes}_{\crt{V}{p}}$. Com base no
desenvolvimento que resulta (\ref{eq:arrayFuncaoTensorial}) e na
proposi��o \ref{prp:produtorioFuncionais}, pode-se realizar o
seguinte:
\begin{eqnarray}
\lco\varsigma_{\hat{\con{T}}^{\otimes}_{\con{V}^{p}}}\rco^{\tilde{\con{T}}^{\otimes}_{\crt{V}{p}}}_{i_1\cdots
i_pj_1\cdots j_p}&=&\fua{\vtf{f}_{i_1\cdots
i_p}^{\tilde{\con{T}}^{\otimes}_{\crt{V}{p}}}}{\fua{\varsigma}{{\vto{v}_{j_1}}^{(1)}
\otimes\cdots\otimes {\vto{v}_{j_p}}^{(p)}}}\nonumber\\
&=&\prod_{k=1}^p
\fua{\vtf{f}_{i_k}^{\con{Z}_k}}{{\vto{w}_{j_k}}^{(k)}}\nonumber\\
&=&\prod_{k=1}^p \lco\lpa\vtf{i}_{V_k}\rpa_{Z_k}\rco^{Z_k}_{i_k
j_k}\,.
\end{eqnarray}
Logo
\begin{equation}\label{eq:mudancaBaseIdentidade}
\lco\varsigma_{\hat{\con{T}}^{\otimes}_{\crt{V}{p}}}\rco^{\tilde{\con{T}}^{\otimes}_{\crt{V}{p}}}=
\lco\lpa\vtf{i}_{\crt{V}{p}}\rpa_{\tilde{\con{T}}^{\otimes}_{\crt{V}{p}}}\rco^{\tilde{\con{T}}^{\otimes}_{\crt{V}{p}}}\,.
\end{equation}
Utilizando procedimento semelhante, obt�m-se as igualdades:
\begin{eqnarray}
\lco\varsigma_{\hat{\con{T}}^{\otimes}_{\crt{V}{p}}}\rco^{\hat{\con{T}}^{\otimes}_{\crt{V}{p}}}&=&
\lco\lpa\vtf{i}_{\crt{V}{p}}\rpa_{\tilde{\con{T}}^{\otimes}_{\crt{V}{p}}}\rco^{\hat{\con{T}}^{\otimes}_{\crt{V}{p}}}\,;\\
\lco{\varsigma^{-1}}_{\tilde{\con{T}}^{\otimes}_{\crt{V}{p}}}\rco^{\hat{\con{T}}^{\otimes}_{\crt{V}{p}}}&=&
\lco\lpa\vtf{i}_{\crt{V}{p}}\rpa_{\hat{\con{T}}^{\otimes}_{\crt{V}{p}}}\rco^{\tilde{\con{T}}^{\otimes}_{\crt{V}{p}}}\,;\\
\lco{\varsigma^{-1}}_{\tilde{\con{T}}^{\otimes}_{\crt{V}{p}}}\rco^{\tilde{\con{T}}^{\otimes}_{\crt{V}{p}}}&=&
\lco\lpa\vtf{i}_{\crt{V}{p}}\rpa_{\hat{\con{T}}^{\otimes}_{\crt{V}{p}}}\rco^{\hat{\con{T}}^{\otimes}_{\crt{V}{p}}}\,.
\end{eqnarray}

Dado um operador tensorial qualquer
$\psi\in\cfl{\cft{\crt{V}{p}}{\con{F}}}{\cft{\crt{V}{p}}{\con{F}}}$,
a partir dos arrays associados � $\varsigma$ e a fun��es
tensoriais compostas, pode-se deduzir facilmente as seguintes
convers�es:
\begin{equation}\label{eq:conversoesTensores}
\begin{array}{rcl}
\mas{\psi}{\tilde{\con{T}}^{\otimes}_{\crt{V}{p}}}{\hat{\con{T}}^{\otimes}_{\crt{V}{p}}}&=&
\mas{\lpa\vtf{i}_{\crt{V}{p}}\rpa}{\tilde{\con{T}}^{\otimes}_{\crt{V}{p}}}{\hat{\con{T}}^{\otimes}_{\crt{V}{p}}}
\mas{\psi}{\hat{\con{T}}^{\otimes}_{\crt{V}{p}}}{\tilde{\con{T}}^{\otimes}_{\crt{V}{p}}}
\mas{\lpa\vtf{i}_{\crt{V}{p}}\rpa}{\tilde{\con{T}}^{\otimes}_{\crt{V}{p}}}{\hat{\con{T}}^{\otimes}_{\crt{V}{p}}}\\
\mas{\psi}{\tilde{\con{T}}^{\otimes}_{\crt{V}{p}}}{\tilde{\con{T}}^{\otimes}_{\crt{V}{p}}}&=&
\mas{\lpa\vtf{i}_{\crt{V}{p}}\rpa}{\tilde{\con{T}}^{\otimes}_{\crt{V}{p}}}{\tilde{\con{T}}^{\otimes}_{\crt{V}{p}}}
\mas{\psi}{\hat{\con{T}}^{\otimes}_{\crt{V}{p}}}{\tilde{\con{T}}^{\otimes}_{\crt{V}{p}}}
\mas{\lpa\vtf{i}_{\crt{V}{p}}\rpa}{\tilde{\con{T}}^{\otimes}_{\crt{V}{p}}}{\hat{\con{T}}^{\otimes}_{\crt{V}{p}}}\\
\mas{\psi}{\hat{\con{T}}^{\otimes}_{\crt{V}{p}}}{\tilde{\con{T}}^{\otimes}_{\crt{V}{p}}}&=&
\mas{\lpa\vtf{i}_{\crt{V}{p}}\rpa}{\hat{\con{T}}^{\otimes}_{\crt{V}{p}}}{\tilde{\con{T}}^{\otimes}_{\crt{V}{p}}}
\mas{\psi}{\tilde{\con{T}}^{\otimes}_{\crt{V}{p}}}{\hat{\con{T}}^{\otimes}_{\crt{V}{p}}}
\mas{\lpa\vtf{i}_{\crt{V}{p}}\rpa}{\hat{\con{T}}^{\otimes}_{\crt{V}{p}}}{\tilde{\con{T}}^{\otimes}_{\crt{V}{p}}}\\
\mas{\psi}{\hat{\con{T}}^{\otimes}_{\crt{V}{p}}}{\hat{\con{T}}^{\otimes}_{\crt{V}{p}}}&=&
\mas{\lpa\vtf{i}_{\crt{V}{p}}\rpa}{\hat{\con{T}}^{\otimes}_{\crt{V}{p}}}{\hat{\con{T}}^{\otimes}_{\crt{V}{p}}}
\mas{\psi}{\tilde{\con{T}}^{\otimes}_{\crt{V}{p}}}{\hat{\con{T}}^{\otimes}_{\crt{V}{p}}}
\mas{\lpa\vtf{i}_{\crt{V}{p}}\rpa}{\hat{\con{T}}^{\otimes}_{\crt{V}{p}}}{\tilde{\con{T}}^{\otimes}_{\crt{V}{p}}}
\end{array}
\end{equation}
Seguindo procedimento semelhante, as demais convers�es entre os
arrays associados a $\psi$ podem ser constru�das alternando-se
convenientemente as bases dos termos � direita da igualdade.



\subsection{Fun��o Tensorial Transposta}\index{fun��o!tensorial!transposta}\label{sec:funcaoTensorialTransposta}
Sejam os espa�os tensoriais $\ete{\crt{V}{p}}{\con{F}}$, $\ete{\crt{W}{q}}{\con{F}}$ e os
tensores quaisquer $\tnr{X}\in\cft{\crt{V}{p}}{\con{F}}$,
$\tnr{Y}\in\cft{\crt{W}{q}}{\con{F}}$. Dados os espa�os vetoriais de fun��es tensoriais
\begin{eqnarray}
\evl{\cft{\crt{V}{p}}{\con{F}}}{\cft{\crt{W}{q}}{\con{F}}}{\con{F}}&\mathrm{e}&
\evl{\cft{\crt{W}{q}}{\con{F}}}{\cft{\crt{V}{p}}{\con{F}}}{\con{F}},\nonumber
\end{eqnarray}
denomina-se fun��o tensorial transposta de
\begin{equation}
{\psi}\in\cfl{\cft{\crt{V}{p}}{\con{F}}}{\cft{\crt{W}{q}}{\con{F}}}\nonumber
\end{equation}
� �nica fun��o tensorial
\begin{equation}
{\psi}^T\in\cfl{\cft{\crt{W}{q}}{\con{F}}}{\cft{\crt{V}{p}}{\con{F}}}\nonumber
\end{equation}
que promove
\begin{equation}\label{eq:funcaoTensorialTransposta}
\fua{\psi}{\tnr{X}}\odot\tnr{Y}=\tnr{X}\odot\fua{\psi^T}{\tnr{Y}}\,.
\end{equation}
Se $\crt{V}{p}=\crt{W}{q}$ e $\con{T}^{\otimes}_{\crt{V}{p}}$ for uma base de $\ete{\crt{V}{p}}{\con{F}}$, pode-se demonstrar facilmente que
\begin{equation}\label{eq:arrayfunTensorialBaseSimetrica}
{ \mas{\psi}{\con{T}^{\otimes}_{\crt{V}{p}}}{\con{T}^{\otimes}_{\crt{V}{p}}}}^T=
\mas{\psi}{\con{T}^{\otimes}_{\crt{V}{p}}}{\con{T}^{\otimes}_{\crt{V}{p}}}\,.
\end{equation}

Caso os espa�os tensoriais considerados forem de Euclidianos, dadas duas bases $\hat{\con{T}}^{\otimes}_{\crt{V}{\times p}}$ e
$\tilde{\con{T}}^{\otimes}_{\crt{V}{\times q}}$ de $\ete{\crt{V}{q}}{\con{F}}$ e
$\ete{\crt{W}{q}}{\con{F}}$ respectivamente, pode-se demonstrar por processo similar � obten��o do resultado (\ref{eq:matrizfunTransposta}) que
\begin{equation}\label{eq:arrayfunTensorialTransposta}
{ \mas{\psi}{\tilde{\con{T}}^{\otimes}_{\crt{W}{q}}}{\hat{\con{T}}^{\otimes}_{\crt{V}{p}}}}^T=
\mas{\psi^T}{\hat{\con{T}}^{\otimes}_{\crt{V}{p}}}{\tilde{\con{T}}^{\otimes}_{\crt{W}{q}}}\,.
\end{equation}


\subsubsection{Operador Tensorial Sim�trico}\index{operador!tensorial!sim�trico}
Considerando as condi��es anteriores, tendo
$\crt{W}{q}=\crt{V}{p}$, um operador tensorial $\psi$ � sim�trico
se $\psi=\psi^T$. Ele �
\emph{anti-sim�trico}\index{operador!tensorial!anti-sim�trico} se
$\psi=-\psi^T$.

\subsubsection{Operador Tensorial Ortogonal}\index{operador!tensorial!ortogonal}
Para $\crt{W}{q}=\crt{V}{p}$, a fun��o $\psi$ � um operador
tensorial ortogonal se
\begin{equation}
\psi^{-1}=\psi^T\,.
\end{equation}
A partir de (\ref{eq:funcaoTensorialTransposta}), similarmente ao
desenvolvimento (\ref{eq:preservaProdInterno}), pode-se demonstrar
que fun��es tensoriais ortogonais preservam a opera��o produto
interno $\odot$.

Com base na defini��o de grupo ortogonal apresentada na se��o
\ref{sec:Transposicao}, pode-se dizer que o subconjunto
$\con{O}_{\cft{\crt{V}{p}}{\con{F}}}$ de
$\cfl{\cft{\crt{V}{p}}{\con{F}}}{\cft{\crt{V}{p}}{\con{F}}}$,
formado somente por operadores tensoriais ortogonais, define
$\gro{\cft{\crt{V}{p}}{\con{F}}}$. Os subconjuntos
$\con{O}^+_{\cft{\crt{V}{p}}{\con{F}}}$ e
$\con{O}^-_{\cft{\crt{V}{p}}{\con{F}}}$ de
$\con{O}_{\cft{\crt{V}{p}}{\con{F}}}$, compostos por operadores
ortogonais pr�prios e impr�prios respectivamente, geram os grupos
ortogonais $\grr{\cft{\crt{V}{p}}{\con{F}}}$ e
$\gri{\cft{\crt{V}{p}}{\con{F}}}\,$.

\section{A Rela��o Tensor e Fun��o Tensorial Linear}

Conforme o teorema apresentado a seguir, pode-se estabelecer uma
rela��o bijetora entre um tensor e uma fun��o tensorial linear,
generalizando a rela��o descrita no teorema
\ref{teo:RepresentacaoRiesz}.
\begin{teo}[Riesz Generalizado]\label{teo:RieszGeneralizado}
Sejam os espa�os tensoriais $\ete{\crt{V}{p}}{\con{F}}$,
$\ete{\con{U}_e}{\con{F}}$, $\ete{\con{W}_e}{\con{F}}$,
$\ete{\con{U}_d}{\con{F}}$ e $\ete{\con{W}_d}{\con{F}}$, tais que
$\crt{V}{p}=\con{U}_e\times\con{W}_e=\con{W}_d\times\con{U}_d$.
Desta forma, dado um escalar $q\,$, $1\leqslant q \leqslant p\,$, sejam
os produtos cartesianos
\begin{eqnarray}
\con{U}_e:=\crt{V}{q}\,, &&
\con{W}_e:=\con{V}_{q+1}\times\cdots\times\con{V}_{p}\,,
\nonumber\\
\con{U}_d:=\con{V}_{p-q+1}\times\cdots\times\con{V}_{p}\,, &&
\con{W}_d:=\crt{V}{p-q}\,.\nonumber
\end{eqnarray}
Sejam os espa�os vetoriais
\begin{eqnarray}
\evl{\cft{\con{U}_e}{\con{F}}}{\cft{\con{W}_e}{\con{F}}}{\con{F}}&e&
\evl{\cft{\con{U}_d}{\con{F}}}{\cft{\con{W}_d}{\con{F}}}{\con{F}}.\nonumber
\end{eqnarray}
Dado um tensor qualquer $\tnr{T}\in\cft{\crt{V}{p}}{\con{F}}$,
sejam as fun��es tensoriais lineares
\begin{eqnarray}
{\stackrel{\rightarrow}{\psi}}_{\tnr{T}}\in\cfl{\cft{\con{U}_e}{\con{F}}}{\cft{\con{W}_e}{\con{F}}}&e&
\stackrel{\leftarrow}{\psi}_{\tnr{T}}\in\cfl{\cft{\con{U}_d}{\con{F}}}{\cft{\con{W}_d}{\con{F}}}\nonumber
\end{eqnarray}
definidas pelas regras
\begin{eqnarray}
\fua{\stackrel{\rightarrow}{\psi}_{\tnr{T}}}{\tnr{X}}=\tnr{X}\odot_q\tnr{T}&e&\fua{\stackrel{\leftarrow}{\psi}_{\tnr{T}}}{\tnr{Y}}=\tnr{T}\odot_q\tnr{Y}.\nonumber
\end{eqnarray}
Os mapeamentos
\begin{eqnarray}
\map{\stackrel{\rightarrow}{\Psi}}{\cft{\crt{V}{p}}{\con{F}}}{\cfl{\cft{\con{U}_e}{\con{F}}}{\cft{\con{W}_e}{\con{F}}}}&e&
\map{\stackrel{\leftarrow}{\Psi}}{\cft{\crt{V}{p}}{\con{F}}}{\cfl{\cft{\con{U}_d}{\con{F}}}{\cft{\con{W}_d}{\con{F}}}}\nonumber
\end{eqnarray}
s�o transforma��es lineares bijetoras se, respectivamente,
\begin{eqnarray}
\fua{\stackrel{\rightarrow}{\Psi}}{\tnr{T}}=\stackrel{\rightarrow}{\psi}_{\tnr{T}}&e&\fua{\stackrel{\leftarrow}{\Psi}}{\tnr{T}}=\stackrel{\leftarrow}{\psi}_{\tnr{T}}.\nonumber
\end{eqnarray}
\end{teo}
\begin{prova}
A demonstra��o deste teorema se processa de maneira similar � do
teorema \ref{teo:RepresentacaoRiesz}, ou seja, dados
$\tnr{T}_1,\tnr{T}_2\in\cft{\crt{V}{p}}{\con{F}}$ e
$\alpha,\beta\in\con{F}$ quaisquer, constata-se que
\begin{itemize}
\item[i.] $\stackrel{\rightarrow}{\Psi}$ � uma fun��o linear:
\begin{eqnarray}
\fua{\stackrel{\rightarrow}{\psi}_{\alpha\tnr{T}_1+\beta\tnr{T}_2}}{\tnr{X}}&=
&\tnr{X}\odot_q\lpa\alpha\tnr{T}_1+\beta\tnr{T}_2\rpa\nonumber\\
&=&\alpha\lpa\tnr{X}\odot_q\tnr{T}_1\rpa+\beta\lpa\tnr{X}\odot_q\tnr{T}_2\rpa\nonumber\\
&=&\alpha\fua{\stackrel{\rightarrow}{\psi}_{\tnr{T}_1}}{\tnr{X}}+\beta\fua{\stackrel{\rightarrow}{\psi}_{\tnr{T}_2}}{\tnr{X}}\nonumber\\
&=&\fua{\lco\alpha\stackrel{\rightarrow}{\psi}_{\tnr{T}_1}
+\beta\stackrel{\rightarrow}{\psi}_{\tnr{T}_2}\rco}{\tnr{X}}\nonumber\\
\stackrel{\rightarrow}{\psi}_{\alpha\tnr{T}_1+\beta\tnr{T}_2}&=&\alpha\stackrel{\rightarrow}{\psi}_{\tnr{T}_1}
+\beta\stackrel{\rightarrow}{\psi}_{\tnr{T}_2}\nonumber\\
\fua{\stackrel{\rightarrow}{\Psi}}{\alpha\tnr{T}_1+\beta\tnr{T}_2}&=&\alpha\fua{\stackrel{\rightarrow}{\Psi}}{\tnr{T}_1}+\beta\fua{\stackrel{\rightarrow}{\Psi}}{\tnr{T}_2}\,.\nonumber
\end{eqnarray}
\item[ii.] $\stackrel{\rightarrow}{\Psi}$ define um mapeamento
injetor:
\begin{eqnarray}
\lco\fua{\stackrel{\rightarrow}{\Psi}}{\tnr{T}_1}\rco\lpa\tnr{X}\rpa&=&\lco\fua{\stackrel{\rightarrow}{\Psi}}{\tnr{T}_2}\rco\lpa\tnr{X}\rpa\nonumber\\
\tnr{X}\odot_q\tnr{T}_1&=&\tnr{X}\odot_q\tnr{T}_2\nonumber\\
\tnr{X}\odot_q\lpa\tnr{T}_1-\tnr{T}_2\rpa&=&\negmath{0}\nonumber\\
\tnr{T}_1&=&\tnr{T}_2\,.\nonumber
\end{eqnarray}
\item[iii.] $\stackrel{\rightarrow}{\Psi}$ define um mapeamento
sobrejetor: dada uma fun��o
$\stackrel{\rightarrow}{\psi}\in\cfl{\cft{\con{U}_e}{\con{F}}}{\cft{\con{W}_e}{\con{F}}}$
qualquer, uma base $\con{T}^{\otimes}_{\crt{V}{q}}$ formada pelos
conjuntos $\{ {\vto{v}_1}^{(i)} ,\cdots, {\vto{v}_{n_i}}^{(i)}\}$
e um tensor
\begin{equation}
\tnr{Z}=\sum_{i_1=1}^{n_1}\cdots\sum_{i_q=1}^{n_q}{\vto{v}_{i_1}}^{(1)}\otimes\cdots\otimes{\vto{v}_{i_q}}^{(q)}
\otimes\fua{\stackrel{\rightarrow}{\psi}}{{\vto{v}_{i_1}^*}^{(1)}\otimes\cdots\otimes{\vto{v}_{i_q}^*}^{(q)}}\,,
\nonumber
\end{equation}
pode-se dizer que
\begin{eqnarray}
\fua{\stackrel{\rightarrow}{\psi}_{\tnr{Z}}}{\tnr{X}}&=&\tnr{X}\odot_q\tnr{Z}\nonumber\\
&=&\lpa\sum_{i_1=1}^{n_1}\cdots\sum_{i_p=1}^{n_q}\fua{\vtf{f}_{i_1\cdots
i_p}^{\con{T}^{\otimes}_{\crt{V}{q}}}}{\tnr{X}}{\vto{v}_{i_1}^*}^{(1)}\otimes\cdots\otimes{\vto{v}_{i_q}^*}^{(q)}\rpa
\odot_q\nonumber\\
&&\lpa\sum_{j_1=1}^{n_1}\cdots\sum_{j_q=1}^{n_q}{\vto{v}_{j_1}}^{(1)}\otimes\cdots\otimes{\vto{v}_{j_q}}^{(q)}
\otimes\fua{\stackrel{\rightarrow}{\psi}}{{\vto{v}_{j_1}^*}^{(1)}\otimes\cdots\otimes{\vto{v}_{j_q}^*}^{(q)}}\rpa\nonumber\\
&=&\sum_{i_1=1}^{n_1}\cdots\sum_{i_p=1}^{n_q}\sum_{j_1=1}^{n_1}\cdots\sum_{j_q=1}^{n_q}\fua{\vtf{f}_{i_1\cdots
i_p}^{\con{T}^{\otimes}_{\crt{V}{q}}}}{\tnr{X}}\delta_{i_1j_1}\cdots\delta_{i_qj_q}
\fua{\stackrel{\rightarrow}{\psi}}{{\vto{v}_{j_1}^*}^{(1)}\otimes\cdots\otimes{\vto{v}_{j_q}^*}^{(q)}}\nonumber\\
&=&\fua{\stackrel{\rightarrow}{\psi}}{\sum_{j_1=1}^{n_1}\cdots\sum_{j_q=1}^{n_q}\fua{\vtf{f}_{j_1\cdots
j_p}^{\con{T}^{\otimes}_{\crt{V}{q}}}}{\tnr{X}}
{\vto{v}_{j_1}^*}^{(1)}\otimes\cdots\otimes{\vto{v}_{j_q}^*}^{(q)}}\nonumber\\
&=&\fua{\stackrel{\rightarrow}{\psi}}{\tnr{X}}\nonumber\\
\fua{\stackrel{\rightarrow}{\Psi}}{\tnr{Z}}&=&\stackrel{\rightarrow}{\psi}\nonumber\,.
\end{eqnarray}
\end{itemize}
A demonstra��o para a fun��o $\stackrel{\leftarrow}{\Psi}$ segue
esta mesma metodologia.
\end{prova}


\begin{prp}\label{prp:propsRieszGeneralizado}
Considerando as condi��es do teorema \ref{teo:RieszGeneralizado},
\begin{itemize}
\item[i.] se $q=p$, ent�o $\stackrel{\rightarrow}{\Psi} \,=\,
\stackrel{\leftarrow}{\Psi}$ ;

\item[ii.] se $2q=p$, os tensores $\tnr{X}$, $\tnr{Y}$,
$\fua{\stackrel{\rightarrow}{\psi}_{\tnr{T}}}{\tnr{X}}$ e
$\fua{\stackrel{\leftarrow}{\psi}_{\tnr{T}}}{\tnr{Y}}$ possuem
ordens iguais;

\item[iii.] se $2q=p$ e $\con{V}_1=\cdots=\con{V}_p$, as fun��es
$\stackrel{\rightarrow}{\psi}_{\tnr{T}}$ e
$\stackrel{\leftarrow}{\psi}_{\tnr{T}}$ s�o operadores lineares;

\item[iv.] Dada uma tupla qualquer
$(\vto{v}_1,\cdots,\vto{v}_p)\in\crt{V}{p}$, � sempre v�lido que
\begin{equation}
\fua{\tnr{T}}{\vto{v}_1,\cdots,\vto{v}_p}=\fua{\stackrel{\rightarrow}{\psi}_{\tnr{T}}}{\vto{v}_1\otimes\cdots\otimes\vto{v}_q}
\odot\lpa\vto{v}_{q+1}\otimes\cdots\otimes\vto{v}_p\rpa\nonumber
\end{equation}
e
\begin{equation}
\fua{\tnr{T}}{\vto{v}_1,\cdots,\vto{v}_p}=\lpa\vto{v}_1\otimes\cdots\otimes\vto{v}_{p-q}\rpa\odot
\fua{\stackrel{\leftarrow}{\psi}_{\tnr{T}}}{\vto{v}_{p-q+1}\otimes\cdots\otimes\vto{v}_p}\,.\nonumber
\end{equation}
\end{itemize}
\end{prp}
\begin{prova}
\begin{itemize}
\item[i.] Se $q=p$, ent�o
$\fua{\stackrel{\rightarrow}{\psi}_{\tnr{T}}}{\tnr{X}}=\fua{\stackrel{\leftarrow}{\psi}_{\tnr{T}}}{\tnr{X}}$,
$\forall\, \tnr{X}$;

\item[ii.] Facilmente comprovada pelas condi��es do teorema
\ref{teo:RieszGeneralizado}.


\item[iii.] Facilmente comprovada pelas condi��es do teorema
\ref{teo:RieszGeneralizado}.


\item[iv.] Considerando
\begin{equation}
\tnr{T}=\sum_{i=1}^m\vto{u}_{1i}\otimes\cdots\otimes\vto{u}_{pi}\,,\,
m\geq 1\nonumber
\end{equation}
e a igualdade (\ref{eq:produtorioPoliadico}), fica evidente que
\begin{equation}
\fua{\tnr{T}}{\vto{v}_1,\cdots,\vto{v}_p}=\tnr{T}\odot
\vto{v}_1\otimes\cdots\otimes\vto{v}_p\,.\nonumber
\end{equation}
Pode-se fazer, ent�o, o seguinte desenvolvimento:
\begin{eqnarray}
\fua{\tnr{T}}{\vto{v}_1,\cdots,\vto{v}_p}&=&\fua{\lco
\sum_{i=1}^m\vto{u}_{1i}\otimes\cdots\otimes\vto{u}_{pi}
\rco}{\vto{v}_1,\cdots,\vto{v}_p}\nonumber\\
&=&\fua{\lco\sum_{i=1}^m
\lpa\vto{u}_{1i}\cdot\vto{v}_{1}\rpa\cdots\lpa\vto{u}_{qi}\cdot\vto{v}_{q}\rpa
 \vto{u}_{q+1i}\otimes\cdots\otimes\vto{u}_{pi}
\rco}{\vto{v}_{q+1},\cdots,\vto{v}_p}\nonumber\\
&=&\fua{\lco\lpa\vto{v}_{1}\otimes\cdots\otimes\vto{v}_{q}\rpa\odot_q\tnr{T}\rco}{\vto{v}_{q+1},\cdots,\vto{v}_p}\nonumber\\
&=&\fua{\stackrel{\rightarrow}{\psi}_{\tnr{T}}}{\vto{v}_1\otimes\cdots\otimes\vto{v}_q}\odot\lpa\vto{v}_{q+1}\otimes\cdots\otimes\vto{v}_p\rpa\,.\nonumber
\end{eqnarray}
Utiliza-se o procedimento semelhante para a igualdade com
$\stackrel{\leftarrow}{\psi}_{\tnr{T}}$.
\end{itemize}
\end{prova}

\begin{prp}\label{prp:funcaoLinearETensores}
Sejam os espa�os tensoriais $\ete{\crt{V}{p}}{\con{F}}$, $\ete{\crt{W}{q}}{\con{F}}$ e
\begin{equation}
\evl{\cft{\crt{V}{p}}{\con{F}}}{\cft{\crt{W}{q}}{\con{F}}}{\con{F}}\,.\nonumber
\end{equation}
Dados os espa�os tensoriais $\ete{\crt{W}{q}\times\crt{V}{p}}{\con{F}}$ e $\ete{\crt{V}{p}\times\crt{W}{q}}{\con{F}}$, uma fun��o qualquer
\begin{equation}
\psi\in\cfl{\cft{\crt{V}{p}}{\con{F}}}{\cft{\crt{W}{q}}{\con{F}}}\,,\nonumber
\end{equation}
nos termos do teorema \ref{teo:RieszGeneralizado}, est� sempre associada a um e somente um tensor de $\cft{\crt{W}{q}\times\crt{V}{p}}{\con{F}}$ e de
$\cft{\crt{V}{p}\times\crt{W}{q}}{\con{F}}$ respectivamente.
\end{prp}
\begin{prova}
Com base no teorema \ref{teo:RieszGeneralizado}, h� um �nico tensor
$\tnr{T}_1\in\cft{\crt{W}{q}\times\crt{V}{p}}{\con{F}}$ associado � fun��o $\psi$, devido
a igualdade $\fua{\stackrel{\leftarrow}{\Psi}}{\tnr{T}_1}=\psi$, onde
$\stackrel{\leftarrow}{\Psi}$ � uma bije��o. Da mesma forma, segundo este mesmo teorema,
existe um �nico $\tnr{T}_2\in\cft{\crt{V}{p}\times\crt{W}{q}}{\con{F}}$ associado a
$\psi$, via $\fua{\stackrel{\rightarrow}{\Psi}}{\tnr{T}_2}=\psi$.
\end{prova}

\subsection{Fun��es Representantes}\index{fun��o!representante} A rela��o entre fun��es
lineares e vetores descrita no teorema \ref{teo:RepresentacaoRiesz} permite substituir a
nota��o da fun��o $\vartheta_{\vto{v}}$ para simplesmente $\vto{v}$, de onde se diz que
um vetor pode tamb�m ser considerado um funcional linear, conforme
(\ref{eq:VetorFuncional}).

No caso de tensores, segundo o teorema
\ref{teo:RieszGeneralizado}, as fun��es
$\stackrel{\rightarrow}{\psi}_{\tnr{T}}$ e
$\stackrel{\leftarrow}{\psi}_{\tnr{T}}$ t�m, cada uma, rela��o
bijetora com o tensor $\tnr{T}$. Com base nisso, as fun��es
\begin{eqnarray}\label{eq:funcaoTensorial}
\rft{T}{q}\,:=\,\stackrel{\leftarrow}{\psi}_{\tnr{T}}&\mathrm{e}&
\lft{T}{q}\,:=\,\stackrel{\rightarrow}{\psi}_{\tnr{T}}
\end{eqnarray}
s�o chamadas fun��es representantes do tensor $\tnr{T}$ com
contra��o $q$ � direita e � esquerda, respectivamente. Se o tensor
$\tnr{T}$ tiver ordem $p$, $\rft{T}{q}$ e $\lft{T}{q}$ s�o fun��es
tensoriais lineares associadas a $\tnr{T}$, segundo
(\ref{eq:funcaoTensorial}), cujas imagens s�o constitu�das por
tensores de ordem $p-q$. Pode-se concluir, ent�o, que
$\vartheta_{\vto{v}}$ � uma fun��o representante de $\vto{v}$ com
contra��o 1 � direita ou � esquerda.

O teorema \ref{teo:RieszGeneralizado} � de fundamental
import�ncia, j� que ele torna poss�vel apresentar o estudo de
tensores como o estudo de uma de suas fun��es representantes. A
livre escolha de uma delas, no entanto, determina qual das duas
igualdades do �tem iv da proposi��o
\ref{prp:propsRieszGeneralizado} � considerada\rodape{Grande parte
dos trabalhos em Mec�nica do Cont�nuo que tratam de Teoria de
Tensores n�o mostra, de forma expl�cita, qual das fun��es
representantes � utilizada para substituir o conceito de tensor.
No caso de tensores de segunda ordem, por exemplo, a cl�ssica
apresenta��o do produto tensorial $\vto{v}\otimes\vto{w}$ como
sendo uma fun��o que mapeia o vetor $\vto{x}$ para o vetor
$(\vto{w}\cdot\vto{x})\vto{v}$ revela, implicitamente, a op��o
pela fun��o representante com contra��o 1 � direita.}.

De agora em diante, toda vez que conceitos baseados em fun��es
representantes forem independentes da abordagem de representa��o
previamente escolhida (com contra��o � direita ou � esquerda),
ser� utilizada a nota��o sem seta $\ftr{T}{q}\,$.

\subsubsection{Fun��o Representante Identidade}\index{fun��o!representante!identidade}
Seja o espa�o tensorial $\ete{\crt{V}{p}\times\crt{V}{p}}{\con{F}}$. A partir da
igualdade (\ref{eq:tensorIdentidade}) que define o tensor identidade de
$\cft{\crt{V}{p}\times\crt{V}{p}}{\con{F}}$, para quaisquer
$\vto{x}_i,\vto{y}_i\in\con{V}_i$, pode-se realizar o seguinte desenvolvimento:
\begin{eqnarray}
\fua{\tnr{I}}{\vto{x}_1,\cdots,\vto{x}_p,\vto{y}_1,\cdots,\vto{y}_p}&=&
\lpa\vto{x}_1\otimes\cdots\otimes\vto{x}_p\rpa\odot
\lpa\vto{y}_1\otimes\cdots\otimes\vto{y}_p\rpa\nonumber\\
&=& \lpa\vto{x}_1\otimes\cdots\otimes\vto{x}_p\rpa\odot\lco
\fua{\vtf{i}_{\crt{V}{p}}}{\vto{y}_1\otimes\cdots\otimes\vto{y}_p}\rco\nonumber\\
&=&
\lco\fua{\vtf{i}_{\crt{V}{p}}}{\vto{x}_1\otimes\cdots\otimes\vto{x}_p}\rco\odot
\lpa\vto{y}_1\otimes\cdots\otimes\vto{y}_p\rpa\nonumber\,.
\end{eqnarray}
Com base no �tem iv da proposi��o
\ref{prp:propsRieszGeneralizado}, pode-se concluir ent�o que a
fun��o representante identidade
\begin{equation}\label{eq:funRepresentanteIdentidade}
\begin{array}{ccccc}
\rft{I}{p}&=&\lft{I}{p}&=&\vtf{i}_{\crt{V}{p}}\,\,.
\end{array}
\end{equation}


\subsubsection{Fun��es Representantes e Tensor Identidade}
Sejam os espa�os tensoriais $\ete{\crt{V}{p}\times\crt{V}{p}}{\con{F}}$,
$\ete{\crt{V}{p}\times\crt{V}{p}\times\crt{W}{q}}{\con{F}}$ e
$\ete{\crt{W}{q}\times\crt{V}{p}\times\crt{V}{p}}{\con{F}}$. Considerando um base
gen�rica qualquer $\con{U}_i=\{ {\vto{u}_1}^{(i)} ,\cdots, {\vto{u}_{n_i}}^{(i)}\}$ de
$\ehr{V_i}{\con{F}}$, todo tensor identidade assume uma das formas
(\ref{eq:TensorIdentidadeExplicito1}) ou (\ref{eq:TensorIdentidadeExplicito2}). Neste
contexto, se $\tnr{I}\in\cft{\crt{V}{p}\times\crt{V}{p}}{\con{F}}$,
$\tnr{A}\in\cft{\crt{V}{p}\times\crt{V}{p}\times\crt{W}{q}}{\con{F}}$ e
$\tnr{B}\in\cft{\crt{W}{q}\times\crt{V}{p}\times\crt{V}{p}}{\con{F}}$, onde
\begin{equation}
\tnr{A}=\sum_{k=1}^m\vto{x}_{1k}\otimes\cdots\otimes\vto{x}_{pk}\otimes
\vto{y}_{1k}\otimes\cdots\otimes\vto{y}_{pk}\otimes\vto{w}_{1k}\otimes\cdots\otimes\vto{w}_{qk}
\end{equation}
e
\begin{equation}
\tnr{B}=\sum_{k=1}^m\vto{w}_{1k}\otimes\cdots\otimes\vto{w}_{qk}\otimes
\vto{x}_{1k}\otimes\cdots\otimes\vto{x}_{pk}\otimes\vto{y}_{1k}\otimes\cdots\otimes\vto{y}_{qk}\,,
\end{equation}
$m\geq 1$, ent�o
\begin{equation}\label{eq:funcaRepresentanteXIdentidade}
\fua{\lft{A}{2p}}{\tnr{I}}=\fua{\rft{B}{2p}}{\tnr{I}}=\sum_{k=1}^m\prod_{i=1}^{p}\vto{x}_{ik}\cdot\vto{y}_{ik}
\vto{w}_{1k}\otimes\cdots\otimes\vto{w}_{qk}\,.
\end{equation}

\subsubsection{Fun��o Representante Composta}\index{fun��o!representante!composta} Dado o
espa�o tensorial $\ete{\crt{V}{p}}{\con{F}}$ e os tensores quaisquer
$\tnr{A}$,$\tnr{B}\in\cft{\crt{V}{p}}{\con{F}}$, onde $p\geqslant s+q$, tal que $s$ e $q$
s�o escalares positivos n�o nulos. Das defini��es (\ref{eq:funcaoTensorial}), dado um
tensor qualquer $\tnr{X}\in\cft{\crt{V}{q}}{\con{F}}$, fica evidente que
\begin{equation}
\fua{\rft{A}{s}}{\fua{\rft{B}{q}}{\tnr{X}}}=\tnr{A}\odot_s\lpa\tnr{B}\odot_q\tnr{X}\rpa
\end{equation}
e
\begin{equation}
\fua{\lft{A}{s}}{\fua{\lft{B}{q}}{\tnr{X}}}=\lpa\tnr{X}\odot_q\tnr{B}\rpa\odot_s\tnr{A}\,.
\end{equation}
Nestas condi��es, o produto contrativo � associativo, logo
\begin{equation}\label{eq:represCompostaDireita}
\rft{A}{s}\circ\rft{B}{q}=\lpa\tnr{A}\odot_s\tnr{B}\rpa\rft{}{q}
\end{equation}
e
\begin{equation}\label{eq:represCompostaEsquerda}
\lft{A}{s}\circ\lft{B}{q}=\lpa\tnr{B}\odot_s\tnr{A}\rpa\lft{}{q}\,.
\end{equation}
Nos casos onde $q=s$, costuma-se utilizar as seguintes
representa��es:
\begin{eqnarray}
\rft{AB}{q}&:=&\lpa\tnr{A}\odot_q\tnr{B}\rpa\rft{}{q}\,\,,\label{eq:funRepCompostaDireita}\\
\lft{AB}{q}&:=&\lpa\tnr{B}\odot_q\tnr{A}\rpa\lft{}{q}\,,\label{eq:funRepCompostaEsquerda}
\end{eqnarray}
onde $\tnr{AB}$ � o tensor resultante dos produtos contrativos em
cada caso. Nota-se que a manipula��o de fun��es representantes compostas com
contra��o � direita � mais simples, j� que n�o � necess�rio
inverter a se\-q��n\-cia dos tensores envolvidos\rodape{Esta � a
principal raz�o pela qual a maioria dos textos introdut�rios optam
por fun��es representantes com contra��o � direita para substituir
o conceito de tensor.}.

Neste contexto, se for poss�vel $n-1$ composi��es de uma mesma fun��o representante $\tnr{A}_{\odot q}$, tem-se
\begin{equation}\label{eq:composicaoMultipla}
\tnr{A^n}_{\odot q}:=(\underbrace{\tnr{A}\odot_q\cdots\odot_q\tnr{A}}_{n\,\,\mathrm{vezes}})_{\odot q}\,.
\end{equation}

\subsubsection{Fun��o Representante Inversa}\index{fun��o!representante!inversa}
Seja o espa�o tensorial $\ete{\crt{V}{p}\times\crt{V}{p}}{\con{F}}$, onde $2p\geqslant
s+q$, tal que $s$ e $q$ s�o positivos n�o nulos. Dado um tensor qualquer
$\tnr{A}\in\cft{\crt{V}{p}\times\crt{V}{p}}{\con{F}}$, uma fun��o $\ftr{B}{q}$, onde
$\tnr{B}\in\cft{\crt{V}{p}\times\crt{V}{p}}{\con{F}}$, � dita inversa de $\ftr{A}{s}$ se
\begin{equation}
\ftr{A}{s}\circ\ftr{B}{q}=\ftr{I}{q}\,.
\end{equation}
Desta forma, com base nos conceitos de fun��o representante
composta, tem-se o seguinte:
\begin{itemize}
\item[i.] Para contra��o � direita,
$\tnr{A}\odot_s\tnr{B}=\tnr{I}$;

\item[ii.] Para contra��o � esquerda,
$\tnr{B}\odot_s\tnr{A}=\tnr{I}$.
\end{itemize}

Pode-se realizar a tradicional representa��o da fun��o inversa
$\ftr{B}{q}$ com $\lpa\ftr{A}{s}\rpa^{-1}$. No entanto, tal
nota��o s� faz sentido se for explicitado o valor de $q$. Quando
$q=s$, utiliza-se a fun��o
\begin{equation}\label{eq:representanteInversaSimples}
\fti{A}{q}:=\lpa\ftr{A}{q}\rpa^{-1}\,.
\end{equation}
Em termos de nota��o, tal defini��o n�o gera ambig�idade, j� que
inexiste o conceito de ``tensor inverso''.

\subsubsection{Fun��o Representante Transposta}\index{fun��o!representante!transposta} Sejam os quatro espa�os tensoriais
$\ete{\crt{V}{p}}{\con{F}}$, $\ete{\crt{W}{q}}{\con{F}}$, $\ete{\crt{U}{s}}{\con{F}}$ e
$\ete{\crt{Z}{s}}{\con{F}}$, onde $s=p+q$. Dados os tensores quaisquer
$\tnr{X}\in\cft{\crt{V}{p}}{\con{F}}$, $\tnr{Y}\in\cft{\crt{W}{q}}{\con{F}}$ e
$\tnr{A}\in\cft{\crt{U}{s}}{\con{F}}$, seja
\begin{equation}
\fua{\ftr{A}{p}}{\tnr{X}}\odot_q\tnr{Y}=\tnr{X}\odot_p\fua{\ftr{B}{q}}{\tnr{Y}}\,,
\end{equation}
onde $\tnr{B}\in\cft{\crt{Z}{s}}{\con{F}}$. A partir da
defini��o de fun��o tensorial transposta, $\ftr{B}{q}$ � denominada fun��o
representante transposta de $\ftr{A}{p}\,$. Nestas circunst�ncias,
pode-se afirmar o seguinte:
\begin{itemize}
\item[i.] Para contra��o � direita,
$\crt{U}{s}=\crt{W}{q}\times\crt{V}{p}$ e
$\crt{Z}{s}=\crt{V}{p}\times\crt{W}{q}$;

\item[ii.] Para contra��o � esquerda,
$\crt{U}{s}=\crt{V}{p}\times\crt{W}{q}$ e
$\crt{Z}{s}=\crt{W}{q}\times\crt{V}{p}$.
\end{itemize}
Nota-se que a ordem dos produtos cartesianos $\crt{V}{p}$ e
$\crt{W}{q}$ que definem $\crt{U}{s}$ e $\crt{Z}{s}$ est� sempre
invertida. Desta forma, se
\begin{equation}\label{eq:decompParaTransposta}
\tnr{A}=\sum_{k=1}^m\vto{w}_{1k}\otimes\cdots\otimes\vto{w}_{qk}\otimes
\vto{v}_{1k}\otimes\cdots\otimes\vto{v}_{pk}\,,
\end{equation}
onde $\vto{w}_{ik}\in\con{W}_i$ e $\vto{v}_{ik}\in\con{V}_i$,
ent�o obrigatoriamente
\begin{equation}\label{eq:decompTransposta}
\tnr{B}=\sum_{k=1}^m\vto{v}_{1k}\otimes\cdots\otimes\vto{v}_{pk}\otimes
\vto{w}_{1k}\otimes\cdots\otimes\vto{w}_{qk}\,.
\end{equation}
Se nas igualdades anteriores $p=q$, segundo a defini��o (\ref{eq:transposicaoSucessiva}), o tensor
\begin{equation}\label{eq:transpostaPermutacao}
\tnr{B}=\tnr{A}_{(1,1+p)(p,2p)}\,.
\end{equation}

Como ocorre no caso de fun��o tensorial inversa, a representa��o
tradicional $(\ftr{A}{p})^T$ para a transposta de $\ftr{A}{p}$
deve vir acompanhada com o valor de $q$. Numa situa��o onde
$\crt{W}{q}=\crt{V}{p}$, n�o h� este problema e pode-se ter uma
fun��o
\begin{equation}\label{eq:transpostaTensor}
\ftt{A}{p}:=\lpa\ftr{A}{p}\rpa^T\,.
\end{equation}
Neste caso, vale salientar que as igualdades
(\ref{eq:decompParaTransposta}), (\ref{eq:decompTransposta}) e (\ref{eq:transpostaPermutacao})
permanecem inalteradas, exceto pelo fato de que agora
$\vto{w}_{ik}\in\con{V}_i$. Particularmente, se
$\vto{w}_{ik}=\vto{v}_{ik}$, a fun��o tensorial $\ftr{A}{p}$ �
sim�trica.

\subsubsection{Tra�o de Fun��o Representante}\label{sec:tracoFuncaoRepresentante}\index{tra�o!de fun��o representante}
Sejam os espa�os tensoriais $\ete{\con{V}_{(p)}}{\con{F}}$  e
$\ete{\con{V}_{(p-2)}}{\con{F}}$ tais que $\con{V}_{(p)}=\crt{V}{p}$,
$\con{V}_r=\con{V}_s$ e
\begin{equation}
\con{V}_{(p-2)}:=\con{V}_1\times\cdots\times\con{V}_{r-1}\times\con{V}_{r+1}
\times\cdots\times\con{V}_{s-1}\times\con{V}_{s+1}\times\cdots\times\con{V}_p\,.
\end{equation}
Nestas condi��es, utilizando (\ref{eq:tracoTensor}), diz-se que a fun��o
\begin{equation}
\fua{\trt{r}{s}}{\ftr{T}{q_1}}:=\lpa\fua{\trt{r}{s}}{\tnr{T}}\rpa_{\odot_{q_2}}\,,
\end{equation}
onde $q_1\leqslant p$ e $q_2\leqslant p-2$, � denominada tra�o da fun��o representante $\ftr{T}{q_1}$ em $r,s$.
Com base nesta defini��o, considerando o caso particular de tensores de segunda ordem, cujo espa�o tensorial � definido por $\con{V}_{(2)}=\con{V}^{2}$, tem-se que o tra�o $\fua{\trt{1}{2}}{\ftr{T}{q_1}}$, onde $q_1\leqslant 2$, resulta uma fun��o escalar, pois
\begin{equation}
 \fua{\trt{1}{2}}{\tnr{T}}\in\con{F}.
\end{equation}
Ainda neste contexto, se $\con{V}$ definir um espa�o vetorial de Hilbert com dimens�o $n$, seja $\con{T}^{\otimes}_{\con{V}{2}}$ base de $\ete{\con{V}^2}{\con{F}}$, gerada pelos tensores poli�dicos $\vto{u}_{i_1}\otimes\vto{u}^*_{i_2}$, $i_j=1,\cdots,n\,$. Dado um tensor qualquer $\tnr{A}\in\cft{\con{V}^{2}}{\con{F}}$, pode-se realizar o seguinte desenvolvimento:
\begin{eqnarray}
\fua{\trt{1}{2}}{\ftr{A}{q_1}}&:=&\fua{\trt{1}{2}}{\sum_{i_1=1}^{n}\sum_{i_2=1}^{n}\fua{\vtf{f}_{i_1
i_2}^{\con{T}^{\otimes}_{\con{V}^{2}}}}{\tnr{A}}\vto{u}_{i_1}\otimes\vto{u}^*_{i_2}}\nonumber\\
&:=&\sum_{i_1=1}^{n}\sum_{i_2=1}^{n}\fua{\vtf{f}_{i_1
i_2}^{\con{T}^{\otimes}_{\con{V}^{2}}}}{\tnr{A}}\fua{\trt{1}{2}}{\vto{u}_{i_1}\otimes\vto{u}^*_{i_2}}\nonumber\\
&:=&\sum_{i_1=1}^{n}\sum_{i_2=1}^{n}\fua{\vtf{f}_{i_1
i_2}^{\con{T}^{\otimes}_{\con{V}^{2}}}}{\tnr{A}}\delta_{i_1i_2}\nonumber\\
&:=&\sum_{i_1=1}^{n}\fua{\vtf{f}_{i_1
i_1}^{\con{T}^{\otimes}_{\con{V}^{2}}}}{\tnr{A}}\nonumber\\
&:=&\trc{\lco\tnr{A}\rco^{\con{T}^{\otimes}_{\con{V}^{2}}}}\,.
\end{eqnarray}




\subsection{Hiperdeterminante de Operador Tensorial Linear}\index{hiperdeterminante!de operador tensorial linear}\label{sec:HiperdeterminanteOperador}
Seja o espa�o tensorial $\ete{\crt{V}{p}}{\con{F}}$, gerado por espa�os de Hilbert
$m$-dimensionais, e o espa�o vetorial de fun��es tensoriais lineares
$\evl{\cft{\crt{V}{p}}{\con{F}}}{\cft{\crt{V}{p}}{\con{F}}}{\con{F}}$. Dado o operador
tensorial $\psi\in\cfl{\cft{\crt{V}{p}}{\con{F}}}{\cft{\crt{V}{p}}{\con{F}}}$, sejam os
tensores anti-sim�tricos n�o nulos $\tnr{P}^\psi,\tnr{P}\in\cfa{(\crt{V}{p})^n}{F}$ e o
inteiro $q=p.n$, tais que
\begin{equation}\label{eq:determinanteUm}
\fua{\tnr{P}^\psi_{\odot_q}}{\tnr{X}_1\otimes\cdots\otimes\tnr{X}_n} :=
\fua{\tnr{P}_{\odot_q}}{
{\fua{\psi}{\tnr{X}_1}\otimes\cdots\otimes\fua{\psi}{\tnr{X}_n}}}\,,\,\forall\,
\tnr{X}_i\in\cft{\crt{V}{p}}{\con{F}}\,.
\end{equation}
Com base na proposi��o \ref{teo:dimensaoEspacoAlternante}, seja $\{\tnr{P}\}$ uma base
do subespa�o $\etn{(\crt{V}{p})^n}{F}$ de $\ete{(\crt{V}{p})^n}{F}$, tal que
\begin{equation}\label{eq:determinanteDois}
\tnr{P}^\psi = \alpha\tnr{P}\,,
\end{equation}
onde $\alpha\in\con{F}$. Nestas condi��es, diz-se que o escalar $\alpha$ � o
hiperdeterminante do operador tensorial linear $\psi$, representado $\hpd{\psi}$. Para
que tal defini��o seja plaus�vel, deve ser mostrado que o valor do escalar $\alpha$ em
(\ref{eq:determinanteDois}) � o mesmo para qualquer tensor do conjunto
$\cfa{(\crt{V}{p})^n}{F}$. Para tal, dado um outro tensor qualquer $\beta\tnr{P}$ deste
conjunto, considerando (\ref{eq:determinanteUm}) e (\ref{eq:determinanteDois}), pode-se
dizer que
\begin{equation}
\lpa\beta\tnr{P}\rpa^\psi = \beta\tnr{P}^\psi =
\beta\alpha\tnr{P}=\alpha\lpa\beta\tnr{P}\rpa\,,\,\forall\,\beta\in\con{F}\,;
\end{equation}
de onde se conclui que $\alpha$ depende somente de $\psi$. Nos termos do Teorema \ref{teo:RieszGeneralizado}, como a fun��o $\psi$ pode ser representante de um tensor $\tnr{S}\in\cft{(\crt{V}{p})^n}{\con{F}}$, ent�o o escalar
\begin{equation}
\hpd{\tnr{S}}:=\hpd{\psi}
\end{equation}
� dito o \emph{hiperdeterminante do tensor} $\tnr{S}$.\index{hiperdeterminante!de tensor}

Com base na defini��o (\ref{eq:determinanteUm}), � poss�vel escrever que
\begin{equation}\label{eq:determinanteTensor}
\fua{\tnr{T}_{\odot_q}}{
{\fua{\psi}{\tnr{X}_1}\otimes\cdots\otimes\fua{\psi}{\tnr{X}_n}}}=(\hpd{\psi})\,\fua{\tnr{T}_{\odot_q}}{\tnr{X}_1\otimes\cdots\otimes\tnr{X}_n}\,,
\end{equation}
para quaisquer $\tnr{X}_i\in\cft{\crt{V}{p}}{\con{F}}$ e
$\tnr{T}\in\cfa{(\crt{V}{p})^n}{F}\,$. A partir desta igualdade, dados os operadores
lineares $\psi_1,\psi_2\in\cfl{\cft{\crt{V}{p}}{\con{F}}}{\cft{\crt{V}{p}}{\con{F}}}$,
utilizando duas bases quaisquer $\hat{\con{T}}^{\otimes}_{\crt{V}{p}}$ e
$\tilde{\con{T}}^{\otimes}_{\crt{V}{p}}$ de $\ete{\crt{V}{p}}{\con{F}}$, tem-se as
seguintes propriedades:
\begin{itemize}
\item[i.] $\hpd{\psi_1\circ\psi_2}=\hpd{\psi_1}\hpd{\psi_2}$\,;
\item[ii.] $\hpd{\vtf{i}_{\crt{V}{p}}}=1$;
\item[iii.] $\psi_1$ autom�rfico $\implies \hpd{\psi_1^{-1}} = \lpa\hpd{\psi_1}\rpa^{-1}$\,;
\item[iv.] $\psi_1$ autom�rfico $\Longleftrightarrow\hpd{\psi_1}\neq 0$;
\item[v.]
$\hpd{\psi_1}=
\hpd{\lco{\psi_1}_{\hat{\con{T}}^{\otimes}_{\crt{V}{p}}}\rco^{\tilde{\con{T}}^{\otimes}_{\crt{V}{p}}}}\,$;
\item[vi.] $\ete{\crt{V}{q}}{\con{F}}$ Euclidiano $\implies\hpd{\psi_1^T} =\hpd{\psi_1}\,$;
\item[vii.] $\psi_1$ ortogonal e $\ete{\crt{V}{q}}{\con{F}}$ Euclidiano $\implies\hpd{\psi_1}=\pm1\,$.
\end{itemize}
Considerando o �ltimo �tem, caso seu hiperdeterminante seja positivo,
$\psi_1$ � chamado operador ortogonal \emph{pr�prio}\index{operador!tensorial!ortogonal pr�prio}, sen�o, ele
� \emph{impr�prio}\index{operador!tensorial!ortogonal impr�prio}.

\noindent\begin{prova}
Vamos demonstrar as propriedades citadas com os itens respectivos apresentados a seguir.
\begin{itemize}
\item[i.] Seja o seguinte desenvolvimento:
\begin{eqnarray*}
(\hpd{\psi_1\circ\psi_2})\,\fua{\tnr{T}_{\odot_q}}{\tnr{X}_1\otimes\cdots\otimes\tnr{X}_n}
&=&\fua{\tnr{T}_{\odot_q}}{
{\fua{\psi_1\circ\psi_2}{\tnr{X}_1}\otimes\cdots\otimes\fua{\psi_1\circ\psi_2}{\tnr{X}_n}}}\noindent\\
&=&(\hpd{\psi_1})\,\fua{\tnr{T}_{\odot_q}}{
{\fua{\psi_2}{\tnr{X}_1}\otimes\cdots\otimes\fua{\psi_2}{\tnr{X}_n}}}\noindent\\
&=&(\hpd{\psi_1})(\hpd{\psi_2})\,\fua{\tnr{T}_{\odot_q}}{
{\tnr{X}_1\otimes\cdots\otimes\tnr{X}_n}}\noindent
\end{eqnarray*}
\item[ii.]  De (\ref{eq:determinanteTensor}), a obten��o da igualdade deste �tem � imediata.

\item[iii.]  Utilizando o resultado dos dois itens anteriores, tem-se que
\begin{equation*}
(\hpd{\psi_1})(\hpd{\psi_1^{-1}}) = \hpd{\psi_1\circ\psi_1^{-1}}  = \hpd{\vtf{i}_{\crt{V}{p}}}= 1\,.
\end{equation*}
\item[iv.] A constata��o de que $\psi_1$ bije��o $\Rightarrow\hpd{\psi_1}\neq 0$ � conseq��ncia imediata da demonstra��o do item iii. Agora vejamos se $\hpd{\psi_1}\neq 0\Rightarrow\psi_1$ bije��o.
Vamos mostrar primeiramente que, dado um tensor qualquer
$\tnr{T}\in\cfa{(\crt{V}{p})^n}{F}$ e um tensor
$\tnr{X}:=\tnr{X}_1\otimes\cdots\otimes\tnr{X}_\alpha\otimes\cdots\otimes\tnr{X}_\beta\otimes\cdots\otimes\tnr{X}_n$,
onde $\tnr{X}_\alpha=\tnr{X}_\beta$,
\begin{equation*}
\fua{\tnr{T}_{\odot_q}}{\tnr{X}}=0\,.
\end{equation*}
Para tal, sabemos que $\tnr{T}=-\tnr{T}_{(\alpha,\beta)}$ e $\tnr{X}_{(\alpha,\beta)}=\tnr{X}_1\otimes\cdots\otimes\tnr{X}_\beta\otimes\cdots\otimes\tnr{X}_\alpha\otimes\cdots\otimes\tnr{X}_n$. Desta forma,
\begin{equation*}
\fua{\tnr{T}_{\odot_q}}{\tnr{X}}=-\fua{{\tnr{T}_{(\alpha,\beta)}}_{\odot_q}}{\tnr{X}}=-\fua{\tnr{T}_{\odot_q}}{\tnr{X}_{(\alpha,\beta)}}\,.
\end{equation*}
Mas como $\tnr{X}_\alpha=\tnr{X}_\beta$, ent�o
$\fua{\tnr{T}_{\odot_q}}{\tnr{X}}=-\fua{\tnr{T}_{\odot_q}}{\tnr{X}}$, c.q.d. Com base
neste resultado, vamos demonstrar agora que se o conjunto
$\con{Y}=\{\tnr{Y}_1,\cdots,\tnr{Y}_n\}$ for linearmente dependente, ent�o
\begin{equation*}
\fua{\tnr{T}_{\odot_q}}{\tnr{Y}_1\otimes\cdots\otimes\tnr{Y}_n}=0\,.
\end{equation*}
Para tal, vamos dizer que um tensor $\tnr{Y}_\alpha$ de $\con{Y}$ � combina��o linear dos demais elementos do conjunto. Pode-se realizar ent�o o seguinte desenvolvimento:
\begin{eqnarray*}
\fua{\tnr{T}_{\odot_q}}{\tnr{Y}_1\otimes\cdots\otimes\tnr{Y}_\alpha\otimes\cdots\otimes\tnr{Y}_n}&=&\nonumber\\
\fua{\tnr{T}_{\odot_q}}{\tnr{Y}_1\otimes\cdots\otimes\sum_{i=1}^n\gamma_i(1-\delta_{i\alpha})\tnr{Y}_i\otimes\cdots\otimes\tnr{Y}_n}&=&\nonumber\\
\sum_{i=1}^n\gamma_i(1-\delta_{i\alpha})\fua{\tnr{T}_{\odot_q}}{\tnr{Y}_1\otimes\cdots\otimes\tnr{Y}_i\otimes\cdots\otimes\tnr{Y}_n}&=&0\nonumber\,.
\end{eqnarray*}
A �ltima igualdade ocorre porque para cada valor do �ndice $i\neq\alpha$, o produto
tensorial, argumento de $\tnr{T}_{\odot_q}$, sempre possui dois tensores iguais, c.q.d.
Tomando este resultado e (\ref{eq:determinanteTensor}), se $\hpd{\psi_1}\neq 0$, ent�o
\begin{equation*}
\fua{\tnr{T}_{\odot_q}}{
{\fua{\psi_1}{\tnr{Z}_1}\otimes\cdots\otimes\fua{\psi_1}{\tnr{Z}_n}}}\neq 0
\end{equation*}
e
\begin{equation*}
\fua{\tnr{T}_{\odot_q}}{ {\tnr{Z}_1\otimes\cdots\otimes\tnr{Z}_n}}\neq 0\,.
\end{equation*}
Como conseq��ncia os conjuntos $\{\tnr{Z}_1,\cdots,\tnr{Z}_n\}$ e
$\{\fua{\psi_1}{\tnr{Z}_1},\cdots,\fua{\psi_1}{\tnr{Z}_n}\}$ s�o linearmente
independentes. Desta conclus�o, como $\tnr{Z}_i$ s�o tensores quaisquer, dada uma base de
tensores poli�dicos $\con{T}^{\otimes}_{\crt{V}{p}}$ de $\ete{\crt{V}{p}}{\con{F}}$,
constru�da a partir de $\con{U}_i=\{{\vto{v}_1}^{(i)} ,\cdots, {\vto{v}_{m}}^{(i)}\}$,
onde $n_i$ � a dimens�o de $\ehr{V_i}{F}$, pode-se afirmar que o conjunto
\begin{equation*}
\lch\fua{\psi_1}{{\vto{v}_{j_1}}^{(1)}\otimes\cdots\otimes{\vto{v}_{j_p}}^{(p)}},\cdots\rch\,,
j_i= 1,\cdots,m
\end{equation*}
� linearmente independente. Desta forma, considerando dois tensores distintos quaisquer
$\tnr{K}_1,\tnr{K}_2\in\cft{\crt{V}{p}}{\con{F}}$, pode-se dizer ent�o que os valores
\begin{equation*}
\fua{\psi_1}{\tnr{K}_1}=\sum_{i_1=1}^{m}\cdots\sum_{i_p=1}^{m}\fua{\vtf{f}_{i_1\cdots
i_p}^{\con{T}^{\otimes}_{\crt{V}{p}}}}{\tnr{K}_1}\fua{\psi_1}{{\vto{v}_{i_1}}^{(1)}\otimes
\cdots\otimes{\vto{v}_{i_p}}^{(p)}}\,
\end{equation*}
e
\begin{equation*}
\fua{\psi_1}{\tnr{K}_2}=\sum_{i_1=1}^{m}\cdots\sum_{i_p=1}^{m}\fua{\vtf{f}_{i_1\cdots
i_p}^{\con{T}^{\otimes}_{\crt{V}{p}}}}{\tnr{K}_2}\fua{\psi_1}{{\vto{v}_{i_1}}^{(1)}\otimes\cdots\otimes{\vto{v}_{i_p}}^{(p)}}\,
\end{equation*}
tamb�m s�o distintos. Se � assim, ent�o o operador $\psi_1$ � uma inje��o. Nos termos da
proposi��o \ref{teo:operadorBijecao}, $\psi_1$ � tamb�m uma bije��o.
\item[v.] Dada uma base qualquer de
tensores poli�dicos $\hat{\con{T}}^{\otimes}_{\crt{V}{p}}$ de $\ete{\crt{V}{p}}{\con{F}}$,
constru�da a partir de $\con{U}_i=\{{\vto{v}_1}^{(i)} ,\cdots, {\vto{v}_{m}}^{(i)}\}$ e
um tensor anti-sim�trico  qualquer $\tnr{T}\in\cfa{(\crt{V}{p})^n}{F}$, onde $n=m^p$.
Dado $i_j=1,\cdots,m$ e sabendo que
\begin{eqnarray*}
\fua{\psi_1}{{\vto{v}_{(i_1)_l}}^{(1)}\otimes
\cdots\otimes{\vto{v}_{(i_p)_l}}^{(p)}}=\sum_{(j_1)_l=1}^{m}\cdots\sum_{(j_p)_l=1}^{m}\lco{\psi_1}_{\hat{\con{T}}^{\otimes}_{\con{V}^{p}}}\rco_{(i_1)_l\cdots(i_p)_1(j_1)_l\cdots(j_p)_l}^{\hat{\con{T}}^{\otimes}_{\con{V}^{p}}}&&\\
{\vto{v}_{(j_1)_l}}^{(1)}\otimes
\cdots\otimes{\vto{v}_{(j_p)_l}}^{(p)}\,,
\end{eqnarray*}
pode-se realizar o seguinte desenvolvimento:
\begin{eqnarray*}
\fua{\tnr{T}_{\odot_q}}{\fua{\psi_1}{{\vto{v}_{(i_1)_1}}^{(1)}\otimes
\cdots\otimes{\vto{v}_{(i_p)_1}}^{(p)}}\otimes\cdots\otimes\fua{\psi_1}{{\vto{v}_{(i_1)_n}}^{(1)}\otimes
\cdots\otimes{\vto{v}_{(i_p)_n}}^{(p)}}}&=&\\
\sum_{(j_1)_1=1}^{m}\cdots\sum_{(j_p)_1=1}^{m}
\lco{\psi_1}_{\hat{\con{T}}^{\otimes}_{\con{V}^{p}}}\rco_{(i_1)_1\cdots(i_p)_1(j_1)_1\cdots(j_p)_1}^{\hat{\con{T}}^{\otimes}_{\con{V}^{p}}}\cdots
&&\\
\cdots\sum_{(j_1)_n=1}^{m}\cdots\sum_{(j_p)_n=1}^{m}
\lco{\psi_1}_{\hat{\con{T}}^{\otimes}_{\con{V}^{p}}}\rco_{(i_1)_n\cdots(i_p)_n(j_1)_n\cdots(j_p)_n}^{\hat{\con{T}}^{\otimes}_{\con{V}^{p}}}
&&\\
\fua{\tnr{T}_{\odot_q}}{{\vto{v}_{(j_1)_1}}^{(1)}\otimes
\cdots\otimes{\vto{v}_{(j_p)_1}}^{(p)}\otimes\cdots\otimes{\vto{v}_{(j_1)_n}}^{(1)}\otimes
\cdots\otimes{\vto{v}_{(j_p)_n}}^{(p)}}\,.&&
\end{eqnarray*}
Por meio de (\ref{eq:somaLeviCivita}), (\ref{eq:tensorAlternante}) e
(\ref{eq:transposicaoTensor}), podemos continuar assim:
\begin{eqnarray*}
\frac{1}{n!}\lco\sum_{(k_1)_1=1}^{m}\cdots\sum_{(k_p)_1=1}^{m}
\lco{\psi_1}_{\hat{\con{T}}^{\otimes}_{\con{V}^{p}}}\rco_{(i_1)_1\cdots(i_p)_1(k_1)_1\cdots(k_p)_1}^{\hat{\con{T}}^{\otimes}_{\con{V}^{p}}}\cdots
\right.&&\\
\left.\cdots\sum_{(k_1)_n=1}^{m}\cdots\sum_{(k_p)_n=1}^{m}
\lco{\psi_1}_{\hat{\con{T}}^{\otimes}_{\con{V}^{p}}}\rco_{(i_1)_n\cdots(i_p)_n(k_1)_n\cdots(k_p)_n}^{\hat{\con{T}}^{\otimes}_{\con{V}^{p}}}\epsilon_{(k_1)_1\cdots(k_p)_1}\cdots\epsilon_{(k_1)_n\cdots(k_p)_n}\rco
&&\\
\fua{\tnr{T}_{\odot_q}}{{\vto{v}_{(i_1)_1}}^{(1)}\otimes
\cdots\otimes{\vto{v}_{(i_p)_1}}^{(p)}\otimes\cdots\otimes{\vto{v}_{(i_1)_n}}^{(1)}\otimes
\cdots\otimes{\vto{v}_{(i_p)_n}}^{(p)}}&&
\end{eqnarray*}
porque o �ltimo termo foi colocado em evid�ncia j� que o tensor $\tnr{T}$ � anti-sim�trico. � importante observar que este resultado fica inalterado se fossem adotadas duas bases diferentes de tensores poli�dicos. Comparando esta igualdade com (\ref{eq:hiperdeterminante}) e com (\ref{eq:determinanteTensor}), conclui-se  a igualdade do �tem.
\item[vi.] A �ltima igualdade da demonstra��o do �tem anterior, escrita para $\psi_1^T$, pode ser apresentada na forma:
\begin{eqnarray*}
\fua{\tnr{T}_{\odot_q}}{\fua{\psi_1^T}{{\vto{v}_{(i_1)_1}}^{(1)}\otimes
\cdots\otimes{\vto{v}_{(i_p)_1}}^{(p)}}\otimes\cdots\otimes\fua{\psi_1^T}{{\vto{v}_{(i_1)_n}}^{(1)}\otimes
\cdots\otimes{\vto{v}_{(i_p)_n}}^{(p)}}}&=&\\
\frac{1}{n!}\lco\sum_{(k_1)_1=1}^{m}\cdots\sum_{(k_p)_1=1}^{m}\cdots\sum_{(k_1)_n=1}^{m}\cdots\sum_{(k_p)_n=1}^{m}
\epsilon_{(k_1)_1\cdots(k_p)_1}\cdots\epsilon_{(k_1)_n\cdots(k_p)_n}\right .&&\\
\left.\underbrace{ \lco{\psi_1^T}_{\hat{\con{T}}^{\otimes}_{\con{V}^{p}}}\rco_{(i_1)_1\cdots(i_p)_1(k_1)_1\cdots(k_p)_1}^{\hat{\con{T}}^{\otimes}_{\con{V}^{p}}}\cdots
\lco{\psi_1^T}_{\hat{\con{T}}^{\otimes}_{\con{V}^{p}}}\rco_{(i_1)_n\cdots(i_p)_n(k_1)_n\cdots(k_p)_n}^{\hat{\con{T}}^{\otimes}_{\con{V}^{p}}}}\rco&&\\
\fua{\tnr{T}_{\odot_q}}{{\vto{v}_{(i_1)_1}}^{(1)}\otimes
\cdots\otimes{\vto{v}_{(i_p)_1}}^{(p)}\otimes\cdots\otimes{\vto{v}_{(i_1)_n}}^{(1)}\otimes
\cdots\otimes{\vto{v}_{(i_p)_n}}^{(p)}}\,.&&
\end{eqnarray*}
Combinando (\ref{eq:arrayfunTensorialTransposta}) com o fato de que
\begin{equation*}
{\lco{\psi_1}_{\hat{\con{T}}^{\otimes}_{\con{V}^{p}}}\rco^{\hat{\con{T}}^{\otimes}_{\con{V}^{p}}}}^T_{(i_1)_l\cdots(i_p)_l(k_1)_l\cdots(k_p)_l}= \lco{\psi_1}_{\hat{\con{T}}^{\otimes}_{\con{V}^{p}}}\rco_{(k_1)_l\cdots(k_p)_l(i_1)_l\cdots(i_p)_l}^{\hat{\con{T}}^{\otimes}_{\con{V}^{p}}}\,,
\end{equation*}
tem-se que o termo destacado na primeira igualdade � igual ao da �ltima igualdade do �tem anterior. Isto ocorre porque os fatores da soma global, promovida pelo diversos somat�rios envolvidos, sofrem apenas altera��o de posi��o.
\item[vii.] Se $\psi_1^T=\psi_1^{-1}$ ent�o
\begin{equation*}
\hpd{\psi_1}\hpd{\psi_1^{-1}}=\hpd{\psi_1}\hpd{\psi_1^{T}}=\lpa\hpd{\psi_1}\rpa^2=1\,.
\end{equation*}
\end{itemize}
\end{prova}



% Seja o espa�o tensorial $\ete{\crt{V}{p}}{\con{F}}$ e duas de suas bases de tensores
% poli�dicos $\hat{\con{T}}^{\otimes}_{\crt{V}{p}}$ e
% $\tilde{\con{T}}^{\otimes}_{\crt{V}{p}}$. Dado o espa�o vetorial de fun��es tensoriais
% lineares $\evl{\cft{\crt{V}{p}}{\con{F}}}{\cft{\crt{V}{p}}{\con{F}}}{\con{F}}$, seja o
% operador tensorial $\psi\in\cfl{\cft{\crt{V}{p}}{\con{F}}}{\cft{\crt{V}{p}}{\con{F}}}$.
% Com base na igualdade (\ref{eq:hiperdertIdentidade}), a partir das convers�es
% (\ref{eq:conversoesTensores}), obt�m-se que
% \begin{eqnarray}
% \hpd{\lco\psi_{\hat{\con{T}}^{\otimes}_{\con{V}^{p}}}\rco^{\tilde{\con{T}}^{\otimes}_{\con{V}^{p}}}}=
% \hpd{\lco\psi_{\tilde{\con{T}}^{\otimes}_{\con{V}^{p}}}\rco^{\hat{\con{T}}^{\otimes}_{\con{V}^{p}}}}=
% \hpd{\lco\psi_{\tilde{\con{T}}^{\otimes}_{\con{V}^{p}}}\rco^{\tilde{\con{T}}^{\otimes}_{\con{V}^{p}}}}=
% \hpd{\lco\psi_{\hat{\con{T}}^{\otimes}_{\con{V}^{p}}}\rco^{\hat{\con{T}}^{\otimes}_{\con{V}^{p}}}}\,.\nonumber\\
% \end{eqnarray}
% Estas igualdades permitem definir o hiperdeterminante
% \begin{equation}
% \hpd{\psi} = \hpd{\lco\psi\rco}\,,
% \end{equation}
% onde o termo � direita significa qualquer um dos arrays associados a $\psi$. Fica
% evidente ent�o que, para
% $\psi_1^T,\psi_2\in\cfl{\cft{\crt{V}{p}}{\con{F}}}{\cft{\crt{V}{p}}{\con{F}}}$,
% \begin{equation}
% \hpd{\psi_1^T\circ\psi_2} = \hpd{\psi_1^T}\hpd{\psi_2} \,.
% \end{equation}
% A partir desta igualdade e de (\ref{eq:hiperdertIdentidade}), caso $\psi_1^T$ seja
% invers�vel, tem-se
% \begin{equation}
% \hpd{\psi_1^T}\hpd{\psi_1^T^{-1}} = \hpd{\vtf{i}_{V_p} } = 1\,,
% \end{equation}
% de onde se conclui que $\hpd{\psi_1^T}\neq0\,$.


\section{Isotropia e Anti-Isotropia}\index{isotropia}
O conceito de isotropia possui estreita rela��o com o de
invari�ncia. A grosso modo, diz-se que uma quantidade ou medida �
isotr�pica se seu valor independe do efeito de qualquer fun��o
ortogonal que eventualmente aja no seu dom�nio de atua��o. Se esta
independ�ncia se restringir somente a fun��es ortogonais pr�prias,
a medida � anti-isotr�pica\index{anti-isotropia}. Medidas que n�o
apresentam nenhuma destas caracter�sticas s�o chamadas
\emph{anisotr�picas}\index{anisotr�prica!medida}.

Considerando um espa�o vetorial Euclidiano, medidas isotr�picas,
representadas por grandezas de segunda ordem (matrizes), por
exemplo, s�o \emph{indiferentes} a rota��es e reflex�es. Ocorrendo
reflex�es, medidas an\-ti-iso\-tr�\-pi\-cas podem sofrer
altera��o.

\subsection{Tensores Invariantes}\index{tensor!invariante}
Seja o espa�o tensorial $\ete{\crt{V}{p}}{\con{F}}$ e o grupo de
operadores tensoriais $\gqq{\cft{\crt{V}{p}}{\con{F}}}{}$.
Considerando que o conjunto
$\con{\tilde{P}}_{\cft{\crt{V}{p}}{\con{F}}}\subseteq
\con{P}_{\cft{\crt{V}{p}}{\con{F}}}$ define o grupo de operadores
$\gqq{\cft{\crt{V}{p}}{\con{F}}}{\backsim}$, um tensor
$\tnr{T}\in\cft{\crt{V}{p}}{\con{F}}$ � invariante � atua��o do
grupo $\gqq{\cft{\crt{V}{p}}{\con{F}}}{\backsim}$ se
\begin{equation}
\fua{\psi}{\tnr{T}}=\tnr{T}\,,\,\forall\,\psi\in\con{\tilde{P}}_{\cft{\crt{V}{p}}{\con{F}}}\,.
\end{equation}
Caso a igualdade anterior seja v�lida para qualquer tensor
$\tnr{T}$ de $\cft{\crt{V}{p}}{\con{F}}$, diz-se que o espa�o
tensorial $\ete{\crt{V}{p}}{\con{F}}$ � invariante ao grupo
$\gqq{\cft{\crt{V}{p}}{\con{F}}}{\backsim}$. Como exemplo, pode-se
observar que $\ete{\crt{V}{p}}{\con{F}}$ � invariante ao grupo
definido pelo conjunto $\{ \vtf{i}_{\crt{V}{p}} \}$.

\subsubsection{Tensor Isotr�pico}\index{tensor!isotr�pico} Um tensor � dito isotr�pico se
ele for invariante ao grupo ortogonal como um todo. Em outras
palavras, se o grupo de operadores tensoriais considerado for
$\gro{\cft{\crt{V}{p}}{\con{F}}}$, um tensor
$\tnr{T}\in\cft{\crt{V}{p}}{\con{F}}$ � isotr�pico se
\begin{equation}
\fua{\psi}{\tnr{T}}=\tnr{T}\,,\,\forall\,\psi\in\con{O}_{\cft{\crt{V}{p}}{\con{F}}}\,.
\end{equation}

\subsubsection{Tensor Anti-Isotr�pico}\index{tensor!anti-isotr�pico} Um tensor � anti-isotr�pico
se ele for invariante ao grupo ortogonal pr�prio. Desta forma, com
base no conjunto $\con{O}^+_{\cft{\crt{V}{p}}{\con{F}}}$ formado
por operadores ortogonais pr�prios, para um tensor anti-isotr�pico
$\tnr{T}\in\cft{\crt{V}{p}}{\con{F}}$ ,
\begin{equation}
\fua{\psi}{\tnr{T}}=\tnr{T}\,,\,\forall\,\psi\in\con{O}^+_{\cft{\crt{V}{p}}{\con{F}}}\,,
\end{equation}
onde o conjunto considerado define o grupo ortogonal
$\grr{\cft{\crt{V}{p}}{\con{F}}}$. Com isso, conclui-se que um
tensor isotr�pico � sempre anti-isotr�pico.

\subsection{Conjuntos de Tensores Invariantes}\index{conjunto!de tensores invariantes}
Seja o espa�o tensorial $\ete{\crt{V}{p}}{\con{F}}$ e o grupo de
operadores tensoriais $\gqq{\cft{\crt{V}{p}}{\con{F}}}{}$.
Considerando o grupo $\gqq{\cft{\crt{V}{p}}{\con{F}}}{\backsim}$
definido por $\con{\tilde{P}}_{\cft{\crt{V}{p}}{\con{F}}}\subseteq
\con{P}_{\cft{\crt{V}{p}}{\con{F}}}$, diz-se que
$\cti{\crt{V}{p}}{\con{F}}\subseteq \cft{\crt{V}{p}}{\con{F}}$ �
conjunto de tensores invariantes ao grupo considerado se
\begin{equation}
\fua{\psi}{\tnr{T}}=\tnr{T}\,,\,\forall\,\psi\in\con{\tilde{P}}_{\cft{\crt{V}{p}}{\con{F}}}\,,\,
\tnr{T}\in\cti{\crt{V}{p}}{\con{F}}\,.
\end{equation}
Se $\con{\tilde{P}}_{\cft{\crt{V}{p}}{\con{F}}}=
\con{O}_{\cft{\crt{V}{p}}{\con{F}}}$, ent�o o conjunto
$\cti{\crt{V}{p}}{\con{F}}$ � formado por tensores isotr�picos,
sendo representado por $\cts{\crt{V}{p}}{\con{F}}$. Para
$\con{\tilde{P}}_{\cft{\crt{V}{p}}{\con{F}}}=
\con{O}^+_{\cft{\crt{V}{p}}{\con{F}}}$, o conjunto de tensores
anti-isotr�picos � representado $\ctt{\crt{V}{p}}{\con{F}}$.

\subsubsection{Exemplos de Tensores Isotr�picos e
Anti-Isotr�picos} Seja o espa�o vetorial $\ehr{V}{\con{F}}$, o
espa�o tensorial $\ete{\con{V}^p}{\con{F}}$ , o tensor identidade
$\tnr{I}\in\cft{\con{V}^2}{\con{F}}$ e o tensor anti-sim�trico n�o
nulo $\tnr{A}\in\cfa{\con{V}^2}{\con{F}}$. Nestas condi��es
particulares, \aut{Backus}\cite{backus_1997_1}, citando
\aut{Weyl}\cite{weyl_1997_1}, afirma que os conjuntos de tensores
isotr�picos n�o vazios s�o gerados (span) pela permuta��o de
tensores identidade. Segundo ele, conjuntos de tensores
anti-isotr�picos n�o vazios s�o gerados pela permuta��o de
tensores identidade e$\backslash$ou de tensores anti-sim�tricos
n�o nulos e$\backslash$ou dos tensores $\tnr{I}\otimes\tnr{A}$,
$\tnr{I}\otimes\tnr{I}\otimes\tnr{A}$,
$\tnr{I}\otimes\tnr{I}\otimes\tnr{I}\otimes\tnr{A}$, etc\ldots Em
termos espec�ficos, para espa�os tensoriais de ordem 0 a 4,
\aut{Backus}\cite{backus_1997_1} demonstra cada um dos resultados
apresentados na tabela \ref{tb:Isotropicos} a seguir.


\begin{table}[!htt]
\centering
\scriptsize{
\begin{tabular}{|c|c|c|c|c|}
  \hline
  % after \\: \hline or \cline{col1-col2} \cline{col3-col4} ...
  \textbf{Ordem de}    & \textbf{Valor de}     & \textbf{Conjunto}    & \textbf{Conjunto}         & \textbf{Elementos de}  \\
  $\ete{\con{V}^p}{\con{F}}$ & $\mathrm{dim}(\ehr{\con{V}}{\con{F}})$  & $\cts{\con{V}^{p}}{\con{F}}$  & $\ctt{\con{V}^{p}}{\con{F}}$  & \textbf{P�s-condi��o} \\
  \hline
  & & & & \\
  $p=0$ & $>0$ & $\con{F}$ & $\con{F}$ & - \\
  & & & & \\
  \hline
& & & & \\
  $p=1$ & $\geqslant 2$ & $\{ \vto{0} \}$ & $\{ \vto{0} \}$ & - \\
  & & & & \\
  \hline
& & & & \\
  $p=2$ & $>0$ & $sp\{ \tnr{I} \}$ & $\emptyset$ & $ \tnr{I}\in\cft{\con{V}^p}{\con{F}},$ \\
& & & & \\
\cline{2-2}\cline{4-4}
& & & & \\
        & $= 2$ &  & $sp\{ \tnr{I} , \tnr{A} \}$ & $\forall \tnr{A}\neq \negmath{0}, $ \\
& & & & \\
\cline{2-2}\cline{4-4}
& & & & \\
        & $\geqslant 3$ &  & $sp\{ \tnr{I} \}$ & $\tnr{A}\in\cfa{\con{V}^p}{\con{F}}$ \\
& & & & \\
\cline{1-4}
& & & & \\
  $p=3$ & $>0$ & $\{ \negmath{0} \}$ & $\emptyset$ &  \\
& & & & \\
\cline{2-2}\cline{4-4}
& & & & \\
        & $= 3$ &  & $sp\{ \tnr{A} \}$ &  \\
& & & & \\
\cline{2-2}\cline{4-4}
& & & & \\
        & $\neq 3$ &  & $\{ \negmath{0} \}$ &  \\
& & & & \\
\cline{1-4}
& & & & \\
  $p=4$ & $>0$ & $sp\{ \tnr{I},\tnr{I}_{(2,3)} , \tnr{I}_{(2,4)} \}$ & $\emptyset$ &  \\
& & & & \\
\cline{2-2}\cline{4-4}
& & & & \\
        & $\in \{ 2, 4 \}$ &  & $sp\{ \tnr{I},\tnr{I}_{(2,3)} , \tnr{I}_{(2,4)} \}$ &  \\
& & & & \\
\cline{2-2}\cline{4-4}
& & & & \\
        & $= 4$ &  & $sp\{ \tnr{I},\tnr{I}_{(2,3)} , \tnr{I}_{(2,4)}, \tnr{A} \}$ &  \\
& & & & \\
\cline{2-5}
& & & & \\
        & $=2$ & $sp\{ \tnr{I}\otimes\tnr{I},(\tnr{I}\otimes\tnr{I})_{(2,3)}, $ & $sp\{ \tnr{I}\otimes\tnr{I},(\tnr{I}\otimes\tnr{I})_{(2,3)},(\tnr{I}\otimes\tnr{I})_{(2,4)}, $ & $ \tnr{I}\in\cft{\con{V}^2}{\con{F}},$ \\
        &  & $(\tnr{I}\otimes\tnr{I})_{(2,4)}\}$ & $ \tnr{I}\otimes\tnr{A},(\tnr{I}\otimes\tnr{A})_{(2,3)},(\tnr{I}\otimes\tnr{A})_{(2,4)},$ & $\forall \tnr{A}\neq \negmath{0}, $ \\
        &  &  & $ \tnr{A}\otimes\tnr{I},(\tnr{A}\otimes\tnr{I})_{(2,3)},(\tnr{A}\otimes\tnr{I})_{(2,4)}\}$ & $\tnr{A}\in\cfa{\con{V}^2}{\con{F}}$ \\

& & & & \\
\hline
\end{tabular}
}  \titfigura{Tensores Isotr�picos e Anti-Isotr�picos em
$\ete{\con{V}^p}{\con{F}}$. Fonte:
\aut{Backus}\cite{backus_1997_1}.}\label{tb:Isotropicos}
\end{table}


\subsection{Operadores Tensoriais de Invari�ncia}\index{operador!tensorial!de invari�ncia}
Seja o espa�o tensorial $\ete{\crt{V}{p}}{\con{F}}$ e o grupo de
operadores tensoriais $\gqq{\cft{\crt{V}{p}}{\con{F}}}{}$. Seja o
conjunto de tensores $\cti{\crt{V}{p}}{\con{F}}$, invariantes ao
grupo $\gqq{\cft{\crt{V}{p}}{\con{F}}}{\backsim}$ definido por
$\con{\tilde{P}}_{\cft{\crt{V}{p}}{\con{F}}}\subseteq
\con{P}_{\cft{\crt{V}{p}}{\con{F}}}$. Um operador tensorial
$\psi\in\con{P}_{\cft{\crt{V}{p}}{\con{F}}}$ � dito de invari�ncia
em $\cti{\crt{V}{p}}{\con{F}}$ se ele definir o mapeamento
\begin{equation}
\map{\psi}{\cti{\crt{V}{p}}{\con{F}}}{\cti{\crt{V}{p}}{\con{F}}}\,.
\end{equation}
O conjunto
$\con{I}_{\cft{\crt{V}{p}}{\con{F}}}\subseteq\con{P}_{\cft{\crt{V}{p}}{\con{F}}}$
de todos os operadores de invari�ncia que atuam em
$\cti{\crt{V}{p}}{\con{F}}$ define o grupo de invari�ncia
$\grn{}{\cft{\crt{V}{p}}{\con{F}}}$.

\subsubsection{Operadores Tensoriais de Isotropia}\index{operador!tensorial!de isotropia}
Dadas as condi��es anteriores, o operador tensorial de isotropia �
um operador de invari�ncia em $\cts{\crt{V}{p}}{\con{F}}$. Com
isso, tem-se o grupo de isotropia
$\gso{}{\cft{\crt{V}{p}}{\con{F}}}$ definido pelo conjunto
$\con{\hat{I}}_{\cft{\crt{V}{p}}{\con{F}}}\subseteq\con{I}_{\cft{\crt{V}{p}}{\con{F}}}$
.

\subsubsection{Operadores Tensoriais de Anti-Isotropia}\index{operador!tensorial!de anti-isotropia}
Neste caso, s�o operadores de invari�ncia em
$\ctt{\crt{V}{p}}{\con{F}}$, elementos do conjunto
$\con{\hat{I}}^+_{\cft{\crt{V}{p}}{\con{F}}}\subseteq\con{I}_{\cft{\crt{V}{p}}{\con{F}}}$
que define o grupo $\gsm{}{\cft{\crt{V}{p}}{\con{F}}}$.


\section{Campo Tensorial}\index{campo!tensorial}
Seja o espa�o m�trico completo $\lpa \con{A}, \varrho \rpa$ e o
espa�o tensorial $\ete{\crt{W}{q}}{\con{F}}$. Dado o subconjunto
$\con{B}\subset\con{A}$, que tamb�m define um espa�o m�trico
completo, a fun��o no mapeamento
\begin{equation}
\map{\fac{F}}{B}{\cft{\crt{W}{q}}{\con{F}}}
\end{equation}
� dita um campo tensorial. Caso $q=1$, tem-se ent�o um \emph{campo
vetorial}\index{campo!tensorial}. Se $q=0$, o campo � dito
\emph{escalar}.\index{campo!escalar}

\subsection{A Rela��o Campo
Tensorial e Fun��o Tensorial}\label{sec:AfimTensorial} Seja
$\saf{U}{S}{\ele{a}}{F}$ um subespa�o afim do espa�o m�trico
completo $\eaf{V}{A}{F}$ e um sistema de coordenadas $(\ele{o} ,
\tilde{\con{U}})$ deste subespa�o. Considerando o espa�o tensorial
$\ete{\crt{W}{q}}{\con{F}}$ e o espa�o vetorial $\evt{U}{F}$ (ou
$\ete{\con{U}}{\con{F}}$), dado um mapeamento
\begin{equation}
\map{\fac{F}}{\epo{S}_\ele{a}}{\cft{\crt{W}{q}}{\con{F}}},
\end{equation}
e a igualdade $\ele{a}=\vto{a}\oplus\ele{o}$, existe uma �nica uma
fun��o tensorial que define
\begin{equation}
\map{\psi}{\con{U}}{\cft{\crt{W}{q}}{\con{F}}}\,,
\end{equation}
cuja regra �
\begin{equation}\label{eq:funcaoCampo}
\fua{\psi}{\vto{x}}=\fua{\fac{F}}{\vto{x}\oplus\vto{a}\oplus\ele{o}}\,.
\end{equation}
Como o dom�nio
$\epo{S}_\ele{a}=\{\vto{x}\oplus\vto{a}\oplus\ele{o}:\vto{x}\in\con{U}\}$,
onde $\vto{a}\oplus\ele{o}$ � um ponto fixo, pode-se dizer que
$\psi$ � a fun��o tensorial associada ao campo tensorial
$\fac{F}$. Desta forma, o estudo de campos tensoriais pode ser substitu�do pelo
estudo de fun��es tensoriais.

\subsubsection{Campo Caracter�stico}\index{campo!caracter�stico}
Seja um espa�o afim m�trico completo $\eaf{V}{A}{F}$ com
dimens�o maior que 2 e dois de seus subespa�os
$\saf{U}{S}{\ele{a}}{F}$ e $\saf{W}{S}{\ele{b}}{F}$, tais que
$\epo{S}_\ele{b}\subset\epo{S}_\ele{a}$. Dado um sistema de
coordenadas $(\ele{o} , \tilde{\con{U}})$ de
$\saf{U}{S}{\ele{a}}{F}$ pelo qual o ponto
$\ele{b}=\vto{b}\oplus\ele{o}$, seja o mapeamento bijetor
\begin{equation}
\map{\fac{F}}{\epo{S}_\ele{b}}{\con{W}}\,,
\end{equation}
onde a fun��o identidade $\fun{i}_\con{W}$ est� associada ao campo
$\fac{F}$. Logo,
\begin{equation}\label{eq:campoDescritor}
\fua{\fac{F}}{\vto{w}\oplus\vto{b}\oplus\ele{o}}=\vto{w}\,,\,\forall\,
\vto{w}\in\con{W}\,.
\end{equation}
Nestas condi��es, diz-se que $\fac{F}$ � um campo caracter�stico de $\epo{S}_\ele{b}$
em $\con{W}$, representado $\fac{F}_{\epo{S}_\ele{b}}^{\con{W}}$.

%Considerando um
%mapeamento sobrejetor
%\begin{equation}
%\map{\chi}{\con{U}}{\con{W}}\,,
%\end{equation}
%pode-se reescrever (\ref{eq:campoDescritor}) na forma
%\begin{equation}
%\fua{\mathcal{F}}{\fua{\chi}{\vto{x}}\oplus\vto{b}\oplus\ele{o}}=\fua{\chi}{\vto{x}}\,.
%\end{equation}
%A partir desta igualdade, a fun��o $\chi$ � dita caracter�stica de
%$\epo{S}_\ele{b}$ em $\con{U}$, representada
%$\fac{F}_{\epo{S}_\ele{b}}^{\con{U}}$. � importante notar que fun��es caracter�sticas permitem substituir o uso de pontos pelo de vetores, viabilizando manipula��es alg�bricas.
\end{comment} 
%    
\chapter{Topics of Affine Geometry}

Although vector spaces are building blocks of the several important concepts presented so far on the two previous chapters, it is still not possible to recognize shapes in their definer sets. However, the study of mathematical shapes does not belong to Linear Algebra, but to the realms of Geometry, whose non empty sets we shall consider here to be constituted by ``shape'' elements called \textsb{points}\index{points}, which are geometric objects of primitive notion, devoid of dimensional features in order to morphologically represent physical locations. If a given set of such points is conveniently related to a vector space by a group action, other shapes can be obtained and thereby further morphological concepts can be developed, when we say that a certain geometry is defined. As physical spaces are usually abstracted by geometric sets, we can not prescind from studying at least the basic topics of the so called Affine Geometry, which includes the well known Euclidean Geometry and is sufficiently governed by the concept of parallelism.


\section{Affine Spaces}\index{affine!space}\index{space!affine}\label{sec:affine}

Recalling the concept of group action presented on chapter \ref{ch:Collect}, let the set of points $\epu{U}$ be the $G$-set of a vector space $U_\cam{F}$ through a simply transitive group action $\oplus$. Given two arbitrary points $a$ and $b$ of this $G$-set $\epu{U}$, now called a \textsb{point space}\index{space!point}, and a vector $\vto{u}\in U_\cam{F}$ where point $\fua{\oplus}{\vto{u}, a}=b$ or $\gloref{actVect}=b$, the axioms of simply transitive group actions can be rewritten the following way:
\begin{itemize}\label{ax:pointSpaces}
\setlength\itemsep{.1em}
\item[i.]  $\vto{0}\oplus a=a\,,\forall \, \ele{a} \in \epu{U}$;
\item[ii.] $\vto{u}\oplus(\vto{v}\oplus a)=(\vto{u}+\vto{v})\oplus a, \forall \, \ele{a}\in \epu{U},\,\forall\, \vto{u},\vto{v}\in U_\cam{F}$;
\item[iii.]  $\exists !\,\, \vto{u}\in U_\cam{F}\textrm{ such that } \vto{u}\oplus a=b$.	
\end{itemize}
In this context, the triple $(U_\cam{F},\epu{U},\oplus)$ is called an affine space, henceforth represented by $\eam{U}{F}$, which will also be used to refer to the point space $\epu{U}$, as we adopted similarly for other cases in order to simplify notation. Moreover, vector space $U_\cam{F}$ is usually called the \textsb{direction space}\index{space!direction} of affine space $\eam{U}{F}$. Therefore, we can say that an affine space is defined by a direction space, a point space and a group action. From axiom iii above, every double of points uniquely identifies a certain vector: if $\vto{u}\oplus a=b$, we define that double $(a,b)$ identifies vector $\vto{u}$, when this vector is represented by $\vv{ab}$ and then $\vv{ab}\oplus a=b$. \emph{The converse is not true: a certain vector can be identified by infinite doubles of points because there is always a $y=\vto{u}\oplus x$ for all $x\in\eam{U}{F}$ and then $\vv{ab}=\vv{xy}$}. Now, from axioms i and ii, considering point $\vto{v}\oplus a=c$, we conclude that $-\vto{v}\oplus c=a$, and then vector $\vto{v}=\vv{ac}=-\vv{ca}$. From this characteristic relationship between points and vectors, the classical pictorial representation of a vector identified by a double of points $(a,b)$ as an arrow whose tail ``starts'' on $a$ and head points to $b$ arises naturally (figure \ref{fg:vetorSeta}).
\begin{figure}[!ht]
\centering
\begin{center}
\scalebox{.72}{\input{partes/figs/vetorSeta.pstex_t}}
\end{center}
\titfigura{Vector as an arrow identified by points $a$ and $b$.}\label{fg:vetorSeta}
\end{figure}
Since an arbitrary vector can be identified by infinite doubles of points, given arbitrary points $x,y,a,b\in\eam{U}{F}$, the second operand of $\vv{ab}+\vv{xy}$ can be described by a vector $\vv{bc}$, where point $c=\vv{xy}\oplus b$, and then equalities $(\vv{ab}+\vv{bc})\oplus a=(\vv{bc}+\vv{ab})\oplus a=\vv{bc}\oplus b=c$ enable us to conclude that if $(\vv{ab}+\vv{bc})\oplus a=c$ then $\vv{ab}+\vv{bc}=\vv{ac}$. From this algebraic property, it is possible to depict graphically a sum of vectors by concatenating each one of the their representative arrows in such a way that a head point to a tail, as shown in the following figure \ref{fg:vetorSoma}.
\begin{figure}[!ht]
\centering
\begin{center}
\scalebox{.72}{\input{partes/figs/vetorSoma.pstex_t}}
\end{center}
\titfigura{Sum of vectors $\protect\vv{ab}$ and $\protect\vv{xy}$ where $\protect\vv{xy}\oplus b=c$.}\label{fg:vetorSoma}
\end{figure}
It is then straightforward to conclude that a vector $\alpha\vv{ab}$, where $\alpha\in\real$, is represented by a ``compressed'' ($\alpha\leq 1$) or ``expanded'' ($\alpha >1$) vector $\vv{ab}$. Moreover, we can say that the coordinates of vector $\alpha\vv{ab}$ are the coordinates of $\vv{ab}$ equally multiplied, that is equally ``compressed'' or ``expanded'' by the real number $\alpha$. For the case of a complex affine space $\eam{U}{C}$, the polar complex coordinates of a vector when multiplied by a complex $\alpha$ are equally multiplied by the real scalar $|\alpha|$ and equally rotated by $\arg{(\alpha)}$ on the complex plane, according to the basic theory of complex number multiplication. Therefore, in affine geometric representation, a complex vector $\alpha\vv{ab}\in U_\cam{C}$ results also in a ``compressed'' or ``expanded'' complex vector $\vv{ab}$.

Considering previous conditions, given an arbitrary point $a\in \eam{U}{F}$ and a vector subspace $S_\cam{F}\subset U_\cam{F}$, the triple $(S_\cam{F},\epu{S}_a,\oplus)$ is called an \textsb{affine subspace}\index{affine!subspace} of $\eam{U}{F}$, represented by $\eams{S}{F}{a}$, if point space $\epu{S}_a:=\{\vto{x}\oplus a:\forall \vto{x}\in S_\cam{F}\}$, to which point $a$, called the \textsb{seed point}\index of $\eams{S}{F}{a}$, also belongs since vector $\vto{0}\in S_\cam{F}$. Note that if a point $b\in\eams{S}{F}{a}$ defines the affine subspace $\eams{S}{F}{b}$, then 
\begin{equation}\label{eq:igualAff}
\eams{S}{F}{a}=\{\vto{x}\oplus a:\forall \vto{x}\in S_\cam{F}\}=\{\vto{x}+\vv{ab}\oplus b:\forall \vto{x}\in S_\cam{F}\}=\eams{S}{F}{b}\,.
\end{equation}
In the case of affine space $\eam{U}{F}$ defined by a $m$-dimensional vector space, an arbitrary affine  subspace $\eams{S}{F}{a}\subset\eam{U}{F}$ becomes another fundamental geometric object called a \textsb{hyperplane}\index{hyperplane} when its definer vector subspace $S_\cam{F}$ is $(m-1)$-dimensional: we specifically call the hyperplane a \textsb{line}\index{line} or a \textsb{plane}\index{plane} when $S_\cam{F}$ is one or two-dimensional respectively. Since it is the vector space which bears a dimensional feature, it is defined that $\dim{(\eam{U}{F})}=\dim{(U_\cam{F})}=m$ and then we use $\gloref{affSub}$ to represent a $m$-dimensional affine space. Now, from the definition of affine spaces using a group action approach, as we have done, angles and distances cannot be obtained, and then it results that requiring these two concepts means two additional features that turn affine geometry into Euclidean geometry, when we state that Euclidean is a more restricted form of affine geometry. On the other hand, our algebraic definition of affine spaces does not require to explicitly present the so called Playfair's Axiom as a restriction, as an axiom itself to be observed because this restriction becomes a natural property, a theorem. In order to present this very important property, we must first deal with the concept of parallelism using previous definitions. Considering our algebraic approach, two affine subspaces are considered to be \textsb{parallel}\index{parallelism} if the direction space of one is an improper subset of the other\footnote{This concept of parallelism admits that a hyperplane (a line or a plane, for example) is parallel to itself.}. In mathematical terms, we define two affine subspaces $\eamsd{S}{F}{a}{n}$ and $\eamsd{V}{F}{b}{r}$ of $\eamd{U}{F}{m}$ to be parallel, represented by $\gloref{affSubPar}\,$, if direction space $S_\cam{F}\subseteq V_\cam{F}$ or $V_\cam{F}\subseteq S_\cam{F}$. From this definition of parallelism, when $\eamsd{S}{F}{a}{n}$ and $\eamsd{V}{F}{b}{r}\,$ are indeed parallel we obtain the following properties:
\begin{itemize}
	\setlength\itemsep{.1em}
	\item[i.] $n=r\iff S_\cam{F}=V_\cam{F}$;
	\item[ii.] $a\in \eamsd{V}{F}{b}{r}\,\wedge\, S_\cam{F}\subseteq V_\cam{F}\iff\eamsd{S}{F}{a}{n}\subseteq\eamsd{V}{F}{b}{r}$;
	\item[iii.] $a\notin \eamsd{V}{F}{b}{r}\,\vee\, b\notin \eamsd{S}{F}{a}{n} \iff \eamsd{S}{F}{a}{n}\cap\eamsd{V}{F}{b}{r}=\emptyset$.
\end{itemize}
Figure \ref{fg:paralelo} depicts a line and plane parallel to each other for the case of a fi\-ve-di\-men\-sio\-nal real affine space. Now, we say that subspaces $\eamsd{W}{F}{c}{r}$ and $\eamsd{S}{F}{a}{n}$ are \textsb{perpendicular}\index{perpendicularity}, represented by $\gloref{affSubPerp}\,$, if vectors of $W_\cam{F}$ and $V_\cam{F}$ do not have an incidence interrelationship, that is, if $W_\cam{F}\perp V_\cam{F}$. Geometrical representation of perpendicularity is more intuitive in the context of affine Euclidean spaces through the concept of angle, which we shall present later in this section.

{\footnotesize
\begin{proof}
For the first item of the previous properties, since $S_\cam{F}\subseteq V_\cam{F}$ or $V_\cam{F}\subseteq S_\cam{F}$, a direction space is a proper subset of the other only if they have different dimensions, otherwise they are equal. The inverse implication is trivial. For the second item, if $a\in \eamsd{V}{F}{b}{r}$ there is a vector $\vto{w}\in V_\cam{F}$ where $a=\vto{w}\oplus b$. Then, from axiom ii on page \pageref{ax:pointSpaces} and since $S_\cam{F}\subseteq V_\cam{F}$, we can write $\vto{x}\oplus a=(\vto{x}+\vto{w})\oplus b$, $\forall \vto{x}\in S_\cam{F}$, from which we conclude that $\eamsd{S}{F}{a}{n}\subseteq\eamsd{V}{F}{b}{r}$. Now, if $\eamsd{S}{F}{a}{n}\subseteq\eamsd{V}{F}{b}{r}$ is true we always have a vector $\vto{y}\in V_\cam{F}$ related to an arbitrary $\vto{x}\in S_\cam{F}$ through $\vto{x}\oplus a=\vto{y}\oplus b$ or $a=(\vto{y}-\vto{x})\oplus b$, from which we conclude that $a\in \eamsd{V}{F}{b}{r}$ and $S_\cam{F}\subseteq V_\cam{F}$. For  property iii, we prove only for $a\notin \eamsd{V}{F}{b}{r}$. In this case, there is no vector in $V_\cam{F}$ relating the pair of points $(b,a)$. Since $S_\cam{F}\subseteq V_\cam{F}$ or $V_\cam{F}\subseteq S_\cam{F}$, we can write for all $\vto{x}\in S_\cam{F}$ and $\vto{y}\in V_\cam{F}$ that
\begin{align*}
a &\neq \vto{y}\oplus b\\
\vto{x}\oplus a &\neq \vto{x}\oplus(\vto{y}\oplus b)\\
\vto{x}\oplus a &\neq (\vto{x}+\vto{y})\oplus b\,,
\end{align*}
from which we conclude that $\eamsd{S}{F}{a}{n}\cap\eamsd{V}{F}{b}{r}=\emptyset$. The inverse implication is verified considering $\eamsd{S}{F}{a}{n}\cap\eamsd{V}{F}{b}{r}=\emptyset$ valid and then, for all $\vto{x}\in S_\cam{F}$ and $\vto{y}\in V_\cam{F}$,
\begin{align*}
\vto{x}\oplus a &\neq \vto{y}\oplus b\\
-\vto{x}\oplus(\vto{x}\oplus a) &\neq -\vto{x}\oplus(\vto{y}\oplus b)\\
a &\neq (\vto{y}-\vto{x})\oplus b\,,
\end{align*}
from which we conclude that $a\notin \eamsd{V}{F}{b}{r}$.
\end{proof}}

\begin{figure}[!ht]
	\centering
	\begin{center}
		\scalebox{.72}{\input{partes/figs/paralelo.pstex_t}}
	\end{center}
	\titfigura{Line and plane parallel to each other.}\label{fg:paralelo}
\end{figure}

\begin{mteo}{Playfair's ``Axiom''}{playfair}
If $\eamd{U}{F}{n+1}$ is an affine space where $\eamsd{S}{F}{a}{n}$ is one of its hyperplanes and $k\notin \eamsd{S}{F}{a}{n}$ one of its points, there is a unique hyperplane $\eamsd{V}{F}{b}{n}\parallel\eamsd{S}{F}{a}{n}$ such that $k\in\eamsd{V}{F}{b}{n}$.
\end{mteo}

{\footnotesize
\begin{proof}
If $\eamsd{S}{F}{a}{n}$ exists, so does $\eamsd{S}{F}{b}{n}$, where $b\neq a$. Since these subspaces are obviously parallel, the existence of subspace $\eamsd{V}{F}{b}{n}=\eamsd{S}{F}{b}{n}$ is proved. Uniqueness is verified by the following rationale: since point $k\in\eamsd{V}{F}{b}{n}$ then $\eamsd{V}{F}{b}{n}=\eamsd{V}{F}{k}{n}$ according to \eqref{eq:igualAff}; and, supposing a third hyperplane $\eamsd{W}{F}{c}{n}\parallel \eamsd{S}{F}{a}{n}$ where $k\in\eamsd{W}{F}{c}{n}$, we also have $\eamsd{W}{F}{c}{n}=\eamsd{W}{F}{k}{n}$. Moreover, from the first property of parallelism, it is clear that $V_\cam{F}=S_\cam{F}=W_\cam{F}$ because the three hyperplanes have obviously the same dimension and then $\eamsd{W}{F}{k}{n}=\eamsd{V}{F}{k}{n}$. Since $\eamsd{V}{F}{b}{n}=\eamsd{V}{F}{k}{n}$ and $\eamsd{W}{F}{c}{n}=\eamsd{W}{F}{k}{n}$, then $\eamsd{V}{F}{b}{n}=\eamsd{W}{F}{c}{n}$. 
\end{proof}}

 Considering previous conditions, given a point $o\in\eamsd{S}{F}{a}{n}$ and a basis $B=\{\vto{v}_1,\cdots,\vto{v}_n\}$ of $S_\cam{F}$, where $n<m$, we call double $\gloref{coordSys}$ an \textsb{affine coordinate system}\index{affine!coordinate system} of $\eamsd{S}{F}{a}{n}$ because point space $\epu{S}_a$ can be obtained from $o$ and $B$ as follows:
\begin{equation}\label{eq:pointSpace}
\epu{S}_a=\{\vto{x}\oplus a:\forall \vto{x}\in S_\cam{F}\}= \{(\vto{x}+\vv{oa})\oplus o:\forall \vto{x}\in \spn{(B)}\}\,.
\end{equation}
In this context, point $o$ is called the \textsb{origin}\index{origin} of the coordinate system and the coordinates of an arbitrary vector $\vto{v}\in S_\cam{F}$ on $B$ are considered to be the coordinates of point $v:=\vto{v}\oplus o$ on $(o,B)$, that is, $\mav{v}{B}:=\mav{\vto{v}}{B}$. Thereby, it is trivial to obtain that the coordinates of point $\vto{u}\oplus v$ are the coordinates of vector $\vto{u}+\vto{v}$ or that $\mav{\vto{u}\oplus v}{B}=\mav{\vto{u}+\vto{v}}{B}$. Moreover, each line $((\mathcal{B}_i)_o)^1_\cam{F}$ defined by vector space $(B_i)_\cam{F}=\spn{(\{\vto{v}_i\})}$ is called an \textsb{axis}\index{axis} of $(o,B)$, which is depicted in figure \ref{fg:coordSystem}.
\begin{figure}[!ht]
	\centering
	\begin{center}
		\scalebox{.72}{\input{partes/figs/coordSystem.pstex_t}}
	\end{center}
	\titfigura{$n$-dimensional affine coordinate system $(o,B)$ and its axes.}\label{fg:coordSystem}
\end{figure}
On section \ref{sec:group}, we said that in a group action a biunivocal relationship between its definer set and its definer group is possible if an element of the set is fixed. For the case of affine spaces, this fixation is established by a coordinate system, which enables to define a point $v$ from a vector $\vto{v}$ through an origin $o$, as we have done. In mathematical terms, considering origin $o$, the mapping $\map{\upsilon_o}{\eamsd{S}{F}{a}{n}}{\eam{S}{F}}$ is bijective if 
\begin{equation}\label{eq:funVect}
\fua{\upsilon_o}{x} = \vv{ox}\,,
\end{equation} 
where bijection $\upsilon_o$ is said to ``vectorize'' its argument and then, since the origin is fixed, we can represent $\fua{\upsilon_o}{v}=\vv{ov}$ by $\vto{v}$. The association $(\eamsd{S}{F}{a}{n},o,B)$ of an affine subspace with a coordinate system can be structured like a vector space and then inherits the same classification, outlined in figure \ref{fig:esquemaEspacos}, from its direction space $S_\cam{F}$ if suitable expressions for metric, norm and inner product are defined. Thereby, for arbitrary points $v$ and $u$ of $\eamsd{S}{F}{a}{n}$, the triple $(\eamsd{S}{F}{a}{n},o,B)$ is called an affine metric space if $\fua{\varrho}{u,v}:=\fua{\varrho}{\vv{ou},\vv{ov}}$, an affine normed space if a norm $\|\vv{ov}\|$ is defined or an affine inner product space if there is the scalar $\vv{ov}\cdot\vv{ou}\,$. Since an affine subspace is itself an affine space, representation of triple $(\eamsd{S}{F}{a}{n},o,B)$ will be shortened by $\eamd{S}{F}{n}$, where the seed point $a$ and the coordinate system are implicit. Since we are dealing in this chapter with morphological matters, what are the geometric manifestations of metric, norm and inner product? We already said that metric refers to the concept of distance and thus scalar $\fua{\varrho}{u,v}$ informs the distance between arbitrary points $u$ and $v$ in the context of affine metric spaces. Norm however is not applicable to points, being geometrically represented by the size of a vector, which also depicts the concept of vector intensity: a ``bigger'' vector is considered to be a more ``intense'' vector. In the context of affine Banach spaces, where distance
\begin{equation}\label{eq:affDistance}
\fua{\varrho}{u,v} := \|\vv{ou} - \vv{ov}\|=\|\vv{vo} + \vv{ou}\|=\|\vv{vu}\|=\|\vv{uv}\|\,,
\end{equation}
norm enables distance measurement. Now, regarding the geometric effects of the inner product, we already said on section \ref{sec:espacoVet} that this product is closely related to vector incidence, being a scalar measure of this fundamental concept\footnote{See page \pageref{sec:incidence}.}. In the context of affine normed inner product spaces, the incidence interrelationship of non zero vectors $\vv{ov}$ and $\vv{ou}$ is expressed both by the \textsb{projection}\index{projection} $\fua{\zeta}{\vv{ov},\vv{ou}}$ of $\vv{ov}$ onto the line defined by the set $\spn{(\{\vv{ou}\})}$ and the projection $\fua{\zeta}{\vv{ou},\vv{ov}}$ of $\vv{ou}$ onto the line defined by $\spn{(\{\vv{ov}\})}$, where $\map{\zeta}{S_\cam{F}^2}{\cam{F}}$ whose function rule is
\begin{equation}\label{eq:projection}
\fua{\zeta}{\vv{ox},\vv{oy}}=\dfrac{\vv{ox}\cdot \vv{oy}}{\|\vv{oy}\|}
\end{equation}
and $S_\cam{F}$ defines an affine normed inner product space. Considering $\vv{ow}:=\fua{\zeta}{\vv{ox},\vv{oy}}\vv{oy}$, we can state that vector $\vv{ow}\in\spn{(\{\vv{oy}\})}$ where points $w$ and $y$ belong to the line defined by $\spn{(\{\vv{oy}\})}$. From the previous rule, if projection is known and not the inner product, we can write that $\vv{ox}\cdot \vv{oy}=\fua{\zeta}{\vv{ox},\vv{oy}}\|\vv{oy}\|$, which is usually and imprecisely regarded as the ``geometric definition'' of the inner product. It is important to note that if arguments $\vv{ox}$ and $\vv{oy}$ are orthogonal, then $\fua{\zeta}{\vv{ox},\vv{oy}}=0$.
 Moreover, in the context of affine Hilbert spaces, where $\|\vv{ox}\|^2=\vv{ox}\cdot \vv{ox}$, when $\vv{oy}=\alpha\vv{ox}$ then
\begin{equation}\label{eq:projColinear}
\fua{\zeta}{\vv{ox},\alpha\vv{ox}}=\dfrac{\overline{\alpha}(\vv{ox}\cdot \vv{ox})}{|\alpha|\|\vv{ox}\|}=\dfrac{\overline{\alpha}\|\vv{ox}\|}{|\alpha|}\,,\,\forall\alpha\in\{\cam{F}\setminus 0\}.
\end{equation}
For the case of affine Euclidean spaces, these equalities lead to
\begin{equation}\label{eq:projColinearEuclid}
\fua{\zeta}{\vv{ox},\alpha\vv{ox}}=\sgn{\alpha}\|\vv{ox}\|\,,\,\,\forall\,\alpha\in\{\real\setminus 0\}\,.
\end{equation}
Concerning projections on affine Hilbert spaces, there is a property, presented on the following theorem, that will be important for future geometric concepts.

\begin{mteo}{Modulus of Projection}{moduProj}
Given the arbitrary vectors $\vv{ox},\vv{oy}\in S_\cam{F}$, where $S_\cam{F}$ defines an affine Hilbert space $\eamd{S}{F}{n}$, the modulus of projection $\fua{\zeta}{\vv{ox},\vv{oy}}$ is never greater than the norm of the projected vector. In other words,
$|\fua{\zeta}{\vv{ox},\vv{oy}}|\leq \|\vv{ox}\|$.
\end{mteo}
\hspace{1pt}
{\footnotesize
\begin{proof}
From the Cauchy-Schwartz Inequality \eqref{eq:Cauchy-Schwartz} and the ``geometric definition'' of the inner product, the following development proves the theorem.
\begin{align*}
|\vv{ox}\cdot\vv{oy}|^2 &\leq (\vv{ox}\cdot\vv{ox})(\vv{oy}\cdot\vv{oy})\\
|\fua{\zeta}{\vv{ox},\vv{oy}}|^2\|\vv{oy}\|^2 &\leq \|\vv{ox}\|^2\|\vv{oy}\|^2\\
|\fua{\zeta}{\vv{ox},\vv{oy}}| &\leq \|\vv{ox}\|\,.
\end{align*}
\end{proof}}

Now, for an affine Euclidean space $\eamd{S}{R}{n}$, if $\theta$ is the smallest angle defined by vectors $\vv{ox},\vv{oy}\in S_\cam{R}$, the linear combination  $\vv{oz}:=\vartheta\vv{oy}/\|\vv{oy}\|$ and the scalar $\vartheta\in\real$ are called respectively the \textsb{vector and scalar geometric projections}\index{projection!vector}\index{projection!scalar} of $\vv{ox}$ onto $\vv{oy}$ when
\begin{equation}\label{eq:projgeo}
\vartheta:=\|\vv{ox}\|\cos\theta\,,
\end{equation}
which can be visualized in figure \ref{fg:geoProject}. In this context, the following corollary establishes a connection between geometric and algebraic projections. From this connection and rule \eqref{eq:projection}, the so called ``geometric definition'' of the Euclidean inner product results the classic expression
\begin{equation}\label{eq:geoProdInt}
\vv{ox}\cdot\vv{oy}=\|\vv{ox}\|\|\vv{oy}\|\cos\theta.
\end{equation}
From this equality and the scalar geometric projection $\vartheta$ of $\vv{ox}$ onto $\vv{oy}$, then
\begin{equation}
\fua{\zeta}{\vv{ox},\vv{oy}}=\dfrac{\vv{ox}\cdot \vv{oy}}{\|\vv{oy}\|} = \dfrac{\|\vv{ox}\|\|\vv{oy}\|\cos\theta }{\|\vv{oy}\|}= \|\vv{ox}\|\cos\theta=\vartheta\,.
\end{equation}

\begin{figure}[!ht]
	\centering
	\begin{center}
		\scalebox{.72}{\input{partes/figs/geoProject.pstex_t}}
	\end{center}
	\titfigura{Vector and scalar geometric projections.}\label{fg:geoProject}
\end{figure}

From the previous equality and definition \eqref{eq:projgeo}, if there is no geometric projection between $\vv{ox}$ and $\vv{oy}$, that is, $\vartheta=0$, we conclude that the angle $\theta=\pi(k+1/2)$, $k=0,1,..\,$. Thereby, if line $\eamsd{W}{\real}{b}{1}$ and plane $\eamsd{V}{\real}{a}{2}$, subspaces of affine Euclidean space $\eamd{S}{\real}{5}$, are perpendicular, we can represent them as depicted on figure \ref{fg:perpend}.
\begin{figure}[!ht]
	\centering
	\begin{center}
		\scalebox{.72}{\input{partes/figs/perpend.pstex_t}}
	\end{center}
	\titfigura{Line and plane perpendicular to each other.}\label{fg:perpend}
\end{figure}


\section{Affinities}
The subject of Geometry is to study not only morphological sets and their elements, as we have done on the previous section, but also the correlation of geometrical objects, that is, functions whose domain and image are morphological sets. Thereby, in the context of Affine Geometry, this section deals exclusively with affinities, here understood as ``geometric'' functions that correlate points, defined by linear vector functions on Hilbert spaces. In more precise terms, given a Hilbert affine space $\eamd{U}{F}{m}$, a bijective mapping $\map{\mathit{g}}{\eamd{U}{F}{m}}{\eamd{U}{F}{m}}$ and a vector space $\evl{\cam{F}}{U}{U}$, we call  $\map{\mathit{f}}{\eamd{U}{F}{m}}{\eamd{U}{F}{m}}$ an \textsb{affine transformation}\index{affine!transformation} and its function an \textsb{affinity}\index{affinity} when there is a unique linear bijection $\vtf{f}\in\evl{\cam{F}}{U}{U}$ where
\begin{equation}\label{eq:affty}
\fua{\mathit{f}}{\vto{x}\oplus a}=\fua{\vtf{f}}{\vto{x}}\oplus\fua{\mathit{g}}{a}\,,\,\,\forall\,a\in\eamd{U}{F}{m}\,,\,\vto{x}\in U_\cam{F}\,.
\end{equation}
It is straightforward to note that since $\vtf{f}$ and $\mathit{g}$ are bijections, so results affinity $\mathit{f}$. Among other properties, it is possible to obtain that every affinity preserves parallelism and dimensionality; in other words, parallel subspaces on the  domain remain parallel on the image and $n$-dimensional subspaces on the domain, $n\leq m$, remain $n$-dimensional on the image: for example, lines are mapped to lines, planes are mapped to planes and so on. From corollary \ref{cor:soTens} and equalities \eqref{eq:Lin2secTens}, we know that operator $\vtf{f}$ is the representative function $\tnr{\mathit{a}}$ of a tensor $\tnr{A}=\vto{a}_1\otimes\vto{a}_2\in\ete{\cam{F}}{U^2}$ in such a way that the previous definition can be rewritten as
\begin{equation}
\fua{\mathit{f}}{\vto{x}\oplus a}=\fua{\tnr{\mathit{a}}}{\vto{x}}\oplus\fua{\mathit{g}}{a}=(\vto{x}\cdot\vto{a}_1)\vto{a}_2\oplus\fua{\mathit{g}}{a}\,,\,\,\forall\,a\in\eamd{U}{F}{m}\,,\,\vto{x}\in U_\cam{F}\,,
\end{equation}
when $\tnr{A}$ is called an \textsb{affinity tensor}\index{tensor!afinity}\index{affinity!tensor}.


{\footnotesize
\begin{proof}
Let's prove that an affinity preserves parallelism and dimension of subspaces. Considering $\eamsd{S}{F}{a}{n}$ and $\eamsd{V}{F}{b}{r}$ parallel subspaces of $\eamd{U}{F}{m}$, where $S_\cam{F}\subseteq V_\cam{F}$, we can say that if $\{\vto{u}_1,\cdots,\vto{u}_n,\cdots,\vto{u}_r,\cdots,\vto{u}_m\}$ is a basis of $U_\cam{F}$, so are $\{\vto{u}_1,\cdots,\vto{u}_n\}$ and $\{\vto{u}_1,\cdots,\vto{u}_n,\cdots,\vto{u}_r\}$ bases of $S_\cam{F}$ and $V_\cam{F}$ respectively. From definition \eqref{eq:affty}, function $\vtf{f}$ is the sole responsible for preserving parallelism or not. Thereby, since $\vtf{f}$ is a bijective linear unary operator, we already know that $\{\fua{\vtf{f}}{\vto{u}_1},\cdots,\fua{\vtf{f}}{\vto{u}_n}\}$ and $\{\fua{\vtf{f}}{\vto{u}_1},\cdots,\fua{\vtf{f}}{\vto{u}_n},\cdots,\fua{\vtf{f}}{\vto{u}_r}\}$ are bases of $S_\cam{F}$ and $V_\cam{F}$ (see p. \pageref{pg:bijecOpe}) and then $S_\cam{F}\subseteq V_\cam{F}$ is preserved. From these same arguments, it is straightforward to conclude that $\vtf{f}$ does not change vector space dimensions.
\end{proof}}


Considering previous conditions, an affinity $\mathit{f}$ is said to be centered at point $a$, represented by $\mathit{f}_a$, when $\mathit{g}$ is the identity function, that is, when point $\fua{\gloref{ctrAff}}{\vto{x}\oplus a}=\fua{\vtf{f}}{\vto{x}}\oplus a$. From this equality, when $\fua{\vtf{f}}{\vto{x}}=\vto{u}+\vto{x}$, we have $\fua{\mathit{f}_a}{\vto{x}\oplus a}=\vto{u}\oplus(\vto{x}\oplus a)$ and then affinity $\mathit{f}_a$ is called a \textsb{translation}\index{translation}, represented by $\gloref{transLat}$, described by the rule
\begin{equation}\label{eq:translt}
\fua{\mathit{t}_\vto{u}}{x}=\vto{u}\oplus x
\end{equation}
and depicted in figure \ref{fg:translat}, where $\mathit{t}_\vto{u}$ preserves the shape of the subspace $\eamd{S}{F}{n}$.
\begin{figure}[!ht]
\centering
\begin{center}
\scalebox{.72}{\input{partes/figs/translat.pstex_t}}
\end{center}
\titfigura{Hyperplane $\eamd{S}{F}{n}$ is mapped to $\eamd{V}{F}{n}$ by translation $\mathit{t}_\vto{u}$.}\label{fg:translat}
\end{figure}
Given an arbitrary vector $\vto{v}\in U_\cam{F}$, from equality $\fua{\mathit{t}_\vto{v}\circ\mathit{t}_\vto{u}}{x}=(\vto{v}+\vto{u})\oplus x$,
it is straightforward to conclude that $\mathit{t}_\vto{v}\circ\mathit{t}_\vto{u}=\mathit{t}_{\vto{v}+\vto{u}}$. Moreover, if $\vto{v}=-\vto{u}$ we can affirm that $\mathit{t}_{-\vto{u}}=\mathit{t}_\vto{u}^{-1}$ and translation $\mathit{t}_\vto{0}$ is the identity function. Previous equality $\mathit{t}_\vto{v}\circ\mathit{t}_\vto{u}=\mathit{t}_{\vto{v}+\vto{u}}$ also allows us to conclude that the set $\mathit{T}$ of all translations on $\eamd{U}{F}{m}$ defines an abelian group $(\mathit{T},\circ)$ because the composition of translations clearly observes the axioms of
\begin{itemize}
\setlength\itemsep{.1em}
\item[i.] associativity, where $\mathit{t}_\vto{u}\circ\lpa \mathit{t}_\vto{v} \circ \mathit{t}_\vto{w} \rpa =
\lpa\mathit{t}_\vto{u}\circ \mathit{t}_\vto{v} \rpa\circ \mathit{t}_\vto{w} \, , \forall \, \vto{u},\vto{v},\vto{w}\in U_\cam{F}$,
\item[ii.] identity element, where $\mathit{t}_\vto{u}\circ\mathit{t}_\vto{0}=\mathit{t}_\vto{0}\circ\mathit{t}_\vto{u}=\mathit{t}_\vto{u}\,,\forall \, \vto{u} \in U_\cam{F}$,
\item[iii.] inverse element, where
$\mathit{t}_\vto{u} \circ  \mathit{t}_{-\vto{u}}= \mathit{t}_{-\vto{u}} \circ \mathit{t}_{\vto{u}} = \mathit{t}_{\vto{0}}$, $\forall \, \vto{u} \in \{U_\cam{F}\setminus\vto{0}\}$, and of
\item[iv.] commutativity, where $\mathit{t}_\vto{u}\circ \mathit{t}_\vto{v}= \mathit{t}_\vto{v}\circ \mathit{t}_\vto{u}\,,\forall \, \vto{u},\vto{v} \in U_\cam{F}$.
\end{itemize}


Still considering previous conditions, the affinity in $\map{\mathit{h}}{\eamd{U}{F}{m}}{\eamd{U}{F}{m}}$ is called a \textsb{dilation}\index{dilation} if bijection $\vtf{h}\in\evl{\cam{F}}{U}{U}$ in
\begin{equation}\label{eq:dilat}
\fua{\mathit{h}}{\vto{x}\oplus a}=\fua{\vtf{h}}{\vto{x}}\oplus\fua{\mathit{g}}{a}\,,\,\,\forall\,a\in\eamd{U}{F}{m}\,,\,\vto{x}\in U_\cam{F}\,,
\end{equation}
is a positive-definite Hermitian operator usually called \textsb{stretch operator}\index{operator!stretch}, which represents, according to inequality \eqref{eq:nonnegTensor}, a positive-definite affinity tensor $\tnr{H}$, called \textsb{stretch tensor}\index{tensor!stretch}. Since the eigenvalues $\lambda_1,\cdots,\lambda_m$ of stretch operator $\vtf{h}$, or of stretch tensor $\tnr{H}$, are positive real scalars\footnote{See section \ref{sec:autoPares}.}, we call $\lambda_i$ a \textsb{stretch coefficient}\index{stretch coefficient}. If coefficient $\lambda_i>1$ or $\lambda_i\leqslant1$, it is called an \textsb{expansion}\index{expansion} or a \textsb{contraction}\index{contraction} respectively and in the case of proportional stretch, where $\lambda_1=\cdots=\lambda_m$, function $\vtf{h}$ (or tensor $\tnr{H}$) is equally called an expansion\index{operator!expansion} or contraction operator(tensor)\index{operator!contraction}. It is important to note that, from corollary \ref{cor:soSymTens}, in the specific context of Euclidean spaces, where $\vtf{h}^\dagger=\vtf{h}^\text{T}$, the stretch tensor $\tnr{H}$ is symmetric. As any other affinity, a dilation $\mathit{h}$ can be centered at point $a$, represented by $\gloref{dilat}$, where $\fua{\mathit{h}_a}{\vto{u}\oplus a}=\fua{\vtf{h}}{\vto{u}}\oplus a$. Figure \ref{fg:dilation} shows different types of dilations.
\begin{figure}[!ht]
\centering
\begin{center}
\scalebox{.72}{\input{partes/figs/dilation.pstex_t}}
\end{center}
\titfigura{Affinities $\mathit{h}$, $\mathit{h}_a$ and $\mathit{s}$ are ordinary, centered and proportional dilations.}\label{fg:dilation}
\end{figure}
Given two arbitrary dilations $\mathit{h}_1$ and $\mathit{h}_2$, equalities
\begin{equation*}
\fua{\mathit{h}_1\circ\mathit{h}_2}{\vto{x}\oplus a}=\fua{\mathit{h}_1}{\fua{\vtf{h}_2}{\vto{x}}\oplus\fua{\mathit{g}_2}{a}}=\fua{\vtf{h}_1\circ\vtf{h}_2}{\vto{x}}\oplus\fua{\mathit{g}_1\circ\mathit{g}_2}{a}
\end{equation*}
prove that $\mathit{h}_1\circ\mathit{h}_2$ is also a dilation and that $\fua{\mathit{h}^{-1}_1}{\vto{x}\oplus a}=\fua{\vtf{h}^{-1}_1}{\vto{x}}\oplus\fua{\mathit{g}^{-1}_1}{a}$. Considering the identity dilation $\mathit{i}$, similarly to the case of translations, it is straightforward to verify that dilations also obey the axioms of associativity, identity element, inverse element and commutativity on the operation of composition. Therefore, the set $H$ of all dilations on $\eamd{U}{F}{m}$ defines the abelian group $(H,\circ)$.

\begin{mteo}{Decompositions of Dilation}{dilaDecomp}
Given a dilation $\mathit{h}$ and a translation $\mathit{t}_\vto{x}$ on an affine space $\eamd{U}{F}{m}$, if vector $\fua{\kappa}{a}:=\vv{\fua{\mathit{g}}{a}a}$ then
\begin{equation*}
\mathit{h} = \mathit{t}_{\fua{\kappa}{a}}\circ\mathit{h}_a=\mathit{h}_{\fua{\mathit{g}}{a}}\circ\mathit{t}_{\fua{\kappa}{a}}\,,\,\forall a\in\eamd{U}{F}{m}\,.
\end{equation*}
\end{mteo}

{\footnotesize
\begin{proof}
Development
\begin{equation*}
\fua{\mathit{h}}{\vto{x}\oplus a} = \fua{\vtf{h}}{\vto{x}}\oplus (\fua{\kappa}{a}\oplus a) = \fua{\kappa}{a}\oplus (\fua{\vtf{h}}{\vto{x}}\oplus a) = \fua{\mathit{t}_{\fua{\kappa}{a}}\circ\mathit{h}_a}{\vto{x}\oplus a}
\end{equation*}
proves the first decomposition and
\begin{equation*}
\fua{\mathit{h}}{\vto{x}\oplus a} = \fua{\vtf{h}}{\vto{x}}\oplus \fua{\mathit{g}}{a} = \fua{\mathit{h}_{\fua{\mathit{g}}{a}}}{\vto{x}\oplus \fua{\mathit{g}}{a}} = \fua{\mathit{h}_{\fua{\mathit{g}}{a}}}{(\fua{\kappa}{a}+\vto{x})\oplus a} = \fua{\mathit{h}_{\fua{\mathit{g}}{a}}\circ\mathit{t}_{\fua{\kappa}{a}}}{\vto{x}\oplus a}
\end{equation*}
proves the second.
\end{proof}}


Still considering previous conditions, an affinity $\mathit{r}$ is called \textsb{isometric}\index{affinity!isometric} when its correspondent linear operator $\vtf{r}$ is isometric, that is, $\vtf{r}\in\grc{I}{\cam{F}}{U}$. In the context of Euclidean spaces, we already know that an isometric operator is also an orthogonal operator\footnote{See p. \pageref{prg:IsometOrth}.}, that is, the isometric group $\grc{I}{\real}{U}$ equals the orthogonal group $\grc{O}{\real}{U}$, whose elements preserve inner product and have determinant $\pm 1$. The centered isometric affinity $\mathit{r}_a$, described by $\fua{\mathit{r}_a}{\vto{x}\oplus a}=\fua{\vtf{r}}{\vto{x}}\oplus a$, is called a \textsb{rotation}\index{rotation} if isometric operator $\vtf{r}$ is proper orthogonal, or $\vtf{r}\in\grc{O^+}{\real}{U}$; otherwise, it is called a \textsb{rotoreflection}\index{rotoreflection}. The affinity tensor $\tnr{R}$ of a rotation is called a \textsb{rotation tensor}\index{rotation!tensor} and, similarly, tensor $\overline{\tnr{R}}$ of a rotoreflection is called a \textsb{rotoreflection tensor}\index{rotoreflection!tensor}. Now, recalling the concept of unimodular tensors presented in section \ref{sec:TensSpac}, let $(U_\real,\tnr{A}_B)$ be an $m$-dimensional oriented Euclidean space where $\tnr{A}_B\in\etng{m}{\real}{U^m}$ and $B=\{\vun{u}_1,\cdots,\vun{u}_m\}$. Considering $\mathit{r}_a$ a rotation and $\overline{\mathit{r}}_a$ a rotoreflection, since every pair of Euclidean orthonormal basis can be related through an orthogonal operator\footnote{See p. \pageref{prg:orthBases}.}, we conclude that rotation  $C=\{\fua{\vtf{r}}{\vun{u}_1},\cdots,\fua{\vtf{r}}{\vun{u}_m}\}$ and rotoreflection $\overline{C}=\{\fua{\overline{\vtf{r}}}{\vun{u}_1},\cdots,\fua{\overline{\vtf{r}}}{\vun{u}_m}\}$ of basis $B$ are positively and negatively oriented bases because $\fua{\tnr{A}_C}{\vun{u}_1,\cdots,\vun{u}_m}=-\fua{\tnr{A}_{\overline{C}}}{\vun{u}_1,\cdots,\vun{u}_m}=1$ from definition \eqref{eq:basisAltern} and equality \eqref{eq:alternDete}. In order to geometrically represent in a simple way basis rotation and rotoreflection, we specify the affine Euclidean space in question to be two-dimensional, that is, $m=2$, and rotoreflection $\overline{\vtf{r}}=-\vtf{r}$. Thereby, for the basis vectors $\vun{u}_1$ and $\vun{u}_2$ the smallest angle $\theta$ between them leads to equalities $0=\vun{u}_1\cdot\vun{u}_2=\|\vun{u}_1\|\|\vun{u}_2\|\cos\theta=\cos\theta$, and since orthogonal functions preserve inner product, then $\theta$ is also the smallest angle between $\fua{\vtf{r}}{\vun{u}_1}$ and $\fua{\vtf{r}}{\vun{u}_2}$, or between $\fua{\overline{\vtf{r}}}{\vun{u}_1}$ and $\fua{\overline{\vtf{r}}}{\vun{u}_2}$, because the following equalities are all valid: $\vun{u}_1\cdot\vun{u}_2=\fua{\vtf{r}}{\vun{u}_1}\cdot\fua{\vtf{r}}{\vun{u}_2}=\fua{\overline{\vtf{r}}}{\vun{u}_1}\cdot\fua{\overline{\vtf{r}}}{\vun{u}_2}=0$. From these angle and size preserving features, we can represent an anticlockwise acute angle $\phi$ rotation and rotoreflection of basis $B$ according to figure \ref{fg:rotoReflex}.
\begin{figure}[!ht]
\centering
\begin{center}
\scalebox{.72}{\input{partes/figs/rotoReflex.pstex_t}}
\end{center}
\titfigura{Two-dimensional rotoreflexion and rotation.}\label{fg:rotoReflex}
\end{figure}
We can verify that the function on the right of this figure is indeed a rotation and on the left a rotoreflection by proving that the rotated and reflected bases are positively and negatively oriented respectively in relation to $(U_\real,\tnr{A}_B)$, that is, by calculating $\fua{\tnr{A}_C}{\vun{u}_1,\vun{u}_2}$ and $\fua{\tnr{A}_{\overline{C}}}{\vun{u}_1,\vun{u}_2}$. From definition \eqref{eq:basisAltern} and equality \eqref{eq:geoProdInt}, development
\begin{align*}
\fua{\tnr{A}_{C}}{\vun{u}_1,\vun{u}_2}&=\sum_{i=1}^2\sum_{j=1}^2\fua{\vtf{f}^\con{C}_{i}}{\vun{u}_1}\fua{\vtf{f}^\con{C}_{j}}{\vun{u}_2}\epsilon_{ij}\\
&=(\vun{u}_1\cdot\fua{\vtf{r}}{\vun{u}_1})(\vun{u}_2\cdot\fua{\vtf{r}}{\vun{u}_2})-(\vun{u}_1\cdot\fua{\vtf{r}}{\vun{u}_2})(\vun{u}_2\cdot\fua{\vtf{r}}{\vun{u}_1})\\
&=(cos\phi cos\phi) - (-\sin\phi\sin\phi)=1
\end{align*}
proves that $C$ is positively oriented and, similarly, $\overline{C}$ results negatively oriented.


Still considering  the oriented Euclidean space $(U_\real,\tnr{A}_B)$ where $B=\{\vun{u}_1,\cdots,\vun{u}_m\}$, the \textsb{simple bivector} $z$ defined by vectors $\vto{x},\vto{y}\in U_\real$ is here understood as the mathematical object, endowed with magnitude and orientation, used to express rotational physical quantities such as angular momentum and torque. In geometrical terms, the magnitude of this simple bivector $z$ is defined to be the area of the parallelogram constructed from line segments $\overline{a\,\vto{x}\oplus a}$ and $\overline{a\,\vto{y}\oplus a}$ while its orientation is related to $\tnr{A}_B$ and to rotating $\overline{a\,\vto{x}\oplus a}$ to $\overline{a\,\vto{y}\oplus a}$, which is the inverse of rotating $\overline{a\,\vto{y}\oplus a}$ to $\overline{a\,\vto{x}\oplus a}$ for the case of the simple bivector $-z$. It is important to say that this new entity is here loosely defined because it is a particular case of a more general concept called multivector, which is out of the scope of this elementary study. Nevertheless, to our good fortune, this simple bivector can be represented as a vector if $\dim(U_\real)=3$, when it is called an \textsb{axial vector}\index{vector!axial}. In this context, considering an axial vector $\vto{z}$ defined by the simple bivector $z$, where $\{\vto{x},\vto{y}\}$ is linearly independent, line $(\{\vto{z}\}_a)_\real^1$ is defined to be perpendicular to the plane $(\{\vto{x},\vto{y}\}_a)_\real^2$ and vector $\vv{a\,\vto{z}\oplus a}$ must be oriented in such a way that basis $\{\vto{z},\vto{x},\vto{y}\}$ is positively oriented, that is, $\fua{\tnr{A}_B}{\vto{z},\vto{x},\vto{y}}>0$.
\begin{figure}[!ht]
\centering
\begin{center}
\scalebox{.72}{\input{partes/figs/axial.pstex_t}}
\end{center}
\titfigura{Axial vector $\vto{z}$ defined by $\vto{x}$, $\vto{y}$ and oriented Euclidean space $(U_\real,\tnr{A}_B)$, where basis $B=\{\vun{u}_1,\vun{u}_2,\vun{u}_3\}$.}\label{fg:axial}
\end{figure}
There is a classic mnemonic rule for obtaining the direction of $\vv{a\,\vto{z}\oplus a}$ that works as follows: considering my right hand, I must point my index and middle fingers to the same directions of $\vv{o\,\vun{u}_1\oplus o}$ and $\vv{o\,\vun{u}_2\oplus o}$ respectively, and then if my thumb points to the same direction of $\vv{o\,\vun{u}_3\oplus o}$, then space $(U_\real,\tnr{A}_B)$ is oriented according to the \textsb{right-hand rule}\index{right-hand rule}; on the other hand, literally, space $(U_\real,\tnr{A}_B)$ is oriented according to the \textsb{left-hand rule}\index{left-hand rule}. Once a hand is found, if I point my index and middle fingers to the same directions of $\vv{a\,\vto{x}\oplus a}$ and $\vv{a\,\vto{y}\oplus a}$ respectively, then the direction of my thumb defines the direction of $\vv{a\,\vto{z}\oplus a}$. For practical purposes, a three-dimensional Euclidean affine space is usually oriented according to the right-hand rule. An axial vector is always the result of a \textsb{cross product}\index{product!cross} of two vectors, that is, $\vto{z}=\gloref{crossProd}$ in the present case, where
\begin{equation}\label{eq:crossProduct}
\vto{x}\times\vto{y} := \tnr{A}_B\odot_2 (\vto{x}\otimes\vto{y})\,.
\end{equation}
For arbitrary vectors $\vto{u},\vto{v},\vto{w}\in U_\real$ and scalars $\alpha,\beta,\theta\in\real$, where $\theta$ is the smallest angle defined by vectors $\vv{ou},\vv{ov}$, the cross product has the following properties.
\begin{itemize}
	\setlength\itemsep{.1em}
	\item[i.] Mixed products: $\vto{u}\cdot(\vto{v}\times\vto{w})=\fua{\tnr{A}_B}{\vto{u},\vto{v},\vto{w}}$;
	\item[ii.] Anticommutativity: $\vto{u}\times\vto{v}=-(\vto{v}\times\vto{u})$;
	\item[iii.] Self cross product: $\vto{u}\times\vto{u}=\vto{0}$;
	\item[iv.] Distributivity over addition: $\vto{u}\times(\vto{v}+\vto{w})=(\vto{u}\times\vto{v})+(\vto{u}\times\vto{v})$;
	\item[v.] Scalar multiplication: $\alpha\beta(\vto{u}\times\vto{v})=(\alpha\vto{u})\times(\beta\vto{v})=(\beta\vto{u})\times(\alpha\vto{v})$;
	\item[vi.] Positive orientation: $\tnr{A}_B(\vto{u}\times\vto{v},\vto{u},\vto{v})>0$;
	\item[vii.] Orthogonality: $(\vto{u}\times\vto{v})\cdot\vto{u}=(\vto{u}\times\vto{v})\cdot\vto{v}=0$;
	\item[viii.] Triple cross product: $\vto{u}\times(\vto{v}\times\vto{w})=(\vto{u}\cdot\vto{w})\vto{v}-(\vto{u}\cdot\vto{v})\vto{w}$;
	\item[ix.] Area of parallelogram: $\|\vto{u}\times\vto{v}\|= \|\vto{u} \|\|\vto{v} \|\sin\theta$.
\end{itemize}

{\footnotesize
\begin{proof}
Item i is verified by  
\begin{equation*}
\vto{u}\cdot(\vto{v}\times\vto{w}) = \vto{u}\cdot[\tnr{A}_B\odot_2 (\vto{v}\otimes\vto{w})] = \tnr{A}_B\odot_3(\vto{u}\otimes\vto{v}\otimes\vto{w}) =\fua{\tnr{A}_B}{\vto{u},\vto{v},\vto{w}}\,.  
\end{equation*}
Given an arbitrary vector $\vto{x}\in U_\real$ and $\tnr{A}_B=\vto{a}\otimes\vto{b}\otimes\vto{c}$, item ii is verified by the following equalities:
\begin{equation*}
\vto{x}\cdot(\vto{u}\times\vto{v})= (\vto{b}\cdot\vto{u})(\vto{c}\cdot\vto{v})(\vto{x}\cdot\vto{a})= \fua{\tnr{A}_B}{\vto{x},\vto{u},\vto{v}}=-\fua{\tnr{A}_B}{\vto{x},\vto{v},\vto{u}}=-\vto{x}\cdot(\vto{v}\times\vto{u})\,.
\end{equation*}
Item iii is a straightforward corollary of item ii. We prove item iv similarly:
\begin{equation*}
\vto{x}\cdot[\vto{u}\times(\vto{v}+\vto{w})]= \fua{\tnr{A}_B}{\vto{x},\vto{u},\vto{v}+\vto{w}}=\fua{\tnr{A}_B}{\vto{x},\vto{v},\vto{u}}+\fua{\tnr{A}_B}{\vto{x},\vto{v},\vto{w}}=\vto{x}\cdot[(\vto{v}\times\vto{u})+(\vto{v}\times\vto{w})]\,.
\end{equation*}
By this same procedure, proof of property v is trivial. Now, property vi is verified by expression
\begin{equation*}
\tnr{A}_B(\vto{u}\times\vto{v},\vto{u},\vto{v})=\tnr{A}_B((\vto{b}\cdot\vto{u})(\vto{c}\cdot\vto{v})\vto{a},\vto{u},\vto{v})=(\vto{a}\cdot\vto{a})^2(\vto{b}\cdot\vto{u})^2(\vto{c}\cdot\vto{v})^2 > 0
\end{equation*}
and, since $\tnr{A}_B$ is antisymmetric, equalities $(\vto{u}\times\vto{v})\cdot\vto{u}=\tnr{A}_B(\vto{u},\vto{u},\vto{v})=0$ prove item vii. Considering an arbitrary orthonormal basis $X=\{\vun{x}_1,\vun{x}_2,\vun{x}_3\}$, equality \eqref{eq:coordAltern} and identity \eqref{eq:LeviKronecker}, development
\begin{align*}
\vto{u}\times(\vto{v}\times\vto{w})&=\tnr{A}_B\odot_2 [\vto{u}\otimes\tnr{A}_B\odot_2 (\vto{v}\otimes\vto{w})]\\
&=\sum_{i,j,k=1}^{m}\epsilon_{ijk}\vun{x}_i^*\otimes\vun{x}_j^*\otimes\vun{x}_k^*\odot_2 [\sum_{j=1}^{m}\fua{\vto{f}^X_j}{\vto{u}}\vun{x}_j\otimes\sum_{k,r,s=1}^{m}\epsilon_{krs}\fua{\vto{f}^X_r}{\vto{v}}\fua{\vto{f}^X_s}{\vto{w}}\vun{x}_k]\\
&=\sum_{i,j,k=1}^{m}\sum_{r,s=1}^{m}-\epsilon_{kji}\epsilon_{krs}\fua{\vto{f}^X_j}{\vto{u}}\fua{\vto{f}^X_r}{\vto{v}}\fua{\vto{f}^X_s}{\vto{w}}\vun{x}_i\\
&=\sum_{i,j,k,r,s=1}^{m}(\delta_{js}\delta_{ir}-\delta_{jr}\delta_{is})\fua{\vto{f}^X_j}{\vto{u}}\fua{\vto{f}^X_r}{\vto{v}}\fua{\vto{f}^X_s}{\vto{w}}\vun{x}_i\\
&=\sum_{i,j,k,r,s=1}^{m}\fua{\vto{f}^X_s}{\vto{u}}\fua{\vto{f}^X_i}{\vto{v}}\fua{\vto{f}^X_s}{\vto{w}}\vun{x}_i-\fua{\vto{f}^X_r}{\vto{u}}\fua{\vto{f}^X_r}{\vto{v}}\fua{\vto{f}^X_i}{\vto{w}}\vun{x}_i\\
&=(\vto{u}\cdot\vto{w})\vto{v}-(\vto{u}\cdot\vto{v})\vto{w}
\end{align*}
proves property viii. Now, in order to prove the last item, equality $\vto{u}\cdot\tnr{A}_B\odot_2\vto{v}\otimes\vto{w}=\tnr{A}_B(\vto{u},\vto{v},\vto{w})$, property ix and equality \eqref{eq:geoProdInt} are required. Then,
\begin{align*}
\|\vto{u}\times\vto{v}\|^2 &= (\vto{u}\times\vto{v})\cdot(\vto{u}\times\vto{v})\\
&= \tnr{A}_B(\vto{u}\times\vto{v},\vto{u},\vto{v})\\
&= -\tnr{A}_B(\vto{v},\vto{u},\vto{u}\times\vto{v})\\
&= -\vto{v}\cdot(\vto{u}\times(\vto{u}\times\vto{v}))\\
&= \vto{v}\cdot[(\vto{u}\cdot\vto{u})\vto{v}-(\vto{u}\cdot\vto{v})\vto{u}]\\
&= \|\vto{u}\|^2\|\vto{v}\|^2-(\vto{u}\cdot\vto{v})^2=\|\vto{u}\|^2\|\vto{v}\|^2-\|\vto{u}\|^2\|\vto{v}\|^2\cos^2\theta=\|\vto{u}\|^2\|\vto{v}\|^2\sin^2\theta\,.
\end{align*}
\end{proof}}

Considering the last property above, the geometrical definition of inner product \eqref{eq:geoProdInt} and arbitrary vectors $\vto{x},\vto{y},\vto{z}\in U$, if $\theta_1$ is the smallest angle defined by $\overline{a\vto{x}\oplus a}$ and $\overline{a(\vto{y}\times\vto{z})\oplus a}$, and angle $\theta_2$ defined by  $\overline{a\vto{y}\oplus a}$ and $\overline{a\vto{z}\oplus a}$, scalar
\begin{equation}
\fua{\tnr{A}_B}{\vto{x},\vto{y},\vto{z}}=\vto{x}\cdot(\vto{y}\times\vto{z})=\|\vto{x}\|(\|\vto{y} \|\|\vto{z} \|\sin\theta_2)\cos\theta_1=\underbrace{\|\vto{y} \|\|\vto{z} \|\sin\theta_2}_{\alpha} \underbrace{\|\vto{x}\|\cos\theta_1}_{\beta}\,,
\end{equation}
where $\alpha$ is the area of parallelogram defined by $\overline{a\vto{y}\oplus a}$ and $\overline{a\vto{z}\oplus a}$, as we already know. Moreover, it is possible to conclude that from vector $\vto{h}=\beta(\vto{y}\times\vto{z})/\|\vto{y}\times\vto{z}\|$, the line segments $\overline{a\vto{x}\oplus a}$, $\overline{a\vto{y}\oplus a}$ and $\overline{a\vto{z}\oplus a}$ define a parallelepiped of height $\overline{a\vto{h}\oplus a}$ and volume $\alpha\beta$, represented in figure \ref{fg:paralelogramo}.
\begin{figure}[!ht]
\centering
\begin{center}
\scalebox{.72}{\input{partes/figs/paralelogramo.pstex_t}}
\end{center}
\titfigura{Parallelogram of volume $\alpha\beta$ defined by $\vto{x},\vto{y},\vto{z}$.}\label{fg:paralelogramo}
\end{figure}
Considering definition \eqref{eq:detTensorSeg2} in the present context of a three-di\-men\-sional Euclidean space and $\{\vto{x},\vto{y},\vto{z}\}\subset U$ a linearly independent set, for a second order tensor $\tnr{T}\in\ete{\cam{F}}{U^2}$ equalities
\begin{equation}
\hdet{\tnr{T}}=\dfrac{\fua{\tnr{A}_B}{\fua{\tnr{\mathit{t}}}{\vto{x}},{\fua{\tnr{\mathit{t}}}{\vto{y}}},{\fua{\tnr{\mathit{t}}}{\vto{z}}}}}{\fua{\tnr{A}_B}{\vto{x},\vto{y},\vto{z}}}=\dfrac{\fua{\tnr{\mathit{t}}}{\vto{x}}\cdot[\,\fua{\tnr{\mathit{t}}}{\vto{y}}\times\fua{\tnr{\mathit{t}}}{\vto{z}}\,]}{\vto{x}\cdot(\vto{y}\times\vto{z})}
\end{equation}
show that $\hdet{\tnr{T}}$ measures the change of volume of a parallelogram ``changed'' by $\tnr{T}$.


%    \chapter{Diferenciação de Funções Tensoriais}

\section{Diferenciabilidade de Funções Tensoriais}

\subsection{Função Tensorial Limitada}\index{função!tensorial!limitada}
Sejam os espaços tensoriais normados $\ete{\crt{V}{p}}{\con{F}}$ e
$\ete{\crt{W}{q}}{\con{F}}$ com as respectivas normas
$\|\bullet\|_{\ete{\crt{V}{p}}{\con{F}}}$ e
$\|\bullet\|_{\ete{\crt{W}{q}}{\con{F}}}$. A função no mapeamento
\begin{equation}\label{eq:mapFuncaoLimitada}
\map{\psi}{{T}_{\con{\crt{V}{p}}\mapsto\con{\con{F}}}}{\cft{\crt{W}{q}}{\con{F}}}
\end{equation}
é dita limitada em  ${T}_{\con{\crt{V}{p}}\mapsto\con{\con{F}}}$
se existir uma constante $\Lambda\in\con{F}$ tal que
\begin{equation}
\|\fua{\psi}{\tnr{X}}\|_{\ete{\crt{W}{q}}{\con{F}}}\leqslant\Lambda\|\tnr{X}\|_{\ete{\crt{V}{p}}{\con{F}}}\,,\,
\forall\,\tnr{X}\in\cft{\crt{V}{p}}{\con{F}}\,.
\end{equation}
Das possíveis constantes $\Lambda$ que obedecem a desigualdade
anterior, seja $\lambda$ a menor delas. Nesta situação, para a
função $\psi$, é possível definir que
\begin{equation}
\| \psi \| = \lambda\,.
\end{equation}
O conjunto de funções limitadas é portanto normado. Além disso, se
os espaços tensoriais $\ete{\crt{V}{p}}{\con{F}}$ e
$\ete{\crt{W}{q}}{\con{F}}$ forem de Banach, toda função limitada
linear é contínua.
\newline

\noindent\begin{prova} Vamos demonstrar agora se a última
afirmação é verídica. Para tal, suponhamos uma seqüência de Cauchy
qualquer $\lim_{i\to\infty}{\fua{\varrho}{\tnr{X}_i,\tnr{T}_0}}=0$
em ${T}_{\con{\crt{V}{p}}\mapsto\con{\con{F}}}$. Isto indica que
$\tnr{X}_i\to\tnr{T}_0$, quando $i\to\infty$. Então, pode-se dizer
que $\lim_{i\to\infty}{\| \tnr{X}_i-\tnr{T}_0 \|}=0$. Com base
nisso e considerando a função em ($\ref{eq:mapFuncaoLimitada}$)
limitada linear, a desigualdade
\begin{equation}
{\|\fua{\psi}{\tnr{X}_i}-\fua{\psi}{\tnr{T}_0}\|}=
{\|\fua{\psi}{\tnr{X}_i-\tnr{T}_0}\|}\leqslant\|\psi\|\|
\tnr{X}_i-\tnr{T}_0\|\,\nonumber
\end{equation}
quando $i\to\infty$, permite concluir que
$\fua{\psi}{\tnr{X}_i}\to\fua{\psi}{\tnr{T}_0}$\,.
\end{prova}



\subsection{Funções Tensoriais
Tangentes}\index{funções!tensoriais!tangentes} Dados os espaços
tensoriais de Banach $\ete{\crt{V}{p}}{\con{F}}$ e
$\ete{\crt{W}{q}}{\con{F}}$ com as respectivas normas
$\|\bullet\|_{\ete{\crt{V}{p}}{\con{F}}}$ e
$\|\bullet\|_{\ete{\crt{W}{q}}{\con{F}}}$, seja
$\con{\tilde{T}}_{\con{\crt{V}{p}}\mapsto\con{\con{F}}}$ um
subconjunto aberto de $\cft{\crt{V}{p}}{\con{F}}$ e o mapeamento
\begin{equation}
\map{\psi}{{\tilde{T}}_{\con{\crt{V}{p}}\mapsto\con{\con{F}}}}{\cft{\crt{W}{q}}{\con{F}}}\,.
\end{equation}
Considerando um tensor
$\tnr{T}_0\in{\tilde{T}}_{\con{\crt{V}{p}}\mapsto\con{\con{F}}}$,
toda e qualquer função $\kappa$, tal que
\begin{equation}
\map{\kappa}{{\tilde{T}}_{\con{\crt{V}{p}}\mapsto\con{\con{F}}}}{\cft{\crt{W}{q}}{\con{F}}}\,,
\end{equation}
é dita tangente à $\psi$ em $\tnr{T}_0$ e vice-versa se, dados o
escalar $\alpha\in\con{F}$ e um tensor qualquer não nulo
$\tnr{H}\in{\tilde{T}}_{\con{\crt{V}{p}}\mapsto\con{\con{F}}}$ ,
\begin{equation}\label{eq:funcaoTangenteGateaux}
\lim_{\alpha\to 0}\frac{\| \fua{\psi}{\alpha\tnr{H}+\tnr{T}_0} -
\fua{\kappa}{\alpha\tnr{H}+\tnr{T}_0}
 \|_{\ete{\crt{W}{q}}{\con{F}}}}
{\|\alpha\tnr{H}\|_{\ete{\crt{V}{p}}{\con{F}}}}=0\,.
\end{equation}
Em outras palavras, o valor no numerador aproxima-se de zero de
forma mais rápida do que no denominador quando o tensor
$\alpha\tnr{H}\to\negmath{0}$, segundo a ``direção'' definida por
$\tnr{H}$. Nestas condições, fica evidente que,
no limite, $\fua{\psi}{\tnr{T}_0}=\fua{\kappa}{\tnr{T}_0}$. Além disso, se
duas funções são tangentes a $\psi$ em $\tnr{T}_0$, então elas são
tangentes entre si.
\newline

\noindent\begin{prova} Para provar a afirmação anterior, sejam
duas funções $\kappa_1$ e $\kappa_2$ tangentes a $\psi$ em
$\tnr{T}_0$. Como a soma dos limites é o limite da soma, tem-se
que
\begin{equation}
\lim_{\alpha\to 0}\frac{\| \fua{\psi}{\alpha\tnr{H}+\tnr{T}_0} -
\fua{\kappa_1}{\alpha\tnr{H}+\tnr{T}_0}
 \|_{\ete{\crt{W}{q}}{\con{F}}}+\| \fua{\kappa_2}{\alpha\tnr{H}+\tnr{T}_0}- \fua{\psi}{\alpha\tnr{H}+\tnr{T}_0}
 \|_{\ete{\crt{W}{q}}{\con{F}}}}
{\|\alpha\tnr{H}\|_{\ete{\crt{V}{p}}{\con{F}}}}=0\,. \nonumber
\end{equation}
A partir da desigualdade triangular, é certo dizer que o valor do
numerador
\begin{eqnarray}
  \lefteqn{\| \fua{\psi}{\alpha\tnr{H}+\tnr{T}_0} -
\fua{\kappa_1}{\alpha\tnr{H}+\tnr{T}_0}
 \|_{\ete{\crt{W}{q}}{\con{F}}}
+} & & \nonumber\\
  &
&\| \fua{\kappa_2}{\alpha\tnr{H}+\tnr{T}_0}-
\fua{\psi}{\alpha\tnr{H}+\tnr{T}_0}
 \|_{\ete{\crt{W}{q}}{\con{F}}}\geqslant\| \fua{\kappa_2}{\alpha\tnr{H}+\tnr{T}_0} -
\fua{\kappa_1}{\alpha\tnr{H}+\tnr{T}_0}
\|_{\ete{\crt{W}{q}}{\con{F}}}\,.\nonumber
\end{eqnarray}
Tal desigualdade permite concluir que
\begin{equation}
\lim_{\alpha\to 0}\frac{\| \fua{\kappa_2}{\alpha\tnr{H}+\tnr{T}_0}
- \fua{\kappa_1}{\alpha\tnr{H}+\tnr{T}_0}
 \|_{\ete{\crt{W}{q}}{\con{F}}}}
{\|\alpha\tnr{H}\|_{\ete{\crt{V}{p}}{\con{F}}}}=0\,. \nonumber
\end{equation}
\end{prova}

\subsubsection{Abordagens Forte e
Fraca}\index{funções!tensoriais!tangentes!fortes} Considerando as
condições anteriores, seja o conjunto $\con{TG}_{\tnr{T}_0}$
formado por pares ordenados de funções tangentes entre si em
$\tnr{T}_0$, cujo domínio é um subconjunto aberto de $\cft{\crt{V}{p}}{\con{F}}$ que contém $\tnr{T}_0$. Pode ocorrer que exista um conjunto
$\overline{\con{TG}}_{\tnr{T}_0}\subseteq\con{TG}_{\tnr{T}_0}$,
tal que seus pares de funções sejam tangentes em $\tnr{T}_0$
independente da forma como o tensor $\alpha\tnr{H}\to\negmath{0}$ em
(\ref{eq:funcaoTangenteGateaux}). Em outras palavras, considerando
$\tnr{Y}:=\alpha\tnr{H}$, para que
$(\psi,\kappa)\in\overline{\con{TG}}_{\tnr{T}_0}$, deve ser válida
a igualdade:
\begin{equation}\label{eq:funcaoTangenteFrechet}
\lim_{\tnr{Y}\to \negmath{0}}\frac{\| \fua{\psi}{\tnr{Y}+\tnr{T}_0} -
\fua{\kappa}{\tnr{Y}+\tnr{T}_0}
 \|_{\ete{\crt{W}{q}}{\con{F}}}}
{\|\tnr{Y}\|_{\ete{\crt{V}{p}}{\con{F}}}}=0\,.
\end{equation}
Pode-se observar que esta condição é mais restritiva ou mais
\emph{forte} do que a condição (\ref{eq:funcaoTangenteGateaux}). A
primeira é então denominada abordagem forte para funções
tensoriais tangentes enquanto a segunda é denominada abordagem
fraca\index{funções!tensoriais!tangentes!fracas}. É importante
enfatizar a diferença entre as duas abordagens: utilizando uma
terminologia geométrica, na abordagem fraca, o ``caminho''
trilhado pela variável no domínio
${\tilde{T}}_{\con{\crt{V}{p}}\mapsto\con{\con{F}}}$ é definido
pelo tensor $\tnr{H}$ arbitrado; na abordagem forte, arbitra-se o
próprio ``caminho'' a ser trilhado. Para fins didáticos, os
``caminhos'' em ambas as abordagens estão representados na figura
a seguir.

\begin{figure}[!htt]
\centering
\includegraphics{partes/parte1/figs/c_diftens/AbordagemForteFraca.eps}
\titfigura{À esquerda, abordagem fraca com tensores arbitrários e
``caminhos'' radiais. À direita, abordagem forte com ``caminhos''
arbitrários.}
\end{figure}

Cabe ressaltar que se o domínio tensorial das funções tangentes
for unidimensional (independente da ordem), as abordagens forte e
fraca ficam idênticas. Desta forma,
\begin{eqnarray}
\dim\lpa\ete{\crt{V}{p}}{\con{F}}\rpa=1&\implies &
\overline{\con{TG}}_{\tnr{T}_0}=\con{TG}_{\tnr{T}_0}\,.\nonumber
\end{eqnarray}

\noindent\begin{prova} Para o caso do espaço tensorial unidimensional $\ete{\crt{V}{p}}{\con{F}}$, seja o conjunto $\lch\tnr{H}\rch$ uma de suas bases. Neste contexto, o tensor $\tnr{Y}$ em (\ref{eq:funcaoTangenteFrechet}) pode sempre ser escrito como a combinação linear $\alpha\tnr{H}$, resultando assim a expressão (\ref{eq:funcaoTangenteGateaux}).
\end{prova}

\subsection{Diferenciabilidade de Gâteaux}\index{Gâteaux!diferenciabilidade de}
Sejam os espaços tensoriais de Banach $\ete{\crt{V}{p}}{\con{F}}$
e $\ete{\crt{W}{q}}{\con{F}}$ e o espaço vetorial de funções
tensoriais lineares
\begin{equation}
\evl{\cft{\crt{V}{p}}{\con{F}}}{\cft{\crt{W}{q}}{\con{F}}}{\con{F}}\,.\nonumber
\end{equation}
Considerando ${\tilde{T}}_{\con{\crt{V}{p}}\mapsto\con{\con{F}}}$
um subconjunto aberto de $\cft{\crt{V}{p}}{\con{F}}$, seja um
tensor
$\tnr{T}_0\in{\tilde{T}}_{\con{\crt{V}{p}}\mapsto\con{\con{F}}}$ e
o conjunto dos pares ordenados de funções tensoriais tangentes
$\con{TG}_{\tnr{T}_0}$. Dado o mapeamento
\begin{equation}
\map{\psi}{{\tilde{T}}_{\con{\crt{V}{p}}\mapsto\con{\con{F}}}}{\cft{\crt{W}{q}}{\con{F}}}\,,
\end{equation}
se existir o par de funções tangentes
$(\psi,\psi_D)\in\con{TG}_{\tnr{T}_0}$, onde
\begin{equation}\label{eq:funcaoGDerivada}
\fua{\psi_D}{\tnr{X}}=\fua{\psi}{\tnr{T}_0}+\fua{\lco\fua{\dvt{\psi}}{\tnr{T}_0}\rco}{\tnr{X}-\tnr{T}_0}\,,
\end{equation}
tal que a função
\begin{equation}
\lco\fua{\dvt{\psi}}{\tnr{T}_0}\rco\in
\cfl{\cft{\crt{V}{p}}{\con{F}}}{\cft{\crt{W}{q}}{\con{F}}}
\end{equation}
é limitada, então este par é único. Nestas condições, a função
$\psi$ é dita \emph{diferenciável de Gâteaux}\index{Gâteaux!função
diferenciável de} ou
\emph{G-diferenciável}\index{função!G-diferenciável} em
$\tnr{T}_0$. Combinando (\ref{eq:funcaoGDerivada}) com
(\ref{eq:funcaoTangenteGateaux}), obtém-se que
\begin{equation}\label{eq:derivadaGateauxIni}
\lim_{\alpha\to 0}\frac{\|
\fua{\psi}{\alpha\tnr{\tnr{H}}+\tnr{T}_0} - \fua{\psi}{\tnr{T}_0}-
\alpha\fua{\lco\fua{\dvt{\psi}}{\tnr{T}_0}\rco}{\tnr{H}}
 \|_{\ete{\crt{W}{q}}{\con{F}}}}
{\|\alpha\tnr{H}\|_{\ete{\crt{V}{p}}{\con{F}}}}=0\,.
\end{equation}
O termo $\alpha\fua{\lco\fua{\dvt{\psi}}{\tnr{T}_0}\rco}{\tnr{H}}$
é denominado o \emph{diferencial de
Gâteaux}\index{Gâteaux!diferencial de} ou o
\emph{G-diferencial}\index{G-diferencial} de $\psi$ em $\tnr{T}_0$
na direção $\tnr{H}$. Chama-se o tensor
$\fua{\lco\fua{\dvt{\psi}}{\tnr{T}_0}\rco}{\tnr{H}}$ de
\emph{derivada direcional}\index{derivada!direcional} de $\psi$ em
$\tnr{T}_0$ na direção $\tnr{H}$. A função tensorial linear
limitada $\lco\fua{\dvt{\psi}}{\tnr{T}_0}\rco$ é chamada
\emph{derivada fraca}\index{derivada!fraca} ou \emph{derivada de
Gâteaux}\index{Gâteaux!derivada de} ou
\emph{G-derivada}\index{G-derivada} de $\psi$ em $\tnr{T}_0$. A função $\dvt{\psi}$ é denominada simplesmente a derivada de
Gâteaux ou a G-derivada de $\psi$. Além disso, se a
função $\psi$ for G-diferenciável em qualquer tensor $\tnr{T}_0$
do seu domínio, diz-se que ela é G-diferenciável em
${\tilde{T}}_{\con{\crt{V}{p}}\mapsto\con{\con{F}}}$.
\newline

\noindent\begin{prova} Vamos mostrar que a função $\psi_D$ é
única. Para tal, admitamos, por hipótese, que existam duas funções
$\psi_{D1}$ e $\psi_{D2}$ tangentes à $\psi$ em $\tnr{T}_0$, tais
que
\begin{equation}
\fua{\psi_{D1}}{\tnr{X}}=
\fua{\psi}{\tnr{T}_0}+\fua{\lco\fua{\dvt{\psi_1}}{\tnr{T}_0}\rco}{\tnr{X}-\tnr{T}_0}\nonumber
\end{equation}
e
\begin{equation}
\fua{\psi_{D2}}{\tnr{X}}=
\fua{\psi}{\tnr{T}_0}+\fua{\lco\fua{\dvt{\psi_2}}{\tnr{T}_0}\rco}{\tnr{X}-\tnr{T}_0}\,.\nonumber
\end{equation}
Desta forma, $\psi_{D1}$ e $\psi_{D2}$ são tangentes entre si em
$\tnr{T}_0$. Aplicando a abordagem fraca de funções tangentes,
obtém-se que
\begin{eqnarray}
\lim_{\alpha\to 0}\frac{\|
\fua{\lco\fua{\dvt{\psi_1}}{\tnr{T}_0}\rco}{\alpha\tnr{H}} -
\fua{\lco\fua{\dvt{\psi_2}}{\tnr{T}_0}\rco}{\alpha\tnr{H}}
 \|_{\ete{\crt{W}{q}}{\con{F}}}}
{\|\alpha\tnr{H}\|_{\ete{\crt{V}{p}}{\con{F}}}}&=&0\,.\nonumber
\end{eqnarray}
Elimina-se, pelas propriedades de funções lineares e normas, o
escalar $\alpha$. Logo,
\begin{equation}
\frac{\|
\fua{\lco\fua{\dvt{\psi_1}}{\tnr{T}_0}-\fua{\dvt{\psi_2}}{\tnr{T}_0}\rco}{\tnr{H}}
 \|_{\ete{\crt{W}{q}}{\con{F}}}}
{\|\tnr{H}\|_{\ete{\crt{V}{p}}{\con{F}}}} = 0\,.\nonumber
\end{equation}
Como $\tnr{H}$, por definição, é não nulo, a igualdade anterior é válida se o numerador for zero. Logo, é possível dizer que para qualquer $\tnr{H}\in{\tilde{T}}_{\con{\crt{V}{p}}\mapsto\con{\con{F}}}$ não nulo,
\begin{equation}
\fua{\lco\fua{\dvt{\psi_1}}{\tnr{T}_0}\rco}{\tnr{H}}=
\fua{\lco\fua{\dvt{\psi_2}}{\tnr{T}_0}\rco}{\tnr{H}}\,.
\nonumber
\end{equation}
\end{prova}

\subsubsection{Diferenciabilidade de Ordem Arbitrária}
Considerando as condições anteriores, sejam o espaço tensorial
$\ete{\lpa\crt{V}{p}\rpa^{k+1}}{\con{F}}$ e o espaço tensorial de funções lineares
\begin{equation}
\evl{\cft{\lpa\crt{V}{p}\rpa^{k+1}}{\con{F}}}{\cft{\crt{W}{q}}{\con{F}}}{\con{F}}\,,\nonumber
\end{equation}
onde $k\geqslant 0$. Considerando $\fua{\dtg{0}{\psi}}{\tnr{T}_0}:=\psi$, se existir o par ordenado de funções tensoriais $\lpa \fua{\dtg{k}{\psi}}{\tnr{T}_0},\psi_{\con{D}^{k+1}} \rpa\in\con{TG}_{\tnr{T}_0}$, tal que
\begin{eqnarray}
  \lefteqn{\fua{\psi_{D^{k+1}}}{\tnr{X}}=\fua{\lco\fua{\dtg{k}{\psi}}{\tnr{T}_0}\rco}{\tnr{H}_1\otimes\cdots\otimes\tnr{H}_{k-1}\otimes\tnr{T}_0}+} & & \nonumber\\
  &
&\fua{\lco\fua{\dtg{k+1}{\psi}}{\tnr{T}_0}\rco}{\tnr{H}_1\otimes\cdots\otimes\tnr{H}_{k}\otimes\lpa\tnr{X}-\tnr{T}_0\rpa}\,,
\end{eqnarray}
onde $\tnr{H}_i\in{\tilde{T}}_{\con{\crt{V}{p}}\mapsto\con{\con{F}}}$ e a função
\begin{equation}
\lco\fua{\dtg{k+1}{\psi}}{\tnr{T}_0}\rco\in
\cfl{\cft{\lpa\crt{V}{p}\rpa^{k+1}}{\con{F}}}{\cft{\crt{W}{q}}{\con{F}}}
\end{equation}
é limitada, então este par é único.

Neste contexto, diz-se que $\psi$ é G-diferenciável de ordem $k+1$ em $\tnr{T}_0$. A função linear $\fua{\dtg{k+1}{\psi}}{\tnr{T}_0}$ é a G-derivada de ordem $k+1$ de $\psi$ em $\tnr{T}_0$. Em termos genéricos, a igualdade (\ref{eq:derivadaGateauxIni}) assume a seguinte forma:
\begin{equation}
\begin{array}{rrr}\label{eq:derivadaGateauxGenerica}
&\lim_{\alpha\to
0}\frac{1}{\|\alpha\tnr{H}\|_{\ete{\crt{V}{p}}{\con{F}}}} \| \fua{\lco\fua{\dtg{k}{\psi}}{\tnr{T}_0}\rco}{\tnr{H}_1\otimes\cdots\otimes\tnr{H}_{k-1}\otimes\lpa\alpha\tnr{H}+\tnr{T}_0\rpa}- & \\
&\fua{\lco\fua{\dtg{k}{\psi}}{\tnr{T}_0}\rco}{\tnr{H}_1\otimes\cdots\otimes\tnr{H}_{k-1}\otimes\tnr{T}_0}- &
\\  &
\alpha\fua{\lco\fua{\dtg{k+1}{\psi}}{\tnr{T}_0}\rco}{\tnr{H}_1\otimes\cdots\otimes\tnr{H}_{k}\otimes\tnr{H}}\|_{\ete{\crt{W}{q}}{\con{F}}}=0\,. &
\end{array}
\end{equation}





\subsubsection{Diferenciabilidade de Fréchet}\index{Fréchet!diferenciabilidade de}
Considerando as condições anteriores, se agora existir o par de
funções tangentes
$(\psi,\psi_D)\in\overline{\con{TG}}_{\tnr{T}_0}$, então a função
$\psi$ é dita \emph{diferenciável de Fréchet}\index{Fréchet!função
diferenciável de} ou
\emph{F-diferenciável}\index{função!F-diferenciável} em
$\tnr{T}_0$. Combinando (\ref{eq:funcaoGDerivada}) e
(\ref{eq:funcaoTangenteFrechet}), obtém-se
\begin{equation}
\lim_{\tnr{Y}\to \negmath{0}}\frac{\|
\fua{\psi}{\tnr{\tnr{Y}}+\tnr{T}_0} - \fua{\psi}{\tnr{T}_0}-
\fua{\lco\fua{\dvt{\psi}}{\tnr{T}_0}\rco}{\tnr{Y}}
 \|_{\ete{\crt{W}{q}}{\con{F}}}}
{\|\tnr{Y}\|_{\ete{\crt{V}{p}}{\con{F}}}}=0\,.
\end{equation}
O tensor $\fua{\lco\fua{\dvt{\psi}}{\tnr{T}_0}\rco}{\tnr{Y}}$ é
denominado o \emph{diferencial de
Fréchet}\index{Fréchet!diferencial de} ou o
\emph{F-di\-fe\-ren\-cial}\index{F-diferencial} de $\psi$ em
$\tnr{T}_0$. A função tensorial limitada linear
$\lco\fua{\dvt{\psi}}{\tnr{T}_0}\rco$ é chamada \emph{derivada
forte}\index{derivada!forte} ou \emph{derivada de
Fréchet}\index{Fréchet!derivada de} ou
\emph{F-derivada}\index{F-derivada}\footnote{A demonstração de que
a F-derivada é única pode ser feita utilizando a mesma metodologia
aplicada à G-derivada, substituindo as ocorrências de $\tnr{Y}$
por $\alpha\tnr{H}$ em (\ref{eq:funcaoTangenteFrechet}).} de
$\psi$ em $\tnr{T}_0$. A função $\dvt{\psi}$ é a derivada de Fréchet ou a F-derivada de $\psi$.

Convém enfatizar que uma função F-diferenciável é sempre
G-di\-fe\-ren\-ciá\-vel\footnote{Uma função
G-di\-fe\-ren\-ciá\-vel é sempre F-diferenciável em condições
especiais. Ver \aut{Wouk}\cite{wouk_1979_1}, pp. 268-270.} e suas
derivadas respectivas são iguais. Apesar disso, é conveniente
utilizar notações diferentes para cada uma das derivadas: as
derivadas $\fua{\dvf{\psi}}{\tnr{T}_0}$ e
$\fua{\dvg{\psi}}{\tnr{T}_0}$ indicam, respectivamente, que $\psi$
é F-diferenciável e G-diferenciável em $\tnr{T}_0$. Caso $\psi$
seja F ou G-diferenciável no seu domínio, então as notações
respectivas para as derivadas são $\fua{\dvf{\psi}}{\tnr{X}}$ e
$\fua{\dvg{\psi}}{\tnr{X}}$.

O conceito de derivadas de ordens superiores também é extensível à abordagem forte de funções tangentes. Desta forma, diz-se que $\fua{\dgg{k}{\psi}}{\tnr{T}_0}$ e $\fua{\dgf{k}{\psi}}{\tnr{T}_0}$ são respectivamente a G-derivada e a F-derivada de ordem $k$ de $\psi$ em $\tnr{T}_0$.

\subsubsection{Calculando a Derivada Direcional}
Ainda sob as condições anteriores, a igualdade
(\ref{eq:derivadaGateauxIni}) revela que o numerador aproxima-se mais rápido de $0$ do que o denominador.
Desta forma, é possível concluir que
\begin{equation}\label{eq:numeradorGateaux}
\lim_{\alpha\to 0}\lco\fua{\psi}{\alpha\tnr{\tnr{H}}+\tnr{T}_0} - \fua{\psi}{\tnr{T}_0}-
\alpha\fua{\lco\fua{\dvg{\psi}}{\tnr{T}_0}\rco}{\tnr{H}}\rco
 =\negmath{0}\,.
\end{equation}
Já que $\alpha$ nunca é $0$, o termo entre colchetes nesta expressão nunca é zero, ou seja, sempre há um \emph{resíduo}. Tal resíduo pode ser definido pela função tensorial no mapeamento
\begin{equation}
\map{\ele{r}_\psi}{{\tilde{T}}_{\con{\crt{V}{p}}\mapsto\con{\con{F}}}}{\cft{\crt{W}{q}}{\con{F}}}\,,
\end{equation}
onde
\begin{equation}
\lim_{\tnr{X}\to \negmath{0}}\fua{\ele{r}_\psi}{\tnr{X}}=\negmath{0}\,.
\end{equation}
Com base nisso, pode-se entender a expressão
(\ref{eq:numeradorGateaux}) segundo a igualdade
\begin{equation}\label{eq:gderivadaResiduo}
\fua{\psi}{\alpha\tnr{\tnr{H}}+\tnr{T}_0} - \fua{\psi}{\tnr{T}_0}-
\alpha\fua{\lco\fua{\dvg{\psi}}{\tnr{T}_0}\rco}{\tnr{H}}
 = \fua{\ele{r}_\psi}{\tnr{H}}\,,
\end{equation}
onde $\ele{r}_\psi$ é dita a \emph{função
resíduo}\index{função!resíduo} de $\psi$ em $\tnr{T}_0\,$.
Neste contexto, isolando o termo da derivada direcional na igualdade anterior e em seguida aplicando $\lim_{\alpha\to 0}$ em ambos os lados da expressão resultante, obtém-se que
\begin{equation}\label{eq:calculoDerivadaDirecional}
\fua{\lco\fua{\dvg{\psi}}{\tnr{T}_0}\rco}{\tnr{H}} =
\lim_{\alpha\to 0}\frac{\fua{\psi}{\alpha\tnr{\tnr{H}}+\tnr{T}_0}
- \fua{\psi}{\tnr{T}_0}}{\alpha}\,.
\end{equation}
Vale ressaltar que para qualquer $\tnr{H}$, esta igualdade fornece
uma regra para a derivada $\fua{\dvg{\psi}}{\tnr{T}_0}$. Além disso, se $\tnr{T}_0$ também for um tensor qualquer, ou seja, se $\psi$ for G-diferenciável no seu domínio, tem-se a regra para a derivada direcional de $\psi$.

COLOCAR UMA PROPOSIÇÂO QUE MOSTRE A DERIVADA DIRECIONAL DE UMA FUNÇÂO ESCALAR

\paragraph{Derivadas Direcionais de Ordem Arbitrária.} O cálculo da
derivada direcional de ordem $k+1$, $k\geqslant 0$, é
feito a partir de (\ref{eq:derivadaGateauxGenerica}), generalizando a igualdade
(\ref{eq:calculoDerivadaDirecional}). Desta forma,
\begin{eqnarray}\label{eq:calculoDerivadaDirecionalOrdemN}
  \lefteqn{\fua{\lco\fua{\dgg{k+1}{\psi}}{\tnr{T}_0}\rco}{\tnr{H}_1\otimes\cdots\otimes\tnr{H}_{k}\otimes\tnr{H}}
=\lim_{\alpha\to
0}\frac{1}{\alpha}} & & \nonumber\\
  &
&\fua{\lco\fua{\dgg{k}{\psi}}{\tnr{T}_0}\rco}{\tnr{H}_1\otimes\cdots\otimes\tnr{H}_{k-1}\otimes\lpa \alpha\tnr{\tnr{H}}+\tnr{T}_0\rpa}
-\nonumber\\  & &
\qquad\fua{\lco\fua{\dgg{k}{\psi}}{\tnr{T}_0}\rco}{\tnr{H}_1\otimes\cdots\otimes\tnr{H}_{k-1}\otimes\tnr{T}_0}\,,
\end{eqnarray}
onde
$\tnr{H}_i\in{\tilde{T}}_{\con{\crt{V}{p}}\mapsto\con{\con{F}}}$ e
$\fua{\dgg{0}{\psi}}{\tnr{T}_0}:=\psi\,$. Para tensores
$\tnr{H},\tnr{H}_1,\cdots,\tnr{H}_k$ quaisquer, obtém-se a regra
da G-derivada $\fua{\dgg{k+1}{\psi}}{\tnr{T}_0}$ e por
conseqüência da F-derivada $\fua{\dgf{k+1}{\psi}}{\tnr{T}_0}$.

\paragraph{Derivadas Parciais.}\index{derivada!parcial}\label{sec:derivadaParcial}
Considerando as condições anteriores, seja o inteiro $m\geqslant 1$ e o conjunto
\begin{equation}
\con{P}={T}_{\con{\crt{V}{p_1}}\mapsto\con{\con{F}}}\times\cdots\times{T}_{\con{\crt{V}{p_m}}\mapsto\con{\con{F}}}
\end{equation}
definidor de um espaço vetorial de Banach. Seja o mapeamento
\begin{equation}
\map{\omega}{\tilde{\con{P}}}{\cft{\crt{W}{q}}{\con{F}}}\,.
\end{equation}
onde $\tilde{\con{P}}$ é subconjunto aberto de $\con{P}\,$.

Com base na proposição \ref{prp:somaPoliadicos}, a condição de existência da função no mapeamento anterior é garantida se, por exemplo,
\begin{equation}
\fua{\omega}{\tnr{X}_1,\cdots,\tnr{X}_m}=
\fua{\psi}{\sum_{k=1}^m\underbrace{\vto{x}_{1k}\otimes\cdots\otimes\vto{x}_{pk}}_{\tnr{X}_k}}\,,
\end{equation}
onde $p=p_1=\cdots=p_m\,$. A derivada direcional (\ref{eq:calculoDerivadaDirecional}) toma então a seguinte forma genérica:
\begin{eqnarray}\label{eq:derivadaDirecionalVariasVariaveis}
\lefteqn{\fua{\lco\fua{\dvg{\omega}}{\tnr{T}_1,\cdots,\tnr{T}_m}\rco}{\tnr{H}_1,\cdots,\tnr{H}_m} =\lim_{\alpha\to 0}\frac{1}{\alpha}} & & \nonumber\\
& &
\fua{\omega}{\alpha\tnr{H}_1+\tnr{T}_1,\cdots,\alpha\tnr{H}_m+\tnr{T}_m}
- \fua{\omega}{\tnr{T}_1,\cdots,\tnr{T}_m}\,.
\end{eqnarray}
A partir daí, diz-se que a função no lado esquerdo da igualdade
\begin{eqnarray}\label{eq:derivadaDirecionalParcial}
\lefteqn{\fua{\lco\fua{\dvg{\omega}}{\tnr{T}_1,\cdots,\tnr{T}_m}\rco}{\tnr{H}_r} =\lim_{\alpha\to
0}\frac{1}{\alpha}} & & \nonumber\\
& &
\fua{\omega}{\tnr{T}_1,\cdots,\alpha\tnr{H}_r+\tnr{T}_r,\cdots,\tnr{T}_m}
- \fua{\omega}{\tnr{T}_1,\cdots,\tnr{T}_m}
\end{eqnarray}
é a G-derivada parcial de $\omega$ em $\tnr{T}_r\,$. Neste caso, a função $\dvg{\omega}$ fica representada por $\dvp{r}{\omega}$, onde $\mathrm{r}$ é o índice do elemento da tupla sobre o qual a derivada parcial é definida. Vale lembrar que os conceitos aqui apresentadas também são válidos no contexto da função $\psi$ F-diferenciável.

\begin{prp}\label{teo:derivadaParcial}
Sejam os espaços de Banach $\ebh{P}{F}$ e $\ete{\crt{V}{p_i}}{\con{F}}$ tais que o conjunto
\begin{equation*}
\con{P}:={T}_{\con{\crt{V}{p_1}}\mapsto\con{\con{F}}}\times\cdots\times{T}_{\con{\crt{V}{p_m}}\mapsto\con{\con{F}}}\,,
\end{equation*}
onde $m\geqslant 1$. Dados um espaço tensorial $\ete{\crt{W}{q}}{\con{F}}$ e $\tilde{\con{P}}$ um subconjunto aberto de $\con{P}\,$, seja a função em  $\map{\omega}{\tilde{\con{P}}}{\cft{\crt{W}{q}}{\con{F}}}$ G-diferenciável na tupla ordenada $(\tnr{T}_1,\cdots,\tnr{T}_m)\,$. Nestas condições, para qualquer elemento   $(\tnr{H}_1,\cdots,\tnr{H}_m)\in\tilde{\con{P}}\,$, a derivada direcional
\begin{equation*}
\fua{\lco\fua{\dvg{\omega}}{\tnr{T}_1,\cdots,\tnr{T}_m}\rco}{\tnr{H}_1,\cdots,\tnr{H}_m} =
\sum_{r=1}^{m}\fua{\lco\fua{\dvp{r}{\omega}}{\tnr{T}_1,\cdots,\tnr{T}_m}\rco}{\tnr{H}_r}\,.
\end{equation*}
\end{prp}
\begin{prova}\rodape{Adaptada de \aut{Zeidler}\cite{zeidler_1995_1}, pp. 232-233.}
O processo mostrado a seguir, que utiliza $m=2\,$, pode ser generalizado para valores maiores. Tomando as definições (\ref{eq:derivadaDirecionalVariasVariaveis}) e (\ref{eq:derivadaDirecionalParcial}), concluimos facilmente as igualdades
\begin{equation*}
\fua{\lco\fua{\dvg{\omega}}{\tnr{T}_1,\tnr{T}_2}\rco}{\tnr{H}_1,\negmath{0}}=\fua{\lco\fua{\dvp{1}{\omega}}{\tnr{T}_1,\tnr{T}_2}\rco}{\tnr{H}_1}
\end{equation*}
e
\begin{equation*}
\fua{\lco\fua{\dvg{\omega}}{\tnr{T}_1,\tnr{T}_2}\rco}{\negmath{0},\tnr{H}_2}=\fua{\lco\fua{\dvp{2}{\omega}}{\tnr{T}_1,\tnr{T}_2}\rco}{\tnr{H}_2}\,,
\end{equation*}
rotuladas respectivamente por (i) e (ii). Somando estas duas expressões, chegamos à igualdade (iii)
\begin{equation*}
\fua{\lco\fua{\dvg{\omega}}{\tnr{T}_1,\tnr{T}_2}\rco}{\lpa\tnr{H}_1,\negmath{0}\rpa+\lpa\negmath{0},\tnr{H}_2\rpa}=\fua{\lco\fua{\dvp{1}{\omega}}{\tnr{T}_1,\tnr{T}_2}\rco}{\tnr{H}_1}+\fua{\lco\fua{\dvp{2}{\omega}}{\tnr{T}_1,\tnr{T}_2}\rco}{\tnr{H}_2}\,,
\end{equation*}
válida para quaisquer $\tnr{H}_1,\tnr{H}_2\in\tilde{\con{P}}\,$. Como estamos tratando com espaços vetoriais, fica claro que a tupla ordenada  $(\tnr{Z}_1,\tnr{Z}_2):=(\tnr{H}_1,\negmath{0})+(\negmath{0},\tnr{H}_2)\,$ é elemento de $\con{P}$. Fazendo $\tnr{H}_2=\negmath{0}$ em (iii), obtemos de (i) que $(\tnr{Z}_1,\tnr{Z}_2)=(\tnr{H}_1,\negmath{0})\,$. Da mesma forma, se $\tnr{H}_1=\negmath{0}\,$, obtemos de (ii) e (iii) que  $(\tnr{Z}_1,\tnr{Z}_2)=(\negmath{0},\tnr{H}_2)\,$. Como estas duas situações confirmam-se mutuamente, fica óbvio que $(\tnr{Z}_1,\tnr{Z}_2)=(\tnr{H}_1,\tnr{H}_2)\,$ para quaisquer $\tnr{H}_1,\tnr{H}_2\in\tilde{\con{P}}\,$.
\end{prova}

\subsection{Funções Tensoriais Suaves}\index{função!tensorial!suave}
Sejam os espaços tensoriais de Banach $\ete{\crt{V}{p}}{\con{F}}$
e $\ete{\crt{W}{q}}{\con{F}}$ e o subconjunto aberto
${\tilde{T}}_{\con{\crt{V}{p}}\mapsto\con{\con{F}}}\subset\cft{\crt{V}{p}}{\con{F}}$.
Sejam os conjuntos $\mathcal{C}_\mathrm{G}^0$, formado por todas as funções contínuas em seu domínio, e $\mathcal{D}_\mathrm{G}^1$, formado por todas as
funções G-diferenciáveis de ordem $1$ em seu domínio. Considerando
uma função qualquer $\psi\in\con\mathcal{D}_\mathrm{G}^1$ que
mapeia
${\tilde{T}}_{\con{\crt{V}{p}}\mapsto\con{\con{F}}}\mapsto\con{\cft{\crt{W}{q}}{\con{F}}}$,
a G-derivada $\dvg{\psi}$ define o mapeamento
\begin{equation}
\map{\dvg{\psi}}{{\tilde{T}}_{\con{\crt{V}{p}}\mapsto\con{\con{F}}}}
{\con{LB}_{{\tilde{T}}_{\con{\crt{V}{p}}\mapsto\con{\con{F}}}\mapsto\cft{\crt{W}{q}}{\con{F}}}}
 \,,
\end{equation}
onde o conjunto
\begin{equation}
\con{LB}_{{\tilde{T}}_{\con{\crt{V}{p}}\mapsto\con{\con{F}}}\mapsto\cft{\crt{W}{q}}{\con{F}}}\subset
\cfl{\cft{\crt{V}{p}}{\con{F}}}{\cft{\crt{W}{q}}{\con{F}}}
\end{equation}
é, por definição, normado. Neste contexto, caso as funções $\psi$ e $\dvg{\psi}$ sejam
elementos de $\mathcal{C}_\mathrm{G}^0$, diz-se que $\psi$ também é elemento do conjunto
$\sug{1}\subset\mathcal{D}_\mathrm{G}^1$, formado por todas as
funções contínuas cuja G-derivada é contínua. É comum dizer também
que $\psi$ é de classe $\sug{1}$ ou que é G-suave de ordem 1. Da
mesma forma, para o caso de funções F-diferenciáveis, define-se o
conjunto $\suf{1}\subset\mathcal{D}_\mathrm{F}$ das funções cuja
\emph{F-derivada} é contínua.

Em termos gerais, para que a função $\psi$ seja G-suave de ordem
$k+1$, ou da classe $\sug{k+1}$, ela e sua G-derivada
$\dgg{k}{\psi}$ devem ser G-suaves de ordem $k$. Em
outras palavras,
\begin{equation}
\psi\in\sug{k}\,\,\mathrm{e}\,\,\dgg{k}{\psi}\in\sug{k}\Longleftrightarrow\psi\in\sug{k+1}\,,\,\,k\geqslant
0\,.
\end{equation}

\subsubsection{Difeomorfismo}\index{difeomorfismo} Considerando as condições anteriores,
dado um tensor
$\tnr{T}_0\in{\tilde{T}}_{\con{\crt{V}{p}}\mapsto\con{\con{F}}}$,
a função tensorial $\psi$ é um difeomorfismo em $\tnr{T}_0$ se ela
for uma bijeção G-diferenciável em $\tnr{T}_0$ e sua função
inversa for G-diferenciável em $\fua{\psi}{\tnr{T}_0}$. Se a
bijeção $\psi$ e sua inversa forem G-suaves de ordem $n$, então diz-se que $\psi$ é um
\emph{$\sug{n}$-difeomorfismo}.

\begin{teo}[Função Inversa Local]\index{Função Inversa Local!Teorema da}\label{teo:FuncaoInversaLocal}
Sejam os espaços tensoriais de Banach $\ete{\crt{V}{p}}{\con{F}}$
e $\ete{\crt{W}{q}}{\con{F}}$. Seja o subconjunto aberto
${\tilde{T}}_{\con{\crt{V}{p}}\mapsto\con{\con{F}}}\subset\cft{\crt{V}{p}}{\con{F}}$
e um tensor
$\tnr{T}_0\in{\tilde{T}}_{\con{\crt{V}{p}}\mapsto\con{\con{F}}}$.
Uma função tensorial $\psi$ é um $\sug{n}$-difeomorfismo em
$\tnr{T}_0$ se e somente se sua derivada
$\fua{\dvg{\psi}}{\tnr{T}_0}$ for uma bijeção.
\end{teo}
\begin{prova}
Este teorema é uma aplicação do Teorema da Função
Implícita\rodape{Ver \aut{Zeidler}\cite{zeidler_1995_1}, pp.
259-260.}.
\end{prova}


\subsection{Regras Fundamentais Para Derivadas}
São apresentadas a seguir algumas regras aqui consideradas
fundamentais nos procedimentos de obtenção de derivadas. A
abordagem aplicada para se conseguir tais regras faz uso do
cálculo da derivada direcional, considerando o argumento de
direção um tensor qualquer.

\begin{teo}[Regra da Soma]\index{Regra da Soma}\label{teo:RegraSome}
Sejam os espaços tensoriais de Banach $\ete{\crt{V}{p}}{\con{F}}$ e
$\ete{\crt{W}{q}}{\con{F}}$. Seja o subconjunto aberto
${\tilde{T}}_{\con{\crt{V}{p}}\mapsto\con{\con{F}}}\subset\cft{\crt{V}{p}}{\con{F}}$.
Sejam as funções $\psi$, $\psi_1$ e $\psi_2$ que mapeiam
${\tilde{T}}_{\con{\crt{V}{p}}\mapsto\con{\con{F}}}\mapsto\cft{\crt{W}{q}}{\con{F}}$. Se
$\psi$ for G-diferenciável no seu domínio com regra
\begin{equation}
\fua{\psi}{\tnr{X}}=\fua{\psi_1}{\tnr{X}}+\fua{\psi_2}{\tnr{X}}\,,\nonumber
\end{equation}
então a função
\begin{equation}
\fua{\dvg{\psi}}{\tnr{X}}=\fua{\dvg{\psi_1}}{\tnr{X}}+
\fua{\dvg{\psi_2}}{\tnr{X}}\,.\nonumber
\end{equation}
\end{teo}
\begin{prova}
Considerando o tensor $\tnr{H}\in{\tilde{T}}_{\con{\crt{V}{p}}\mapsto\con{\con{F}}}$ uma
direção qualquer, seja o seguinte desenvolvimento:
\begin{eqnarray}
\fua{\lco\fua{\dvg{\psi}}{\tnr{X}}\rco}{\tnr{H}}&=&
\lim_{\alpha\to
0}\frac{\fua{\psi_1}{\alpha\tnr{\tnr{H}}+\tnr{X}}+\fua{\psi_2}{\alpha\tnr{\tnr{H}}+\tnr{X}}
- \fua{\psi_1}{\tnr{X}}-\fua{\psi_2}{\tnr{X}}}{\alpha} \nonumber\\
&=&\lim_{\alpha\to 0}\frac{\fua{\psi_1}{\alpha\tnr{\tnr{H}}+\tnr{X}} -
\fua{\psi_1}{\tnr{X}}}{\alpha}+\lim_{\alpha\to
0}\frac{\fua{\psi_2}{\alpha\tnr{\tnr{H}}+\tnr{X}} - \fua{\psi_2}{\tnr{X}}}{\alpha}
\nonumber\\
&=&\fua{\lco\fua{\dvg{\psi_1}}{\tnr{X}}\rco}{\tnr{H}}+\fua{\lco\fua{\dvg{\psi_2}}{\tnr{X}}\rco}{\tnr{H}}\nonumber\\
&=&\fua{\lco\fua{\dvg{\psi_1}}{\tnr{X}}+\fua{\dvg{\psi_2}}{\tnr{X}}\rco}{\tnr{H}}\,.\nonumber
\end{eqnarray}
A última igualdade é obtida porque as derivadas $\fua{\dvg{\psi_1}}{\tnr{X}}$ e
$\fua{\dvg{\psi_2}}{\tnr{X}}$ são elementos de um espaço vetorial de funções lineares.
\end{prova}

\begin{teo}[Regra do Produto]\index{Regra do Produto}\label{teo:RegraProduto}
Sejam os quatro espaços tensoriais de Banach
$\ete{\crt{V}{p}}{\con{F}}$,
$\ete{\crt{K}{r}\times\crt{Z}{s}}{\con{F}}$,
$\ete{\crt{K}{r}\times\crt{W}{q}}{\con{F}}$ e
$\ete{\crt{W}{q}\times\crt{Z}{s}}{\con{F}}$. Sejam o subconjunto
aberto
${\tilde{T}}_{\con{\crt{V}{p}}\mapsto\con{\con{F}}}\subset\cft{\crt{V}{p}}{\con{F}}$
e os mapeamentos:
\begin{eqnarray}
&
\map{\psi}{{\tilde{T}}_{\con{\crt{V}{p}}\mapsto\con{\con{F}}}}{\cft{\crt{K}{r}\times\crt{Z}{s}}{\con{F}}}\,,
&\nonumber\\
&
\map{\psi_1}{{\tilde{T}}_{\con{\crt{V}{p}}\mapsto\con{\con{F}}}}{\cft{\crt{K}{r}\times\crt{W}{q}}{\con{F}}}\,,
&\nonumber\\
&
\map{\psi_2}{{\tilde{T}}_{\con{\crt{V}{p}}\mapsto\con{\con{F}}}}{\cft{\crt{W}{q}\times\crt{Z}{s}}{\con{F}}}\,.
&\nonumber
\end{eqnarray}
Se $\psi$ for G-diferenciável no seu domínio com
regra
\begin{equation}
\fua{\psi}{\tnr{X}}=\fua{\psi_1}{\tnr{X}}\odot_q\fua{\psi_2}{\tnr{X}}\,,\nonumber
\end{equation}
então
\begin{eqnarray}
\fua{\lco\fua{\dvg{\psi}}{\tnr{X}}\rco}{\tnr{H}}=\fua{\psi_1}{\tnr{X}}\odot_q\fua{\lco\fua{\dvg{\psi_2}}{\tnr{X}}\rco}{\tnr{H}}+\nonumber\\
\fua{\lco\fua{\dvg{\psi_1}}{\tnr{X}}\rco}{\tnr{H}}\odot_q\fua{\psi_2}{\tnr{X}}\,,\forall\,\tnr{H}\in{\tilde{T}}_{\con{\crt{V}{p}}\mapsto\con{\con{F}}}\,.\nonumber
\end{eqnarray}
\end{teo}
\begin{prova}
A igualdade anterior é o resultado do seguinte desenvolvimento:
\begin{eqnarray}
\fua{\lco\fua{\dvg{\psi}}{\tnr{X}}\rco}{\tnr{H}}=\lim_{\alpha\to
0}\frac{1}{\alpha}\,\,\fua{\psi_1}{\alpha\tnr{\tnr{H}}+\tnr{X}}\odot_q\fua{\psi_2}{\alpha\tnr{\tnr{H}}+\tnr{X}}
- \fua{\psi_1}{\tnr{X}}\odot_q\fua{\psi_2}{\tnr{X}}\,,\nonumber
\end{eqnarray}
adicionando e subtraindo o termo
$\fua{\psi_1}{\alpha\tnr{\tnr{H}}+\tnr{X}}\odot_q\fua{\psi_2}{\tnr{X}}$,
tem-se
\begin{eqnarray}
\fua{\lco\fua{\dvg{\psi}}{\tnr{X}}\rco}{\tnr{H}}=\lim_{\alpha\to
0}\frac{1}{\alpha}\,\,\fua{\psi_1}{\alpha\tnr{\tnr{H}}+\tnr{X}}\odot_q\fua{\psi_2}{\alpha\tnr{\tnr{H}}+\tnr{X}}
-
\fua{\psi_1}{\alpha\tnr{\tnr{H}}+\tnr{X}}\odot_q\fua{\psi_2}{\tnr{X}}+\nonumber\\
\fua{\psi_1}{\alpha\tnr{\tnr{H}}+\tnr{X}}\odot_q\fua{\psi_2}{\tnr{X}}-
\fua{\psi_1}{\tnr{X}}\odot_q\fua{\psi_2}{\tnr{X}}=\nonumber\\
\lim_{\alpha\to
0}\frac{1}{\alpha}\,\,\fua{\psi_1}{\alpha\tnr{\tnr{H}}+\tnr{X}}\odot_q\lpa
\fua{\psi_2}{\alpha\tnr{\tnr{H}}+\tnr{X}} -
\fua{\psi_2}{\tnr{X}}\rpa + \nonumber\\
\lpa \fua{\psi_1}{\alpha\tnr{\tnr{H}}+\tnr{X}} -
\fua{\psi_1}{\tnr{X}}\rpa\odot_q\fua{\psi_2}{\tnr{X}}=\,\nonumber\\
\lim_{\alpha\to
0}\frac{1}{\alpha}\,\fua{\psi_1}{\alpha\tnr{\tnr{H}}+\tnr{X}}\odot_q\lpa
\fua{\psi_2}{\alpha\tnr{\tnr{H}}+\tnr{X}} -
\fua{\psi_2}{\tnr{X}}\rpa + \nonumber\\
\lim_{\alpha\to 0}\frac{1}{\alpha}\,\lpa
\fua{\psi_1}{\alpha\tnr{\tnr{H}}+\tnr{X}} -
\fua{\psi_1}{\tnr{X}}\rpa\odot_q\fua{\psi_2}{\tnr{X}}\,.\nonumber
\end{eqnarray}
\end{prova}


\begin{teo}[Regra da Cadeia]\index{Regra da Cadeia}\label{teo:RegraCadeia}
Sejam os espaços tensoriais de Banach $\ete{\crt{V}{p}}{\con{F}}$,
$\ete{\crt{K}{s}}{\con{F}}$ e $\ete{\crt{W}{q}}{\con{F}}$. Sejam
os subconjunto abertos
${\tilde{T}}_{\con{\crt{V}{p}}\mapsto\con{\con{F}}}\subset\cft{\crt{V}{p}}{\con{F}}$
e
${\tilde{T}}_{\con{\crt{K}{s}}\mapsto\con{\con{F}}}\subset\cft{\crt{K}{s}}{\con{F}}$
sobre os quais definem-se mapeamentos:
\begin{eqnarray}
&
\map{\psi}{{\tilde{T}}_{\con{\crt{V}{p}}\mapsto\con{\con{F}}}}{\cft{\crt{W}{q}}{\con{F}}}\,,
&\nonumber\\
&
\map{\psi_2}{{\tilde{T}}_{\con{\crt{V}{p}}\mapsto\con{\con{F}}}}{{\tilde{T}}_{\con{\crt{K}{s}}\mapsto\con{\con{F}}}}\,,
&\nonumber\\
&
\map{\psi_1}{{\tilde{T}}_{\con{\crt{K}{s}}\mapsto\con{\con{F}}}}{\cft{\crt{W}{q}}{\con{F}}}\,.
&\nonumber
\end{eqnarray}
Se $\psi$ for G-diferenciável no seu domínio com regra
\begin{equation}
\fua{\psi}{\tnr{X}}=\fua{\psi_1\circ\psi_2}{\tnr{X}}\,,\nonumber
\end{equation}
então
\begin{eqnarray}
\fua{\dvg{\psi}}{\tnr{X}}=\fua{\dvg{\psi_1}}{\fua{\psi_2}{\tnr{X}}}\circ\fua{\dvg{\psi_2}}{\tnr{X}}\,.\nonumber
\end{eqnarray}
\end{teo}
\begin{prova}\rodape{Adaptada de
\aut{Zeidler}\cite{zeidler_1995_1}, pg. 248.} Segundo a igualdade
(\ref{eq:gderivadaResiduo}), tem-se que
\begin{equation}
\alpha\fua{\lco\fua{\dvg{\psi_1}}{\tnr{Y}}\rco}{\tnr{Z}}
 = \fua{\psi_1}{\alpha\tnr{\tnr{Z}}+\tnr{Y}} - \fua{\psi_1}{\tnr{Y}}-
\fua{\ele{r}_{\psi_1}}{\tnr{Z}}\,.\nonumber
\end{equation}
Como $\tnr{Y}$ e $\tnr{Z}$ são tensores quaisquer, então seja
$\tnr{Y}=\fua{\psi_2}{\tnr{X}}$ , onde
\begin{equation} \fua{\psi_2}{\tnr{X}}
 = \fua{\psi_2}{\beta\tnr{\tnr{H}}+\tnr{X}} - \beta\fua{\lco\fua{\dvg{\psi_2}}{\tnr{X}}\rco}{\tnr{H}} -
\fua{\ele{r}_{\psi_2}}{\tnr{H}}\,,\nonumber
\end{equation}
e
$\tnr{Z}=\frac{\beta}{\alpha}\fua{\lco\fua{\dvg{\psi_2}}{\tnr{X}}\rco}{\tnr{H}}$.
Desenvolvendo a primeira igualdade, chega-se a
\begin{eqnarray}
\fua{\lco\fua{\lco\dvg{\psi_1}\rco\circ\psi_2}{\tnr{X}}\rco}{\fua{\lco\fua{\dvg{\psi_2}}{\tnr{X}}\rco}{\tnr{H}}}
 = \fua{\psi_1}{\fua{\psi_2}{\beta\tnr{\tnr{H}}+\tnr{X}}-
 \fua{\ele{r}_{\psi_2}}{\tnr{H}}} - \nonumber\\
 \fua{\psi_1\circ\psi_2}{\tnr{X}}-
\fua{\ele{r}_{\psi_1}}{\fua{\lco\fua{\dvg{\psi_2}}{\tnr{X}}\rco}{\tnr{H}}}\,.\nonumber
\end{eqnarray}
Fazendo $\beta \to 0$, utilizando o teorema \ref{teo:RieszGeneralizado} e aplicando o
conceito de funções representantes, pode-se realizar o seguinte desenvolvimento, para
qualquer $\tnr{H}\in{\tilde{T}}_{\con{\crt{V}{p}}\mapsto\con{\con{F}}}$:
\begin{eqnarray}
\fua{\lco\fua{\dvg{\psi}}{\tnr{X}}\rco}{\tnr{H}}&=&\fua{\lco\fua{\lco\dvg{\psi_1}\rco\circ\psi_2}{\tnr{X}}\rco}
{\fua{\lco\fua{\dvg{\psi_2}}{\tnr{X}}\rco}{\tnr{H}}}\nonumber\\
\fua{\lco\fua{\dvg{\psi}}{\tnr{X}}\rco}{\tnr{H}}&=&\ftr{A}{s}\circ\fua{\ftr{B}{p}}{\tnr{H}}\nonumber\\
\fua{\dvg{\psi}}{\tnr{X}}&=&\ftr{A}{s}\circ\ftr{B}{p}\,.\nonumber
\end{eqnarray}
\end{prova}


\subsection{Gradiente e Divergente}\index{gradiente}\label{sec:GradienteDivergente} Sejam os espaços tensoriais de Banach
$\ete{\crt{V}{p}}{\con{F}}$ e $\ete{\crt{W}{q}}{\con{F}}$. Seja o
subconjunto aberto
${\tilde{T}}_{\con{\crt{V}{p}}\mapsto\con{\con{F}}}\subset\cft{\crt{V}{p}}{\con{F}}$
e a função em
\begin{equation}
\map{\psi}{{\tilde{T}}_{\con{\crt{V}{p}}\mapsto\con{\con{F}}}}{\cft{\crt{W}{q}}{\con{F}}}
\end{equation}
G-diferenciável em
$\tnr{T}_0\in{\tilde{T}}_{\con{\crt{V}{p}}\mapsto\con{\con{F}}}$.
Segundo o teorema \ref{teo:RieszGeneralizado}, pode-se interpretar
a G-derivada de $\psi$ nas formas
\begin{eqnarray}
\fua{\dvg{\psi}}{\tnr{T}_0}=\rft{\bar{G}}{p}&\mathrm{ou}&
\fua{\dvg{\psi}}{\tnr{T}_0}=\lft{\hat{G}}{p}\,.
\end{eqnarray}
Os tensores
$\tnr{\bar{G}}\in{\con{T}}_{\con{\crt{W}{q}}\times\con{\crt{V}{p}}\mapsto\con{\con{F}}}$
e
$\tnr{\hat{G}}\in{\con{T}}_{\con{\crt{V}{p}}\times\con{\crt{W}{q}}\mapsto\con{\con{F}}}$,
presentes nas funções representantes, com notações alteradas para
$\grd{r}{\psi}{\tnr{T}_0}$ e $\grd{e}{\psi}{\tnr{T}_0}$
respectivamente, são denominados gradiente \emph{à
direita}\index{gradiente!à direita} e \emph{à
esquerda}\index{gradiente!à esquerda} de $\psi$ em $\tnr{T}_0$.
Neste contexto, a regra da G-derivada de $\psi$, para qualquer
tensor
$\tnr{T}_0\in{\con{T}}_{\con{\crt{V}{p}}\mapsto\con{\con{F}}}$,
pode ser escrita nas seguintes formas:
\begin{equation}
\fua{\lco\fua{\dvg{\psi}}{\tnr{T}_0}\rco}{\tnr{X}}=\grd{r}{\psi}{\tnr{T}_0}\odot_p\tnr{X}
\end{equation}
ou
\begin{equation}
\fua{\lco\fua{\dvg{\psi}}{\tnr{T}_0}\rco}{\tnr{X}}=\tnr{X}\odot_p\grd{e}{\psi}{\tnr{T}_0}\,.
\end{equation}
As funções $\gqu{r}{\psi}$ e
$\gqu{e}{\psi}$ são chamadas os
gradientes à direita e à esquerda  de $\psi$ respectivamente. Além disso, se eventuais conceitos posteriores
independem das abordagens à direita ou à esquerda, será utilizada
a notação $\gqu{}{\psi}$.

Para os casos onde $q\geqslant p$, dado o tensor
identidade\index{divergente}
$\tnr{I}\in{\con{T}}_{\con{\crt{V}{p}\times\crt{V}{p}}\mapsto\con{\con{F}}}$,
diz-se que o tensor
\begin{equation}
\drd{r}{\psi}{\tnr{T}_0}:=\grd{r}{\psi}{\tnr{T}_0}\odot_{2p}\tnr{I}
\end{equation}
de ordem $q-p$ é o divergente \emph{à direita}\index{divergente!à
direita} de $\psi$ em $\tnr{T}_0$ se
$\crt{W}{q}=\crt{Z}{q-p}\times\crt{V}{p}$. Da mesma forma, o
tensor de ordem $q-p$
\begin{equation}
\drd{e}{\psi}{\tnr{T}_0}:=\tnr{I}\odot_{2p}\grd{e}{\psi}{\tnr{T}_0}
\end{equation}
é o divergente \emph{à esquerda}\index{divergente!à esquerda} de
$\psi$ em $\tnr{T}_0$ se
$\crt{W}{q}=\crt{V}{p}\times\crt{U}{q-p}$. As funções
$\dqu{r}{\psi}$ e $\dqu{e}{\psi}$ são os divergentes à direita e à esquerda de $\psi$ respectivamente. Além
disso, se os conceitos à direita e à esquerda forem irrelevantes,
utiliza-se a notação $\dqu{}{\psi}$.

\begin{prp}
Sejam os quatro espaços tensoriais de Banach
$\ete{\crt{V}{p}}{\con{F}}$,
$\ete{\crt{K}{r}\times\crt{Z}{s}}{\con{F}}$,
$\ete{\crt{K}{r}\times\crt{W}{q}}{\con{F}}$ e
$\ete{\crt{W}{q}\times\crt{Z}{s}}{\con{F}}$. Sejam o subconjunto
aberto
${\tilde{T}}_{\con{\crt{V}{p}}\mapsto\con{\con{F}}}\subset\cft{\crt{V}{p}}{\con{F}}$
e os mapeamentos:
\begin{eqnarray}
&
\map{\psi}{{\tilde{T}}_{\con{\crt{V}{p}}\mapsto\con{\con{F}}}}{\cft{\crt{K}{r}\times\crt{Z}{s}}{\con{F}}}\,,
&\nonumber\\
&
\map{\psi_1}{{\tilde{T}}_{\con{\crt{V}{p}}\mapsto\con{\con{F}}}}{\cft{\crt{K}{r}\times\crt{W}{q}}{\con{F}}}\,,
&\nonumber\\
&
\map{\psi_2}{{\tilde{T}}_{\con{\crt{V}{p}}\mapsto\con{\con{F}}}}{\cft{\crt{W}{q}\times\crt{Z}{s}}{\con{F}}}\,.
&\nonumber
\end{eqnarray}
Se $\psi$ for G-diferenciável no seu domínio com regra
\begin{equation}
\fua{\psi}{\tnr{X}}=\fua{\psi_1}{\tnr{X}}\odot_q\fua{\psi_2}{\tnr{X}}\,,\nonumber
\end{equation}
então são válidos os seguintes itens:
\begin{itemize}
\item[i.] para qualquer
$\tnr{H}\in{T}_{\con{\crt{V}{p}}\mapsto\con{\con{F}}}$,
\begin{eqnarray}
\grd{r}{\psi}{\tnr{X}}\odot_p\tnr{H}=\fua{\psi_1}{\tnr{X}}\odot_q\grd{r}{\psi_2}{\tnr{X}}\odot_p\tnr{H}+\nonumber\\
\tnr{H}\odot_p\grd{e}{\psi_1}{\tnr{X}}\odot_q\fua{\psi_2}{\tnr{X}}=\tnr{H}\odot_p\grd{e}{\psi}{\tnr{X}}\,;
 \nonumber
\end{eqnarray}
\item[ii.] se $p=q=r=s$ ,
\begin{eqnarray}
\grd{r}{\psi}{\tnr{X}}=\fua{\psi_1}{\tnr{X}}\odot_p\grd{r}{\psi_2}{\tnr{X}}+\nonumber\\
\lco\fua{\psi_2}{\tnr{X}}\rco_{(1,p+1)(p,2p)}\odot_p\lco\grd{e}{\psi_1}{\tnr{X}}\rco_{(1,2p+1)(p,3p)}
 \nonumber
\end{eqnarray}
e
\begin{eqnarray}
\grd{e}{\psi}{\tnr{X}}=\lco\grd{r}{\psi_2}{\tnr{X}}\rco_{(1,2p+1)(p,3p)}\odot_p\lco\fua{\psi_1}{\tnr{X}}\rco_{(1,p+1)(p,2p)}+\nonumber\\
\grd{e}{\psi_1}{\tnr{X}}\odot_q\fua{\psi_2}{\tnr{X}}\,.
 \nonumber
\end{eqnarray}


\item[iii.] se $p=q=r=s$ e $\con{V}_i=\con{K}_i=\con{Z}_i$ ,
\begin{equation}
\drd{r}{\psi}{\tnr{X}}=\fua{\psi_1}{\tnr{X}}\odot_p\drd{r}{\psi_2}{\tnr{X}}+
\lco\fua{\psi_2}{\tnr{X}}\rco_{(1,p+1)(p,2p)}\odot_p\drd{r}{\psi_1}{\tnr{X}}\nonumber
\end{equation}
e
\begin{equation}
\drd{e}{\psi}{\tnr{X}}=\drd{e}{\psi_2}{\tnr{X}}\odot_p\lco\fua{\psi_1}{\tnr{X}}\rco_{(1,p+1)(p,2p)}+
\drd{e}{\psi_1}{\tnr{X}}\odot_p\fua{\psi_2}{\tnr{X}}\nonumber\,.
\end{equation}
\end{itemize}
\end{prp}
\begin{prova} Os itens a seguir referem-se aos itens respectivos
do teorema.
\begin{itemize}
\item[i.] É uma conseqüência direta da aplicação da regra do
produto com a definição de gradiente.

\item[ii.] Dadas as condições do ítem, o tensor
$\grd{e}{\psi_1}{\tnr{X}}\in\cft{\crt{V}{p}\times\crt{K}{p}\times\crt{W}{p}}{\con{F}}$
no ítem i. pode ser transposto para
$\lco\grd{e}{\psi_1}{\tnr{X}}\rco_{(1,2p+1)(p,3p)}\in\cft{\crt{W}{p}\times\crt{K}{p}\times\crt{V}{p}}{\con{F}}$.
Desta forma, o produto contrativo com o tensor $\tnr{H}$ na
igualdade
\begin{eqnarray}
\grd{r}{\psi}{\tnr{X}}\odot_p\tnr{H}=\fua{\psi_1}{\tnr{X}}\odot_p\grd{r}{\psi_2}{\tnr{X}}\odot_p\tnr{H}+\nonumber\\
\lco\fua{\psi_2}{\tnr{X}}\rco_{(1,p+1)(p,2p)}\odot_p\lco\grd{r}{\psi_1}{\tnr{X}}\rco_{(1,2p+1)(p,3p)}\odot_p\tnr{H}\,,\forall\,\tnr{H}\in\crt{V}{p}\,.
 \nonumber
\end{eqnarray}
pode ser eliminado. A igualdade para o gradiente à esquerda é
obtida da mesma forma.

\item[iii.]
Pelas condições do ítem, pode-se verificar facilmente que
\begin{equation}
\lco\grd{e}{\psi_1}{\tnr{X}}\rco_{(1,2p+1)(p,3p)}= \grd{r}{\psi_1}{\tnr{X}}\,.\nonumber
\end{equation}
Tomando a igualdade do gradiente à direita no ítem ii, tem-se,
para um tensor identidade
$\tnr{I}\in{\con{T}}_{\con{\crt{V}{p}\times\crt{V}{p}}\mapsto\con{\con{F}}}$,
que
\begin{eqnarray}
\grd{r}{\psi}{\tnr{X}}\odot_{2p}\tnr{I}&=&\fua{\psi_1}{\tnr{X}}\odot_q\grd{r}{\psi_2}{\tnr{X}}\odot_{2p}\tnr{I}+\nonumber\\
&&\lco\fua{\psi_2}{\tnr{X}}\rco_{(1,p+1)(p,2p)}\odot_q\grd{r}{\psi_1}{\tnr{X}}\odot_{2p}\tnr{I}\,.
 \nonumber
\end{eqnarray}
Tal desenvolvimento também pode ser realizado para a igualdade do
gradiente à esquerda.
\end{itemize}
\end{prova}



\begin{prp}\label{teo:gradienteConstante}
Sejam os espaços tensoriais de Banach
$\ete{\crt{V}{p}}{\con{F}}$ e $\ete{\crt{W}{q}}{\con{F}}$. Seja o
subconjunto aberto ${\tilde{T}}_{\con{\crt{V}{p}}\mapsto\con{\con{F}}}\subset\cft{\crt{V}{p}}{\con{F}}$
e a função no mapeamento
\begin{equation}\nonumber
\map{\psi}{{\tilde{T}}_{\con{\crt{V}{p}}\mapsto\con{\con{F}}}}{\cft{\crt{W}{q}}{\con{F}}}
\end{equation}
G-diferenciável no seu domínio. Considerando os gradientes $\gqu{r}{\psi}$ e
$\gqu{e}{\psi}$ constantes em ${\tilde{T}}_{\con{\crt{V}{p}}\mapsto\con{\con{F}}}$ com regras respectivas  $\fua{\gqu{r}{\psi}}{\tnr{X}}=\tnr{G}_r$ e $\fua{\gqu{e}{\psi}}{\tnr{X}}=\tnr{G}_e$, então
\begin{equation}\nonumber
 \tnr{G}_r \odot_p \lpa \tnr{H}_1 - \tnr{H}_2 \rpa = \lpa \tnr{H}_1 - \tnr{H}_2 \rpa \odot_p \tnr{G}_e = \fua{\psi}{\tnr{H}_1} - \fua{\psi}{\tnr{H}_2}\,,
\end{equation}
onde $\tnr{H}_1$ e $\tnr{H}_2$ são tensores quaisquer de ${\tilde{T}}_{\con{\crt{V}{p}}\mapsto\con{\con{F}}}\,$.
\end{prp}
\begin{prova}
Consideremos, primeiramente, uma regra qualquer para $\psi$ tal que sua G-derivada, com regra $\fua{\dvg{\psi}}{\tnr{X}}=\tnr{G}$, é constante. Desta forma, seja a regra $\fua{\psi}{\tnr{X}}=\tnr{G}\odot_p \tnr{X} + \tnr{C}$, onde $\tnr{C}\in{\tilde{T}}_{\con{\crt{V}{p}}\mapsto\con{\con{F}}}$ é também constante. Aplicando (\ref{eq:calculoDerivadaDirecional}), tem-se que
\begin{equation}\nonumber
 \fua{\lco\fua{\dvg{\psi}}{\tnr{X}}\rco}{\tnr{H}}=\fua{\psi}{\tnr{H}}-\tnr{C}\,,\,\forall\,\tnr{H}\in{\tilde{T}}_{\con{\crt{V}{p}}\mapsto\con{\con{F}}}\,.
\end{equation}
Tomando os tensores $\tnr{H}_1$ e $\tnr{H}_2$ quaisquer, tem-se duas igualdades no formato na igualdade anterior. Quando tais igualdades são subtraídas uma da outra obtém-se
\begin{equation}\nonumber
 \fua{\lco\fua{\dvg{\psi}}{\tnr{X}}\rco}{\tnr{H}_1}-\fua{\lco\fua{\dvg{\psi}}{\tnr{X}}\rco}{\tnr{H}_2}=\fua{\psi}{\tnr{H}_1}-\fua{\psi}{\tnr{H}_2}\,.
\end{equation}
A partir daí, com base na definição de gradiente, obtém-se as últimas igualdades da proposição.
\end{prova}

\subsection{Contornos Regulares}\index{contorno!regular}\label{sec:contornoRegular}
Sejam os subespaços afins $\saf{U}{S}{\ele{a}}{F}$ e $\saf{
\partial U}{\partial S}{\ele{b}}{F}$ do espaço afim de Hilbert
$\eaf{V}{A}{F}$, cuja dimensão é maior que 2, onde $\partial U$ e
$\epo{\partial S}_\ele{b}$ são os contornos dos conjuntos
$\con{U}$ e $\epo{S}_\ele{a}$ respectivamente. Seja $\ehr{W}{F}$
subespaço bidimensional de $\ehr{V}{F}$. O contorno $\epo{\partial S}_\ele{b}$ é
dito regular se para cada
$\ele{u}\in\epo{\partial S}_\ele{b}$ existir um mapeamento bijetor
\begin{equation}
\map{\zeta}{\con{Z}}{\viz{\lch\ele{u}\rch}\cap\epo{\partial S}_\ele{b}}\,,
\end{equation}
onde
 \begin{itemize}
    \item[i.] o conjunto $\con{Z}\subset\con{W}$ é aberto;
    \item[ii.] o conjunto $\viz{\lch\ele{u}\rch}\subset\epo{A}$;
    \item[iii.] a bijeção $\zeta$ é G-suave de ordem 1 em $\con{Z}$;
    \item[iv.] a função $\zeta^{-1}$ é contínua em
    $\viz{\lch\ele{u}\rch}\cap\epo{\partial S}_\ele{b}$;
    \item[v.] a derivada $\fua{\dvg{\zeta}}{\fua{\zeta^{-1}}{\ele{u}}}$ é inversível.
\end{itemize}


\subsubsection{Função Normal Unitária}
Considerando as condições do ítem anterior, seja o campo característico
$\fac{F}_{\epo{\partial S}_\ele{b}}^{\partial\con{U}}$, tal que os pontos de
$\epo{\partial S}_\ele{b}$ são relacionados, de forma biunívoca, aos vetores de
$\partial\con{U}$. Seja $\lch \vun{w}_1 , \vun{w}_2\rch$ uma das bases ortonormais de
$\ehr{W}{F}$. Adotando
\begin{equation}
\overline{\zeta}:=\fac{F}_{\epo{\partial
S}_\ele{b}}^{\partial\con{U}}\circ\zeta\,,
\end{equation}
dado $\vto{z}_\vto{u}=\fua{\overline{\zeta}^{\,-1}}{\vto{u}}$, onde $\vto{u}=\fua{\fac{F}_{\epo{\partial
S}_\ele{b}}^{\partial\con{U}}}{\ele{u}}$, o vetor polar
\begin{equation}
\vto{n}_\vto{u}:=\frac{\fua{\lco\fua{\dvg{\overline{\zeta}}}{\vto{z}_\vto{u}}\rco}{\vun{w}_1}\barwedge
\fua{\lco\fua{\dvg{\overline{\zeta}}}{\vto{z}_\vto{u}}\rco}{\vun{w}_2}}{\|
\fua{\lco\fua{\dvg{\overline{\zeta}}}{\vto{z}_\vto{u}}\rco}{\vun{w}_1}\barwedge
\fua{\lco\fua{\dvg{\overline{\zeta}}}{\vto{z}_\vto{u}}\rco}{\vun{w}_2}\|_{\ehr{V}{F}}}
\end{equation}
é denominado \emph{vetor normal unitário}\index{vetor!normal
unitário} ao conjunto $\viz{\lch\vto{u}\rch}\cap\partial\con{U}$. Se
existir pelo menos mais uma função $\overline{\zeta}'\neq\overline{\zeta}$ definida a
partir de $\vto{u}$ e $\con{Z'}\subset\con{W}$ e
\begin{equation}
\vto{n}'_\vto{u}:=\frac{\fua{\lco\fua{\dvg{\overline{\zeta}'}}{\vto{z}'_\vto{u}}\rco}{\vun{w}_1}\barwedge
\fua{\lco\fua{\dvg{\overline{\zeta}'}}{\vto{z}'_\vto{u}}\rco}{\vun{w}_2}}{\|
\fua{\lco\fua{\dvg{\overline{\zeta}'}}{\vto{z}'_\vto{u}}\rco}{\vun{w}_1}\barwedge
\fua{\lco\fua{\dvg{\overline{\zeta}'}}{\vto{z}'_\vto{u}}\rco}{\vun{w}_2}\|_{\ehr{V}{F}}}=\vto{n}_\vto{u}\,,\,\forall\,\vto{u}\in\partial\con{U}\,,
\end{equation}
diz-se que o contorno regular $\partial\con{U}$ é
\emph{orientável}\index{contorno!orientável}. Neste contexto, é
possível definir o mapeamento
\begin{equation}
\map{\fac{N}_{\partial \con{U}}}{\partial \con{U}}{\con{V}}\,,
\end{equation}
com regra
$\fua{\fac{N}_{\partial\con{U}}}{\vto{x}}=\vto{n}_\vto{x}$. A
função $\fac{N}_{\partial \con{U}}$ é denominada normal unitária
ao contorno orientável $\partial \con{U}$. Se o vetor $\fua{\fac{N}_{\partial
U}}{\vto{u}}+\vto{u}\notin \con{U}$ para todo $\vto{u}\in\partial
\con{U}$, tem-se a representação $\fac{N}^+$  e o contorno orientável
$\partial \con{U}$ é dito estar \emph{positivamente
orientado}\index{contorno!positivamente orientado}. Caso
$\fua{\fac{N}_{\partial U}}{\vto{u}}+\vto{u}\in \con{U}$ para todo
$\vto{u}\in\partial \con{U}$, tem-se $\fac{N}^-$ e o contorno
$\partial \con{U}$ está \emph{negativamente
orientado}\index{contorno!negativamente orientado}.

%   \chapter{Teoria B�sica da Medida}

\section{Espa�o Mensur�vel}

\subsection{Classe}\index{classe}
Trata-se de um conjunto cujos elementos s�o conjuntos. Se o
conjunto $\con{A}$ � elemento da classe $\css{M}$, ent�o
$\con{A}\in\css{M}$ e sua representa��o se torna $\cor{A}$.
Se os conjuntos de $\css{M}$ s�o disjuntos
ent�o tal classe � dita \emph{disjunta}\index{classe!disjunta}. Se
$\css{M}$ � formada por subconjuntos do conjunto $\con{M}$ diz-se
que $\css{M}$ � classe de $\con{M}$. Neste caso, a representa��o de $\con{M}$ muda para $\cor{M}$.

\subsubsection{Classe Cont�vel}\index{classe!cont�vel}
Uma classe � cont�vel se existir uma bije��o entre seus elementos
e o conjunto dos naturais $\nat$.

\subsection{Anel de Conjunto}\index{anel!de conjunto}
Seja uma classe $\css{M}$ do conjunto n�o vazio $\cor{M}$. Tal
classe � dita um anel do conjunto $\cor{M}$ se
\begin{itemize}
\item[i.] o conjunto $\emptyset\in\css{M}$,

\item[ii.] dada uma classe cont�vel
$\lch\cor{M}_1,\cdots,\cor{M}_n\rch\subseteq\css{M}$, a uni�o
$\bigcup_{i=1}^{n}\cor{M}_i\in\css{M}$ e

\item[iii.] dados $\cor{A},\cor{B}\in\css{M}$, o conjunto
diferen�a $\cor{A}/\cor{B}\in\css{M}$.
\end{itemize}

\subsubsection{Anel Sigma}\index{anel!sigma} Um anel $\css{M}$ do
conjunto $\cor{M}$ � dito um anel sigma ou
\emph{$\sigma$-anel}\index{$\sigma$-anel} ou
\emph{$\sigma$-�lgebra}\index{$\sigma$-�lgebra} do conjunto
considerado se, dada a classe cont�vel
$\lch\cor{M}_1,\cor{M}_2,\cdots\rch\subseteq\css{M}$, a uni�o
$\bigcup_{i=1}^{\infty}\cor{M}_i\in\css{M}$.

\subsection{Espa�o Mensur�vel}\index{espa�o!mensur�vel}
O par ordenado $\lpa\cor{M},\css{M}\rpa$, onde $\css{M}$ � um
$\sigma$-anel de $\cor{M}$, � denominado espa�o mensur�vel. Se o
conjunto $\cor{M}$ define um campo, ent�o o espa�o mensur�vel �
dito um \emph{campo mensur�vel}\index{campo!mensur�vel}.

\section{Espa�o Medida}

\subsection{Fun��o de Conjunto}\index{fun��o!de conjunto}
Denomina-se fun��o de conjunto o par ordenado $\lpa
\css{M}_\fuc{f},\fuc{f} \rpa$, onde a classe $\css{M}_\fuc{f}$ � o dom�nio
da regra $\fuc{f}$. Em outras palavras, uma fun��o de conjunto �
aquela que possui um conjunto como argumento.

\subsubsection{Fun��o de Conjunto Aditiva}\index{fun��o!de
conjunto!aditiva} Dada uma classe $\css{M}$ do conjunto $\cor{M}$,
seja o mapeamento $\map{\fuc{m}}{\css{M}}{\real}$. A fun��o de
conjunto $\fuc{m}$ � chamada aditiva em $\css{M}$ se, dada a
classe disjunta
$\lch\cor{M}_1,\cdots,\cor{M}_n\rch\subseteq\css{M}$, tem-se a
igualdade
\begin{equation}
\fua{\fuc{m}}{\bigcup_{i=1}^{n}\cor{M}_i}=\sum_{i=1}^n\fua{\fuc{m}}{\cor{M}_i}\,.
\end{equation}
Vale ressaltar que $\fua{\fuc{m}}{\emptyset}=0$, pois
$\fua{\fuc{m}}{\emptyset}=\fua{\fuc{m}}{\emptyset\cup\emptyset}=\fua{\fuc{m}}{\emptyset}+\fua{\fuc{m}}{\emptyset}$.

\paragraph{Conte�do.} A fun��o de conjunto aditiva $\fuc{m}$ no
anel $\css{M}$ do conjunto $\cor{M}$ � dita um conte�do neste anel
se
\begin{equation}
\fua{\fuc{m}}{\cor{A}}\geqslant0\,,\,\forall\,\cor{A}\in\css{M}\,.
\end{equation}

\paragraph{Medida.} O conte�do $\fuc{m}$ no $\sigma$-anel
$\css{M}$ do conjunto $\cor{M}$ � denominado uma medida ou um
\emph{$\sigma$-conte�do}\index{$\sigma$-conte�do} neste anel. Em
outras palavras, dada a classe disjunta
$\lch\cor{M}_1,\cor{M}_2,\cdots\rch\subseteq\css{M}$,
\begin{equation}
\fua{\fuc{m}}{\bigcup_{i=1}^{\infty}\cor{M}_i}=\sum_{i=1}^\infty\fua{\fuc{m}}{\cor{M}_i}\,.
\end{equation}

\subsection{Espa�o Medida}\index{espa�o!medida}
Dado o espa�o mensur�vel $\lpa\cor{M},\css{M}\rpa$ e uma medida
$\fuc{m}$ em $\css{M}$, chama-se o par ordenado
$\lpa\lpa\cor{M},\css{M}\rpa,\fuc{m}\rpa$ de espa�o medida, cuja
representa��o � abreviada para $\ems{M}{\fuc{m}}{M}\,$.

\section{Fun��es Mensur�veis}

\subsection{Fun��o Mensur�vel}\index{fun��o!mensur�vel}
Sejam o espa�o mensur�vel $\lpa\cor{M},\css{M}\rpa$ e o campo
mensur�vel $\lpa\con{F},\css{F}\rpa$. A fun��o no mapeamento
$\map{f}{\cor{M}}{\con{F}}$ � dita mensur�vel se a preimagem
\begin{equation}
\con{R}^{-1}_{\con{E},\fun{f}}\in\css{M}\,,\,\forall\,\con{E}\in\css{F}\,.
\end{equation}

\subsubsection{Fun��o Simples}\index{fun��o!simples} Sejam o
espa�o mensur�vel $\lpa\cor{M},\css{M}\rpa$ e o campo mensur�vel
$\lpa\con{F},\css{F}\rpa$. Considerando a classe disjunta
$\lch\cor{M}_1,\cdots,\cor{M}_n\rch\subseteq\css{M}$, a fun��o
mensur�vel no mapeamento $\map{s}{\cor{M}}{\con{F}}$ � dita simples se
sua regra for
\begin{equation}
\fua{s}{\ele{x}}=\sum_{i=1}^n\alpha_i\fua{\delta_{\cor{M}_i}}{\ele{x}}\,,
\end{equation}
onde $\alpha_i\in\con{F}$ e a regra
\begin{equation}
\fua{\delta_{\cor{M}_i}}{\ele{x}}= \left\{
\begin{array}{ll}
1 & \textrm{se $\ele{x}\in\cor{M}_i$}\\
0 & \textrm{se $\ele{x}\notin\cor{M}_i$}
\end{array} \right.\,.
\end{equation}
A fun��o $\delta_{\cor{M}_i}$ � conhecida como \emph{fun��o
caracter�stica}\index{fun��o!caracter�stica de conjunto} de
$\cor{M}_i$.


\subsection{Converg�ncia de Fun��es Mensur�veis}\index{converg�ncia!de fun��es mensur�veis}
Sejam o espa�o medida $\ems{M}{\fuc{m}}{\css{M}}$ e o campo
mensur�vel $\lpa\con{F},\css{F}\rpa$. Seja o mapeamento
$\map{f}{\cor{M}}{\con{F}}$, onde $\fun{f}$ � mensur�vel. Uma seq�encia
de fun��es mensur�veis $\fun{f}_1,\fun{f}_2,\cdots$ que tamb�m
fazem $\cor{M}\mapsto\con{F}$ � dita convergente na medida para
$\fun{f}$ se, dado um n�mero real $\epsilon>0$,
\begin{equation}
\lim_{i\to\infty}\fua{\fuc{m}}{\lch\ele{x}\,:\,|\fua{f_i}{\ele{x}}-\fua{f}{\ele{x}}
|\leqslant\epsilon  \rch}=0\,.
\end{equation}

\subsection{Integra��o de Fun��o Simples}\index{integra��o!de fun��o simples}
Sejam o espa�o medida $\ems{M}{\fuc{m}}{\css{M}}$ e o campo
mensur�vel $\lpa\con{F},\css{F}\rpa$. Seja o mapeamento
$\map{s}{\cor{M}}{\con{F}}$, onde $\fun{s}$ � fun��o simples com regra
\begin{equation}
\fua{s}{\ele{x}}=\sum_{i=1}^n\alpha_i\fua{\delta_{\cor{M}_i}}{\ele{x}}\,.
\end{equation}
 A fun��o $\fun{s}$ � dita integr�vel em um conjunto
 $\cor{E}\in\css{M}$ se
\begin{equation}
\fua{\fuc{m}}{\bigcup_{i=1}^n\cor{E}\cap\cor{M}_i}<\infty\,.
\end{equation}
Se $\fun{s}$  � integr�vel, diz-se ent�o que o escalar
\begin{equation}
\int_{\cor{E}}\fun{s}\,:=\,\sum_{i=1}^n\alpha_i\fua{\fuc{m}}{\cor{E}\cap\cor{M}_i}
\end{equation}
� a \emph{integral}\index{integral!de fun��o simples} de $\fun{s}$
em $\cor{E}\,$.

\subsection{Seq��ncia Fundamental M�dia}\index{seq��ncia!fundamental m�dia}
Dados o espa�o medida $\ems{M}{\fuc{m}}{\css{M}}$ e o campo
mensur�vel $\lpa\con{F},\css{F}\rpa$, seja a seq��ncia de fun��es
simples $\fun{s}_1,\fun{s}_2,\cdots$ que mapeiam $\cor{M}$ para
$\con{F}$. Tal seq��ncia � dita fundamental m�dia em
$\cor{E}\in\css{M}$ se suas fun��es forem integr�veis em $\cor{E}$
e
\begin{equation}
\lim_{\min{\lpa i,j \rpa}\to\infty}
\int_{\cor{E}}\fun{s}_i-\int_{\cor{E}}\fun{s}_j =0\,.
\end{equation}
Em outras palavras, ao se percorrer as fun��es de uma seq��ncia
fundamental m�dia, as integrais dos termos $\fun{s}_i$ e
$\fun{s}_j$ tendem � igualdade.

\subsection{Integra��o de Fun��o Mensur�vel}\index{integra��o!de fun��o mensur�vel}
A partir do espa�o medida $\ems{M}{\fuc{m}}{\css{M}}$ e do campo
mensur�vel $\lpa\con{F},\css{F}\rpa$, seja a fun��o em
$\map{f}{\cor{M}}{\con{F}}$ mensur�vel. Se existir uma seq��ncia
fundamental m�dia $\fun{s}_1,\fun{s}_2,\cdots$, convergente na
medida para $\fun{f}$, tal que suas fun��es simples s�o todas
integr�veis em $\cor{E}\in\css{M}$, ent�o a fun��o $\fun{f}$ �
dita integr�vel em $\cor{E}$. Caso isto ocorra, o valor
\begin{equation}
\int_{\cor{E}}\fun{f}:=\lim_{i\to\infty} \int_{\cor{E}}\fun{s}_i
\end{equation}
� a \emph{integral}\index{integral!de fun��o mensur�vel} de
$\fun{f}$ em $\cor{E}\,$.

\subsubsection{Propriedades Fundamentais de Integrais}
Os teoremas a seguir apresentam as principais
propriedades\footnote{Demonstra��es destas propriedades s�o
encontradas em \aut{Munroe}\cite{munroe_1971_1}, pp. 127-129.} de
integrais de fun��es mensur�veis.


\begin{teo}
Dados o espa�o medida $\ems{M}{\fuc{m}}{\css{M}}$ e o campo
mensur�vel $\lpa\con{F},\css{F}\rpa$, seja a fun��o em
$\map{f}{\con{M}}{\con{F}}$ mensur�vel e integr�vel em cada conjunto da
classe disjunta
$\lch\cor{M}_1,\cdots,\cor{M}_n\rch\subseteq\css{M}$. Nestas
condi��es,
\begin{equation}
\int_{\bigcup_{i=1}^{n}\cor{M}_i}\fun{f}\,=\,\sum_{i=1}^n\int_{\cor{M}_i}\fun{f}\,.\nonumber
\end{equation}
\end{teo}


\begin{teo}\label{teo:IntegralLinear}
Sejam o espa�o medida $\ems{M}{\fuc{m}}{\css{M}}$ e o campo
mensur�vel $\lpa\con{F},\css{F}\rpa$. Sejam os mapeamentos
$\map{f}{\cor{M}}{\con{F}}$ e $\map{g}{\cor{M}}{\con{F}}$, onde $\fun{f}$ e
$\fun{g}$ s�o elementos de espa�os vetoriais de fun��es mensur�veis e integr�veis em $\cor{E}\in\css{M}$.
Dados $\alpha,\beta\in\con{F}$, a fun��o
$\alpha\fun{f}+\beta\fun{g}$ tamb�m � integr�vel em $\cor{E}$ e
\begin{equation}
\int_{\cor{E}}\lpa\alpha\fun{f}+\beta\fun{g}\rpa\,=\,
\alpha\int_{\cor{E}}\fun{f}+
\beta\int_{\cor{E}}\fun{g}\,.\nonumber
\end{equation}
\end{teo}

\begin{teo}
Sejam o espa�o medida $\ems{M}{\fuc{m}}{\css{M}}$ e o campo
mensur�vel $\lpa\con{F},\css{F}\rpa$. Dada a fun��o em
$\map{f}{\cor{M}}{\con{F}}$ mensur�vel e integr�vel em
$\cor{E}\in\css{M}$, se $\fua{\fuc{m}}{\cor{E}}=0$, ent�o
\begin{equation}
\int_{\cor{E}}\fun{f}\,=\,0\,.\nonumber
\end{equation}
\end{teo}

\begin{teo}
Sejam o espa�o medida $\ems{M}{\fuc{m}}{\css{M}}$ e o campo
mensur�vel $\lpa\con{F},\css{F}\rpa$. Sejam os mapeamentos
$\map{f}{\cor{M}}{\con{F}}$ e $\map{g}{\cor{M}}{\con{F}}$, onde $\fun{f}$ e
$\fun{g}$ s�o mensur�veis e integr�veis em $\cor{E}\in\css{M}$.
Dado o conjunto $\cor{A}\subseteq\cor{E}$, onde
$\fua{\fuc{m}}{\cor{A}}=0$, se
\begin{equation}
\fua{\fun{f}}{\ele{x}}\geqslant\fua{\fun{g}}{\ele{x}}\,,\,\forall\,\ele{x}\in\cor{A}'\,,\nonumber
\end{equation}
ent�o
\begin{equation}
\int_{\cor{E}}\fun{f}\,\geqslant\,\int_{\cor{E}}\fun{g}\,.\nonumber
\end{equation}
\end{teo}

\section{Integra��o de Fun��es Tensoriais}

\subsection{Fun��o Tensorial Mensur�vel}\index{fun��o!tensorial!mensur�vel}
Dado o espa�o tensorial $\ete{\crt{V}{p}}{F}$, sejam o espa�o
mensur�vel $\lpa\cft{\crt{V}{p}}{F},\css{V}\rpa$ e o campo
mensur�vel $\lpa\con{F},\css{F}\rpa$. Nestas condi��es, a fun��o
tensorial em
\begin{equation}
\map{\nu}{\cft{\crt{V}{p}}{F}}{\con{F}}
\end{equation}
� dita mensur�vel. Desta forma, dado um tensor qualquer
$\tnr{T}\in\cft{\crt{V}{p}}{F}$, a fun��o representante
$\ftr{T}{p}$ � mensur�vel.

\subsection{Fun��o Tensorial Integr�vel}\index{fun��o!tensorial!integr�vel}\label{sec:funcaoIntegravel}
Dados os espa�os tensoriais $\ete{\crt{V}{p}}{F}$ e
$\ete{\crt{W}{q}}{F}$, seja o mapeamento
\begin{equation}
\map{\psi}{\cft{\crt{V}{p}}{F}}{\cft{\crt{W}{q}}{F}}\,.
\end{equation}
Dado um tensor qualquer $\tnr{X}\in\cft{\crt{W}{q}}{F}$, a fun��o
$\psi$ � dita integr�vel em
${\tilde{T}}_{\con{\crt{V}{p}}\mapsto\con{\con{F}}}\subseteq\cft{\crt{V}{p}}{F}$
se a fun��o no mapeamento
\begin{equation}
\map{\ftr{X}{q}\circ\psi}{\cft{\crt{V}{p}}{F}}{\con{F}}
\end{equation}
for mensur�vel e tamb�m integr�vel em
${\tilde{T}}_{\con{\crt{V}{p}}\mapsto\con{\con{F}}}$. Nestas
condi��es, com base no teorema \ref{teo:IntegralLinear},
conclui-se que a fun��o tensorial $\mathcal{I}_\psi$ com regra
\begin{equation}\label{eq:funcaoI}
\fua{\mathcal{I}_\psi}{\tnr{X}}=
\int_{{\tilde{T}}_{\con{\crt{V}{p}}\mapsto\con{\con{F}}}}\ftr{X}{q}\circ\psi\,,
\end{equation}
� linear. Em outras palavras, dado o espa�o vetorial
$\evl{\cft{\crt{W}{q}}{\con{F}}}{\con{F}}{\con{F}}$ ,
\begin{equation}
\mathcal{I}_\psi\in\cfl{\cft{\crt{W}{q}}{\con{F}}}{\con{F}}.
\end{equation}
O teorema \ref{teo:RieszGeneralizado} permite afirmar que existe
um �nico tensor $\tnr{E}\in\cft{\crt{W}{q}}{F}$ tal que
\begin{equation}\label{eq:tensorFuncaoI}
\fua{\ftr{E}{q}}{\tnr{T}}=\fua{\mathcal{I}_\psi}{\tnr{T}}\,,\,\forall\,\tnr{T}\in\cft{\crt{W}{q}}{F}\,.
\end{equation}
Tal tensor � denominado a \emph{integral}\index{integral!de fun��o
tensorial} de $\psi$ em
${\tilde{T}}_{\con{\crt{V}{p}}\mapsto\con{\con{F}}}$, representado
por
\begin{equation}\label{eq:tensorIntegral}
\int_{{\tilde{T}}_{\con{\crt{V}{p}}\mapsto\con{\con{F}}}}\psi\,.
\end{equation}

\begin{prp}\label{prp:integralIdentidade}
Dados os espa�os tensoriais $\ete{\crt{V}{p}}{F}$ e
$\ete{\crt{U}{r}\times\crt{W}{s}}{F}$, seja o mapeamento
\begin{equation}
\map{\psi}{\cft{\crt{V}{p}}{F}}{\cft{\crt{U}{r}\times\crt{W}{s}}{F}}\,.\nonumber
\end{equation}
Considerando os tensores $\tnr{A}\in\cft{\crt{U}{r}}{F}$ e $\tnr{B}\in\cft{\crt{W}{s}}{F}$,
s�o v�lidas as seguintes igualdades:
\begin{eqnarray}
\tnr{A}\odot_r\int_{\cft{\crt{V}{p}}{F}}\psi&=&\int_{\cft{\crt{V}{p}}{F}}\rft{A}{r}\circ\psi\,;\nonumber\\
\lco\int_{\cft{\crt{V}{p}}{F}}\psi\rco\odot_s\tnr{B}&=&\int_{\cft{\crt{V}{p}}{F}}\lft{B}{s}\circ\psi
\,.\nonumber
\end{eqnarray}
\end{prp}
\begin{prova}
A partir da regra (\ref{eq:funcaoI}), tem-se o desenvolvimento a seguir:
\begin{eqnarray}
\fua{\mathcal{I}_{\lco\rft{A}{r}\circ\psi\rco}}{\tnr{X}}&=&
\int_{\cft{\crt{V}{p}}{F}}\ftr{X}{s}\circ\lco\rft{A}{r}\circ\psi\rco\nonumber\\
&=&
\int_{\cft{\crt{V}{p}}{F}}\lco \tnr{X}\odot_s\tnr{A}\rco_{\overleftarrow{\odot} r}\circ\psi\nonumber\\
&=&
\fua{\mathcal{I}_\psi}{\tnr{X}\odot_s\tnr{A}}\,.\nonumber
\end{eqnarray}
Associadas �s fun��es lineares $\fua{\mathcal{I}_{\lco\rft{A}{r}\circ\psi\rco}}{\tnr{X}}$ e $\fua{\mathcal{I}_\psi}{\tnr{X}\odot_s\tnr{A}}$, sejam respectivamente as integrais $\tnr{E_1}\in\cft{\crt{W}{s}}{F}$ e $\tnr{E_2}\in\cft{\crt{U}{r}}{F}$ abrevia��es de $\int_{\cft{\crt{V}{p}}{F}}\rft{A}{r}\circ\psi$ e $\int_{\cft{\crt{V}{p}}{F}}\psi$. Da igualdade anterior, � poss�vel concluir que
\begin{equation}
\fua{\lft{{E_1}}{s}}{\tnr{X}}=\fua{\lft{{E_2}}{r}}{\tnr{X}\odot_s\tnr{A}}=
\lpa\tnr{X}\odot_s\tnr{A}\rpa \odot_r  \tnr{{E_2}} = \fua{\lco \tnr{A}\odot_r\tnr{{E_2}}\rco_{\overrightarrow{\odot} s}}{\tnr{X}}\,,\,\forall \tnr{X}\in\cft{\crt{W}{s}}{F}\,.
\nonumber
\end{equation}
A partir da nota��o adotada (\ref{eq:tensorIntegral}), conclui-se a primeira igualdade do teorema. A demonstra��o da segunda igualdade � feita de forma similar.
\end{prova}

\subsubsection{Identidades de Green}\index{Green!identidades de}
A partir de um conjunto fechado $\con{A}$, tal que seu contorno � orient�vel, as identidades de Green s�o igualdades que apresentam uma importante rela��o entre integrais definidas sobre o interior $\widehat{\con{A}}$ e integrais definidas sobre o contorno $\partial{\con{A}}$.

Dentre as identidades de Green, a mais geral � aquela que resulta do chamado Teorema de Green. De um corol�rio deste teorema, denominado Teorema de Gauss-Ostrogradsky ou \emph{Teorema da Diverg�ncia}\index{Diverg�ncia!Teorema da}, resulta uma outra identidade de Green, fundamental no estudo da Din�mica de corpos cont�nuos. Estes dois teoremas s�o apresentados a seguir.


\begin{teo}[Green]\index{Green!teorema de}\label{teo:Green}
Seja o espa�o de Hilbert $\ehr{V}{F}$ com dimens�o maior que 2 e
um conjunto fechado $\con{U}\subset\con{V}$, tal que seu contorno
${\partial\con{U}}$ � orient�vel. Dado o espa�o tensorial
$\ete{\crt{W}{q}}{F}$, seja o mapeamento
\begin{equation}
\map{\psi}{\con{V}}{\cft{\crt{W}{q}}{\con{F}}}\nonumber\,,
\end{equation}
onde a fun��o $\psi$ � cont�nua em $\con{U}$ e G-suave de ordem 1 no interior
$\widehat{U}$. Considerando o contorno ${\partial\con{U}}$
positivamente orientado, sejam os mapeamentos
\begin{eqnarray}
\map{\psi^\otimes_e}{\partial\con{U}}{\cft{\con{V}\times\crt{W}{q}}{\con{F}}}
\,\,\,&e&\,\,\,
\map{\psi^\otimes_r}{\partial\con{U}}{\cft{\crt{W}{q}\times\con{V}}{\con{F}}}\,,\nonumber
\end{eqnarray}
onde
\begin{eqnarray}
\fua{\psi^\otimes_e}{\vto{x}}= \fua{\fac{N}^+_{{\partial\con{U}}}}{\vto{x}} \otimes \fua{\psi}{\vto{x}}
\,\,\,&e&\,\,\,
\fua{\psi^\otimes_r}{\vto{x}}= \fua{\psi}{\vto{x}}\otimes\fua{\fac{N}^+_{{\partial\con{U}}}}{\vto{x}}\,.\nonumber
\end{eqnarray}
Neste contexto, s�o v�lidas as igualdades
\begin{eqnarray}
\int_{\widehat{U}}\gqu{e}{\psi}\,=\,\int_{\partial\con{U}}\psi^\otimes_e\nonumber\,\,\,
&e&\,\,\,
\int_{\widehat{U}}\gqu{r}{\psi}\,=\,\int_{\partial\con{U}}\psi^\otimes_r\nonumber\,.
\end{eqnarray}
\end{teo}

\begin{teo}[Gauss-Ostrogradsky]\index{Gauss-Ostrogradsky!teorema de}
Considerando as condi��es do teorema \ref{teo:Green}, seja $\crt{W}{q}=\con{V}\times\crt{Z}{p}\times\con{V}$. Sejam os mapeamentos
\begin{eqnarray}
\map{\psi^\odot_e}{\partial\con{U}}{\cft{\crt{Z}{p}\times\con{V}}{\con{F}}}
\,\,\,&e&\,\,\,
\map{\psi^\odot_r}{\partial\con{U}}{\cft{\con{V}\times\crt{Z}{p}}{\con{F}}}\,,\nonumber
\end{eqnarray}
onde
\begin{eqnarray}
\fua{\psi^\odot_e}{\vto{x}}= \fua{\fac{N}^+_{{\partial\con{U}}}}{\vto{x}} \odot \fua{\psi}{\vto{x}}
\,\,\,&e&\,\,\,
\fua{\psi^\odot_r}{\vto{x}}= \fua{\psi}{\vto{x}}\odot\fua{\fac{N}^+_{{\partial\con{U}}}}{\vto{x}}\,.\nonumber
\end{eqnarray}
Desta forma, tem-se que
\begin{eqnarray}
\int_{\widehat{U}}\dqu{e}{\psi}\,=\,\int_{\partial\con{U}}\psi^\odot_e\nonumber\,\,\,
&e&\,\,\,
\int_{\widehat{U}}\dqu{r}{\psi}\,=\,\int_{\partial\con{U}}\psi^\odot_r\nonumber\,.
\end{eqnarray}
\end{teo}
\begin{prova}
Dado o tensor identidade $\tnr{I}\in\cft{V^2}{\con{F}}$, a partir da identidade de Green $\int_{\widehat{U}}\gqu{e}{\psi}=\int_{\partial\con{U}}\psi^\otimes_e$, pode-se dizer que o produto contrativo
\begin{equation}
\tnr{I}\odot_2\int_{\widehat{U}}\gqu{e}{\psi}=\tnr{I}\odot_2\int_{\partial\con{U}}\psi^\otimes_e\,.\nonumber
\end{equation}
Com base na proposi��o \ref{prp:integralIdentidade}, � poss�vel dizer que
\begin{equation}
\int_{\widehat{U}}\rft{I}{2}\circ\gqu{e}{\psi}=\int_{\partial\con{U}}\rft{I}{2}\circ\psi^\otimes_e\,.\nonumber
\end{equation}
A partir da igualdade (\ref{eq:funcaRepresentanteXIdentidade}), para quaisquer vetores $\vto{x}\in\widehat{U}$ e $\vto{y}\in\partial\con{U}$,
\begin{equation}
\fua{\rft{I}{2}\circ\gqu{e}{\psi}}{\vto{x}}=\tnr{I}\odot_2\fua{\gqu{e}{\psi}}{\vto{x}}=\fua{\dqu{e}{\psi}}{\vto{x}}\nonumber
\end{equation}
e
\begin{equation}
\fua{\rft{I}{2}\circ\psi^\otimes_e}{\vto{y}}=\tnr{I}\odot_2\lco\fua{\fac{N}^+_{{\partial\con{U}}}}{\vto{y}} \otimes \fua{\psi}{\vto{y}}\rco=\fua{\fac{N}^+_{{\partial\con{U}}}}{\vto{y}} \odot \fua{\psi}{\vto{y}}\,.\nonumber
\end{equation}
A demonstra��o para $\int_{\widehat{U}}\dqu{r}{\psi}\,=\,\int_{\partial\con{U}}\psi^\odot_r$ segue a mesma metodologia.
\end{prova}

    \begin{thebibliography}{99.}

\bibitem{adams_2003_2}\aut{Adams, R.; Fournier J.} \nob{Sobolev Spaces.} 2 ed. Kidlington: Academic Press, 2003.

\bibitem{agranovitch_2015_1}\aut{Agranovich, M. S.} \nob{Sobolev Spaces, Their Generalizations, and Elliptic Problems
in Smooth and Lipschitz Domains.} 1 ed. London: Springer-Verlag, 2015.

\bibitem{appleby_1987}\aut{Appleby, P. G. et al} \textit{On the Classification of Isotropic Tensors} in \nob{Glasgow Mathematical Journal}, Vol. 29, No. 2. Cambridge: Cambridge University Press, 1987, pp. 185-196.

\bibitem{artzy_1965_1}\aut{Artzy, R.} \nob{Linear Geometry.} 1 ed. Reading: Addison-Wesley Publishing Company, 1965.

\bibitem{backus_1997_1}\aut{Backus, G.} \nob{Continuum Mechanics.} Available at
<\url{http:// samizdat.mines.edu/backus/}>, Accessed on feb 2004.

\bibitem{bishop_1980_1}\aut{Bishop, R. L.; Goldberg, S. I.} \nob{Tensor Analysis on Manifolds.} 1 ed. Mineola: Dover Publications, 1980.

\bibitem{bowen_2008_1}\aut{Bowen, R. M.; Wang, C. -C.} \nob{Introduction to Vectors and Tensors.} 1 ed. Mineola: Dover Publications, 2008.

\bibitem{cameron_1999_1}\aut{Cameron, P. J.} \nob{Sets, Logic and Categories.} 1 ed. London: Springer-Verlag, 1999.

\bibitem{cervantes_2010}\aut{Cervantes, M.} \nob{Don Quijote De La Mancha} 1 ed. Madrid: Castalia, 2010.

\bibitem{chadwick_1999_2}\aut{Chadwick, P.} \nob{Continuum Mechanics: Concise Theory and Problems.} 2 ed. Mineola: Dover Publications, 1999.

\bibitem{christoffel_1869_1}\aut{Christoffel, E. B.} \textit{Ueber die Transformation der homogenen Differentialausdr�cke zweiten Grades} in \nob{Journal f�r die Reine und Angewandte Mathematik.} Vol. 70, 1869, pp. 46-70.

\bibitem{ciarlet_1988_2}\aut{Ciarlet, P. G.} \nob{Mathematical Elasticity - Volume 1:
Three Dimensional Elasticity.} 1. ed. Amsterdam: Elsevier Science Publishers BV, 1988.

\bibitem{connel_1999_1}\aut{Connel, E. H.} \nob{Elements of Abstract and Linear Algebra.}
Available at <\url{http://www.math.miami.edu/~ec/book/}>, Accessed on feb 2004.

\bibitem{conrad_XXXX_1}\aut{Conrad, F.} \nob{Tensor Products.} Available at
<\url{http://www.math.uconn.edu/~kconrad/blurbs/linmultialg/tensorprod.pdf}>, Accessed on sept 2017.

\bibitem{derbyshire_2006_1}\aut{Derbyshire, J.} \nob{Unknown Quantity: a Real and Imaginary History of Algebra.}
1 ed. Washington,D.C.: Joseph Henry Press, 2006.

\bibitem{dieudonne_1969_1}\aut{Dieudonn�, J.} \nob{Foundations Of Modern Analysis.}
1 ed. New York: Academic Press, 1969.

\bibitem{docarmo_2016}\aut{Do Carmo, M. P.} \nob{Differential Geometry of Curves and Surfaces.}
2 ed. Mineola: Dover Publications, 2016.

\bibitem{dodson_1991}\aut{Dodson, C.T.J.; Poston, T.} \nob{Tensor Geomery: The Geometric Viewpoint and its Uses.}
1 ed. Berlin: Springer-Verlag Inc., 1991.

\bibitem{einstein_1915_1}\aut{Einstein, A.} \textit{Zur allgemeinen Relativit�tstheorie} in \nob{Preussische Akademie der Wissenschaften, Sitzungsberichte.} Berlim, 1915, pp. 778-786, 799-801.

\bibitem{figueiredo_1964_1}\aut{Figueiredo, D. G.} \textit{A Simplified Proof of the Divergence Theorem} in \nob{The American Mathematical Monthly}, Vol. 71, No. 6, 1964, pp. 619-622.

\bibitem{flanders_1989_1}\aut{Flanders, H.} \nob{Differential Forms With the Applications to the Physical Sciences.} 1 ed. Mineola: Dover Publications, 1989.

\bibitem{gibbs_1881}\aut{Gibbs, J. W.} \nob{Elements of Vector Analysis Arranged for the Use of Students in Physics.} 1. ed.  New Haven: Tuttle, Morehouse \& Taylor, 1881.

\bibitem{gurtin_1981}\aut{Gurtin, M. E.} \nob{An Introduction to Continuum Mechanics.} 1. ed. San Diego: Academic Press, 1981.

\bibitem{gurtin_2010}\aut{Gurtin, M. E. et al.} \nob{The Mechanics and Thermodynamics of Continua.} 1. ed. Cambridge: Cambridge University Press, 2010.

\bibitem{halmos_1974_1}\aut{Halmos, P. R.} \nob{Measure Theory.} 1 ed. New York: Springer-Verlag Inc., 1974.

\bibitem{halmos_19742_1}\aut{Halmos, P. R.} \nob{Naive Set Theory.} 1 ed. New York: Springer-Verlag Inc., 1974.

\bibitem{hamilton_1846_1}\aut{Hamilton, W. R.} \textit{On Quaternions; or on a new System of Imaginaries in Algebra} in \nob{ The London, Edinburgh and Dublin Philosophical Magazine and Journal of Science: Volume XXIX.} 1846, pp. 26-31.

\bibitem{hermann_1975}\aut{Hermann, R.} \nob{Ricci and Levi-Civita's Tensor Analysis Paper.} 1 ed. Brookline: Math Sci Press, 1975.

\bibitem{hermes_1974_1}\aut{Hermes, H.; Markwald, W.} \textit{Foundations of Mathematics} in \nob{ Fundamentals of Mathematics: Volume I.} 1 ed. Massachusetts: MIT Press, 1987, pp. 1-86.

\bibitem{johnson_1970}\aut{Johnson, C. R.} \textit{Positive Definite Matrices} in \nob{The American Mathematical Monthly} Vol. 77, No. 3. New York: Mathematical Assossiation of America, 1970, pp. 259-264.

\bibitem{knuth_1997_1}\aut{Knuth, D. E.} \nob{The Art of Computer Programming, Volume 1.} 1 ed. Boston: Addison-Wesley, 1997.

\bibitem{kreyszig_1978_1}\aut{Kreyszig, E.} \nob{Introductory Functional Analysis With Applications.} 1 ed. Boston: John Wiley \& Sons, 1978.

\bibitem{lanczos_1970_1}\aut{Lanczos, C.} \nob{Space Through The Ages: The Evolution of Geometrical Ideas From Pythagoras to Hilbert and Einstein.} 1 ed. London: Academic Press, 1970.

\bibitem{lang_1993_3}\aut{Lang, S.} \nob{Real and Functional Analysis.} 3 ed. Berlin: Springer-Verlag, 1993.

\bibitem{levi_2013_1}\aut{Levi-Civita, T.} \nob{The Absolute Differential Calculus (Calculus of Tensors).} 1 ed. Mineola: Dover Publications, 2013.

\bibitem{loomis_2014_1}\aut{Loomis, L. H.; Sternberg, S.} \nob{Advanced Calculus.} 1
ed. Hackensack: World Scientific Publishing Co., 2014.

\bibitem{luque_2006_1}\aut{Luque, J.; Thibon, J.} \nob{Hyperdeterminantal Calculations of Selberg's And Aomoto's Integrals.} Available at <\url{https://arxiv.org/pdf/math/0607410.pdf}>, Accessed on july 2018.

\bibitem{malament_2005_1}\aut{Malament, D. B.} \nob{Notes on Geometry and Spacetime.}
Available at
<\url{http://www.lps.uci.edu/home/fac-staff/faculty/malament/geometryspacetimedocs/}>,
Accessed on aug 2006.

\bibitem{marcus_1973_1}\aut{Marcus, M.} \nob{Finite Dimensional Multilinear Algebra, Part I.} 1 ed. New York: Marcel Dekker Inc., 1973.

\bibitem{milne_2003_1}\aut{Milne J. S.} \nob{Group Theory.} Available at
<\url{http:// www.jmilne.org/math/CourseNotes/}>, Accessed on feb 2004.

\bibitem{munroe_1971_1}\aut{Munroe, M. E.} \nob{Measure and Integration} 1 ed.
Reading: Addison-Wesley Publishing Company, 1971.

\bibitem{ogden_1997}\aut{Ogden, R.W.} \nob{Non-Linear Elastic Deformations.} 1 ed. Mineola: Dover Publications, 1997.

\bibitem{ricci_1900_1}\aut{Ricci, G.; Levi-Civita, T.} \textit{M�thodes de Calcul Diff�rentiel Absolu et Leurs Applications} in \nob{Mathematische Annalen.} Vol 54, No. 1-2. Berlin: Springer, 1900, pp. 125-201.

\bibitem{segel_1999_1}\aut{Segel, L.} \nob{Mathematics Applied to Continuum Mechanics.} 1 ed. New York: Macmillan Publishing Co., 1977.

\bibitem{shen_2002_1}\aut{Shen, A.; Vereshchagin, N. K.} \nob{Basic Set Theory} 1 ed. Providence:  American Mathematical Society, 2002.

\bibitem{snapper_1971_1}\aut{Snapper, E.; Troyer, R. J.} \nob{Metric Affine Geometry.} 1
ed. New York: Academic Press, 1971.

\bibitem{strang_2006_4}\aut{Strang, G.} \nob{Linear Algebra and Its Applications.} 4 ed. Pacific Grove: Brooks-Cole Publishing, 2006.

\bibitem{soutasLittle_1999_1}\aut{Soutas-Little, R. W.} \nob{Elasticity.} 1 ed. Mineola: Dover Publications, 1999.

\bibitem{spivak_1965_1}\aut{Spivak, M.} \nob{Calculus on Manifolds: a Modern Approach to Classical Theorems of Advanced Calculus} 1 ed. Princeton: Perseus Books Publishing, 1965.

\bibitem{truesdell_1992}\aut{Truesdell, C. A.; Noll, W.} \nob{The Non-Linear Field Theories of Mechanics.} 1 ed. Berlin: Springer-Verlag Inc., 1992.

\bibitem{vaisman_1980_1}\aut{Vaisman, I.} \nob{Foundations of Three Dimensional Euclidean
Geometry.} New York: Marcel Dekker Inc., 1980.

\bibitem{vinberg_2003_1}\aut{Vinberg, E. B.} \nob{A Course in Algebra.} Rhode Island: American Mathematical Society, 2003.

\bibitem{voigt_1898_1}\aut{Voigt, W.} \nob{Die Fundamentalen Physikalischen Eigenschaften der Krystalle in Elementarer Darstellung.} Leipzig: Verlag Von Veit \& Comp., 1898.

\bibitem{waerden_1985_1}\aut{Waerden, B.} \nob{A History of Algebra: From al-Khw\=arizm\={\i} to Emmy Noether} 1 ed. Berlin: Springer-Verlag Inc., 1985.

\bibitem{waleffe_1997_1}\aut{Waleffe, F.} \nob{Tensor Product and Tensors.} Available at
<\url{http://www.math.wisc.edu/~milewski/}>, Accessed on october 2004.

\bibitem{weil_1953_2}\aut{Weyl, H.} \nob{The Classical Groups: Their Invariants and Representations.} 2 ed. Princeton: Princeton University Press, 1953.

\bibitem{wouk_1979_1}\aut{Wouk, A.} \nob{A Course of Applied Functional Analysis.} 1 ed.
New York: John Wiley \& Sons Inc., 1979.

\bibitem{zeidler_1995_1}\aut{Zeidler, E.} \nob{Applied Functional Analysis: Main Principles and Their Applications.}
1 ed. New York: Springer-Verlag Inc., 1995.

\end{thebibliography}

%    %\pagenumbering{arabic}
{\let\newpage\relax \part{Elementary Continuum Mechanics}\label{par:continuum}}



%    
\chapter{A Mec�nica do Cont�nuo}\label{sec:MecanicaContinuo}


\section{Elementos Fundamentais}

\subsection{Mat�ria}\index{mat�ria}
O conceito de mat�ria envolve tudo aquilo que � constitu�do por part�culas chamadas
\emph{f�rmions}. Denomina-se f�rmion\index{f�rmion} todo elemento sub-at�mico que obedece o chamado
Princ�pio da Exclus�o de Pauli\index{Pauli!Princ�pio da Exclus�o de}. Basicamente, este princ�pio
diz que em um determinado instante, para um determinado �tomo, dado o conjunto de par�metros da
equa��o de onda de seus f�rmions id�nticos, n�o h� dois conjuntos iguais de valores destes
par�metros. Tais par�metros s�o denominados \emph{n�meros qu�nticos}\index{n�mero qu�ntico}. Em
outras palavras, um par qualquer de f�rmions id�nticos de um �tomo, num dado instante, sempre
define um par de n�meros qu�nticos diferentes.

O estudo da estrutura do �tomo revela que entre suas part�culas, das quais os f�rmions fazem parte,
h� movimento relativo. Al�m disso, foi observado que somente uma parte extremamente pequena do
volume do �tomo � ocupado por seus elementos. Uma conclus�o natural destes fatos � a afirma��o
�bvia de que existem grandes ``vazios'' nesta estrutura. A mat�ria como um todo � portanto
intrinsecamente discreta, ou seja, descont�nua.

\subsubsection{Massa}\index{massa}

En\-ten\-de-se por massa toda grandeza que se relaciona, de maneira diretamente
proporcional, � quantidade de mat�ria presente em uma determinada parcela selecionada para estudo. O termo ``quantidade de mat�ria'', embora vago,
depende, de alguma forma, do n�mero e do tipo das part�culas constitutivas
desta parcela de mat�ria.

\subsection{Espa�o e Tempo}\label{sec:EspacoTempo}
Entende-se por \emph{morfologia}\index{morfologia} o estudo da forma, da ``estrutura externa'' de
algo. No �mbito do presente estudo, a morfologia de algo se baseia na caracteriza��o de seu
\emph{formato}, definido pela geometria, e de sua \emph{posi��o}, descrita pelo lugar que ocupa.
S�o exemplos de vari�veis morfol�gicas aqui consideradas, a dist�ncia, a �rea, o volume, o
paralelismo, a perpendicularidade, entre outras. Neste estudo, denomina-se
espa�o\index{espa�o!morfol�gico} o contexto que viabiliza a utiliza��o de vari�veis morfol�gicas.

Al�m de morfologia, � fundamental o conceito de
\emph{cronologia}\index{cronologia}; entendida como o estudo do
tempo. A consci�ncia humana de tempo traz consigo a no��o da
sucessividade de eventos: atrav�s dela, pode-se dizer que algo
ocorreu antes ou depois de outro. No entanto, para que se tenha o
``antes'' e o ``depois'', faz-se necess�rio definir o ``agora'' ou
o \emph{instante}\index{instante} atual. � do seq�enciamento
ilimitado de infinitos instantes sucessivos que se constitui a
estrutura cronol�gica. Uma parcela finita desta estrutura, tomada
para estudo, � denominada \emph{dura��o}\index{dura��o}.

� conveniente agrupar as estruturas morfol�gica e cronol�gica em um contexto mais
abrangente, chamado \emph{espa�o-tempo}\index{espa�o-tempo}, sobre o qual se estuda a
intera��o entre o espa�o e o tempo. Na abordagem da Mec�nica Cl�ssica, tal intera��o se
fundamenta na independ�ncia completa entre a morfologia e a cronologia; neste contexto,
tem-se o \emph{espa�o-tempo Newtoniano}\index{espa�o-tempo!Newtoniano}. A abordagem
Relativ�stica, por sua vez, torna a cronologia parte integrante da morfologia e invalida,
desta maneira, a separa��o conceitual destas duas estruturas.

\subsubsection{Movimento}\index{movimento}
Pode-se relacionar cada instante cronol�gico com uma determinada
estrutura morfol�gica. Denomina-se movimento a sucess�o cronol�gica
limitada ou ilimitada de infinitas estruturas morfol�gicas.


\subsubsection{Observador}\index{observador}
O estudo do movimento dos corpos � relativo, ou seja, ele s� � realizado quando se
estabelece o ponto de vista a partir do qual referenciam-se as quantidades mensur�veis
definidas. Desta forma, toda a an�lise quantitativa e tamb�m qualitativa do movimento s�o
feitas com base neste ponto de vista, denominado observador ou
\emph{referencial}\index{referencial}\rodape{Neste trabalho, o termo ``observador'' �
utilizado no contexto f�sico enquanto ``referencial'' aparace no contexto matem�tico.}.
Na pr�tica, um observador descreve o movimento de um corpo a partir do registro de uma
estrutura morfol�gica associada a um instante na dura��o de tempo considerada.


\subsection{Corpo}\index{corpo}
Denomina-se corpo uma parcela de mat�ria, selecionada para estudo, que possui uma
determinada estrutura morfol�gica em um determinado instante cronol�gico. A partir desta
defini��o, estabelece-se uma importante rela��o entre espa�o, tempo e
mat�ria\rodape{Sobre esta rela��o \aut{Weyl}\cite{weyl_1952_2}, p. 1, diz que
``\textsl{Espa�o e tempo s�o geralmente considerados como formas de exist�ncia do
mundo [natural], enquanto a mat�ria � sua subst�ncia.}''}; fundamental para a
modelagem f�sico-matem�tica do mundo natural\rodape{Neste trabalho, considera-se que os
termos ``mundo natural'' e ``mundo das id�ias'' fazem parte do chamado ``mundo real''.}.



\subsubsection{A Impenetrabilidade dos Corpos}
Sabe-se, intuitivamente, que um mesmo corpo pode assumir infinitas formas. No entanto, a
Impenetrabilidade dos Corpos estabelece que um corpo, em um determinado instante do
tempo, possui uma e somente uma estrutura morfol�gica. Fica evidente que as parcelas de
mat�ria que constituem o corpo, seus \emph{subcorpos}\index{subcorpos}, tamb�m seguem
este mesmo princ�pio.


\subsubsection{A Hip�tese dos Corpos Monol�ticos}\index{corpo!monol�tico} Em termos pr�ticos,
pode-se estudar um corpo sob a forma de um agregado composto de um grande n�mero de
part�culas. Dos fen�menos que podem ser modelados por este corpo, muito poucos s�o
aqueles cuja descri��o do comportamento particular de cada part�cula pode ser generalizada para o
comportamento de um conjunto de part�culas. Na grande maioria dos problemas que se apresentam, a
obten��o de valores m�dios gerais para grandezas medidas em cada part�cula n�o � tarefa simples:
al�m do comportamento da part�cula individual, deve-se levar em conta sua intera��o com as demais
part�culas. Devido ao seu elevado n�vel de complexidade, o procedimento para c�lculo de valores
m�dios � geralmente feito, a partir de t�cnicas estat�sticas, sobre parcelas de mat�ria com n�mero
relativamente pequeno de part�culas, de forma espec�fica.

M�todos estat�sticos de c�lculo mostram-se, na maioria das vezes, insuficientes e inadequados para
atender �s necessidades de in�meros problemas envolvendo parcelas de mat�ria com elevado n�mero de
part�culas. Uma abordagem diferenciada para solu��o destes problemas abstrai o conceito de mat�ria,
retirando-lhe a caracter�stica intr�nseca da descontinuidade. A mat�ria, nesta hip�tese, possui
estrutura monol�tica, sem vazios. Com esta aproxima��o, � poss�vel usufruir das seguintes
vantagens:
\begin{itemize}
  \item[a)] Um corpo pode ser relacionado com a id�ia matem�tica de cont�nuo ou espa�o
  m�trico completo;
  \item[b)] Fun��es e mapeamentos podem atuar indiretamente sobre um corpo;
  \item[c)] Descri��es e interpreta��es geom�tricas s�o aplic�veis;
  \item[d)] Conceitos de C�lculo Diferencial e Integral podem ser utilizados.
\end{itemize}

Conv�m esclarecer que n�o h� regras dogm�ticas que estabele�am as circunst�ncias pelas quais a
mat�ria � aproximada para uma estrutura monol�tica: tudo ir� depender das condi��es do problema em estudo. Pode-se afirmar, entretanto, que o n�mero de part�culas de mat�ria deve ser elevado o suficiente para que esta aproxima��o produza resultados aceit�veis.


\section{Mec�nica}

\subsection{Defini��o}
Diz-se que Mec�nica � a ci�ncia que realiza o estudo descritivo e preditivo do movimento dos
corpos. Descrever um fen�meno natural, neste caso, significa simplesmente medi-lo, enquanto
predizer significa relacionar tal medi��o com suas causas. O estudo descritivo na Mec�nica � geralmente denominado \emph{Cinem�tica}.

O principal objetivo da Mec�nica � propor modelos matem�ticos que, al�m de descrever o movimento
dos corpos com base nos conceitos de forma, tempo e temperatura, possam predizer tal movimento a
partir dos conceitos de for�a, energia e calor. Com base nas rela��es entre estes dois grupos de
conceitos, classificadas a seguir, o estudo preditivo � realizado.
\begin{itemize}
\item[a)] Rela��es gen�ricas: s�o as rela��es a que todo e qualquer corpo est� sujeito. A \emph{Est�tica}
objetiva estudar estas rela��es assumindo que n�o h� movimento. A \emph{Din�mica} aborda tais
rela��es admitindo a possibilidade de movimento
\item[b)] Rela��es constitutivas: s�o aquelas espec�ficas para determinados tipos ou grupos de corpos.
\end{itemize}


\subsection{A Mec�nica do Cont�nuo}
Trata-se do estudo descritivo e preditivo do movimento dos corpos monol�ticos. � a Mec�nica dos
corpos que s�o relacionados ao conceito matem�tico de cont�nuo. No �mbito da Mec�nica do Cont�nuo, toda a modelagem matem�tica baseia-se fundamentalmente em tr�s a��es:
\begin{itemize}
\item[a)] interpretar o corpo como um conjunto composto por infinitas part�culas e quantificar numericamente sua massa;
\item[b)] rotular cada um dos pontos do espa�o que s�o ocupados pelas part�culas do corpo;
\item[c)] rotular cada um dos infinitos instantes do tempo.
\end{itemize}

\subsubsection{Elementos Fundamentais}

Com o objetivo de possibilitar uma modelagem matem�tica adequada para as finalidades deste trabalho, s�o impostas algumas restri��es:
\begin{itemize}
\item[a)] a massa deve ser sempre um escalar n�o negativo;
\item[b)] a proximidade das part�culas do corpo deve estar refletida na sua estrutura morfol�gica;
\item[c)] os corpos s�o impenetr�veis;
\item[d)] as estruturas morfol�gica e cronol�gica se relacionam de maneira independente;
\item[e)] para todo e qualquer fim, o \emph{tempo � absoluto}\index{tempo!absoluto}, ou seja, a evolu��o dos infinitos instantes de tempo � constante.
\end{itemize}

\paragraph{Modelando a Mat�ria.}
A modelagem matem�tica da mat�ria � feita atrav�s de espa�os medida, definidos por
espa�os m�tricos completos. Os valores da medida quantificam a massa das parcelas de
mat�ria consideradas.

\paragraph{Modelando o Espa�o.}
A estrutura morfol�gica � modelada atrav�s do conceito de espa�o
afim Euclidiano. Nele est�o considerados tanto aspectos
geom�tricos quanto aspectos alg�bricos, subsidiados,
respectivamente, pelos conceitos de espa�o puntual e de espa�o
vetorial.

\paragraph{Modelando o Tempo.}
Para o presente estudo, o tempo nada mais � do que o campo Real.
Os instantes de tempo s�o n�meros reais e a dura��o do tempo � um subconjunto limitado por dois de seus instantes.

\paragraph{Modelando o Corpo.}
A modelagem do corpo resulta da jun��o dos modelos matem�ticos da
mat�ria, do espa�o e do tempo. Com base nas restri��es
apresentadas, diz-se que um corpo cont�nuo � um subespa�o medida
que est� associado, de maneira biun�voca, a um e somente um
subespa�o puntual de um subespa�o afim Euclidiano tridimensional,
em um determinado instante de tempo. O conjunto das diversas
associa��es deste tipo, em uma determinada dura��o de tempo,
constitui um movimento do corpo\rodape{Como a associa��o entre os
subespa�os medida e puntual � �nica para um instante de tempo,
pode-se dizer tamb�m que um movimento do corpo � o conjunto dos
diversos espa�os puntuais numa dura��o de tempo.}.


% fazer tabela da rela��o dos conceitos f�sicos com os matem�ticos

% Dizer como a MC ser� estudada no trabalho: princ�pios gerais e espec�ficos

%     \chapter{A Cinem�tica do Cont�nuo}



\section{O Corpo e o Espa�o-Tempo Newtoniano}


\subsection{Espa�o-Tempo Newtoniano}\index{espa�o-tempo!Newtoniano}
Considerando o espa�o afim Euclidiano tridimensional $\eaf{V}{A}{\real}$ e o conjunto dos
reais $\real$, representando o tempo, diz-se que o conjunto $\epo{A}\times\real$ � um
espa�o-tempo Newtoniano. Um subconjunto deste espa�o significa o produto cartesiano  dos
subespa�os de apenas um ou de ambos os termos constituintes. Em outras palavras, dado o
conjunto $\con{U}\subseteq\con{V}$, o espa�o-tempo Newtoniano
$\epo{S}_{\ele{a}}\times\tempo$, onde $\saf{U}{S}{\ele{a}}{\real}$ � subespa�o de
$\eaf{V}{A}{\real}$ e $\tempo\subseteq\real$ � uma dura��o do tempo, � subconjunto de
$\epo{A}\times\real$.



Em termos gen�ricos, de agora em diante, conv�m representar o espa�o-tempo Newtoniano
gerado a partir de um espa�o afim Euclidiano tridimensional qualquer $\eaf{M}{M}{\real}$
pela nota��o $\enw{M}{\tempo}{M}\,$, onde $\tempo$ � uma dura��o do tempo.


\subsubsection{Referencial}\index{referencial}

Seja o espa�o-tempo Newtoniano $\enw{M}{\tempo}{M}\,$, tal que $\con{M}$ define o espa�o orientado Euclidiano $(\eeu{M}{3},\hat{\tnr{P}})$. Seja o par ordenado
$(\ele{o},\tilde{\con{U}})$ um sistema de coordenadas afim de $\eaf{M}{M}{\real}\,$, onde
$\tilde{\con{U}}$ � base de $\eeu{M}{3}$. Considerando o conjunto $\lch t_{\ele{o}} \rch$
uma base de $\tempo$, o par ordenado $\rfr{o}{\tilde{\con{U}}}$ � denominado uma
base\index{espa�o-tempo!Newtoniano!base do} ou um referencial de $\enw{M}{\tempo}{M}\,$. Se a base $\tilde{\con{U}}$ � positivamente ou negativamente orientada, o referencial por ela definido recebe a mesma classifica��o.

O referencial � a modelagem matem�tica do conceito de observador, registrando um dado
evento f�sico por meio dos pares ponto e instante do tempo. Al�m disso, ele viabiliza a
eventual substitui��o do estudo de pontos pelo de vetores, bem como o de campos
tensoriais pelo de fun��es tensoriais\rodape{Ver se��es \ref{sec:AfimMetrico} e
\ref{sec:AfimTensorial}.}.


\subsection{Configura��es do Corpo}

\subsubsection{Corpo}\index{corpo}

Seja o espa�o de Banach $\ebh{\cor{M}}{\real}$, o espa�o medida $\ems{M}{\fuc{m}}{M}$ e
o espa�o-tempo Newtoniano $\enw{M}{\tempo}{M}\,$. Seja
$\overline{\epo{B}}\subseteq\epo{M}$ o fechamento de um conjunto aberto conexo
$\epo{B}_{\ele{a}}$, cujo contorno $\partial\epo{B}_{\ele{b}}$ � regular\rodape{Trata-se
de uma restri��o matem�tica, a partir da qual � poss�vel definir as Identidades de Green,
fundamentais para o estudo da Din�mica. Ver se��es \ref{sec:contornoRegular} e
\ref{sec:funcaoIntegravel}.}. Sejam os espa�os vetoriais $\eeu{B}{3}$ e $\eeu{\partial
B}{3}$, definidores dos subespa�os afins Euclidianos $\saf{B}{B}{\ele{a}}{\real}$ e
$\saf{\partial B}{\partial B}{\ele{b}}{\real}$. Um subconjunto $\cor{B}\subseteq\cor{M}$
� denominado corpo se existir, para um determinado instante $t\in\tempo$, um mapeamento
bijetor
\begin{equation}\label{eq:configuracao}
\map{\chi_t}{\cor{B}}{\overline{\epo{B}}}\,,
\end{equation}
onde a fun��o $\chi_t$ � um $\sug{1}$-difeomorfismo\rodape{Em outras palavras, $\chi_t$ e sua inversa s�o homeomorfismos G-diferenci�veis em seus dom�nios respectivos.}. Nestas condi��es, diz-se que $\chi_t$ � uma \emph{configura��o}\index{configura��o!de corpo} de $\cor{B}$ e a imagem
$\overline{\epo{B}}$, representada $\overline{\epo{B}}_t$, � dita a \emph{regi�o}\index{regi�o!de corpo} ocupada por $\cor{B}$. A medida $\fuc{m}$ � chamada \emph{fun��o
massa}\index{fun��o!massa} e o escalar real $\fua{\fuc{m}}{\cor{B}}$ representa a massa
de $\cor{B}$.

\subsubsection{Universo}\index{universo} Considerando as condi��es anteriores, se
o mapeamento (\ref{eq:configuracao}) for v�lido para qualquer $\cor{B}\subseteq\cor{M}$,
diz-se que $\cor{M}$ � um universo\rodape{Na verdade, o mundo natural � um espa�o-tempo,
relacionado a um espa�o afim m�trico quadridimensional, definido pelo chamado espa�o
vetorial de Minkowski.}. Em outras palavras, $\cor{M}$ � considerado um universo se ele e
seus subconjuntos pr�prios forem corpos. Como conseq��ncia, conclui-se que todo universo
� tridimensional.


\subsubsection{Configura��es Vetoriais}\index{configura��o!vetorial}
As diversas configura��es de um corpo $\cor{B}$, definidas segundo
(\ref{eq:configuracao}), n�o permitem eventuais manipula��es alg�bricas, necess�rias ao
estudo da Cinem�tica, entre os diferentes pontos relacionados a suas part�culas. Tal
manipula��o se d� atrav�s do uso de vetores. Para este fim, seja um referencial
$\rfr{o}{\tilde{\con{U}}}$ do espa�o-tempo Newtoniano $\enw{M}{\tempo}{M}$, onde
$\epo{M}$ � a regi�o de um universo $\cor{M}$. Seja $\overline{\epo{B}}_t\subset\epo{M}$
a regi�o de $\cor{B}$ no instante $t$ e o espa�os afins Euclidianos tridimensionais
$\saf{B}{B}{\ele{a}}{\real}$ e $\saf{\partial B}{\partial C}{\ele{b}}{\real}$, tal que
$\partial\epo{B}_{\ele{b}}$ � o contorno regular de $\epo{B}_{\ele{a}}$ cujo fechamento �
$\overline{\epo{B}_t}$. Nestas condi��es, pode-se definir um mapeamento caracter�stico
com regra
\begin{equation}
\fua{\fac{F}_{\overline{\epo{B}}_t}^{\overline{C}}}{x}= \lch
\begin{array}{l}
\fua{\fac{F}_{\partial\epo{B}_\ele{b}}^{\partial\ele{B}}}{x}\,,\,\,\forall\ele{x}\in\partial\epo{B}_\ele{b}\\
\fua{\fac{F}_{\epo{B}_\ele{a}}^{\ele{B}}}{x}\,,\,\,\forall\ele{x}\in\epo{B}_\ele{a}
\end{array}\right.\,.
\end{equation}
A partir da�, com base em (\ref{eq:configuracao}), pode-se definir o mapeamento bijetor
\begin{equation}\label{eq:configuracaoVetorial}
\map{\fac{F}_{\overline{\epo{B}}_t}^{\overline{B}}\circ\chi_t}{\cor{B}}{\overline{\con{B}}}\,,
\end{equation}
onde, no instante $t$, a fun��o $\fac{F}_{\overline{\epo{B}}}^{\overline{B}}\circ\chi_t$ � dita uma
configura��o vetorial de $\cor{B}$ e a imagem $\overline{B} \subset \con{M}$ sua regi�o vetorial, representada por $\overline{B}_t$.

\paragraph{As Configura��es Material e Espacial.}\index{configura��o!material}\index{configura��o!espacial} A configura��o vetorial resolve
em parte o problema da manipula��o alg�brica de vetores necess�ria ao estudo da
Cinem�tica. Quando a perspectiva de an�lise � a pr�pria part�cula do corpo e n�o os
diversos pontos que ela ocupa em cada instante de tempo, persiste a impossibilidade
alg�brica, j� que no mapeamento ($\ref{eq:configuracaoVetorial}$) os elementos do dom�nio
n�o s�o vetores.

A identifica��o das part�culas de um corpo como sendo vetores � feita adotando-se uma
configura��o vetorial arbitr�ria. Isto ocorre da seguinte forma: considerando as
condi��es do item anterior, seja $C_\chi$ o conjunto formado por todas as configura��es
poss�veis do corpo $\cor{B}$ e uma configura��o qualquer $\chi_r\in C_\chi\,$,
arbitrariamente definida. Nos termos de
(\ref{eq:configuracaoVetorial}), a configura��o vetorial de $\cor{B}$ feita por $\chi_r$
define
\begin{equation}\label{eq:configuracaoVetorialRef}
\map{\fac{F}_{\overline{\epo{B}}_r}^{\overline{B}_r}\circ\chi_r}{\cor{B}}{\overline{\con{B}}_r}\,.
\end{equation}
Quando esta configura��o vetorial arbitr�ria � utilizada para
identificar ou \emph{rotular} as part�culas do corpo, ela � denominada configura��o
material ou \emph{de refer�ncia}\index{configura��o!de refer�ncia}\rodape{� importante
ressaltar que n�o h� interdepend�ncia entre os conceitos de configura��o de refer�ncia e
de referencial.} e a imagem $\overline{\con{B}}_r$ � dita a \emph{regi�o material ou de refer�ncia}\index{regi�o!de refer�ncia}\index{regi�o!material} de $\cor{B}$. Neste contexto, diz-se que a part�cula
\begin{equation}
\prt{u}:=\fua{\lco\chi_r^{-1}\circ{\fac{F}_{\overline{\epo{B}}_r}^{\overline{B}_r}}^{-1}\rco}{\vto{u}}
\end{equation}
� rotulada pelo vetor $\vto{u}$. Os vetores da regi�o que rotula um corpo s�o
denominados \emph{vetores materiais ou de refer�ncia}\index{vetor!material}\index{vetor!de refer�ncia}. Da mesma forma, pode-se definir, a partir da configura��o $\chi_t\,$, o mapeamento bijetor
\begin{equation}\label{eq:configuracaoVetorialEsp}
\map{\fac{F}_{\overline{\epo{B}}_t}^{\overline{B}_t}\circ\chi_t}{\cor{B}}{\overline{\con{B}}_t}\,,
\end{equation}
onde a fun��o nele definida � dita a configura��o espacial do
corpo $\cor{B}$ e $\overline{B}_t$ � sua \emph{regi�o espacial}\index{regi�o!espacial}. Os vetores desta regi�o s�o ditos \emph{espaciais}\index{vetor!espacial}.

Combinando os conceitos de configura��o material e espacial, pode-se definir o mapeamento
\begin{equation}
\map{\fac{F}_{\overline{\epo{B}}_t}^{\overline{B}_t}\circ\chi_t\circ\chi_r^{-1}\circ{\fac{F}_{\overline{\epo{B}}_r}^{\overline{B}_r}}^{-1}}{\overline{\con{B}}_r}{\overline{\con{B}}_t}\,,
\end{equation}
onde a bije��o
\begin{equation}\label{eq:configEspacialDescMaterial}
\negsymbol{\chi}_t:=\fac{F}_{\overline{\epo{B}}_t}^{\overline{B}_t}\circ\chi_t\circ\chi_r^{-1}\circ{\fac{F}_{\overline{\epo{B}}_r}^{\overline{B}_r}}^{-1}\,,
\end{equation}
no instante $t$, � dita a descri��o material da configura��o espacial do
corpo $\cor{B}$. Da mesma forma, a fun��o inversa
\begin{equation}
\negsymbol{\chi}_t^{-1}:=\fac{F}_{\overline{\epo{B}}_r}^{\overline{B}_r}\circ\chi_r\circ\chi_t^{-1}\circ{\fac{F}_{\overline{\epo{B}}_t}^{\overline{B}_t}}^{-1}
\end{equation}
� a descri��o espacial da configura��o material de $\cor{B}$. Em termos gerais,
a descri��o no dom�nio material de toda e qualquer quantidade mec�nica mensur�vel, selecionada para estudo, � qualificada como \emph{Lagrangiana}\index{descri��o!Lagrangiana}. Se este dom�nio for uma regi�o
espacial, tem-se a descri��o \emph{Euleriana}\index{descri��o!Euleriana}.

As configura��es vetoriais presentes em (\ref{eq:configuracaoVetorial}) e (\ref{eq:configuracaoVetorialRef}) pressup�em a exist�ncia dos respectivos referenciais
$\rfr{o}{\tilde{\con{U}}}$ e $\rfr{o_r}{\tilde{\con{U}}_r}\,$ sincronizados, ou seja, $t_{o_r}=t_o\,$. O primeiro � denominado um \emph{referencial espacial}\index{referencial!espacial} enquanto o segundo um \emph{referencial material}\index{referencial!material}.

Com base no que foi exposto, a figura \ref{fg:configvet} apresenta, para um dado instante $t$, um esquema envolvento as configura��es do corpo $\cor{B}$ e suas descri��es material e espacial.
\begin{figure}[!ht]
\centering
\begin{center}
\input{partes/parte2/figs/c_cinemat/configvet.pstex_t}
\end{center}
\titfigura{Configura��es e Regi�es do Corpo $\cor{B}$ no instante $t$.}\label{fg:configvet}
\end{figure}


\begin{prp}
Sejam o espa�o medida $\ems{\con{M}}{\fuc{m}}{M}$ e o espa�o-tempo Newtoniano
$\enw{M}{\tempo}{M}\,$, onde $\cor{M}$ � um universo e $\epo{M}$ � formado pelas regi�es
dos corpos de $\cor{M}$. Dado um corpo qualquer $\cor{B}\subseteq\cor{M}$ e sua regi�o
$\epo{B}_r\subseteq\epo{M}$, tal que uma configura��o arbitr�ria
$\map{\chi_r}{\cor{B}}{\epo{B}_r}\,$, � poss�vel definir o espa�o medida
$\ems{\epo{M}}{\fuc{m^*}}{M^*}$, onde
 a massa $\fua{\fuc{m}^*}{\epo{B}_r}:=\fua{\fuc{m}}{\cor{B}}$ e a classe $\css{M^*}$ � formada pelas regi�es
dos conjuntos que constituem $\css{M}$.
\end{prp}
\begin{prova} Para simplificar as representa��es subseq�entes, admitiremos a nota��o $\fua{\chi_r}{\cor{A}}$ para
a regi�o definida pela configura��o $\chi_r$. Temos que mostrar que $\css{M^*}$ � um
$\sigma$-anel. Para tal, consideremos uma classe
$\lch\fua{\chi_r}{\cor{B}_1},\fua{\chi_r}{\cor{B}_2},\cdots\rch\subseteq\css{M^*}$, onde
$\lch\cor{B}_1,\cor{B}_2,\cdots\rch\subseteq\css{M}$. Como a fun��o $\chi_t$ � uma
bije��o, pode-se afirmar que o conjunto
\begin{equation}
\bigcup_{i=1}^{\infty}\fua{\chi_r}{\cor{B}_i}=\fua{\chi_r}{\bigcup_{i=1}^{\infty}\cor{B}_i}\,.\nonumber
\end{equation}
J� que a classe $\css{M}$ � um $\sigma$-anel e $\css{M^*}$ cont�m as regi�es dos
conjuntos de $\css{M}$, ent�o
\begin{equation}
\bigcup_{i=1}^{\infty}\cor{B}_i\in\css{M}\implies\fua{\chi_r}{\bigcup_{i=1}^{\infty}\cor{B}_i}\in\css{M^*}\nonumber\,.
\end{equation}

Sabe-se que $\fua{\chi_r}{\emptyset}=\emptyset$. Logo, se $\emptyset\in\css{M}$ ent�o
$\emptyset\in\css{M^*}$. Al�m disso, dados os conjuntos $\fua{\chi_r}{\cor{B}_i}$ e
$\fua{\chi_r}{\cor{B}_j}$ quaisquer,
\begin{equation}
\fua{\chi_r}{\cor{B}_i}/\fua{\chi_r}{\cor{B}_j}=\fua{\chi_r}{\cor{B}_i/\cor{B}_j}\,.\nonumber
\end{equation}
Como $\cor{B}_i/\cor{B}_j\in\css{M}$ ent�o
$\fua{\chi_r}{\cor{B}_i/\cor{B}_j}\in\css{M^*}$.
\end{prova}

\section{Deforma��es do Corpo}\index{deforma��o}\label{sec:deformacao}
Seja um corpo $\cor{B}$ de um universo $\cor{M}$ e $\enw{M}{\tempo}{M}$ um espa�o-tempo
Newtoniano, onde $\epo{M}$ � a regi�o dos corpos de $\cor{M}$. A descri��o material $\negsymbol{\chi}_t$ de uma configura��o espacial qualquer de $\cor{B}$, nos termos de
(\ref{eq:configEspacialDescMaterial}), � denominada uma deforma��o do corpo $\cor{B}$.
Deformar um corpo, neste contexto, significa alterar-lhe seu estado morfol�gico, ou seja,
n�o apenas seu formato mas tamb�m sua posi��o\rodape{Ver se��o \ref{sec:EspacoTempo} .}.
Neste contexto, um corpo pass�vel de sofrer deforma��es � dito
\emph{deform�vel}\index{corpo!deform�vel}. No �mbito deste trabalho, todo e qualquer
corpo � deform�vel; desta forma, na denomina��o ``corpo'', de agora em diante, ficar�
subentendido o adjetivo ``deform�vel''.

\subsection{O Gradiente de Deforma��o}\index{gradiente!de deforma��o}
Uma necessidade fundamental do estudo aqui realizado � a determina��o do comportamento da varia��o da
deforma��o $\negsymbol{\chi}_t$ no seu dom�nio de atua��o. Em termos matem�ticos, esta varia��o pode ser medida pela derivada da deforma��o $\negsymbol{\chi}_t\,$, representada, nos termos da se��o
\ref{sec:GradienteDivergente}, pelo conceito de gradiente.

Em �ltima an�lise, o tensor de segunda ordem
$\fua{\gqu{}{\negsymbol{\chi}_t}}{\vto{x}_r}\in{\con{T}}_{\con{M}^2\mapsto\con{\real}}$
determina, em termos quantitativos e qualitativos, as caracter�sticas de deforma��o
sofridas pelo conjunto de pontos do corpo. Devido a sua import�ncia, o tensor
$\subolds{\tnr{F}}{x}{r}:=\fua{\gqu{}{\negsymbol{\chi}_t}}{\vto{x}_r}$ recebe o nome �bvio de gradiente de deforma��o em $\vto{x}_r\in\overline{B}_r$.

Levando-se em conta a defini��o de configura��o de um corpo, a deforma��o
$\negsymbol{\chi}_t$ � tamb�m um $\sug{1}$-difeomorfismo e sua derivada
$\fua{\dvg{\negsymbol{\chi}_t}}{\vto{x}_r}$ � portanto invers�vel. Com base no conceito de
gradiente, pode-se dizer que a fun��o representante do gradiente de deforma��o
\begin{equation}
{\subolds{\tnr{F}}{x}{r}}_{\odot 1} = \fua{\dvg{\negsymbol{\chi}_t}}{\vto{x}_r}\,.
\end{equation}
Neste contexto, utilizando a defini��o (\ref{eq:representanteInversaSimples}),
\begin{equation}
{\subolds{\tnr{F}}{x}{r}^{-1}}_{\odot1} = \lco\fua{\dvg{\negsymbol{\chi}_t}}{\vto{x}_r}\rco^{-1}\,,
\end{equation}
de onde se pode concluir, nos termos da se��o \ref{sec:HiperdeterminanteOperador}, que a
quantidade escalar $\det{{\subolds{\tnr{F}}{x}{r}}_{\odot 1}}$ � n�o nula.

Uma deforma��o � classificada como \emph{homog�nea}\index{deforma��o!homog�nea},
representada $\overline{\negsymbol{\chi}_t}$,  se seu gradiente for constante ao
longo do dom�nio. Neste caso, com base na proposi��o \ref{teo:gradienteConstante},
pode-se dizer, em termos gen�ricos, que
\begin{equation}\label{eq:deformacaoHomogenea}
\fua{\ftr{{F}}{1}}{\vto{h}_1-\vto{h}_2}=
\fua{\overline{\negsymbol{\chi}_t}}{\vto{h}_1} -
\fua{\overline{\negsymbol{\chi}_t}}{\vto{h}_2}\,,
\end{equation}
onde $\tnr{F}$ possui o mesmo valor ao longo de $\con{M}$, sendo denominado gradiente de deforma��o homog�nea. Os vetores $\vto{h}_1$ e $\vto{h}_2$ s�o elementos quaisquer de $\overline{B}_r$. Desta forma, deforma��es homog�neas possuem, invariavelmente, a regra
\begin{equation}\label{eq:regraDeformacaoHomogenea}
\fua{\overline{\negsymbol{\chi}_t}}{\vto{x}_r} =  \fua{\ftr{{F}}{1}}{\vto{x}_r} +
\vto{c}\,,
\end{equation}
onde $\vto{c}\in\con{M}$ representa um vetor fixo qualquer.

\subsubsection{Deforma��o Isoc�rica}\index{deforma��o!isoc�rica}
Uma deforma��o � dita isoc�rica quando ela preserva o volume do corpo por ela deformado.
Em termos matem�ticos, isto � definido conforme apresentado a seguir. Considerando as
condi��es anteriores e um referencial positivamente orientado, seja
$\{\vto{u},\vto{v},\vto{w}\}\subset\overline{B}_r$ um conjunto linearmente independente
qualquer. Com base nas representa��es geom�tricas apresentadas na se��o
\ref{sec:produtoVetorial}, uma deforma��o $\negsymbol{\chi}_t$ � isoc�rica se e somente
se
\begin{equation}
\fua{\negsymbol{\chi}_t}{\vto{u}}\cdot(\fua{\negsymbol{\chi}_t}{\vto{v}}\wedge\fua{\negsymbol{\chi}_t}{\vto{w}})=\vto{u}\cdot(\vto{v}\wedge\vto{w})\,.
\end{equation}
Considerando o caso de deforma��o homog�nea, a igualdade
(\ref{eq:regraDeformacaoHomogenea}) gera, para qualquer $\vto{c}\in\con{M}$,
\begin{equation}
\lpa\fua{\ftr{{F}}{1}}{\vto{u}}+\vto{c}\rpa\cdot
\lco\lpa\fua{\ftr{{F}}{1}}{\vto{v}}+\vto{c}\rpa\wedge\lpa\fua{\ftr{{F}}{1}}{\vto{w}}+\vto{c}\rpa\rco=\vto{u}\cdot(\vto{v}\wedge\vto{w})\,.
\end{equation}
Sendo assim, especificando $\vto{c}=\vto{0}$ nesta igualdade, pode-se dizer, com base na
proposi��o \ref{teo:produtoTriplo}, que numa deforma��o homog�nea isoc�rica
\begin{equation}
\det\tnr{{F}}=1\,.
\end{equation}


\subsection{Deforma��es Afins}\index{deforma��o!afim}
Considerando as condi��es definidas na se��o \ref{sec:deformacao}, seja a transforma��o afim
\begin{equation}
\map{\mathsf{H}}{\epo{M}}{\epo{M}}\,,
\end{equation}
onde a afinidade $\mathsf{H}$ n�o inclui
reflex�es\rodape{Ver se��o \ref{sec:afinidade}.}. Sejam os referenciais
sincronizados $\rfr{o_r}{\tilde{\con{U}_r}}$ e $\rfr{o}{\tilde{\con{U}}}$, promotores dos campos caracter�sticos
$\fac{F}_{\overline{\epo{B}}_r}^{\overline{B}_r}$ e
$\fac{F}_{\overline{\epo{B}}_t}^{\overline{B}_t}$, onde $\overline{\epo{B}}_r$ e
$\overline{\epo{B}}_t$ s�o as regi�es material e espacial de um corpo $\cor{B}$,
respectivamente. Uma configura��o espacial qualquer $\negsymbol{\chi}_t$ de $\cor{B}$ �
dita uma deforma��o afim ou uma \emph{afinidade vetorial}\index{afinidade!vetorial},
representada $\negsymbol{\vartheta}_t$, se
\begin{equation}
\negsymbol{\vartheta}_t=\fac{F}_{\overline{\epo{B}}_t}^{\overline{B}_t}\circ\mathsf{H}\circ
{\fac{F}_{\overline{\epo{B}}_r}^{\overline{B}_r}}^{-1}.
\end{equation}
A partir da�, pode-se realizar o seguinte desenvolvimento
\begin{eqnarray}\label{eq:defAfimRegra}
\fua{\negsymbol{\vartheta}_t}{\vto{x}} & = &
\fua{\fac{F}_{\overline{\epo{B}}_t}^{\overline{B}_t}\circ\mathsf{H}}{\vto{x}_r\oplus
o_r}\nonumber\\
& = &
\fua{\fac{F}_{\overline{\epo{B}}_t}^{\overline{B}_t}}{\fua{\vtf{l}}{\vto{x}_r}+\fua{\mathsf{H}}{\ele{o_r}}}\nonumber\\
& = &
\fua{\fac{F}_{\overline{\epo{B}}_t}^{\overline{B}_t}}{\fua{\vtf{l}}{\vto{x}_r}+\vto{z}\oplus
o}\nonumber\\
& = & \lpa\tnr{A}\odot_1\vto{x}_r\rpa+\vto{z}\,,
\end{eqnarray}
onde $\vto{z}\in\con{M}$ � o vetor que associa os pontos $\ele{o}$ e $\fua{\mathsf{H}}{\ele{o_r}}$. Com base na �ltima igualdade, a diferencia��o da deforma��o afim
$\negsymbol{\vartheta}_t$ permite dizer que o gradiente de deforma��o afim
\begin{equation}\label{eq:gradDeformacaoAfim}
\fua{\gqu{}{\negsymbol{\vartheta}_t}}{\vto{x}_r}=\tnr{A}\,.
\end{equation}
Tomando esta igualdade e o resultado de (\ref{eq:defAfimRegra}), pode-se dizer, com base
em (\ref{eq:regraDeformacaoHomogenea}), que toda deforma��o afim
$\negsymbol{\vartheta}_t$ � homog�nea. Desta forma,
\begin{equation}
\fua{\ftr{\tnr{A}}{1}}{\vto{h}_1-\vto{h}_2}= \fua{\negsymbol{\vartheta}_t}{\vto{h}_1} -
\fua{\negsymbol{\vartheta}_t}{\vto{h}_2}\,,\,\forall\, \vto{h}_1,\vto{h}_2\in\overline{B}_r\,.
\end{equation}

\subsubsection{Decomposi��o Polar do Gradiente de Deforma��o Afim}\index{Decomposi��o Polar!do gradiente de deforma��o afim}
Considerando as condi��es da se��o anterior, do desenvolvimento que resulta na igualdade
(\ref{eq:defAfimRegra}), pode-se dizer que o vetor
$\fua{\negsymbol{\vartheta}_t}{\vto{x}_r}=\fua{\vtf{l}}{\vto{x}_r}+\vto{z}\,$. Como as regras poss�veis para a afinidade $\mathsf{H}$ excluem as reflex�es, ent�o $\det \vtf{l} > 0$. De acordo com o teorema \ref{teo:decompPolar}, o operador linear
\begin{equation}\label{eq:decompOperadorAfim}
\vtf{l}=\vtf{p}_1\circ\vtf{o}=\vtf{o}\circ\vtf{p}_2\,,
\end{equation}
onde $\vtf{p}_1,\vtf{p}_2\in\cfl{M}{M}$ s�o operadores sim�tricos positivos-definidos
e o operador $\vtf{o}\in\cfl{M}{M}$ � ortogonal pr�prio. Com base nas defini��es (\ref{eq:funRepCompostaDireita}) e (\ref{eq:funRepCompostaEsquerda}), como os operadores lineares em (\ref{eq:decompOperadorAfim}) s�o fun��es representantes de tensores, tal igualdade pode ser reescrita na forma
\begin{equation}
\tnr{A}_{\odot_1}=\lpa\tnr{P}_1\tnr{O}\rpa_{\odot_1}=\lpa\tnr{O}\tnr{P}_2\rpa_{\odot_1}\,,
\end{equation}
onde, obviamente, $\tnr{P}_1,\tnr{P}_2\in{\con{T}}_{\con{M}^2\mapsto\con{\real}}$ s�o
tensores sim�tricos po\-si\-ti\-vo-se\-mi\-de\-fi\-ni\-dos e
$\tnr{O}\in{\con{T}}_{\con{M}^2\mapsto\con{\real}}$ � tensor ortogonal pr�prio.
Em termos gerais, conclui-se que o tensor
\begin{equation}\label{eq:decompGradAfim}
\tnr{A}=\tnr{P}_1\tnr{O}=\tnr{O}\tnr{P}_2\,.
\end{equation}
A partir destas igualdades e de (\ref{eq:gradDeformacaoAfim}), diz-se que houve a decomposi��o polar do gradiente de deforma��o afim $\tnr{A}$.

Neste contexto, o objetivo agora � interpretar, em termos geom�tricos, a decomposi��o
(\ref{eq:decompGradAfim}). Para tal, seja a deforma��o afim
\begin{equation}\label{eq:rotacaoDilacaoRotacao}
\negsymbol{\vartheta}_t=\fac{F}_{\overline{\epo{B}}_t}^{\overline{B}_t}\circ
\mathsf{D}_1\circ\mathsf{K}\circ\subold{\mathsf{T}}{v} \circ
{\fac{F}_{\overline{\epo{B}}_r}^{\overline{B}_r}}^{-1}=\fac{F}_{\overline{\epo{B}}_t}^{\overline{B}_t}\circ
\mathsf{K}\circ\mathsf{D}_2\circ\subold{\mathsf{T}}{v} \circ
{\fac{F}_{\overline{\epo{B}}_r}^{\overline{B}_r}}^{-1}\,,
\end{equation}
onde $\mathsf{T}_\vto{v}$ � uma transla��o associada a um vetor fixo $\vto{v}\in\con{M}$,
$\mathsf{D}_1$ e $\mathsf{D}_2$ s�o dila��es e $\mathsf{K}$ uma rota��o. A partir da�,
pode-se realizar o seguinte desenvolvimento:
\begin{eqnarray}\label{eq:decompDefAfim}
\fua{\negsymbol{\vartheta}_t}{\vto{x}_r} & = &
\fua{\fac{F}_{\overline{\epo{B}}_t}^{\overline{B}_t}\circ\mathsf{D}_1\circ\mathsf{K}}{\vto{v}+\vto{x}_r\oplus
o_r}\nonumber\\
& = &
\fua{\fac{F}_{\overline{\epo{B}}_t}^{\overline{B}_t}\circ\mathsf{D}_1}{\fua{\tnr{R}_{\odot_1}}{\vto{v}+\vto{x}_r}\oplus
\fua{\mathsf{K}}{o_r}}\nonumber\\
& = &
\fua{\fac{F}_{\overline{\epo{B}}_t}^{\overline{B}_t}}{\fua{{\tnr{S}_1\tnr{R}}_{\odot_1}}{\vto{v}+\vto{x}_r}\oplus
\fua{\mathsf{D}_1\circ\mathsf{K}}{o_r}}\nonumber\\
& = &
\fua{\fac{F}_{\overline{\epo{B}}_t}^{\overline{B}_t}}{\fua{{\tnr{S}_1\tnr{R}}_{\odot_1}}{\vto{v}+\vto{x}_r}+
\vto{z}\oplus\ele{o}}\nonumber\\
& = &
\fua{{\tnr{S}_1\tnr{R}}_{\odot_1}}{\vto{x}_r}+
\underbrace{\fua{{\tnr{S}_1\tnr{R}}_{\odot_1}}{\vto{v}}+\vto{z}}_{\overline{\vto{v}}}\,.\label{eq:dilacaoRotacao}
\end{eqnarray}
De maneira similar, desenvolvendo o termo mais � direita em
(\ref{eq:rotacaoDilacaoRotacao}), chega-se facilmente a
\begin{equation}\label{eq:rotacaoDilacao}
\fua{\negsymbol{\vartheta}_t}{\vto{x}_r} =
\fua{{\tnr{R}\tnr{S}_2}_{\odot_1}}{\vto{x}_r}+\overline{\vto{v}}\,.
\end{equation}
Diferenciando $\negsymbol{\vartheta}_t$ a partir das igualdades
(\ref{eq:dilacaoRotacao}) e (\ref{eq:rotacaoDilacao}), com base em
(\ref{eq:gradDeformacaoAfim}) e (\ref{eq:rotacaoDilacaoRotacao}), pode-se afirmar que
\begin{equation}\label{eq:decompGeoGradAfim}
\tnr{A}=\tnr{S}_1\tnr{R}=\tnr{R}\tnr{S}_2\,.
\end{equation}
� f�cil observar este resultado n�o � alterado pela mudan�a da posi��o de
$\mathsf{T}_\vto{v}$ na composi��o de afinidades em (\ref{eq:rotacaoDilacaoRotacao}).
Desta forma, os tensores em (\ref{eq:decompGeoGradAfim}) s�o �nicos. Comparando tais
tensores com os de (\ref{eq:decompGradAfim}), que tamb�m s�o �nicos, conclui-se que a
decomposi��o polar do gradiente de deforma��o afim $\tnr{A}$ resulta do efeito combinado
de uma rota��o com uma dila��o. Nos termos da se��o \ref{sec:funcaoTensorialTransposta},
dados os tensores $\tnr{V}:=\tnr{S}_1$ e $\tnr{U}:=\tnr{S}_2$, para fun��es
representantes com contra��o � direita,
\begin{equation}\label{eq:decompPolarFinal1}
\tnr{A}=\tnr{V}\odot_1\tnr{R}=\tnr{R}\odot_1\tnr{U}\,,
\end{equation}
onde o tensor $\tnr{V}$ � denominado \emph{tensor de estiramento p�s-rotacional}\index{tensor!de estiramento p�s-rotacional}\index{tensor!de estiramento � esquerda} e
$\tnr{U}$ � o \emph{tensor de estiramento pr�-rotacional}\index{tensor!de estiramento pr�-rotacional}\index{tensor!de estiramento � direita}\rodape{\aut{Noll}\cite{noll_1974_2} ``batizou'' os tensores $\tnr{V}$ e $\tnr{U}$ \emph{tensor de estiramento � esquerda} e \emph{tensor de estiramento � direita} respectivamente. Ao contr�rio da quase unanimidade dos textos relacionados � Mec�nica do Cont�nuo, tal denomina��o, onde fica impl�cito um tensor de rota��o, n�o ser� utilizada neste trabalho.}. Se a igualdade anterior � verdadeira, utilizando as defini��es (\ref{eq:composicaoMultipla}), (\ref{eq:representanteInversaSimples}) e (\ref{eq:transpostaTensor}), fica f�cil obter que
\begin{equation}\label{eq:veDois}
\tnr{A}\odot_1\tnr{A^T}=\tnr{V^2}
\end{equation}
e
\begin{equation}\label{eq:uDois}
\tnr{A^T}\odot_1\tnr{A}=\tnr{U^2}\,.
\end{equation}
Para fun��es com contra��o � esquerda, (\ref{eq:decompPolarFinal1}) � reescrita na forma
\begin{equation}\label{eq:decompPolarFinal2}
\tnr{A}=\tnr{R}\odot_1\tnr{V}=\tnr{U}\odot_1\tnr{R}\,.
\end{equation}

� importante salientar que, com base nas igualdades anteriores, como o dom�nio de atua��o da deforma��o afim $\negsymbol{\vartheta}_t$ � $\overline{\con{B}}_r$, ent�o $\tnr{V}_{\odot 1}$ � uma fun��o material enquanto $\tnr{U}_{\odot 1}$ � espacial.

\subsubsection{Exemplos}
As descri��es das deforma��es afins a seguir s�o realizadas por meio de um �nico
referencial $\rfr{o}{\hat{\con{U}}}$, onde o conjunto
$\hat{\con{U}}=\lch\vun{e}_1,\vun{e}_2,\vun{e}_3\rch$ � base ortonormal positivamente
orientada do espa�o $\eeu{M}{3}\,$. Para as representa��es de cada um dos tipos de
deforma��o mostradas a seguir, a regi�o
$\overline{{\epo{B}}}_r$, definidora da regi�o material, � um cubo com os  seguintes pontos: $\ele{x}_1:=o$,
$\ele{x}_2:=\vun{e}_1\oplus o$, $\ele{x}_3:=(\vun{e}_1+\vun{e}_3)\oplus o$,
$\ele{x}_4:=\vun{e}_3\oplus o$, $\ele{x}_5:=(\vun{e}_1+\vun{e}_2)\oplus o$,
$\ele{x}_6:=\vun{e}_2\oplus o$, $\ele{x}_7:=(\vun{e}_2+\vun{e}_3)\oplus o$ e
$\ele{x}_8:=(\vun{e}_1+\vun{e}_2+\vun{e}_3)\oplus o\,$.

\paragraph{Deforma��o R�gida.}\index{deforma��o!r�gida}
Uma deforma��o afim $\negsymbol{\vartheta}_t$ � chamada deforma��o r�gida se, nos termos
de (\ref{eq:decompDefAfim}), possuir a seguinte regra:
\begin{equation}
\fua{\negsymbol{\vartheta}_t}{\vto{x}_r} =
\fua{{\tnr{R}}_{\odot_1}}{\vto{x}_r}+\overline{\vto{v}}\,.
\end{equation}
O gradiente de deforma��o r�gida � portanto um tensor de rota��o. Com base nesta regra, o
exemplo proposto est� representado na figura \ref{fg:deformacaoRigida} tal que as coordenadas
\begin{equation}
\lco\fua{\negsymbol{\vartheta}_t\circ\fac{F}_{\overline{\epo{B}}_r}^{\overline{B}_r}}
{\ele{x}_i}\rco^{\hat{\con{U}}}=\lco
\begin{array}{ccc}
\cos\theta & -\sin\theta & 0 \\
\sin\theta & \cos\theta & 0 \\
0 & 0 & 1
\end{array}\rco\lco
\fua{\fac{F}_{\overline{\epo{B}}_r}^{\overline{B}_r}}
{\ele{x}_i}\rco^{\hat{\con{U}}} + \sum_{j=1}^3\lco\vun{e}_j\rco^{\hat{\con{U}}}\,,
\end{equation}
onde $i=1,\dots,8$. Nesta igualdade, a matriz est� associada ao tensor de rota��o
$\tnr{R}$ que promove um giro anti-hor�rio de �ngulo $\theta$ em torno do eixo
$\overline{ox_4}$ do cubo. O �ltimo termo da soma s�o as coordenadas do vetor
$\overline{\vto{v}}$ da transla��o envolvida.
\begin{figure}[!ht]
\centering
\begin{center}
\input{partes/parte2/figs/c_cinemat/rigida.pstex_t}
\end{center}
\titfigura{Deforma��o R�gida.}\label{fg:deformacaoRigida}
\end{figure}

\paragraph{Alongamento Puro.}
Uma deforma��o afim $\negsymbol{\vartheta}_t$ � dita um alongamento puro se sua regra for
\begin{equation}\label{eq:estiramentoPuro}
\fua{\negsymbol{\vartheta}_t}{\vto{x}_r}=\fua{{\tnr{U}}_{\odot_1}}{\vto{x}_r}\,,
\end{equation}
onde
\begin{equation}
\tnr{U}=\tnr{V}=\sum_{i=1}^3\alpha_{ii} {\vun{e}_{i}} \otimes{\vun{e}_{i}}\,,
\end{equation}
tal que $\vun{e}_{i}$ e $\alpha_{ii}$ s�o os autovetores e os autovalores de $\tnr{U}$,
respectivamente. Para a situa��o mostrada na figura \ref{fg:alongamentoPuro},
\begin{equation}
\lco\fua{\negsymbol{\vartheta}_t\circ\fac{F}_{\overline{\epo{B}}_r}^{\overline{B}_r}}
{\ele{x}_i}\rco^{\hat{\con{U}}}=\lco
\begin{array}{ccc}
\alpha_{11} & 0 & 0 \\
0 & \alpha_{22} & 0 \\
0 & 0 & \alpha_{33}
\end{array}\rco\lco
\fua{\fac{F}_{\overline{\epo{B}}_r}^{\overline{B}_r}} {\ele{x}_i}\rco^{\hat{\con{U}}}\,,
\end{equation}
\begin{figure}[!ht]
\centering
\begin{center}
\input{partes/parte2/figs/c_cinemat/estira.pstex_t}
\end{center}
\titfigura{Alongamento Puro}\label{fg:alongamentoPuro}
\end{figure}

Se o volume do cubo for preservado pelos valores $\alpha_{ii}$, o alongamento puro � isoc�rico.


\paragraph{Cisalhamento Simples.}
A deforma��o n�o isoc�rica
$\negsymbol{\vartheta}_t$ � um cisalhamento simples na dire��o de $\vun{e}_i$ ao longo
do vetor $\vun{e}_j$, $i\neq j$, se o gradiente de deforma��o
\begin{equation}
\tnr{A}=\tnr{I}+\gamma{\vun{e}_{i}} \otimes{\vun{e}_{j}}\,.
\end{equation}
Para fins de exemplo, adotando $i=2$ e $j=3$, tem-se
\begin{equation}
\lco\fua{\negsymbol{\vartheta}_t\circ\fac{F}_{\overline{\epo{B}}_r}^{\overline{B}_r}}
{\ele{x}_i}\rco^{\hat{\con{U}}}=\lco
\begin{array}{ccc}
1 & 0 & 0 \\
0 & 1 & \gamma \\
0 & 0 & 1
\end{array}\rco\lco
\fua{\fac{F}_{\overline{\epo{B}}_r}^{\overline{B}_r}} {\ele{x}_i}\rco^{\hat{\con{U}}}\,,
\end{equation}
cuja representa��o � mostrada na figura \ref{fg:cisalhamentoSimples}.
\begin{figure}[!ht]
\centering
\begin{center}
\input{partes/parte2/figs/c_cinemat/cisalha.pstex_t}
\end{center}
\titfigura{Cisalhamento Simples.}\label{fg:cisalhamentoSimples}
\end{figure}


\subsection{Medidas de Estiramento}\index{estiramento!medidas de}
Considerando um deforma��o afim qualquer, uma medida de estiramento\rodape{Neste
trabalho, ``medida de estiramento'' � a tradu��o de ``strain measure''.} deve informar,
de alguma maneira, o n�vel ou a quantidade de estiramento envolvida em tal deforma��o.
Al�m disso, fica claro que a evolu��o desta medida deve se equiparar a evolu��o do
estiramento e no caso de deforma��es r�gidas ela deve ser nula.

Existe ainda uma terceira restri��o para se definir uma medida de estiramento: para
n�veis de estiramento muito baixos, deve haver uma proporcionalidade direta entre o
estiramento propriamente dito e sua medida. Tal restri��o compatibiliza o enfoque geral
aqui abordado com o conceito cl�ssico de medida de estiramento, apresentado a seguir.

A partir das condi��es definidas na se��o \ref{sec:deformacao}, seja um tensor de estiramento qualquer $\tnr{S}\in{\con{T}}_{\con{M}^2\mapsto\con{\real}}$. Na reta definida por um de seus autovetores $\vun{e}_i$, dado $\real^+$ o conjunto dos reais positivos, \aut{Love}\cite{love_1944_2} apresenta a fun��o $\overline{e}$, denominada ``extens�o'', onde
$\map{\overline{e}}{\real^+}{\real}$ e
\begin{equation}\label{eq:extensao}
\fua{\overline{e}}{\alpha_{ii}}:=\alpha_{ii}-1
\end{equation}
� um escalar muito pequeno. A express�o anterior estabeleceu-se ao longo dos anos e se
tornou base para a defini��o cl�ssica da medida dos pequenos estiramentos. No contexto
desta se��o, uma caracter�stica importante desta fun��o � o fato de
$\fua{\dvg{\overline{e}_{ii}}}{\ele{x}}=1\,$.

Sejam os subconjuntos $\con{TS}\subset{\con{T}}_{\con{M}^2\mapsto\con{\real}}$, formado por tensores sim�tricos, e $\con{TS}^+\subset\con{TS}$, formado por tensores sim�tricos positivos-definidos. Em termos gen�ricos, compilando as informa��es anteriores, diz-se que a fun��o tensorial no mapeamento $\map{\epsilon}{\con{TS}^+}{\con{TS}}$ � uma medida de estiramento e seus valores s�o \emph{tensores medida de estiramento}\index{tensor!medida de estiramento} se
\begin{equation}\label{eq:medidaEstiramento}
\fua{\epsilon}{\tnr{S}}=\sum_{i=1}^3\fua{e}{\alpha_{ii}}\vun{e}_{i}\otimes\vun{e}_{i}\,,
\end{equation}
tal que $e$, denominada \emph{fun��o medida de estiramento}\index{fun��o!medida de estiramento}, define o mapeamento $\map{e}{\real^+}{\real}$ e obedece as seguintes restri��es:
\begin{itemize}
 \item [i.] $e$ deve ser mon�tona, ou seja, $x_1>x_2\implies\fua{e}{x_1}>\fua{e}{x_2}$, $\forall x_1,x_2\in\real^+\,$;
 \item [ii.]$\fua{e}{1}=0\,$;
 \item [iii.]$e$ deve ser G-suave de grau um ou seja $e\in\sug{1}$;
 \item [iv.] a fun��o $\fua{\dvg{e}}{1}=1\,$.
\end{itemize}

Pode-se definir ent�o o \emph{tensor medida de estiramento}\index{tensor!medida de estiramento} $\tnr{E}_{\tnr{S}}=\fua{\epsilon}{\tnr{S}}$. Desta forma, tem-se os tensores $\tnr{E}_{\tnr{U}}=\fua{\epsilon}{\tnr{U}}$ e $\tnr{E}_{\tnr{V}}=\fua{\epsilon}{\tnr{V}}$, a partir dos quais podem ser constru�das fun��es representantes do tipo material e espacial respectivamente. De (\ref{eq:decompPolarFinal1}), (\ref{eq:decompPolarFinal2}) e (\ref{eq:medidaEstiramento}), obt�m-se que
\begin{eqnarray}
\fua{\epsilon}{\tnr{V}} & = & \fua{\epsilon}{\tnr{R}\odot_1\tnr{U}\odot_1\tnr{R}^T}\nonumber\\
& = & \fua{\epsilon}{\sum_{i=1}^3\alpha_{ii}(\tnr{R}\odot_1\vun{e}_{i})\otimes(\vun{e}_{i}\odot_1\tnr{R}^T)}\nonumber\\
& = & \sum_{i=1}^3\fua{e}{\alpha_{ii}}(\tnr{R}\odot_1\vun{e}_{i})\otimes(\vun{e}_{i}\odot_1\tnr{R}^T)\nonumber\\
 & = & \tnr{R}\odot_1\fua{\epsilon}{\tnr{U}}\odot_1\tnr{R}^T\,.
\end{eqnarray}

\begin{figure}[!b]
\centering
\begin{center}
\input{partes/parte2/figs/c_cinemat/sethhill.pstex_t}
\end{center}
\titfigura{Fun��es Medidas de Estiramento de Seth-Hill.}\label{fg:SethHill}
\end{figure}
\subsubsection{Medidas de Seth-Hill}\index{Seth-Hill!medidas de}
Considerando as condi��es anteriores, uma medida de estiramento $\epsilon^{(k)}$ � dita uma medida de Seth-Hill se a regra de sua fun��o medida de estiramento for
\begin{equation}
\fua{e^{(k)}}{x}=\frac{1}{k}\lpa x^{k} - 1 \rpa\,,
\end{equation}
onde $k\neq0$. Notar que esta defini��o � uma generaliza��o de (\ref{eq:extensao}), onde  $k=1$. Quando $k$ for um valor inteiro, com base em (\ref{eq:composicaoMultipla}) e  (\ref{eq:medidaEstiramento}), pode-se realizar o seguinte desenvolvimento:
\begin{eqnarray}
\fua{\epsilon^{(k)}}{\tnr{S}}&=&\sum_{i=1}^3\frac{1}{k}\lpa \alpha_{ii}^{k} - 1 \rpa\vun{e}_{i}\otimes\vun{e}_{i}\nonumber\\
&=&\frac{1}{k}\lpa\sum_{i=1}^3\alpha_{ii}^{k} \vun{e}_{i}\otimes\vun{e}_{i} - \tnr{I}\rpa\nonumber\\
&=&\frac{1}{k}\lpa \tnr{S^k} - \tnr{I}\rpa\label{eq:tensorHill}\,.
\end{eqnarray}
A plotagem de alguns valores $\fua{e^{(k)}}{x}$ contra pares de valores $(x,k)$ � apresentada na figura \ref{fg:SethHill}.


\paragraph{Exemplos.}
Com base nas igualdades (\ref{eq:veDois}), (\ref{eq:uDois}) e (\ref{eq:tensorHill}),  a tabela \ref{tb:medidasHill} mostra os exemplos mais utilizados de tensores medida de Seth-Hill no �mbito das deforma��es afins.
\begin{table}[!htt]
\centering
\begin{tabular}{|c|c|c|c|}
\hline
& & &\\
\textbf{   Valor de $k$   } & \textbf{   Tensor  de } &\textbf{   Express�o de    } & \textbf{   Tensor   } \\
& \textbf{   Estiramento } & \textbf{   Seth-Hill   } & \textbf{   medida de...   }\\
& & & \\
\hline
& & & \\
-2 & $\tnr{U}$   & $\tnr{E}_{\tnr{U}}^{(-2)}=\frac{1}{2}\lpa\tnr{I} - \tnr{A}^{-1}\odot_1\tnr{A}^{-T}\rpa$   & Almansi\index{Almansi!tensor medida de}\\
& & & \\
\hline
& & & \\
-2 & $\tnr{V}$  & $\tnr{E}_{\tnr{V}}^{(-2)}=\frac{1}{2}\lpa\tnr{I} - \tnr{A}^{-T}\odot_1\tnr{A}^{-1}\rpa$   & Almansi-Hamel\index{Almansi-Hamel!tensor medida de} \\
& & & \\
\hline
& & & \\
-2 & $\tnr{V}^{-1}$  & $\tnr{E}_{\tnr{V}^{-1}}^{(-2)}=\frac{1}{2}\lpa\tnr{I} - \tnr{A}\odot_1\tnr{A}^{T}\rpa$   & Finger\index{Finger!tensor medida de} \\
& & & \\
\hline
& & &\\
$\to 0$ & - & $\tnr{E}^{(0)}=\sum_{i=1}^3\ln\alpha_{ii}\vun{e}_{i}\otimes\vun{e}_{i}$   & Henky\index{Henky!tensor medida de} \\
& & &\\
\hline
& & &\\
$1$ &  $\tnr{U}$ &$\tnr{E}_{\tnr{U}}^{(1)}=\tnr{U}-\tnr{I}$ & Biot\index{Biot!tensor medida de} \\
& & &\\
\hline
& & & \\
$2$ & $\tnr{U}$ & $\tnr{E}_{\tnr{U}}^{(2)}=\frac{1}{2}\lpa \tnr{A}^{T}\odot_1\tnr{A}-\tnr{I}\rpa$ & Green-St Venant\index{Green-St Venant!tensor medida de} \\
& & & \\
\hline
& & &\\
$2$ & $\tnr{U}^{-1}$ & $\tnr{E}_{\tnr{U}^{-1}}^{(2)}=\frac{1}{2}\lpa \tnr{A}^{-1}\odot_1\tnr{A}^{-T}-\tnr{I}\rpa$ & Piola\index{Piola!tensor medida de} \\
& & & \\
\hline
\end{tabular}
\titfigura{Exemplos de tensores medidas de Seth-Hill.}\label{tb:medidasHill}
\end{table}

\subsection{Fun��es-Deslocamento}\index{fun��o!-deslocamento}
No contexto das condi��es apresentadas na se��o \ref{sec:deformacao}, para uma deforma��o
qualquer $\negsymbol{\chi}_t\,$ do corpo $\cor{B}$, a fun��o no mapeamento
$\map{\negsymbol{\delta}_t}{\overline{B}_r}{\con{M}}\,$ com regra
\begin{equation}\label{eq:deslocamento}
\fua{\negsymbol{\delta}_t}{\vto{x}_r}=\fua{\negsymbol{\chi}_t}{\vto{x}_r}-\vto{x}_r
\end{equation}
� denominada uma fun��o-deslocamento do corpo $\cor{B}$ sob a deforma��o $\negsymbol{\chi}_t\,$. O valor desta fun��o no vetor $\vto{u}$ � dito o \emph{deslocamento}\index{deslocamento} da part�cula $\prt{u}\,$.

Para realizar a descri��o Euleriana da fun��o-deslocamento, seja $\vto{x}=\fua{\negsymbol{\chi}_t}{\vto{x}_r}$ em
(\ref{eq:deslocamento}). Desta forma, pode-se definir a fun��o-deslocamento espacial
\begin{equation}
\negsymbol{\Delta}_t=\negsymbol{\delta}_t\circ\negsymbol{\chi}_t^{-1}\,
\end{equation}
onde
\begin{equation}\label{eq:deslocamentoEuleriano}
\fua{\negsymbol{\Delta}_t}{\vto{x}}=\vto{x}-\fua{\negsymbol{\chi}_t^{-1}}{\vto{x}}\,.
\end{equation}
O valor desta fun��o indica o deslocamento da part�cula que ocupa o ponto $\vto{x}\oplus o\,$ num instante fixo $t$.

Muitas vezes, � conveniente investigar o comportamento ou a distribui��o do
deslocamento de um corpo deformado ao longo de seus pontos. Para tal, deriva-se, em
alguma dire��o, a fun��o-deslocamento no seu dom�nio de atua��o. A fun��o resultante
representa o tensor de segunda ordem
\begin{equation}\label{eq:gradienteDeslocamento}
\tnr{J}:=\fua{\gqu{}{\negsymbol{\delta}_t}}{\vto{x}_r}=\tnr{F}-\tnr{I}\,,
\end{equation}
denominado \emph{gradiente de fun��o-deslocamento}\index{gradiente!de fun��o-deslocamento}.
Com base nesta igualdade, tomando uma deforma��o homog�nea qualquer definida por
(\ref{eq:regraDeformacaoHomogenea}), pode-se realizar, a partir de (\ref{eq:deslocamento}), o seguinte desenvolvimento:
\begin{eqnarray}
\fua{\overline{\negsymbol{\delta}}_t}{\vto{x}_r}&=&
\fua{\ftr{{F}}{1}}{\vto{x}_r} -\vto{x}_r +
\vto{c}\nonumber\\
&=&
\lpa\overline{\tnr{F}}-\tnr{I}\rpa\odot_1\vto{x}_r + \vto{c}\nonumber\\
&=&\fua{\ftr{\overline{J}}{1}}{\vto{x}_r} +
\vto{c}\,.\label{eq:deslocamentoHomogeneo}
\end{eqnarray}
Da �ltima igualdade resulta que a regra da chamada \emph{fun��o-deslocamento homog�neo}\index{fun��o!-deslocamento homog�neo} $\overline{\negsymbol{\delta}}_t$ possui forma similar ao da sua deforma��o homog�nea associada.

A inten��o agora � estabelecer uma importante rela��o entre o gradiente de fun��o-deslocamento e
algumas medidas de Seth-Hill. Tomando o tensor de Green-St Venant, com base nas
igualdades (\ref{eq:funRepresentanteIdentidade}) e (\ref{eq:gradienteDeslocamento}) ,
tem-se o seguinte desenvolvimento:
\begin{eqnarray}
\tnr{E}_{\tnr{U}}^{(2)}&=&\frac{1}{2}\lco \lpa\tnr{J}+\tnr{I}\rpa^{T}\odot_1\lpa\tnr{J}+\tnr{I}\rpa-\tnr{I}\rco\nonumber\\
&=&\frac{1}{2}\lco \lpa\tnr{J}^T\odot_1\tnr{J}+\tnr{J}^T+\tnr{J}+\tnr{I}\rpa-\tnr{I}\rco\nonumber\\
&=&\frac{1}{2}\lpa \tnr{J}+\tnr{J}^T+\underbrace{\tnr{J}^T\odot_1\tnr{J}}\rpa\label{eq:greenPequeno}\,.
\end{eqnarray}
Procedendo de maneira similar com o tensor de Almansi, chega-se a
\begin{equation}\label{eq:almansiPequeno}
\tnr{E}_{\tnr{U}}^{(-2)}=-\frac{1}{2}\lpa \tnr{J}^{-1}+\tnr{J}^{-T}+\underbrace{\tnr{J}^{-1}\odot_1\tnr{J}^{-T}}\rpa\,.
\end{equation}

Na pr�tica, medidas de estiramento s�o vari�veis fundamentais para o estudo de grande parte dos problemas da Mec�nica do Cont�nuo. Alguns destes problemas, sob certas circunst�ncias, podem ser simplificados de maneira significativa admitindo que medidas de estiramento, notadamente n�o lineares, sejam linearizadas. A retirada do termo destacado em (\ref{eq:greenPequeno}) lineariza o lado direito desta express�o. Para o caso de (\ref{eq:almansiPequeno}), entretanto, tal lineariza��o � v�lida se a vari�vel de an�lise for $\tnr{J}^{-1}$. Desta forma, tem-se como resultado
\begin{equation}
\tnr{\tilde{E}}_{\tnr{U}}^{(2)}:=\frac{1}{2}\lpa \tnr{J}+\tnr{J}^T\rpa
\end{equation}
e
\begin{equation}
\tnr{\tilde{E}}_{\tnr{U}}^{(-2)}:=-\frac{1}{2}\lpa \tnr{J}^{-1}+\tnr{J}^{-T}\rpa\,,
\end{equation}
onde os tensores $\tnr{\tilde{E}}_{\tnr{U}}^{(2)}$  e  $\tnr{\tilde{E}}_{\tnr{U}}^{(-2)}$ s�o as vers�es linearizadas dos tensores medidas originais correspondentes. Quando for o caso, esta lineariza��o tamb�m pode ser feita em outras express�es de medidas de estiramento.

� importante dizer que a omiss�o dos termos n�o lineares s� � sustent�vel se eles n�o
forem significativos, ou seja, se as deforma��es envolvidas forem pequenas ou
\emph{infinitesimais}\index{deforma��o!infinitesimal}. Por conta disso, tais tensores
linearizados s�o chamados \emph{tensores medida de estiramento
infinitesimal}\index{tensor!medida de estiramento infinitesimal}.

\section{Movimentos do Corpo}
Seja o espa�o-tempo Newtoniano $\enw{M}{\tempo}{M}$ e um corpo qualquer $\cor{B}$ de um
universo $\cor{M}$, tal que $\epo{M}$ � formado pelas regi�es dos corpos de $\cor{M}$. Seja o conjunto $C_\chi$ formado por todas as configura��es poss�veis
deste corpo. Uma fun��o $\varphi_{\cor{B}}$, com regra
$\fua{\varphi_{\cor{B}}}{t}=\chi_t\,$, � dita um movimento de $\cor{B}$ se ela define o
mapeamento bijetor
\begin{equation}
\map{\varphi_{\cor{B}}}{\tempo}{B_\chi}\,,
\end{equation}
onde a imagem $B_\chi\subseteq C_\chi$.

\subsection{Movimentos Vetoriais}\index{movimento!vetorial}
Dadas as condi��es anteriores e as condi��es que permitem chegar � defini��o
(\ref{eq:configuracaoVetorialEsp}), tem-se a configura��o espacial
\begin{equation}
\fua{\phi_{\cor{B}}}{t}:=\fac{F}_{\overline{\epo{B}}_t}^{\overline{B}_t}\circ\lco\fua{\varphi_{\cor{B}}}{t}\rco\,.
\end{equation}
A fun��o $\phi_{\cor{B}}$ � dita um movimento vetorial de $\cor{B}$ e define o mapeamento bijetor
\begin{equation}
\map{\phi_{\cor{B}}}{\tempo}{B_{\fac{F}{}{}\circ\chi}}\,,
\end{equation}
onde a imagem $B_{\fac{F}{}{}\circ\chi}$ � o conjunto das configura��es espaciais de
$\cor{B}$ ao longo da dura��o $\tempo$. A descri��o Lagrangiana
$\negsymbol{\phi}_{\cor{B}}$ do movimento vetorial de $\cor{B}$ define a regra
\begin{equation}\label{eq:movimentoLagrangiano}
\fua{\negsymbol{\phi}_{\cor{B}}}{t}=\fua{\phi_{\cor{B}}}{t}\circ\chi_r^{-1}\circ{\fac{F}_{\overline{\epo{B}}_r}^{\overline{B}_r}}^{-1}\,,
\end{equation}
a partir da qual tem-se o mapeamento
\begin{equation}
\map{\negsymbol{\phi}_{\cor{B}}}{\tempo}{B_{\negsymbol{\chi}}}\,,
\end{equation}
onde $B_{\negsymbol{\chi}}$ � conjunto das descri��es Lagrangianas das configura��es espaciais de $\cor{B}$. De maneira similar, o movimento vetorial $\negsymbol{\phi}_{\cor{B}}^{-1}$ � resultado da descri��o Euleriana do movimento vetorial de $\cor{B}$ cuja regra �
\begin{equation}\label{eq:movimentoEuleriano}
\fua{\negsymbol{\phi}_{\cor{B}}^{-1}}{t}={\fac{F}_{\overline{\epo{B}}_r}^{\overline{B}_r}}\circ\chi_r\circ\lco\fua{\phi_{\cor{B}}}{t}\rco^{-1}\,.
\end{equation}
Este movimento define
\begin{equation}
\map{\negsymbol{\phi}_{\cor{B}}^{-1}}{\tempo}{B_{\negsymbol{\chi}^{-1}}}\,,
\end{equation}
onde $B_{\negsymbol{\chi}^{-1}}$ � o conjunto das descri��es Eulerianas da configura��o
material do corpo $\cor{B}$.


\subsection{Trajet�rias}\index{trajet�ria}
Considerando as condi��es anteriores, dado um corpo $\cor{B}$, seja o mapeamento $\map{\subolds{\negsymbol{\delta}}{u}{r}}{\tempo}{\con{M}}$, onde $\vto{u}_r$ � um vetor qualquer da regi�o material $\overline{B}_r$. A fun��o $\subolds{\negsymbol{\delta}}{u}{r}$ � dita a trajet�ria da part�cula $\prt{u}$ sob uma fun��o-deslocamento $\negsymbol{\delta}_t$ de $\cor{B}$, se sua regra for
\begin{equation}\label{eq:trajetoria}
\fua{\subolds{\negsymbol{\delta}}{u}{r}}{t}=\fua{\negsymbol{\delta}_t}{\vto{u}_r}\,.
\end{equation}
Em termos Eulerianos, esta regra fica
\begin{equation}\label{eq:trajetoriaEuleriana}
\fua{\subold{\negsymbol{\Delta}}{u}}{t}=\fua{\negsymbol{\Delta}_t}{\vto{u}}\,,
\end{equation}
tal que $\vto{u}$ � um vetor qualquer da regi�o espacial $\overline{B}$.
Tomando as defini��es (\ref{eq:deslocamento}), (\ref{eq:deslocamentoEuleriano}), (\ref{eq:movimentoLagrangiano}) e (\ref{eq:movimentoEuleriano}), pode-se reescrever as express�o anteriores nas formas
\begin{equation}\label{eq:trajetoriaDetalhada}
\fua{\subolds{\negsymbol{\delta}}{u}{r}}{t}=\fua{\lco\fua{\negsymbol{\phi}_{\cor{B}}}{t}\rco}{\vto{u}_r}-\vto{u}_r
\end{equation}
e
\begin{equation}\label{eq:trajetoriaEspacialDetalhada}
\fua{\subold{\negsymbol{\Delta}}{u}}{t}=\vto{u}-\fua{\lco\fua{\negsymbol{\phi}^{-1}_{\cor{B}}}{t}\rco}{\vto{u}}\,.
\end{equation}


\subsubsection{Taxas das Trajet�rias}\index{taxa!da trajet�ria}
O conceito de taxa de uma vari�vel diz respeito ao comportamento de sua varia��o ao longo
do tempo. Em outras palavras, a taxa de uma vari�vel � uma medida da rapidez de sua
evolu��o. Na pr�tica, tal medida � obtida pela derivada da vari�vel em estudo na dura��o
do tempo. Seja ent�o a trajet�ria esta vari�vel de estudo. Considerando as condi��es
da defini��o (\ref{eq:trajetoria}), a taxa
\begin{equation}
\subolds{\negsymbol{\upsilon}}{u}{r} :=
\fua{\dvg{\subolds{\negsymbol{\delta}}{u}{r}}}{t}\,,
\end{equation}
que promove $\map{\subolds{\negsymbol{\upsilon}}{u}{r}}{\tempo}{\con{M}}\,$, � chamada
\emph{fun��o-velocidade material}\index{fun��o!-velocidade material} da part�cula
$\prt{u}$, no instante $t$, da trajet�ria $\subolds{\negsymbol{\delta}}{u}{r}$. O vetor
$\fua{\subolds{\negsymbol{\upsilon}}{u}{r}}{t_1}=\fua{\lco\fua{\dvg{\subolds{\negsymbol{\delta}}{u}{r}}}{t_1}\rco}{t_1}$
� a \emph{velocidade material}\index{velocidade!material} de $\prt{u}$ no instante
$t_1\,$. Genericamente, uma taxa � representada acentuando a vari�vel com um ponto. Logo,
$\fua{\dot{\subolds{\negsymbol{\delta}}{u}{r}}}{t}:=\fua{\subolds{\negsymbol{\upsilon}}{u}{r}}{t}\,$.
Pode-se concluir ent�o que a fun��o-velocidade material descreve a rapidez da evolu��o do
deslocamento de uma part�cula.

A fim de proceder � deriva��o da trajet�ria, a igualdade (\ref{eq:trajetoriaDetalhada})
pode ser reinterpretada na seguinte forma:
\begin{equation}
\fua{\subolds{\negsymbol{\delta}}{u}{r}}{t}=\subolds{\lco\fua{\negsymbol{\phi}_{\cor{B}}}{t}\rco}{u}{r}-\vto{u}_r\,.
\end{equation}
Como $\vto{u}_r$ � uma constante nesta igualdade, � poss�vel dizer que a
fun��o-velocidade
\begin{equation}\label{eq:funcaoVelocidadeMaterial}
\subolds{\negsymbol{\upsilon}}{u}{r}=\subolds{\lco\dvg{\fua{\negsymbol{\phi}_{\cor{B}}}{t}}\rco}{u}{r}
=\subolds{\lco\fua{\dot{\negsymbol{\phi}_{\cor{B}}}}{t}\rco}{u}{r}
\end{equation}
para qualquer instante de tempo $t$. A descri��o Euleriana desta fun��o toma a forma
\begin{equation}\label{eq:funcaoVelocidadeEleriana}
\subold{\negsymbol{\upsilon}}{u}=\subold{\lco\fua{\fua{\dot{\negsymbol{\phi}_{\cor{B}}}}{t}\circ\negsymbol{\phi}^{-1}_{\cor{B}}}{t}\rco}{u}\,,
\end{equation}
onde $\subold{\negsymbol{\upsilon}}{u}$ � a fun��o-velocidade material no ponto $\vto{u}\oplus o$, onde $\vto{u}\in\con{R}_{\subolds{\negsymbol{\delta}}{u}{r}}$. O valor $\fua{\subold{\negsymbol{\upsilon}}{u}}{t_1}$ � a velocidade material em $\vto{u}\oplus o$ no instante $t_1$ ou a velocidade material da part�cula que ocupa o ponto $\vto{u}\oplus o$ neste mesmo instante. Cabe ressaltar que embora esta fun��o-velocidade seja relativa a um vetor da regi�o espacial, a perspectiva de an�lise continua sendo a part�cula na sua trajet�ria.

Para um dada part�cula em uma dada trajet�ria, tamb�m � fundamental a medi��o da rapidez
da evolu��o de sua velocidade. Neste contexto, a fun��o no mapeamento
$\map{\subolds{\negsymbol{\alpha}}{u}{r}}{\tempo}{\con{M}}$, com regra
\begin{equation}
\fua{\subolds{\negsymbol{\alpha}}{u}{r}}{t} =
\fua{\lco\fua{\dgg{2}{{\subolds{\negsymbol{\delta}}{u}{r}}}}{t}\rco}{t}\,,
\end{equation}
� denominada a \emph{fun��o-acelera��o material}\index{fun��o!-acelera��o material} da
part�cula $\prt{u}$, no instante $t$, na trajet�ria $\subolds{\negsymbol{\delta}}{u}{r}$.
O vetor $\fua{\subolds{\negsymbol{\alpha}}{u}{r}}{t_1}$ � a
\emph{acelera��o}\index{acelera��o} de $\prt{u}$ no instante $t_1$. Neste caso, define-se
$\ddot{\subolds{\negsymbol{\delta}}{u}{r}}=\subolds{\negsymbol{\alpha}}{u}{r}\,$. Por
procedimento similar ao da fun��o-velocidade material, obt�m-se que fun��o-acelera��o
material �
\begin{equation}
\subolds{\negsymbol{\alpha}}{u}{r}
=\subolds{\lco\fua{\ddot{\negsymbol{\phi}_{\cor{B}}}}{t}\rco}{u}{r}\,.
\end{equation}
A partir da�, a fun��o-acelera��o material no ponto $\vto{u}\oplus o$,
$\vto{u}\in\con{R}_{\subolds{\negsymbol{\delta}}{u}{r}}$,  toma a forma
\begin{equation}
\subold{\negsymbol{\alpha}}{u}=\subold{\lco\fua{\ddot{\negsymbol{\phi}_{\cor{B}}}}{t}\circ\fua{\negsymbol{\phi}^{-1}_{\cor{B}}}{t}\rco}{u}\,.
\end{equation}

Agora, tomando por base a descri��o (\ref{eq:trajetoriaEspacialDetalhada}), pode-se
definir a taxa da descri��o Euleriana da trajet�ria $\subolds{\negsymbol{\delta}}{u}{r}$
como sendo
\begin{equation}
\subold{\negsymbol{\Upsilon}}{u} = \fua{\dvg{\subold{\negsymbol{\Delta}}{u}}}{t}\,.
\end{equation}
Esta taxa � a \emph{fun��o-velocidade espacial}\index{fun��o!-velocidade espacial}, no
instante $t$, do ponto $\vto{u}\oplus o$. Diz-se que seu valor � a \emph{velocidade
espacial}\index{velocidade!espacial} de $\vto{u}\oplus o$ num dado instante. Aqui, a
perspectiva de an�lise deixa de ser a part�cula na sua trajet�ria e passa a ser um ponto
qualquer. Em outras palavras, n�o se observa o corpo, mas o espa�o.

Novamente, pode-se representar a fun��o $\subold{\negsymbol{\Upsilon}}{u}$ com
$\dot{\subold{\negsymbol{\Delta}}{u}}\,$. Como o vetor  $\vto{u}$ � constante em
(\ref{eq:trajetoriaEspacialDetalhada}), por procedimento similiar ao que resultou
(\ref{eq:funcaoVelocidadeMaterial}), chega-se � fun��o
\begin{equation}\label{eq:velocidadeEspacial}
\subold{\negsymbol{\Upsilon}}{u}=-\subold{\lco\fua{\lco\negsymbol{\phi}^{-1}_{\cor{B}}\rco^{\cdot}}{t}\rco}{u}\,.
\end{equation}
Para a descri��o Lagrangiana desta fun��o, tem-se a fun��o-velocidade espacial
\begin{equation}\label{eq:velocidadeEspacialLagrangiana}
\subolds{\negsymbol{\Upsilon}}{u}{r}=-\subolds{\lco\fua{\lco\negsymbol{\phi}^{-1}_{\cor{B}}\rco^{\cdot}}{t}
\circ\fua{\negsymbol{\phi}_{\cor{B}}}{t}\rco}{u}{r}
\end{equation}
da posi��o da part�cula $\prt{u}\,$ no instante $t$. O vetor
$\fua{\subolds{\negsymbol{\Upsilon}}{u}{r}}{t_1}$ � a velocidade espacial de $\prt{u}\,$
no instante $t_1$. Em outras palavras, neste mesmo instante, o vetor citado � a
velocidade espacial do ponto ocupado por $\prt{u}$.

Definindo-se \emph{fun��o-acelera��o espacial}\index{fun��o!-acelera��o espacial} do
ponto $\vto{u}\oplus o$ no instante $t$ com a regra
\begin{equation}
\fua{\subold{\negsymbol{\Lambda}}{u}}{t} =
\fua{\lco\fua{\dgg{2}{\subold{\negsymbol{\Delta}}{u}}}{t}\rco}{t}\,,
\end{equation}
obt�m-se, via procedimento j� apresentado, que a fun��o
\begin{equation}
\subold{\negsymbol{\Lambda}}{u}=-\subold{\lco\fua{\lco\negsymbol{\phi}^{-1}_{\cor{B}}\rco^{\cdot\cdot}}{t}\rco}{u}
\end{equation}
e a fun��o-acelera��o espacial da posi��o da part�cula $\prt{u}$ �
\begin{equation}
\subolds{\negsymbol{\Lambda}}{u}{r}=-\subolds{\lco\fua{\lco\negsymbol{\phi}^{-1}_{\cor{B}}\rco^{\cdot\cdot}}{t}\circ\fua{\negsymbol{\phi}_{\cor{B}}}{t}\rco}{u}{r}\,.
\end{equation}

\subsection{Taxas e Distribui��es}

Nesta se��o, o objetivo � estudar fun��es gen�ricas cujas regras dependem do tempo e das
regi�es do corpo em estudo. A elas s�o aplicados os conceitos de taxa e de varia��o nas
distribui��es material e espacial.

Para tal, s�o considerados os mapeamentos material
$\map{\negsymbol{\psi}}{\overline{B}_r\times\tempo}{\con{M}}$ e espacial
$\map{\negsymbol{\Psi}}{\overline{B}\times\tempo}{\con{M}}$. No contexto da se��o
anterior, os subscritos $E$ e $L$ nas regras
\begin{equation}\label{eq:descricaoEulerianapsi}
\fua{\negsymbol{\psi}_E}{\vto{x},t}=\fua{\negsymbol{\psi}}{\fua{\lco\fua{\negsymbol{\phi}^{-1}_{\cor{B}}}{t}\rco}{\vto{x}},t}
\end{equation}
e
\begin{equation}\label{eq:descricaoLagrangianaPsi}
\fua{\negsymbol{\Psi}_L}{\vto{x}_r,t}=\fua{\negsymbol{\Psi}}{\fua{\lco\fua{\negsymbol{\phi}_{\cor{B}}}{t}\rco}{\vto{x}_r},t}
\end{equation}
definem as descri��es Euleriana e Lagrangiana de suas fun��es respectivas.

\subsubsection{Distribui��es Material e Espacial}
Dadas as condi��es anteriores, para medir a distribui��o material de $\negsymbol{\psi}$, seja um instante de tempo $t_1\,$, onde a regra
\begin{equation}
\fua{\negsymbol{\psi}_{t_1}}{\vto{x}_r}=\fua{\negsymbol{\psi}}{\vto{x}_r,t_1}
\end{equation}
apresenta a fun��o material $\map{\negsymbol{\psi}_{t_1}}{\overline{B}_r}{\con{M}}$, descrevendo $\negsymbol{\psi}$ em um dado instante de tempo. Nos termos da se��o
\ref{sec:GradienteDivergente}, a varia��o da distribui��o material $\negsymbol{\psi}_{t_1}$ na part�cula $\prt{u}$ � medida pela fun��o $\fua{\dvg{{\negsymbol{\psi}}_{t_1}}}{\vto{u}_r}$, que representa o tensor de segunda ordem $\fua{\gqu{}{\negsymbol{\psi}_{t_1}}}{\vto{u}_r}$.

Utilizando a mesma metodologia, a fun��o vetorial em $\map{\negsymbol{\Psi}_{t_1}}{\overline{B}}{\con{M}}\,$, cuja regra �
\begin{equation}
\fua{\negsymbol{\Psi}_{t_1}}{\vto{x}}=\fua{\negsymbol{\Psi}}{\vto{x},t_1}\,,
\end{equation}
descreve a distribui��o espacial de $\negsymbol{\Psi}$ no instante $t_1$. A medida de sua varia��o no ponto $\vto{u}\oplus o$ � obtida por meio do tensor $\fua{\gqu{}{\negsymbol{\Psi}_{t_1}}}{\vto{u}}\,$.

\subsubsection{Evolu��es Temporais}
Considerando as condi��es anteriores, a fim de analisar a evolu��o temporal da fun��o $\negsymbol{\psi}$, seja um vetor $\vto{u}_r\in\overline{B}_r$ a partir do qual define-se a regra
\begin{equation}
\fua{\subolds{\negsymbol{\psi}}{u}{r}}{t}=\fua{\negsymbol{\psi}}{\vto{u}_r,t}\,,
\end{equation}
onde $\map{\subolds{\negsymbol{\psi}}{u}{r}}{\tempo}{\con{M}}$ e descreve a fun��o
$\negsymbol{\psi}$ relativa a uma part�cula $\prt{u}$. Aplicando o conceito de taxa
apresentado na se��o anterior, pode-se dizer que a fun��o
\begin{equation}\label{eq:evolucaoMaterialFuncao}
\dot{\subolds{\negsymbol{\psi}}{u}{r}} := \fua{\dvg{\subolds{\negsymbol{\psi}}{u}{r}}}{t}
\end{equation}
mede a varia��o da evolu��o temporal de $\subolds{\negsymbol{\psi}}{u}{r}$ relativa �
part�cula $\prt{u}$ no instante $t$. De maneira similar, a fun��o na regra
\begin{equation}
\fua{\subold{\negsymbol{\Psi}}{u}}{t}=\fua{\negsymbol{\Psi}}{\vto{u},t}\,
\end{equation}
est� relacionada ao ponto $\vto{u}\oplus o\,$. Neste contexto, diz-se que a taxa
\begin{equation}
\dot{\subold{\negsymbol{\Psi}}{u}} := \fua{\dvg{\subold{\negsymbol{\Psi}}{u}}}{t}
\end{equation}
� a varia��o da evolu��o temporal de $\subold{\negsymbol{\Psi}}{u}$ no ponto
$\vto{u}\oplus o\,$ no instante $t$.

\paragraph{Taxas Materiais e Espaciais.}
Nos termos da proposi��o \ref{teo:derivadaParcial}, seja um par ordenado qualquer
$(\vto{x}_r,t)\in\overline{B}_r\times\tempo$ e a defini��o
(\ref{eq:evolucaoMaterialFuncao}), a partir dos quais o desenvolvimento a seguir �
realizado.
\begin{eqnarray}
\fua{\lco\dvg{\fua{\negsymbol{\psi}}{\vto{x}_r,t}}\rco}{\vto{x}_r,t}&=&
\fua{\lco\fua{\dvp{1}{\negsymbol{\psi}}}{\vto{x}_r,t}\rco}{\vto{x}_r}+
\fua{\lco\fua{\dvp{2}{\negsymbol{\psi}}}{\vto{x}_r,t}\rco}{t}\nonumber\\
&=&\fua{\lco\fua{\dvg{\negsymbol{\psi}_t}}{\vto{x}_r}\rco}{\vto{x}_r}+
\fua{\lco\fua{\dvg{\subolds{\negsymbol{\psi}}{x}{r}}}{t}\rco}{t}\nonumber\\
&=&\fua{\gqu{}{\negsymbol{\psi}_{t}}}{\vto{x}_r}\odot_1\vto{x}_r+
\fua{\dot{\subolds{\negsymbol{\psi}}{x}{r}}}{t}\label{eq:desenvTaxaMaterial}\,.
\end{eqnarray}
Procedendo de maneira semelhante para o caso da taxa espacial de $\negsymbol{\Psi}$, dado um par qualquer $(\vto{h},h)\in\overline{B}\times\tempo$, chega-se �
\begin{equation}
 \fua{\lco\dvg{\fua{\negsymbol{\Psi}}{\vto{x},t}}\rco}{\vto{x},t}=\fua{\gqu{}{\negsymbol{\Psi}_{t}}}{\vto{x}}\odot_1\vto{x}+\fua{\dot{\subold{\negsymbol{\Psi}}{x}}}{t}\label{eq:desenvTaxaEspacial}\,.
\end{equation}
Utilizando a Regra da Cadeia (teorema \ref{teo:RegraCadeia}), pode-se realizar o seguinte desenvolvimento:
\begin{eqnarray}
\fua{\lco\dvg{\fua{\negsymbol{\psi}_E}{\vto{x},t}}\rco}{\vto{x},t}&=&\fua{\lco\fua{\dvp{1}{\negsymbol{\psi}}}{\fua{\lco\fua{\negsymbol{\phi}^{-1}_{\cor{B}}}{t}\rco}{\vto{x}}
,t}\rco}{\vto{x}}+\fua{\lco\fua{\dvp{2}{\negsymbol{\psi}_E}}{\vto{x},t}\rco}{t}\nonumber\\
&=&\fua{\lco\fua{\dvg{\negsymbol{\psi}_t\circ\lco\fua{\negsymbol{\phi}^{-1}_{\cor{B}}}{t}\rco}}{\vto{x}}\rco}{\vto{x}}+
\fua{\lco\fua{\dvg{\subold{\negsymbol{\psi}_E}{x}}}{t}\rco}{t}\nonumber\\
&=&\fua{\lco\fua{\lco\dvg{{\negsymbol{\psi}}_{t}}\rco_E}{\vto{x}}\rco}{
\fua{\lco\dvg{\fua{\lco\fua{\negsymbol{\phi}^{-1}_{\cor{B}}}{t}\rco}{\vto{x}}}\rco}{\vto{x}}
}+\fua{\dot{\subold{\negsymbol{\psi}_E}{x}}}{t}\nonumber\\
&=&\fua{\lco\gqu{}{{\negsymbol{\psi}}_{t}}\rco_E}{\vto{x}}\odot_1
\subold{\lco\fua{\lco\negsymbol{\phi}^{-1}_{\cor{B}}\rco^{\cdot}}{t}\rco}{x}+
\fua{\dot{\subold{\negsymbol{\psi}_E}{x}}}{t}\nonumber\\
&=&-\fua{\lco\gqu{}{{\negsymbol{\psi}}_{t}}\rco_E}{\vto{x}}\odot_1\fua{\subold{\negsymbol{\Upsilon}}{h}}{t}+
\fua{\dot{\subold{\negsymbol{\psi}_E}{x}}}{t}\label{eq:desenvTaxaMaterialEuleriana}\,.
\end{eqnarray}
Procedendo de maneira similar, obt�m-se o vetor
\begin{equation}
\fua{\lco\dvg{\fua{\negsymbol{\Psi}_L}{\vto{x}_r,t}}\rco}{\vto{x}_r,t}=\fua{\lco\gqu{}{{\negsymbol{\Psi}}_{t}}\rco_L}{\vto{x}_r}\odot_1\fua{\subolds{\negsymbol{\upsilon}}{h}{r}}{t}+h\fua{\dot{\subolds{\negsymbol{\Psi}_L}{x}{r}}}{t}\label{eq:desenvTaxaEspacialLagrangiana}\,.
\end{equation}

 Diz-se que as fun��es $\dvgm{\negsymbol{\Psi}}$ e $\dvge{\negsymbol{\psi}}$ s�o as taxas material e espacial de $\negsymbol{\Psi}$ e $\negsymbol{\psi}$ respectivamente, caso
$\dvgm{\negsymbol{\Psi}}=\lco\dvg{\negsymbol{\Psi}_L}\rco_E$ e
$\dvge{\negsymbol{\psi}}=\lco\dvg{\negsymbol{\psi}_E}\rco_L$. Nestas condi��es,
utilizando os resultados (\ref{eq:desenvTaxaMaterialEuleriana}) e
(\ref{eq:desenvTaxaEspacialLagrangiana}), tem-se que
\begin{equation}
\fua{\lco\dvge{\fua{\negsymbol{\psi}}{\vto{x}_r,t}}\rco}{\vto{x}_r,t}=h\fua{\dot{\subolds{\negsymbol{\psi}}{x}{r}}}{t}-\fua{\gqu{}{{\negsymbol{\psi}}_{t}}}{\vto{x}_r}\odot_1\fua{\subolds{\negsymbol{\Upsilon}}{h}{r}}{t}
\end{equation}
e
\begin{equation}
\fua{\lco\dvgm{\fua{\negsymbol{\Psi}}{\vto{x},t}}\rco}{\vto{x},h}=h\fua{\dot{\subold{\negsymbol{\Psi}}{x}}}{t}+\fua{\gqu{}{{\negsymbol{\Psi}}_{t}}}{\vto{x}}\odot_1\fua{\subold{\negsymbol{\upsilon}}{h}}{t}\,.
\end{equation}
Com base nas igualdades (\ref{eq:desenvTaxaMaterial}) e (\ref{eq:desenvTaxaEspacial}),
pode-se  concluir que a taxa material $\dvgm{\negsymbol{\psi}}=\dvg{\negsymbol{\psi}}$ e
a taxa espacial $\dvge{\negsymbol{\Psi}}=\dvg{\negsymbol{\Psi}}$.

%    \begin{thebibliography}{99.}

\bibitem{backus_1997_2}\aut{Backus, G.} \nob{ Continuum Mechanics.} Available at
<\url{http:// samizdat.mines.edu/backus/}>, Accessed on feb 2004.

\bibitem{benvenuto_1991}\aut{Benvenuto, E.} \nob{An Introduction to the History of Structural Mechanics, Part I: Statics and Resistance of Solids}. 1. ed. New York: Springer-Verlag, 1991.

\bibitem{jakob_1694}\aut{Bernoulli, Jakob.} \textit{Curvatura Laminae Elasticae. Ejus Identitas cum Curvatura Lintei a pondere inclusi fluidi expansi, \&c.} in \nob{Acta Eruditorum}, 1694, pp. 262-276.

\bibitem{jakob_1695}\aut{Bernoulli, Jakob.} \textit{Explicationes, annotationes et additiones ad ea, quae in actis sup. anni de curva elastica, isochrona paracentrica, et velaria, \&c.} in \nob{Acta Eruditorum}, 1695, pp. 537-553.

\bibitem{belhoste_1991_1}\aut{Belhoste, B.} \nob{Augustin-Louis Cauchy: A Biography.} 1. ed. Berlim: Springer-Verlag, 1991.

\bibitem{bradley_2009}\aut{Bradley, R.; Sandifer, C.} \nob{Cauchy's Cours d'analyse: An Annotated Translation.} 1. ed. New York: Springer-Verlag, 2009.

\bibitem{brandon_2003_2}\aut{Brandon, R.} \nob{ Large Deformation Kinematics.} Available at
<\url{http://www.me.unm.edu/~rmbrann/gobag.html}>, Accessed on jan 2007.

\bibitem{calinger_2016_1}\aut{Calinger, R.} \nob{Leonhard Euler: Mathematical Genius in the Enlightenment.}
1. ed. Princeton: Princeton University Press, 2016.

\bibitem{cauchy_1823}\aut{Cauchy, A.} \textit{Recherches sur L'�quilibre et Lemouvement Int\'erieur des Corps Solides ou Fluides, \'Elastiques ou Non \'Elastiques} in \nob{Bulletin Des Sciences de La Soci�t� Philomatique
de Paris}, January, 1823, pp. 9-13.

\bibitem{cauchy_1827}\aut{Cauchy, A.} \textit{Sur La Pression ou Tension Dans Les Corps Solides} in \nob{Exercices de Math\'ematiques}, March, 1827, pp. 42-57.

\bibitem{cauchy_1889}\aut{Cauchy, A.} \textit{Sur Quelques Propri\'et\'es des Poly\`edres} in \nob{\OE uvres Compl\'etes d'Augustin Cauchy}, S\'erie II, Tome VII, 1889, pp. 55-59.

\bibitem{cauchySolid_1889}\aut{Cauchy, A.} \textit{De La Pression ou Tension Dans Un Corps Solide} in \nob{\OE uvres Compl\'etes d'Augustin Cauchy}, S\'erie II, Tome VII, 1889, pp. 60-81.

\bibitem{chadwick_1999_1}\aut{Chadwick, P.} \nob{ Continuum Mechanics: Concise Theory and Problems.}
1. ed. Mineola: Dover Publications, 1999.

\bibitem{ciarlet_1988_2_2}\aut{Ciarlet, P. G.} \nob{ Mathematical Elasticity - Volume 1:
Three Dimensional Elasticity.} 1. ed. Amsterdam: Elsevier Science Publishers BV, 1988.


\bibitem{corben_1994_2}\aut{Corben, H. C.; Stehle, P.} \nob{ Classical Mechanics.} 2. ed. New York: Dover Publications, 1994.

\bibitem{dugas_1988_1}\aut{Dugas, R.} \nob{A History of Mechanics.} 1. ed. New York: Dover Publications, 1988.

\bibitem{euler_1749_1}\aut{Euler, L.} \nob{ Scientia Navalis, Volume 1.} 1. ed. St. Petersburg: Academy of Sciences Press, 1749.

\bibitem{fellmann_2007_1}\aut{Fellmann, E.} \nob{Leonhard Euler.} 1. ed. Basel: Birkh\"auser Verlag, 2007.

\bibitem{galileo_1954_2}\aut{Galileo Galilei.} \nob{Dialogs Concerning Two New Sciences.} 1. ed. New York: Dover Publications, 1954.


\bibitem{gurtin_1981_2}\aut{Gurtin, M. E.} \nob{ An Introduction to Continuum Mechanics.} 1. ed. San Diego: Academic Press, 1981.

\bibitem{hanyga_1985_2}\aut{Hanyga, A. } \nob{ Mathematical Theory of Non-Linear Elasticity.} Translated by
R. W. Ogden. 1. ed. Chichester/New York/Warsaw: Elis Horwood Limited/John Wiley \&
Sons/PWN-Polish Scientific Publishers, 1985.

\bibitem{haupt_2002_2}\aut{Haupt, P.} \nob{ Continuum Mechanics and Theory of Materials.} 2. ed. Berlim: Springer-Verlag, 2002.

\bibitem{heilbron_2010_1}\aut{Heilbron, J.L.} \nob{Galileo.} 1. ed. New York: Oxford University Press, 2010.

\bibitem{higdon_1981_3}\aut{Higdon, A. et al.} \nob{Mec�nica dos Materiais.} 3. ed. Rio de Janeiro: Guanabara Dois, 1981.

\bibitem{hill_1978_2}\aut{Hill, R.} Aspects of Invariance in Solid Mechanics. \nob{ Advances in Applied Mechanics.} New York: Academic Press, vol. 18, pp. 1-75, 1978.

\bibitem{hooke_1678_1}\aut{Hooke, R.} \nob{Lectures \emph{de Potentia Restitutiva}, Or of Spring: Explaining the Power of Springing Bodies.} 1. ed. London: John Martyn, 1678.

\bibitem{hooke_1705_1}\aut{Hooke, R.} \nob{The Posthumous Works of Robert Hooke.} 1. ed. London: Richard Waller, 1705.


\bibitem{love_1944_2}\aut{Love, A. E. H.} \nob{ A Treatise on the Mathematical Theory of Elasticity.} 4. ed. New York: Dover Publications, 1944.

\bibitem{lubarda_2002_2}\aut{Lubarda, V. A.} \nob{ Elastoplasticity Theory.} 1. ed. Boca Raton: CRC Press,   2002.

\bibitem{lund_2000_1}\aut{Lund, J.; Byrne, J.} \textit{Leonardo Da Vinci's Tensile Strength Tests: Implications For The Discovery Of Engineering Mechanics} in \nob{Civil. Eng. and Env. Syst.} Vol. 00, 2000, pp. 1-8.


\bibitem{malvern_1969_2}\aut{Malvern, L. E.} \nob{ Introduction to the Mechanics of a Continuous Medium.}
1. ed. Englewood Cliffs: Prentice-Hall, 1969.

\bibitem{mariotte_1740_1}\aut{Mariotte, E.} \nob{Trait� du mouvement des eaux et des autres corps fluides.} 2. ed. Paris: Jean N�aulme, 1700.

\bibitem{marsden_1994_2}\aut{Marsden, J. E.; Hughes, T. J. R.} \nob{ Mathematical
Foundations of Elasticity.} 1. ed. Mineola: Dover Publications, 1994.


\bibitem{maugin_1993_2}\aut{Maugin, G.} \nob{Material Inhomogeneities in Elasticity.} 1. ed. New York: Chapman \& Hall, 1993.

\bibitem{maugin_2013}\aut{Maugin, G.} \nob{Continuum Mechanics Through the Twentieth Century: A Concise Historical Perspective.} 1. ed. Dordrecht: Springer, 2013.

\bibitem{maugin_2014}\aut{Maugin, G.} \nob{Continuum Mechanics Through the Eighteenth and Nineteenth Centuries.} 1. ed. Switzerland: Springer, 2014.

\bibitem{maugin_2016}\aut{Maugin, G.} \nob{Continuum Mechanics Through the Ages - From the Renaissance to the Twentieth Century.} 1. ed. Switzerland: Springer, 2016.

\bibitem{maxwell_1991_2}\aut{Maxwell, J. C.} \nob{ Matter and Motion.} 1. ed. New York: Dover Publications, 1991.

\bibitem{mikhailov_2005}\aut{Mikhailov, G.} \textit{Daniel Bernoulli, Hydrodynamica (1738)} in \nob{Landmark Writings in Western Mathematics 1640-1940.} 1. ed. Amsterdam: Elsevier Science, 2005, pp. 131-142.

\bibitem{newton_1999_1}\aut{Newton, I.} \nob{The Principia: Mathematical Principles of Natural Philosophy.}, translated by I. Bernard Cohen and Anne Whitman, 1. ed. Berkeley: University of California Press, 1999.

\bibitem{noll_1974_2}\aut{Noll, W.} \nob{ The Foundations of Mechanics and Thermodynamics: Selected Papers.} 1. ed. Berlim: Springer-Verlag, 1974.


\bibitem{ogden_1997_2}\aut{Ogden, R. W.} \nob{ Non-Linear Elastic Deformations.} 1. ed. Mineola: Dover Publications, 1997.

\bibitem{oldfather_1933_1}\aut{Oldfather, W. et al.} \textit{Leonhard Euler's Elastic Curves} in \nob{Isis}, Vol. 20, No. 1,  Chicago: The University of Chicago Press, 1933, pp. 72-160.

\bibitem{podio_2000_2}\aut{Podio-Guidugli, P.} A Primer on Elasticity. \nob{ Journal of Elasticity.} Berlim: Springer-Verlag, vol. 40, pp. 1-104, 2000.

\bibitem{popov_1990_1}\aut{Popov, E.} \nob{ Engineering Mechanics of Solids.} 1. ed. Englewood Cliffs: Prentice Hall, 1990.

\bibitem{prager_1997_2}\aut{Prager, W.} \nob{ Introduction to Mechanics of Continua.} 1. ed. Mineola: Dover Publications, 2004.

\bibitem{sedov_1997_2}\aut{Sedov, L. I.} \nob{ Mechanics of Continuous Media - Volume 1.} Translated by J. P. Nowacki. 1. ed. Farrer Road: World Scientific Publishing Co., 1997.

\bibitem{silhavy_1997_2}\aut{\v{S}ilhav�, M.} \nob{ The Mechanics and Thermomechanics of Continuous Media.} 1. ed. Berlim: Springer-Verlag, 1997.

\bibitem{sokolnikoff_1983_2}\aut{Sokolnikoff, I. S.} \nob{ Mathematical Theory of Elasticity.}
2. ed. Malabar: Robert Krieger Publishing Company, 1983.

\bibitem{talpaert_2002_2}\aut{Talpaert, Y. R.} \nob{ Tensor Analysis and Continuum Mechanics.} 1. ed. Dordrecht: Kluwer Academic Publishers, 2002.

\bibitem{teeman_2005_2}\aut{Temam, R. M.; Miranville, A. M.} \nob{ Mathematical Modeling in Continuum Mechanics.} 2. ed. Cambridge: Cambridge University Press, 2005.

\bibitem{timoshenko_1983_1}\aut{Timoshenko, S.} \nob{History of Strength of Materials.} 1 ed. Mineola: Dover Publications, 1983.

\bibitem{tolstoi_2014_1}\aut{Tolstoy, L.} \nob{War \& Peace.} 1 ed. London: The Folio Society Ltd, 2014.

\bibitem{truesdell_1954_1}\aut{Truesdell, C. A.} \textit{Rational fluid mechanics, 1687-1765: Editor's introduction to vol. II 12 of Euler's works} in \nob{Euleri Opera Omnia, Series II, vol. 12.} 1. ed. Zurich: F�ssli, 1954, pp. 1-125.

\bibitem{truesdell_1960}\aut{Truesdell, C. A.} \nob{The Rational Mechanics of Flexible or Elastic Bodies: 1638-1788}. 1. ed. Basel: Birkh�user, 1960.

\bibitem{truesdell_1968}\aut{Truesdell, C. A.} \nob{Essays in the History of Mechanics}. 1. ed. New York: Springer-Verlag, 1968.

\bibitem{truesdell_1977_2}\aut{Truesdell, C. A.} \nob{ A First Course in Rational Continuum Mechanics - Volume 1}. 1. ed. New York: Academic Press, 1977.


\bibitem{truesdellNoll_1997_2}\aut{Truesdell, C. A.; Noll, W.} \nob{ The Non-Linear Field Theories
of Mechanics.} 3. ed. New York: Springer-Verlag, 2003.

\bibitem{wallace_1998_1}\aut{Wallace, W.} \textit{Galileo's Pisan studies in science and philosophy} in \nob{ The Cambridge Companion to Galileo.} 1. ed. Cambridge: Cambridge University Press, 1998, pp. 27-52.


\bibitem{wangTruesdell_1973_2}\aut{Wang, C.-C; Truesdell, C. A.} \nob{ Introduction to Rational Elasticity.} 1. ed. Leyden: Noordhoff International Publishing, 1973.


\bibitem{weyl_1952_2}\aut{Weyl, H.} \nob{ Space Time Matter.} 4. ed. Mineola: Dover Publications, 1952.

\bibitem{zammattio_1980}\aut{Zammattio, C. et al.} \nob{Leonardo The Scientist.} 1. ed. New York: McGraw-Hill, 1980.



\end{thebibliography}



\backmatter%%%%%%%%%%%%%%%%%%%%%%%%%%%%%%%%%%%%%%%%%%%%%%%%%%%%%%%


\printglossary[title=Nota��o e S�mbolos,style=super]


\printindex

\end{document}


