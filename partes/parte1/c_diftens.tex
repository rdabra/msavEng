\chapter{Diferenciação de Funções Tensoriais}

\section{Diferenciabilidade de Funções Tensoriais}

\subsection{Função Tensorial Limitada}\index{função!tensorial!limitada}
Sejam os espaços tensoriais normados $\ete{\crt{V}{p}}{\con{F}}$ e
$\ete{\crt{W}{q}}{\con{F}}$ com as respectivas normas
$\|\bullet\|_{\ete{\crt{V}{p}}{\con{F}}}$ e
$\|\bullet\|_{\ete{\crt{W}{q}}{\con{F}}}$. A função no mapeamento
\begin{equation}\label{eq:mapFuncaoLimitada}
\map{\psi}{{T}_{\con{\crt{V}{p}}\mapsto\con{\con{F}}}}{\cft{\crt{W}{q}}{\con{F}}}
\end{equation}
é dita limitada em  ${T}_{\con{\crt{V}{p}}\mapsto\con{\con{F}}}$
se existir uma constante $\Lambda\in\con{F}$ tal que
\begin{equation}
\|\fua{\psi}{\tnr{X}}\|_{\ete{\crt{W}{q}}{\con{F}}}\leqslant\Lambda\|\tnr{X}\|_{\ete{\crt{V}{p}}{\con{F}}}\,,\,
\forall\,\tnr{X}\in\cft{\crt{V}{p}}{\con{F}}\,.
\end{equation}
Das possíveis constantes $\Lambda$ que obedecem a desigualdade
anterior, seja $\lambda$ a menor delas. Nesta situação, para a
função $\psi$, é possível definir que
\begin{equation}
\| \psi \| = \lambda\,.
\end{equation}
O conjunto de funções limitadas é portanto normado. Além disso, se
os espaços tensoriais $\ete{\crt{V}{p}}{\con{F}}$ e
$\ete{\crt{W}{q}}{\con{F}}$ forem de Banach, toda função limitada
linear é contínua.
\newline

\noindent\begin{prova} Vamos demonstrar agora se a última
afirmação é verídica. Para tal, suponhamos uma seqüência de Cauchy
qualquer $\lim_{i\to\infty}{\fua{\varrho}{\tnr{X}_i,\tnr{T}_0}}=0$
em ${T}_{\con{\crt{V}{p}}\mapsto\con{\con{F}}}$. Isto indica que
$\tnr{X}_i\to\tnr{T}_0$, quando $i\to\infty$. Então, pode-se dizer
que $\lim_{i\to\infty}{\| \tnr{X}_i-\tnr{T}_0 \|}=0$. Com base
nisso e considerando a função em ($\ref{eq:mapFuncaoLimitada}$)
limitada linear, a desigualdade
\begin{equation}
{\|\fua{\psi}{\tnr{X}_i}-\fua{\psi}{\tnr{T}_0}\|}=
{\|\fua{\psi}{\tnr{X}_i-\tnr{T}_0}\|}\leqslant\|\psi\|\|
\tnr{X}_i-\tnr{T}_0\|\,\nonumber
\end{equation}
quando $i\to\infty$, permite concluir que
$\fua{\psi}{\tnr{X}_i}\to\fua{\psi}{\tnr{T}_0}$\,.
\end{prova}



\subsection{Funções Tensoriais
Tangentes}\index{funções!tensoriais!tangentes} Dados os espaços
tensoriais de Banach $\ete{\crt{V}{p}}{\con{F}}$ e
$\ete{\crt{W}{q}}{\con{F}}$ com as respectivas normas
$\|\bullet\|_{\ete{\crt{V}{p}}{\con{F}}}$ e
$\|\bullet\|_{\ete{\crt{W}{q}}{\con{F}}}$, seja
$\con{\tilde{T}}_{\con{\crt{V}{p}}\mapsto\con{\con{F}}}$ um
subconjunto aberto de $\cft{\crt{V}{p}}{\con{F}}$ e o mapeamento
\begin{equation}
\map{\psi}{{\tilde{T}}_{\con{\crt{V}{p}}\mapsto\con{\con{F}}}}{\cft{\crt{W}{q}}{\con{F}}}\,.
\end{equation}
Considerando um tensor
$\tnr{T}_0\in{\tilde{T}}_{\con{\crt{V}{p}}\mapsto\con{\con{F}}}$,
toda e qualquer função $\kappa$, tal que
\begin{equation}
\map{\kappa}{{\tilde{T}}_{\con{\crt{V}{p}}\mapsto\con{\con{F}}}}{\cft{\crt{W}{q}}{\con{F}}}\,,
\end{equation}
é dita tangente à $\psi$ em $\tnr{T}_0$ e vice-versa se, dados o
escalar $\alpha\in\con{F}$ e um tensor qualquer não nulo
$\tnr{H}\in{\tilde{T}}_{\con{\crt{V}{p}}\mapsto\con{\con{F}}}$ ,
\begin{equation}\label{eq:funcaoTangenteGateaux}
\lim_{\alpha\to 0}\frac{\| \fua{\psi}{\alpha\tnr{H}+\tnr{T}_0} -
\fua{\kappa}{\alpha\tnr{H}+\tnr{T}_0}
 \|_{\ete{\crt{W}{q}}{\con{F}}}}
{\|\alpha\tnr{H}\|_{\ete{\crt{V}{p}}{\con{F}}}}=0\,.
\end{equation}
Em outras palavras, o valor no numerador aproxima-se de zero de
forma mais rápida do que no denominador quando o tensor
$\alpha\tnr{H}\to\negmath{0}$, segundo a ``direção'' definida por
$\tnr{H}$. Nestas condições, fica evidente que,
no limite, $\fua{\psi}{\tnr{T}_0}=\fua{\kappa}{\tnr{T}_0}$. Além disso, se
duas funções são tangentes a $\psi$ em $\tnr{T}_0$, então elas são
tangentes entre si.
\newline

\noindent\begin{prova} Para provar a afirmação anterior, sejam
duas funções $\kappa_1$ e $\kappa_2$ tangentes a $\psi$ em
$\tnr{T}_0$. Como a soma dos limites é o limite da soma, tem-se
que
\begin{equation}
\lim_{\alpha\to 0}\frac{\| \fua{\psi}{\alpha\tnr{H}+\tnr{T}_0} -
\fua{\kappa_1}{\alpha\tnr{H}+\tnr{T}_0}
 \|_{\ete{\crt{W}{q}}{\con{F}}}+\| \fua{\kappa_2}{\alpha\tnr{H}+\tnr{T}_0}- \fua{\psi}{\alpha\tnr{H}+\tnr{T}_0}
 \|_{\ete{\crt{W}{q}}{\con{F}}}}
{\|\alpha\tnr{H}\|_{\ete{\crt{V}{p}}{\con{F}}}}=0\,. \nonumber
\end{equation}
A partir da desigualdade triangular, é certo dizer que o valor do
numerador
\begin{eqnarray}
  \lefteqn{\| \fua{\psi}{\alpha\tnr{H}+\tnr{T}_0} -
\fua{\kappa_1}{\alpha\tnr{H}+\tnr{T}_0}
 \|_{\ete{\crt{W}{q}}{\con{F}}}
+} & & \nonumber\\
  &
&\| \fua{\kappa_2}{\alpha\tnr{H}+\tnr{T}_0}-
\fua{\psi}{\alpha\tnr{H}+\tnr{T}_0}
 \|_{\ete{\crt{W}{q}}{\con{F}}}\geqslant\| \fua{\kappa_2}{\alpha\tnr{H}+\tnr{T}_0} -
\fua{\kappa_1}{\alpha\tnr{H}+\tnr{T}_0}
\|_{\ete{\crt{W}{q}}{\con{F}}}\,.\nonumber
\end{eqnarray}
Tal desigualdade permite concluir que
\begin{equation}
\lim_{\alpha\to 0}\frac{\| \fua{\kappa_2}{\alpha\tnr{H}+\tnr{T}_0}
- \fua{\kappa_1}{\alpha\tnr{H}+\tnr{T}_0}
 \|_{\ete{\crt{W}{q}}{\con{F}}}}
{\|\alpha\tnr{H}\|_{\ete{\crt{V}{p}}{\con{F}}}}=0\,. \nonumber
\end{equation}
\end{prova}

\subsubsection{Abordagens Forte e
Fraca}\index{funções!tensoriais!tangentes!fortes} Considerando as
condições anteriores, seja o conjunto $\con{TG}_{\tnr{T}_0}$
formado por pares ordenados de funções tangentes entre si em
$\tnr{T}_0$, cujo domínio é um subconjunto aberto de $\cft{\crt{V}{p}}{\con{F}}$ que contém $\tnr{T}_0$. Pode ocorrer que exista um conjunto
$\overline{\con{TG}}_{\tnr{T}_0}\subseteq\con{TG}_{\tnr{T}_0}$,
tal que seus pares de funções sejam tangentes em $\tnr{T}_0$
independente da forma como o tensor $\alpha\tnr{H}\to\negmath{0}$ em
(\ref{eq:funcaoTangenteGateaux}). Em outras palavras, considerando
$\tnr{Y}:=\alpha\tnr{H}$, para que
$(\psi,\kappa)\in\overline{\con{TG}}_{\tnr{T}_0}$, deve ser válida
a igualdade:
\begin{equation}\label{eq:funcaoTangenteFrechet}
\lim_{\tnr{Y}\to \negmath{0}}\frac{\| \fua{\psi}{\tnr{Y}+\tnr{T}_0} -
\fua{\kappa}{\tnr{Y}+\tnr{T}_0}
 \|_{\ete{\crt{W}{q}}{\con{F}}}}
{\|\tnr{Y}\|_{\ete{\crt{V}{p}}{\con{F}}}}=0\,.
\end{equation}
Pode-se observar que esta condição é mais restritiva ou mais
\emph{forte} do que a condição (\ref{eq:funcaoTangenteGateaux}). A
primeira é então denominada abordagem forte para funções
tensoriais tangentes enquanto a segunda é denominada abordagem
fraca\index{funções!tensoriais!tangentes!fracas}. É importante
enfatizar a diferença entre as duas abordagens: utilizando uma
terminologia geométrica, na abordagem fraca, o ``caminho''
trilhado pela variável no domínio
${\tilde{T}}_{\con{\crt{V}{p}}\mapsto\con{\con{F}}}$ é definido
pelo tensor $\tnr{H}$ arbitrado; na abordagem forte, arbitra-se o
próprio ``caminho'' a ser trilhado. Para fins didáticos, os
``caminhos'' em ambas as abordagens estão representados na figura
a seguir.

\begin{figure}[!htt]
\centering
\includegraphics{partes/parte1/figs/c_diftens/AbordagemForteFraca.eps}
\titfigura{À esquerda, abordagem fraca com tensores arbitrários e
``caminhos'' radiais. À direita, abordagem forte com ``caminhos''
arbitrários.}
\end{figure}

Cabe ressaltar que se o domínio tensorial das funções tangentes
for unidimensional (independente da ordem), as abordagens forte e
fraca ficam idênticas. Desta forma,
\begin{eqnarray}
\dim\lpa\ete{\crt{V}{p}}{\con{F}}\rpa=1&\implies &
\overline{\con{TG}}_{\tnr{T}_0}=\con{TG}_{\tnr{T}_0}\,.\nonumber
\end{eqnarray}

\noindent\begin{prova} Para o caso do espaço tensorial unidimensional $\ete{\crt{V}{p}}{\con{F}}$, seja o conjunto $\lch\tnr{H}\rch$ uma de suas bases. Neste contexto, o tensor $\tnr{Y}$ em (\ref{eq:funcaoTangenteFrechet}) pode sempre ser escrito como a combinação linear $\alpha\tnr{H}$, resultando assim a expressão (\ref{eq:funcaoTangenteGateaux}).
\end{prova}

\subsection{Diferenciabilidade de Gâteaux}\index{Gâteaux!diferenciabilidade de}
Sejam os espaços tensoriais de Banach $\ete{\crt{V}{p}}{\con{F}}$
e $\ete{\crt{W}{q}}{\con{F}}$ e o espaço vetorial de funções
tensoriais lineares
\begin{equation}
\evl{\cft{\crt{V}{p}}{\con{F}}}{\cft{\crt{W}{q}}{\con{F}}}{\con{F}}\,.\nonumber
\end{equation}
Considerando ${\tilde{T}}_{\con{\crt{V}{p}}\mapsto\con{\con{F}}}$
um subconjunto aberto de $\cft{\crt{V}{p}}{\con{F}}$, seja um
tensor
$\tnr{T}_0\in{\tilde{T}}_{\con{\crt{V}{p}}\mapsto\con{\con{F}}}$ e
o conjunto dos pares ordenados de funções tensoriais tangentes
$\con{TG}_{\tnr{T}_0}$. Dado o mapeamento
\begin{equation}
\map{\psi}{{\tilde{T}}_{\con{\crt{V}{p}}\mapsto\con{\con{F}}}}{\cft{\crt{W}{q}}{\con{F}}}\,,
\end{equation}
se existir o par de funções tangentes
$(\psi,\psi_D)\in\con{TG}_{\tnr{T}_0}$, onde
\begin{equation}\label{eq:funcaoGDerivada}
\fua{\psi_D}{\tnr{X}}=\fua{\psi}{\tnr{T}_0}+\fua{\lco\fua{\dvt{\psi}}{\tnr{T}_0}\rco}{\tnr{X}-\tnr{T}_0}\,,
\end{equation}
tal que a função
\begin{equation}
\lco\fua{\dvt{\psi}}{\tnr{T}_0}\rco\in
\cfl{\cft{\crt{V}{p}}{\con{F}}}{\cft{\crt{W}{q}}{\con{F}}}
\end{equation}
é limitada, então este par é único. Nestas condições, a função
$\psi$ é dita \emph{diferenciável de Gâteaux}\index{Gâteaux!função
diferenciável de} ou
\emph{G-diferenciável}\index{função!G-diferenciável} em
$\tnr{T}_0$. Combinando (\ref{eq:funcaoGDerivada}) com
(\ref{eq:funcaoTangenteGateaux}), obtém-se que
\begin{equation}\label{eq:derivadaGateauxIni}
\lim_{\alpha\to 0}\frac{\|
\fua{\psi}{\alpha\tnr{\tnr{H}}+\tnr{T}_0} - \fua{\psi}{\tnr{T}_0}-
\alpha\fua{\lco\fua{\dvt{\psi}}{\tnr{T}_0}\rco}{\tnr{H}}
 \|_{\ete{\crt{W}{q}}{\con{F}}}}
{\|\alpha\tnr{H}\|_{\ete{\crt{V}{p}}{\con{F}}}}=0\,.
\end{equation}
O termo $\alpha\fua{\lco\fua{\dvt{\psi}}{\tnr{T}_0}\rco}{\tnr{H}}$
é denominado o \emph{diferencial de
Gâteaux}\index{Gâteaux!diferencial de} ou o
\emph{G-diferencial}\index{G-diferencial} de $\psi$ em $\tnr{T}_0$
na direção $\tnr{H}$. Chama-se o tensor
$\fua{\lco\fua{\dvt{\psi}}{\tnr{T}_0}\rco}{\tnr{H}}$ de
\emph{derivada direcional}\index{derivada!direcional} de $\psi$ em
$\tnr{T}_0$ na direção $\tnr{H}$. A função tensorial linear
limitada $\lco\fua{\dvt{\psi}}{\tnr{T}_0}\rco$ é chamada
\emph{derivada fraca}\index{derivada!fraca} ou \emph{derivada de
Gâteaux}\index{Gâteaux!derivada de} ou
\emph{G-derivada}\index{G-derivada} de $\psi$ em $\tnr{T}_0$. A função $\dvt{\psi}$ é denominada simplesmente a derivada de
Gâteaux ou a G-derivada de $\psi$. Além disso, se a
função $\psi$ for G-diferenciável em qualquer tensor $\tnr{T}_0$
do seu domínio, diz-se que ela é G-diferenciável em
${\tilde{T}}_{\con{\crt{V}{p}}\mapsto\con{\con{F}}}$.
\newline

\noindent\begin{prova} Vamos mostrar que a função $\psi_D$ é
única. Para tal, admitamos, por hipótese, que existam duas funções
$\psi_{D1}$ e $\psi_{D2}$ tangentes à $\psi$ em $\tnr{T}_0$, tais
que
\begin{equation}
\fua{\psi_{D1}}{\tnr{X}}=
\fua{\psi}{\tnr{T}_0}+\fua{\lco\fua{\dvt{\psi_1}}{\tnr{T}_0}\rco}{\tnr{X}-\tnr{T}_0}\nonumber
\end{equation}
e
\begin{equation}
\fua{\psi_{D2}}{\tnr{X}}=
\fua{\psi}{\tnr{T}_0}+\fua{\lco\fua{\dvt{\psi_2}}{\tnr{T}_0}\rco}{\tnr{X}-\tnr{T}_0}\,.\nonumber
\end{equation}
Desta forma, $\psi_{D1}$ e $\psi_{D2}$ são tangentes entre si em
$\tnr{T}_0$. Aplicando a abordagem fraca de funções tangentes,
obtém-se que
\begin{eqnarray}
\lim_{\alpha\to 0}\frac{\|
\fua{\lco\fua{\dvt{\psi_1}}{\tnr{T}_0}\rco}{\alpha\tnr{H}} -
\fua{\lco\fua{\dvt{\psi_2}}{\tnr{T}_0}\rco}{\alpha\tnr{H}}
 \|_{\ete{\crt{W}{q}}{\con{F}}}}
{\|\alpha\tnr{H}\|_{\ete{\crt{V}{p}}{\con{F}}}}&=&0\,.\nonumber
\end{eqnarray}
Elimina-se, pelas propriedades de funções lineares e normas, o
escalar $\alpha$. Logo,
\begin{equation}
\frac{\|
\fua{\lco\fua{\dvt{\psi_1}}{\tnr{T}_0}-\fua{\dvt{\psi_2}}{\tnr{T}_0}\rco}{\tnr{H}}
 \|_{\ete{\crt{W}{q}}{\con{F}}}}
{\|\tnr{H}\|_{\ete{\crt{V}{p}}{\con{F}}}} = 0\,.\nonumber
\end{equation}
Como $\tnr{H}$, por definição, é não nulo, a igualdade anterior é válida se o numerador for zero. Logo, é possível dizer que para qualquer $\tnr{H}\in{\tilde{T}}_{\con{\crt{V}{p}}\mapsto\con{\con{F}}}$ não nulo,
\begin{equation}
\fua{\lco\fua{\dvt{\psi_1}}{\tnr{T}_0}\rco}{\tnr{H}}=
\fua{\lco\fua{\dvt{\psi_2}}{\tnr{T}_0}\rco}{\tnr{H}}\,.
\nonumber
\end{equation}
\end{prova}

\subsubsection{Diferenciabilidade de Ordem Arbitrária}
Considerando as condições anteriores, sejam o espaço tensorial
$\ete{\lpa\crt{V}{p}\rpa^{k+1}}{\con{F}}$ e o espaço tensorial de funções lineares
\begin{equation}
\evl{\cft{\lpa\crt{V}{p}\rpa^{k+1}}{\con{F}}}{\cft{\crt{W}{q}}{\con{F}}}{\con{F}}\,,\nonumber
\end{equation}
onde $k\geqslant 0$. Considerando $\fua{\dtg{0}{\psi}}{\tnr{T}_0}:=\psi$, se existir o par ordenado de funções tensoriais $\lpa \fua{\dtg{k}{\psi}}{\tnr{T}_0},\psi_{\con{D}^{k+1}} \rpa\in\con{TG}_{\tnr{T}_0}$, tal que
\begin{eqnarray}
  \lefteqn{\fua{\psi_{D^{k+1}}}{\tnr{X}}=\fua{\lco\fua{\dtg{k}{\psi}}{\tnr{T}_0}\rco}{\tnr{H}_1\otimes\cdots\otimes\tnr{H}_{k-1}\otimes\tnr{T}_0}+} & & \nonumber\\
  &
&\fua{\lco\fua{\dtg{k+1}{\psi}}{\tnr{T}_0}\rco}{\tnr{H}_1\otimes\cdots\otimes\tnr{H}_{k}\otimes\lpa\tnr{X}-\tnr{T}_0\rpa}\,,
\end{eqnarray}
onde $\tnr{H}_i\in{\tilde{T}}_{\con{\crt{V}{p}}\mapsto\con{\con{F}}}$ e a função
\begin{equation}
\lco\fua{\dtg{k+1}{\psi}}{\tnr{T}_0}\rco\in
\cfl{\cft{\lpa\crt{V}{p}\rpa^{k+1}}{\con{F}}}{\cft{\crt{W}{q}}{\con{F}}}
\end{equation}
é limitada, então este par é único.

Neste contexto, diz-se que $\psi$ é G-diferenciável de ordem $k+1$ em $\tnr{T}_0$. A função linear $\fua{\dtg{k+1}{\psi}}{\tnr{T}_0}$ é a G-derivada de ordem $k+1$ de $\psi$ em $\tnr{T}_0$. Em termos genéricos, a igualdade (\ref{eq:derivadaGateauxIni}) assume a seguinte forma:
\begin{equation}
\begin{array}{rrr}\label{eq:derivadaGateauxGenerica}
&\lim_{\alpha\to
0}\frac{1}{\|\alpha\tnr{H}\|_{\ete{\crt{V}{p}}{\con{F}}}} \| \fua{\lco\fua{\dtg{k}{\psi}}{\tnr{T}_0}\rco}{\tnr{H}_1\otimes\cdots\otimes\tnr{H}_{k-1}\otimes\lpa\alpha\tnr{H}+\tnr{T}_0\rpa}- & \\
&\fua{\lco\fua{\dtg{k}{\psi}}{\tnr{T}_0}\rco}{\tnr{H}_1\otimes\cdots\otimes\tnr{H}_{k-1}\otimes\tnr{T}_0}- &
\\  &
\alpha\fua{\lco\fua{\dtg{k+1}{\psi}}{\tnr{T}_0}\rco}{\tnr{H}_1\otimes\cdots\otimes\tnr{H}_{k}\otimes\tnr{H}}\|_{\ete{\crt{W}{q}}{\con{F}}}=0\,. &
\end{array}
\end{equation}





\subsubsection{Diferenciabilidade de Fréchet}\index{Fréchet!diferenciabilidade de}
Considerando as condições anteriores, se agora existir o par de
funções tangentes
$(\psi,\psi_D)\in\overline{\con{TG}}_{\tnr{T}_0}$, então a função
$\psi$ é dita \emph{diferenciável de Fréchet}\index{Fréchet!função
diferenciável de} ou
\emph{F-diferenciável}\index{função!F-diferenciável} em
$\tnr{T}_0$. Combinando (\ref{eq:funcaoGDerivada}) e
(\ref{eq:funcaoTangenteFrechet}), obtém-se
\begin{equation}
\lim_{\tnr{Y}\to \negmath{0}}\frac{\|
\fua{\psi}{\tnr{\tnr{Y}}+\tnr{T}_0} - \fua{\psi}{\tnr{T}_0}-
\fua{\lco\fua{\dvt{\psi}}{\tnr{T}_0}\rco}{\tnr{Y}}
 \|_{\ete{\crt{W}{q}}{\con{F}}}}
{\|\tnr{Y}\|_{\ete{\crt{V}{p}}{\con{F}}}}=0\,.
\end{equation}
O tensor $\fua{\lco\fua{\dvt{\psi}}{\tnr{T}_0}\rco}{\tnr{Y}}$ é
denominado o \emph{diferencial de
Fréchet}\index{Fréchet!diferencial de} ou o
\emph{F-di\-fe\-ren\-cial}\index{F-diferencial} de $\psi$ em
$\tnr{T}_0$. A função tensorial limitada linear
$\lco\fua{\dvt{\psi}}{\tnr{T}_0}\rco$ é chamada \emph{derivada
forte}\index{derivada!forte} ou \emph{derivada de
Fréchet}\index{Fréchet!derivada de} ou
\emph{F-derivada}\index{F-derivada}\footnote{A demonstração de que
a F-derivada é única pode ser feita utilizando a mesma metodologia
aplicada à G-derivada, substituindo as ocorrências de $\tnr{Y}$
por $\alpha\tnr{H}$ em (\ref{eq:funcaoTangenteFrechet}).} de
$\psi$ em $\tnr{T}_0$. A função $\dvt{\psi}$ é a derivada de Fréchet ou a F-derivada de $\psi$.

Convém enfatizar que uma função F-diferenciável é sempre
G-di\-fe\-ren\-ciá\-vel\footnote{Uma função
G-di\-fe\-ren\-ciá\-vel é sempre F-diferenciável em condições
especiais. Ver \aut{Wouk}\cite{wouk_1979_1}, pp. 268-270.} e suas
derivadas respectivas são iguais. Apesar disso, é conveniente
utilizar notações diferentes para cada uma das derivadas: as
derivadas $\fua{\dvf{\psi}}{\tnr{T}_0}$ e
$\fua{\dvg{\psi}}{\tnr{T}_0}$ indicam, respectivamente, que $\psi$
é F-diferenciável e G-diferenciável em $\tnr{T}_0$. Caso $\psi$
seja F ou G-diferenciável no seu domínio, então as notações
respectivas para as derivadas são $\fua{\dvf{\psi}}{\tnr{X}}$ e
$\fua{\dvg{\psi}}{\tnr{X}}$.

O conceito de derivadas de ordens superiores também é extensível à abordagem forte de funções tangentes. Desta forma, diz-se que $\fua{\dgg{k}{\psi}}{\tnr{T}_0}$ e $\fua{\dgf{k}{\psi}}{\tnr{T}_0}$ são respectivamente a G-derivada e a F-derivada de ordem $k$ de $\psi$ em $\tnr{T}_0$.

\subsubsection{Calculando a Derivada Direcional}
Ainda sob as condições anteriores, a igualdade
(\ref{eq:derivadaGateauxIni}) revela que o numerador aproxima-se mais rápido de $0$ do que o denominador.
Desta forma, é possível concluir que
\begin{equation}\label{eq:numeradorGateaux}
\lim_{\alpha\to 0}\lco\fua{\psi}{\alpha\tnr{\tnr{H}}+\tnr{T}_0} - \fua{\psi}{\tnr{T}_0}-
\alpha\fua{\lco\fua{\dvg{\psi}}{\tnr{T}_0}\rco}{\tnr{H}}\rco
 =\negmath{0}\,.
\end{equation}
Já que $\alpha$ nunca é $0$, o termo entre colchetes nesta expressão nunca é zero, ou seja, sempre há um \emph{resíduo}. Tal resíduo pode ser definido pela função tensorial no mapeamento
\begin{equation}
\map{\ele{r}_\psi}{{\tilde{T}}_{\con{\crt{V}{p}}\mapsto\con{\con{F}}}}{\cft{\crt{W}{q}}{\con{F}}}\,,
\end{equation}
onde
\begin{equation}
\lim_{\tnr{X}\to \negmath{0}}\fua{\ele{r}_\psi}{\tnr{X}}=\negmath{0}\,.
\end{equation}
Com base nisso, pode-se entender a expressão
(\ref{eq:numeradorGateaux}) segundo a igualdade
\begin{equation}\label{eq:gderivadaResiduo}
\fua{\psi}{\alpha\tnr{\tnr{H}}+\tnr{T}_0} - \fua{\psi}{\tnr{T}_0}-
\alpha\fua{\lco\fua{\dvg{\psi}}{\tnr{T}_0}\rco}{\tnr{H}}
 = \fua{\ele{r}_\psi}{\tnr{H}}\,,
\end{equation}
onde $\ele{r}_\psi$ é dita a \emph{função
resíduo}\index{função!resíduo} de $\psi$ em $\tnr{T}_0\,$.
Neste contexto, isolando o termo da derivada direcional na igualdade anterior e em seguida aplicando $\lim_{\alpha\to 0}$ em ambos os lados da expressão resultante, obtém-se que
\begin{equation}\label{eq:calculoDerivadaDirecional}
\fua{\lco\fua{\dvg{\psi}}{\tnr{T}_0}\rco}{\tnr{H}} =
\lim_{\alpha\to 0}\frac{\fua{\psi}{\alpha\tnr{\tnr{H}}+\tnr{T}_0}
- \fua{\psi}{\tnr{T}_0}}{\alpha}\,.
\end{equation}
Vale ressaltar que para qualquer $\tnr{H}$, esta igualdade fornece
uma regra para a derivada $\fua{\dvg{\psi}}{\tnr{T}_0}$. Além disso, se $\tnr{T}_0$ também for um tensor qualquer, ou seja, se $\psi$ for G-diferenciável no seu domínio, tem-se a regra para a derivada direcional de $\psi$.

COLOCAR UMA PROPOSIÇÂO QUE MOSTRE A DERIVADA DIRECIONAL DE UMA FUNÇÂO ESCALAR

\paragraph{Derivadas Direcionais de Ordem Arbitrária.} O cálculo da
derivada direcional de ordem $k+1$, $k\geqslant 0$, é
feito a partir de (\ref{eq:derivadaGateauxGenerica}), generalizando a igualdade
(\ref{eq:calculoDerivadaDirecional}). Desta forma,
\begin{eqnarray}\label{eq:calculoDerivadaDirecionalOrdemN}
  \lefteqn{\fua{\lco\fua{\dgg{k+1}{\psi}}{\tnr{T}_0}\rco}{\tnr{H}_1\otimes\cdots\otimes\tnr{H}_{k}\otimes\tnr{H}}
=\lim_{\alpha\to
0}\frac{1}{\alpha}} & & \nonumber\\
  &
&\fua{\lco\fua{\dgg{k}{\psi}}{\tnr{T}_0}\rco}{\tnr{H}_1\otimes\cdots\otimes\tnr{H}_{k-1}\otimes\lpa \alpha\tnr{\tnr{H}}+\tnr{T}_0\rpa}
-\nonumber\\  & &
\qquad\fua{\lco\fua{\dgg{k}{\psi}}{\tnr{T}_0}\rco}{\tnr{H}_1\otimes\cdots\otimes\tnr{H}_{k-1}\otimes\tnr{T}_0}\,,
\end{eqnarray}
onde
$\tnr{H}_i\in{\tilde{T}}_{\con{\crt{V}{p}}\mapsto\con{\con{F}}}$ e
$\fua{\dgg{0}{\psi}}{\tnr{T}_0}:=\psi\,$. Para tensores
$\tnr{H},\tnr{H}_1,\cdots,\tnr{H}_k$ quaisquer, obtém-se a regra
da G-derivada $\fua{\dgg{k+1}{\psi}}{\tnr{T}_0}$ e por
conseqüência da F-derivada $\fua{\dgf{k+1}{\psi}}{\tnr{T}_0}$.

\paragraph{Derivadas Parciais.}\index{derivada!parcial}\label{sec:derivadaParcial}
Considerando as condições anteriores, seja o inteiro $m\geqslant 1$ e o conjunto
\begin{equation}
\con{P}={T}_{\con{\crt{V}{p_1}}\mapsto\con{\con{F}}}\times\cdots\times{T}_{\con{\crt{V}{p_m}}\mapsto\con{\con{F}}}
\end{equation}
definidor de um espaço vetorial de Banach. Seja o mapeamento
\begin{equation}
\map{\omega}{\tilde{\con{P}}}{\cft{\crt{W}{q}}{\con{F}}}\,.
\end{equation}
onde $\tilde{\con{P}}$ é subconjunto aberto de $\con{P}\,$.

Com base na proposição \ref{prp:somaPoliadicos}, a condição de existência da função no mapeamento anterior é garantida se, por exemplo,
\begin{equation}
\fua{\omega}{\tnr{X}_1,\cdots,\tnr{X}_m}=
\fua{\psi}{\sum_{k=1}^m\underbrace{\vto{x}_{1k}\otimes\cdots\otimes\vto{x}_{pk}}_{\tnr{X}_k}}\,,
\end{equation}
onde $p=p_1=\cdots=p_m\,$. A derivada direcional (\ref{eq:calculoDerivadaDirecional}) toma então a seguinte forma genérica:
\begin{eqnarray}\label{eq:derivadaDirecionalVariasVariaveis}
\lefteqn{\fua{\lco\fua{\dvg{\omega}}{\tnr{T}_1,\cdots,\tnr{T}_m}\rco}{\tnr{H}_1,\cdots,\tnr{H}_m} =\lim_{\alpha\to 0}\frac{1}{\alpha}} & & \nonumber\\
& &
\fua{\omega}{\alpha\tnr{H}_1+\tnr{T}_1,\cdots,\alpha\tnr{H}_m+\tnr{T}_m}
- \fua{\omega}{\tnr{T}_1,\cdots,\tnr{T}_m}\,.
\end{eqnarray}
A partir daí, diz-se que a função no lado esquerdo da igualdade
\begin{eqnarray}\label{eq:derivadaDirecionalParcial}
\lefteqn{\fua{\lco\fua{\dvg{\omega}}{\tnr{T}_1,\cdots,\tnr{T}_m}\rco}{\tnr{H}_r} =\lim_{\alpha\to
0}\frac{1}{\alpha}} & & \nonumber\\
& &
\fua{\omega}{\tnr{T}_1,\cdots,\alpha\tnr{H}_r+\tnr{T}_r,\cdots,\tnr{T}_m}
- \fua{\omega}{\tnr{T}_1,\cdots,\tnr{T}_m}
\end{eqnarray}
é a G-derivada parcial de $\omega$ em $\tnr{T}_r\,$. Neste caso, a função $\dvg{\omega}$ fica representada por $\dvp{r}{\omega}$, onde $\mathrm{r}$ é o índice do elemento da tupla sobre o qual a derivada parcial é definida. Vale lembrar que os conceitos aqui apresentadas também são válidos no contexto da função $\psi$ F-diferenciável.

\begin{prp}\label{teo:derivadaParcial}
Sejam os espaços de Banach $\ebh{P}{F}$ e $\ete{\crt{V}{p_i}}{\con{F}}$ tais que o conjunto
\begin{equation*}
\con{P}:={T}_{\con{\crt{V}{p_1}}\mapsto\con{\con{F}}}\times\cdots\times{T}_{\con{\crt{V}{p_m}}\mapsto\con{\con{F}}}\,,
\end{equation*}
onde $m\geqslant 1$. Dados um espaço tensorial $\ete{\crt{W}{q}}{\con{F}}$ e $\tilde{\con{P}}$ um subconjunto aberto de $\con{P}\,$, seja a função em  $\map{\omega}{\tilde{\con{P}}}{\cft{\crt{W}{q}}{\con{F}}}$ G-diferenciável na tupla ordenada $(\tnr{T}_1,\cdots,\tnr{T}_m)\,$. Nestas condições, para qualquer elemento   $(\tnr{H}_1,\cdots,\tnr{H}_m)\in\tilde{\con{P}}\,$, a derivada direcional
\begin{equation*}
\fua{\lco\fua{\dvg{\omega}}{\tnr{T}_1,\cdots,\tnr{T}_m}\rco}{\tnr{H}_1,\cdots,\tnr{H}_m} =
\sum_{r=1}^{m}\fua{\lco\fua{\dvp{r}{\omega}}{\tnr{T}_1,\cdots,\tnr{T}_m}\rco}{\tnr{H}_r}\,.
\end{equation*}
\end{prp}
\begin{prova}\rodape{Adaptada de \aut{Zeidler}\cite{zeidler_1995_1}, pp. 232-233.}
O processo mostrado a seguir, que utiliza $m=2\,$, pode ser generalizado para valores maiores. Tomando as definições (\ref{eq:derivadaDirecionalVariasVariaveis}) e (\ref{eq:derivadaDirecionalParcial}), concluimos facilmente as igualdades
\begin{equation*}
\fua{\lco\fua{\dvg{\omega}}{\tnr{T}_1,\tnr{T}_2}\rco}{\tnr{H}_1,\negmath{0}}=\fua{\lco\fua{\dvp{1}{\omega}}{\tnr{T}_1,\tnr{T}_2}\rco}{\tnr{H}_1}
\end{equation*}
e
\begin{equation*}
\fua{\lco\fua{\dvg{\omega}}{\tnr{T}_1,\tnr{T}_2}\rco}{\negmath{0},\tnr{H}_2}=\fua{\lco\fua{\dvp{2}{\omega}}{\tnr{T}_1,\tnr{T}_2}\rco}{\tnr{H}_2}\,,
\end{equation*}
rotuladas respectivamente por (i) e (ii). Somando estas duas expressões, chegamos à igualdade (iii)
\begin{equation*}
\fua{\lco\fua{\dvg{\omega}}{\tnr{T}_1,\tnr{T}_2}\rco}{\lpa\tnr{H}_1,\negmath{0}\rpa+\lpa\negmath{0},\tnr{H}_2\rpa}=\fua{\lco\fua{\dvp{1}{\omega}}{\tnr{T}_1,\tnr{T}_2}\rco}{\tnr{H}_1}+\fua{\lco\fua{\dvp{2}{\omega}}{\tnr{T}_1,\tnr{T}_2}\rco}{\tnr{H}_2}\,,
\end{equation*}
válida para quaisquer $\tnr{H}_1,\tnr{H}_2\in\tilde{\con{P}}\,$. Como estamos tratando com espaços vetoriais, fica claro que a tupla ordenada  $(\tnr{Z}_1,\tnr{Z}_2):=(\tnr{H}_1,\negmath{0})+(\negmath{0},\tnr{H}_2)\,$ é elemento de $\con{P}$. Fazendo $\tnr{H}_2=\negmath{0}$ em (iii), obtemos de (i) que $(\tnr{Z}_1,\tnr{Z}_2)=(\tnr{H}_1,\negmath{0})\,$. Da mesma forma, se $\tnr{H}_1=\negmath{0}\,$, obtemos de (ii) e (iii) que  $(\tnr{Z}_1,\tnr{Z}_2)=(\negmath{0},\tnr{H}_2)\,$. Como estas duas situações confirmam-se mutuamente, fica óbvio que $(\tnr{Z}_1,\tnr{Z}_2)=(\tnr{H}_1,\tnr{H}_2)\,$ para quaisquer $\tnr{H}_1,\tnr{H}_2\in\tilde{\con{P}}\,$.
\end{prova}

\subsection{Funções Tensoriais Suaves}\index{função!tensorial!suave}
Sejam os espaços tensoriais de Banach $\ete{\crt{V}{p}}{\con{F}}$
e $\ete{\crt{W}{q}}{\con{F}}$ e o subconjunto aberto
${\tilde{T}}_{\con{\crt{V}{p}}\mapsto\con{\con{F}}}\subset\cft{\crt{V}{p}}{\con{F}}$.
Sejam os conjuntos $\mathcal{C}_\mathrm{G}^0$, formado por todas as funções contínuas em seu domínio, e $\mathcal{D}_\mathrm{G}^1$, formado por todas as
funções G-diferenciáveis de ordem $1$ em seu domínio. Considerando
uma função qualquer $\psi\in\con\mathcal{D}_\mathrm{G}^1$ que
mapeia
${\tilde{T}}_{\con{\crt{V}{p}}\mapsto\con{\con{F}}}\mapsto\con{\cft{\crt{W}{q}}{\con{F}}}$,
a G-derivada $\dvg{\psi}$ define o mapeamento
\begin{equation}
\map{\dvg{\psi}}{{\tilde{T}}_{\con{\crt{V}{p}}\mapsto\con{\con{F}}}}
{\con{LB}_{{\tilde{T}}_{\con{\crt{V}{p}}\mapsto\con{\con{F}}}\mapsto\cft{\crt{W}{q}}{\con{F}}}}
 \,,
\end{equation}
onde o conjunto
\begin{equation}
\con{LB}_{{\tilde{T}}_{\con{\crt{V}{p}}\mapsto\con{\con{F}}}\mapsto\cft{\crt{W}{q}}{\con{F}}}\subset
\cfl{\cft{\crt{V}{p}}{\con{F}}}{\cft{\crt{W}{q}}{\con{F}}}
\end{equation}
é, por definição, normado. Neste contexto, caso as funções $\psi$ e $\dvg{\psi}$ sejam
elementos de $\mathcal{C}_\mathrm{G}^0$, diz-se que $\psi$ também é elemento do conjunto
$\sug{1}\subset\mathcal{D}_\mathrm{G}^1$, formado por todas as
funções contínuas cuja G-derivada é contínua. É comum dizer também
que $\psi$ é de classe $\sug{1}$ ou que é G-suave de ordem 1. Da
mesma forma, para o caso de funções F-diferenciáveis, define-se o
conjunto $\suf{1}\subset\mathcal{D}_\mathrm{F}$ das funções cuja
\emph{F-derivada} é contínua.

Em termos gerais, para que a função $\psi$ seja G-suave de ordem
$k+1$, ou da classe $\sug{k+1}$, ela e sua G-derivada
$\dgg{k}{\psi}$ devem ser G-suaves de ordem $k$. Em
outras palavras,
\begin{equation}
\psi\in\sug{k}\,\,\mathrm{e}\,\,\dgg{k}{\psi}\in\sug{k}\Longleftrightarrow\psi\in\sug{k+1}\,,\,\,k\geqslant
0\,.
\end{equation}

\subsubsection{Difeomorfismo}\index{difeomorfismo} Considerando as condições anteriores,
dado um tensor
$\tnr{T}_0\in{\tilde{T}}_{\con{\crt{V}{p}}\mapsto\con{\con{F}}}$,
a função tensorial $\psi$ é um difeomorfismo em $\tnr{T}_0$ se ela
for uma bijeção G-diferenciável em $\tnr{T}_0$ e sua função
inversa for G-diferenciável em $\fua{\psi}{\tnr{T}_0}$. Se a
bijeção $\psi$ e sua inversa forem G-suaves de ordem $n$, então diz-se que $\psi$ é um
\emph{$\sug{n}$-difeomorfismo}.

\begin{teo}[Função Inversa Local]\index{Função Inversa Local!Teorema da}\label{teo:FuncaoInversaLocal}
Sejam os espaços tensoriais de Banach $\ete{\crt{V}{p}}{\con{F}}$
e $\ete{\crt{W}{q}}{\con{F}}$. Seja o subconjunto aberto
${\tilde{T}}_{\con{\crt{V}{p}}\mapsto\con{\con{F}}}\subset\cft{\crt{V}{p}}{\con{F}}$
e um tensor
$\tnr{T}_0\in{\tilde{T}}_{\con{\crt{V}{p}}\mapsto\con{\con{F}}}$.
Uma função tensorial $\psi$ é um $\sug{n}$-difeomorfismo em
$\tnr{T}_0$ se e somente se sua derivada
$\fua{\dvg{\psi}}{\tnr{T}_0}$ for uma bijeção.
\end{teo}
\begin{prova}
Este teorema é uma aplicação do Teorema da Função
Implícita\rodape{Ver \aut{Zeidler}\cite{zeidler_1995_1}, pp.
259-260.}.
\end{prova}


\subsection{Regras Fundamentais Para Derivadas}
São apresentadas a seguir algumas regras aqui consideradas
fundamentais nos procedimentos de obtenção de derivadas. A
abordagem aplicada para se conseguir tais regras faz uso do
cálculo da derivada direcional, considerando o argumento de
direção um tensor qualquer.

\begin{teo}[Regra da Soma]\index{Regra da Soma}\label{teo:RegraSome}
Sejam os espaços tensoriais de Banach $\ete{\crt{V}{p}}{\con{F}}$ e
$\ete{\crt{W}{q}}{\con{F}}$. Seja o subconjunto aberto
${\tilde{T}}_{\con{\crt{V}{p}}\mapsto\con{\con{F}}}\subset\cft{\crt{V}{p}}{\con{F}}$.
Sejam as funções $\psi$, $\psi_1$ e $\psi_2$ que mapeiam
${\tilde{T}}_{\con{\crt{V}{p}}\mapsto\con{\con{F}}}\mapsto\cft{\crt{W}{q}}{\con{F}}$. Se
$\psi$ for G-diferenciável no seu domínio com regra
\begin{equation}
\fua{\psi}{\tnr{X}}=\fua{\psi_1}{\tnr{X}}+\fua{\psi_2}{\tnr{X}}\,,\nonumber
\end{equation}
então a função
\begin{equation}
\fua{\dvg{\psi}}{\tnr{X}}=\fua{\dvg{\psi_1}}{\tnr{X}}+
\fua{\dvg{\psi_2}}{\tnr{X}}\,.\nonumber
\end{equation}
\end{teo}
\begin{prova}
Considerando o tensor $\tnr{H}\in{\tilde{T}}_{\con{\crt{V}{p}}\mapsto\con{\con{F}}}$ uma
direção qualquer, seja o seguinte desenvolvimento:
\begin{eqnarray}
\fua{\lco\fua{\dvg{\psi}}{\tnr{X}}\rco}{\tnr{H}}&=&
\lim_{\alpha\to
0}\frac{\fua{\psi_1}{\alpha\tnr{\tnr{H}}+\tnr{X}}+\fua{\psi_2}{\alpha\tnr{\tnr{H}}+\tnr{X}}
- \fua{\psi_1}{\tnr{X}}-\fua{\psi_2}{\tnr{X}}}{\alpha} \nonumber\\
&=&\lim_{\alpha\to 0}\frac{\fua{\psi_1}{\alpha\tnr{\tnr{H}}+\tnr{X}} -
\fua{\psi_1}{\tnr{X}}}{\alpha}+\lim_{\alpha\to
0}\frac{\fua{\psi_2}{\alpha\tnr{\tnr{H}}+\tnr{X}} - \fua{\psi_2}{\tnr{X}}}{\alpha}
\nonumber\\
&=&\fua{\lco\fua{\dvg{\psi_1}}{\tnr{X}}\rco}{\tnr{H}}+\fua{\lco\fua{\dvg{\psi_2}}{\tnr{X}}\rco}{\tnr{H}}\nonumber\\
&=&\fua{\lco\fua{\dvg{\psi_1}}{\tnr{X}}+\fua{\dvg{\psi_2}}{\tnr{X}}\rco}{\tnr{H}}\,.\nonumber
\end{eqnarray}
A última igualdade é obtida porque as derivadas $\fua{\dvg{\psi_1}}{\tnr{X}}$ e
$\fua{\dvg{\psi_2}}{\tnr{X}}$ são elementos de um espaço vetorial de funções lineares.
\end{prova}

\begin{teo}[Regra do Produto]\index{Regra do Produto}\label{teo:RegraProduto}
Sejam os quatro espaços tensoriais de Banach
$\ete{\crt{V}{p}}{\con{F}}$,
$\ete{\crt{K}{r}\times\crt{Z}{s}}{\con{F}}$,
$\ete{\crt{K}{r}\times\crt{W}{q}}{\con{F}}$ e
$\ete{\crt{W}{q}\times\crt{Z}{s}}{\con{F}}$. Sejam o subconjunto
aberto
${\tilde{T}}_{\con{\crt{V}{p}}\mapsto\con{\con{F}}}\subset\cft{\crt{V}{p}}{\con{F}}$
e os mapeamentos:
\begin{eqnarray}
&
\map{\psi}{{\tilde{T}}_{\con{\crt{V}{p}}\mapsto\con{\con{F}}}}{\cft{\crt{K}{r}\times\crt{Z}{s}}{\con{F}}}\,,
&\nonumber\\
&
\map{\psi_1}{{\tilde{T}}_{\con{\crt{V}{p}}\mapsto\con{\con{F}}}}{\cft{\crt{K}{r}\times\crt{W}{q}}{\con{F}}}\,,
&\nonumber\\
&
\map{\psi_2}{{\tilde{T}}_{\con{\crt{V}{p}}\mapsto\con{\con{F}}}}{\cft{\crt{W}{q}\times\crt{Z}{s}}{\con{F}}}\,.
&\nonumber
\end{eqnarray}
Se $\psi$ for G-diferenciável no seu domínio com
regra
\begin{equation}
\fua{\psi}{\tnr{X}}=\fua{\psi_1}{\tnr{X}}\odot_q\fua{\psi_2}{\tnr{X}}\,,\nonumber
\end{equation}
então
\begin{eqnarray}
\fua{\lco\fua{\dvg{\psi}}{\tnr{X}}\rco}{\tnr{H}}=\fua{\psi_1}{\tnr{X}}\odot_q\fua{\lco\fua{\dvg{\psi_2}}{\tnr{X}}\rco}{\tnr{H}}+\nonumber\\
\fua{\lco\fua{\dvg{\psi_1}}{\tnr{X}}\rco}{\tnr{H}}\odot_q\fua{\psi_2}{\tnr{X}}\,,\forall\,\tnr{H}\in{\tilde{T}}_{\con{\crt{V}{p}}\mapsto\con{\con{F}}}\,.\nonumber
\end{eqnarray}
\end{teo}
\begin{prova}
A igualdade anterior é o resultado do seguinte desenvolvimento:
\begin{eqnarray}
\fua{\lco\fua{\dvg{\psi}}{\tnr{X}}\rco}{\tnr{H}}=\lim_{\alpha\to
0}\frac{1}{\alpha}\,\,\fua{\psi_1}{\alpha\tnr{\tnr{H}}+\tnr{X}}\odot_q\fua{\psi_2}{\alpha\tnr{\tnr{H}}+\tnr{X}}
- \fua{\psi_1}{\tnr{X}}\odot_q\fua{\psi_2}{\tnr{X}}\,,\nonumber
\end{eqnarray}
adicionando e subtraindo o termo
$\fua{\psi_1}{\alpha\tnr{\tnr{H}}+\tnr{X}}\odot_q\fua{\psi_2}{\tnr{X}}$,
tem-se
\begin{eqnarray}
\fua{\lco\fua{\dvg{\psi}}{\tnr{X}}\rco}{\tnr{H}}=\lim_{\alpha\to
0}\frac{1}{\alpha}\,\,\fua{\psi_1}{\alpha\tnr{\tnr{H}}+\tnr{X}}\odot_q\fua{\psi_2}{\alpha\tnr{\tnr{H}}+\tnr{X}}
-
\fua{\psi_1}{\alpha\tnr{\tnr{H}}+\tnr{X}}\odot_q\fua{\psi_2}{\tnr{X}}+\nonumber\\
\fua{\psi_1}{\alpha\tnr{\tnr{H}}+\tnr{X}}\odot_q\fua{\psi_2}{\tnr{X}}-
\fua{\psi_1}{\tnr{X}}\odot_q\fua{\psi_2}{\tnr{X}}=\nonumber\\
\lim_{\alpha\to
0}\frac{1}{\alpha}\,\,\fua{\psi_1}{\alpha\tnr{\tnr{H}}+\tnr{X}}\odot_q\lpa
\fua{\psi_2}{\alpha\tnr{\tnr{H}}+\tnr{X}} -
\fua{\psi_2}{\tnr{X}}\rpa + \nonumber\\
\lpa \fua{\psi_1}{\alpha\tnr{\tnr{H}}+\tnr{X}} -
\fua{\psi_1}{\tnr{X}}\rpa\odot_q\fua{\psi_2}{\tnr{X}}=\,\nonumber\\
\lim_{\alpha\to
0}\frac{1}{\alpha}\,\fua{\psi_1}{\alpha\tnr{\tnr{H}}+\tnr{X}}\odot_q\lpa
\fua{\psi_2}{\alpha\tnr{\tnr{H}}+\tnr{X}} -
\fua{\psi_2}{\tnr{X}}\rpa + \nonumber\\
\lim_{\alpha\to 0}\frac{1}{\alpha}\,\lpa
\fua{\psi_1}{\alpha\tnr{\tnr{H}}+\tnr{X}} -
\fua{\psi_1}{\tnr{X}}\rpa\odot_q\fua{\psi_2}{\tnr{X}}\,.\nonumber
\end{eqnarray}
\end{prova}


\begin{teo}[Regra da Cadeia]\index{Regra da Cadeia}\label{teo:RegraCadeia}
Sejam os espaços tensoriais de Banach $\ete{\crt{V}{p}}{\con{F}}$,
$\ete{\crt{K}{s}}{\con{F}}$ e $\ete{\crt{W}{q}}{\con{F}}$. Sejam
os subconjunto abertos
${\tilde{T}}_{\con{\crt{V}{p}}\mapsto\con{\con{F}}}\subset\cft{\crt{V}{p}}{\con{F}}$
e
${\tilde{T}}_{\con{\crt{K}{s}}\mapsto\con{\con{F}}}\subset\cft{\crt{K}{s}}{\con{F}}$
sobre os quais definem-se mapeamentos:
\begin{eqnarray}
&
\map{\psi}{{\tilde{T}}_{\con{\crt{V}{p}}\mapsto\con{\con{F}}}}{\cft{\crt{W}{q}}{\con{F}}}\,,
&\nonumber\\
&
\map{\psi_2}{{\tilde{T}}_{\con{\crt{V}{p}}\mapsto\con{\con{F}}}}{{\tilde{T}}_{\con{\crt{K}{s}}\mapsto\con{\con{F}}}}\,,
&\nonumber\\
&
\map{\psi_1}{{\tilde{T}}_{\con{\crt{K}{s}}\mapsto\con{\con{F}}}}{\cft{\crt{W}{q}}{\con{F}}}\,.
&\nonumber
\end{eqnarray}
Se $\psi$ for G-diferenciável no seu domínio com regra
\begin{equation}
\fua{\psi}{\tnr{X}}=\fua{\psi_1\circ\psi_2}{\tnr{X}}\,,\nonumber
\end{equation}
então
\begin{eqnarray}
\fua{\dvg{\psi}}{\tnr{X}}=\fua{\dvg{\psi_1}}{\fua{\psi_2}{\tnr{X}}}\circ\fua{\dvg{\psi_2}}{\tnr{X}}\,.\nonumber
\end{eqnarray}
\end{teo}
\begin{prova}\rodape{Adaptada de
\aut{Zeidler}\cite{zeidler_1995_1}, pg. 248.} Segundo a igualdade
(\ref{eq:gderivadaResiduo}), tem-se que
\begin{equation}
\alpha\fua{\lco\fua{\dvg{\psi_1}}{\tnr{Y}}\rco}{\tnr{Z}}
 = \fua{\psi_1}{\alpha\tnr{\tnr{Z}}+\tnr{Y}} - \fua{\psi_1}{\tnr{Y}}-
\fua{\ele{r}_{\psi_1}}{\tnr{Z}}\,.\nonumber
\end{equation}
Como $\tnr{Y}$ e $\tnr{Z}$ são tensores quaisquer, então seja
$\tnr{Y}=\fua{\psi_2}{\tnr{X}}$ , onde
\begin{equation} \fua{\psi_2}{\tnr{X}}
 = \fua{\psi_2}{\beta\tnr{\tnr{H}}+\tnr{X}} - \beta\fua{\lco\fua{\dvg{\psi_2}}{\tnr{X}}\rco}{\tnr{H}} -
\fua{\ele{r}_{\psi_2}}{\tnr{H}}\,,\nonumber
\end{equation}
e
$\tnr{Z}=\frac{\beta}{\alpha}\fua{\lco\fua{\dvg{\psi_2}}{\tnr{X}}\rco}{\tnr{H}}$.
Desenvolvendo a primeira igualdade, chega-se a
\begin{eqnarray}
\fua{\lco\fua{\lco\dvg{\psi_1}\rco\circ\psi_2}{\tnr{X}}\rco}{\fua{\lco\fua{\dvg{\psi_2}}{\tnr{X}}\rco}{\tnr{H}}}
 = \fua{\psi_1}{\fua{\psi_2}{\beta\tnr{\tnr{H}}+\tnr{X}}-
 \fua{\ele{r}_{\psi_2}}{\tnr{H}}} - \nonumber\\
 \fua{\psi_1\circ\psi_2}{\tnr{X}}-
\fua{\ele{r}_{\psi_1}}{\fua{\lco\fua{\dvg{\psi_2}}{\tnr{X}}\rco}{\tnr{H}}}\,.\nonumber
\end{eqnarray}
Fazendo $\beta \to 0$, utilizando o teorema \ref{teo:RieszGeneralizado} e aplicando o
conceito de funções representantes, pode-se realizar o seguinte desenvolvimento, para
qualquer $\tnr{H}\in{\tilde{T}}_{\con{\crt{V}{p}}\mapsto\con{\con{F}}}$:
\begin{eqnarray}
\fua{\lco\fua{\dvg{\psi}}{\tnr{X}}\rco}{\tnr{H}}&=&\fua{\lco\fua{\lco\dvg{\psi_1}\rco\circ\psi_2}{\tnr{X}}\rco}
{\fua{\lco\fua{\dvg{\psi_2}}{\tnr{X}}\rco}{\tnr{H}}}\nonumber\\
\fua{\lco\fua{\dvg{\psi}}{\tnr{X}}\rco}{\tnr{H}}&=&\ftr{A}{s}\circ\fua{\ftr{B}{p}}{\tnr{H}}\nonumber\\
\fua{\dvg{\psi}}{\tnr{X}}&=&\ftr{A}{s}\circ\ftr{B}{p}\,.\nonumber
\end{eqnarray}
\end{prova}


\subsection{Gradiente e Divergente}\index{gradiente}\label{sec:GradienteDivergente} Sejam os espaços tensoriais de Banach
$\ete{\crt{V}{p}}{\con{F}}$ e $\ete{\crt{W}{q}}{\con{F}}$. Seja o
subconjunto aberto
${\tilde{T}}_{\con{\crt{V}{p}}\mapsto\con{\con{F}}}\subset\cft{\crt{V}{p}}{\con{F}}$
e a função em
\begin{equation}
\map{\psi}{{\tilde{T}}_{\con{\crt{V}{p}}\mapsto\con{\con{F}}}}{\cft{\crt{W}{q}}{\con{F}}}
\end{equation}
G-diferenciável em
$\tnr{T}_0\in{\tilde{T}}_{\con{\crt{V}{p}}\mapsto\con{\con{F}}}$.
Segundo o teorema \ref{teo:RieszGeneralizado}, pode-se interpretar
a G-derivada de $\psi$ nas formas
\begin{eqnarray}
\fua{\dvg{\psi}}{\tnr{T}_0}=\rft{\bar{G}}{p}&\mathrm{ou}&
\fua{\dvg{\psi}}{\tnr{T}_0}=\lft{\hat{G}}{p}\,.
\end{eqnarray}
Os tensores
$\tnr{\bar{G}}\in{\con{T}}_{\con{\crt{W}{q}}\times\con{\crt{V}{p}}\mapsto\con{\con{F}}}$
e
$\tnr{\hat{G}}\in{\con{T}}_{\con{\crt{V}{p}}\times\con{\crt{W}{q}}\mapsto\con{\con{F}}}$,
presentes nas funções representantes, com notações alteradas para
$\grd{r}{\psi}{\tnr{T}_0}$ e $\grd{e}{\psi}{\tnr{T}_0}$
respectivamente, são denominados gradiente \emph{à
direita}\index{gradiente!à direita} e \emph{à
esquerda}\index{gradiente!à esquerda} de $\psi$ em $\tnr{T}_0$.
Neste contexto, a regra da G-derivada de $\psi$, para qualquer
tensor
$\tnr{T}_0\in{\con{T}}_{\con{\crt{V}{p}}\mapsto\con{\con{F}}}$,
pode ser escrita nas seguintes formas:
\begin{equation}
\fua{\lco\fua{\dvg{\psi}}{\tnr{T}_0}\rco}{\tnr{X}}=\grd{r}{\psi}{\tnr{T}_0}\odot_p\tnr{X}
\end{equation}
ou
\begin{equation}
\fua{\lco\fua{\dvg{\psi}}{\tnr{T}_0}\rco}{\tnr{X}}=\tnr{X}\odot_p\grd{e}{\psi}{\tnr{T}_0}\,.
\end{equation}
As funções $\gqu{r}{\psi}$ e
$\gqu{e}{\psi}$ são chamadas os
gradientes à direita e à esquerda  de $\psi$ respectivamente. Além disso, se eventuais conceitos posteriores
independem das abordagens à direita ou à esquerda, será utilizada
a notação $\gqu{}{\psi}$.

Para os casos onde $q\geqslant p$, dado o tensor
identidade\index{divergente}
$\tnr{I}\in{\con{T}}_{\con{\crt{V}{p}\times\crt{V}{p}}\mapsto\con{\con{F}}}$,
diz-se que o tensor
\begin{equation}
\drd{r}{\psi}{\tnr{T}_0}:=\grd{r}{\psi}{\tnr{T}_0}\odot_{2p}\tnr{I}
\end{equation}
de ordem $q-p$ é o divergente \emph{à direita}\index{divergente!à
direita} de $\psi$ em $\tnr{T}_0$ se
$\crt{W}{q}=\crt{Z}{q-p}\times\crt{V}{p}$. Da mesma forma, o
tensor de ordem $q-p$
\begin{equation}
\drd{e}{\psi}{\tnr{T}_0}:=\tnr{I}\odot_{2p}\grd{e}{\psi}{\tnr{T}_0}
\end{equation}
é o divergente \emph{à esquerda}\index{divergente!à esquerda} de
$\psi$ em $\tnr{T}_0$ se
$\crt{W}{q}=\crt{V}{p}\times\crt{U}{q-p}$. As funções
$\dqu{r}{\psi}$ e $\dqu{e}{\psi}$ são os divergentes à direita e à esquerda de $\psi$ respectivamente. Além
disso, se os conceitos à direita e à esquerda forem irrelevantes,
utiliza-se a notação $\dqu{}{\psi}$.

\begin{prp}
Sejam os quatro espaços tensoriais de Banach
$\ete{\crt{V}{p}}{\con{F}}$,
$\ete{\crt{K}{r}\times\crt{Z}{s}}{\con{F}}$,
$\ete{\crt{K}{r}\times\crt{W}{q}}{\con{F}}$ e
$\ete{\crt{W}{q}\times\crt{Z}{s}}{\con{F}}$. Sejam o subconjunto
aberto
${\tilde{T}}_{\con{\crt{V}{p}}\mapsto\con{\con{F}}}\subset\cft{\crt{V}{p}}{\con{F}}$
e os mapeamentos:
\begin{eqnarray}
&
\map{\psi}{{\tilde{T}}_{\con{\crt{V}{p}}\mapsto\con{\con{F}}}}{\cft{\crt{K}{r}\times\crt{Z}{s}}{\con{F}}}\,,
&\nonumber\\
&
\map{\psi_1}{{\tilde{T}}_{\con{\crt{V}{p}}\mapsto\con{\con{F}}}}{\cft{\crt{K}{r}\times\crt{W}{q}}{\con{F}}}\,,
&\nonumber\\
&
\map{\psi_2}{{\tilde{T}}_{\con{\crt{V}{p}}\mapsto\con{\con{F}}}}{\cft{\crt{W}{q}\times\crt{Z}{s}}{\con{F}}}\,.
&\nonumber
\end{eqnarray}
Se $\psi$ for G-diferenciável no seu domínio com regra
\begin{equation}
\fua{\psi}{\tnr{X}}=\fua{\psi_1}{\tnr{X}}\odot_q\fua{\psi_2}{\tnr{X}}\,,\nonumber
\end{equation}
então são válidos os seguintes itens:
\begin{itemize}
\item[i.] para qualquer
$\tnr{H}\in{T}_{\con{\crt{V}{p}}\mapsto\con{\con{F}}}$,
\begin{eqnarray}
\grd{r}{\psi}{\tnr{X}}\odot_p\tnr{H}=\fua{\psi_1}{\tnr{X}}\odot_q\grd{r}{\psi_2}{\tnr{X}}\odot_p\tnr{H}+\nonumber\\
\tnr{H}\odot_p\grd{e}{\psi_1}{\tnr{X}}\odot_q\fua{\psi_2}{\tnr{X}}=\tnr{H}\odot_p\grd{e}{\psi}{\tnr{X}}\,;
 \nonumber
\end{eqnarray}
\item[ii.] se $p=q=r=s$ ,
\begin{eqnarray}
\grd{r}{\psi}{\tnr{X}}=\fua{\psi_1}{\tnr{X}}\odot_p\grd{r}{\psi_2}{\tnr{X}}+\nonumber\\
\lco\fua{\psi_2}{\tnr{X}}\rco_{(1,p+1)(p,2p)}\odot_p\lco\grd{e}{\psi_1}{\tnr{X}}\rco_{(1,2p+1)(p,3p)}
 \nonumber
\end{eqnarray}
e
\begin{eqnarray}
\grd{e}{\psi}{\tnr{X}}=\lco\grd{r}{\psi_2}{\tnr{X}}\rco_{(1,2p+1)(p,3p)}\odot_p\lco\fua{\psi_1}{\tnr{X}}\rco_{(1,p+1)(p,2p)}+\nonumber\\
\grd{e}{\psi_1}{\tnr{X}}\odot_q\fua{\psi_2}{\tnr{X}}\,.
 \nonumber
\end{eqnarray}


\item[iii.] se $p=q=r=s$ e $\con{V}_i=\con{K}_i=\con{Z}_i$ ,
\begin{equation}
\drd{r}{\psi}{\tnr{X}}=\fua{\psi_1}{\tnr{X}}\odot_p\drd{r}{\psi_2}{\tnr{X}}+
\lco\fua{\psi_2}{\tnr{X}}\rco_{(1,p+1)(p,2p)}\odot_p\drd{r}{\psi_1}{\tnr{X}}\nonumber
\end{equation}
e
\begin{equation}
\drd{e}{\psi}{\tnr{X}}=\drd{e}{\psi_2}{\tnr{X}}\odot_p\lco\fua{\psi_1}{\tnr{X}}\rco_{(1,p+1)(p,2p)}+
\drd{e}{\psi_1}{\tnr{X}}\odot_p\fua{\psi_2}{\tnr{X}}\nonumber\,.
\end{equation}
\end{itemize}
\end{prp}
\begin{prova} Os itens a seguir referem-se aos itens respectivos
do teorema.
\begin{itemize}
\item[i.] É uma conseqüência direta da aplicação da regra do
produto com a definição de gradiente.

\item[ii.] Dadas as condições do ítem, o tensor
$\grd{e}{\psi_1}{\tnr{X}}\in\cft{\crt{V}{p}\times\crt{K}{p}\times\crt{W}{p}}{\con{F}}$
no ítem i. pode ser transposto para
$\lco\grd{e}{\psi_1}{\tnr{X}}\rco_{(1,2p+1)(p,3p)}\in\cft{\crt{W}{p}\times\crt{K}{p}\times\crt{V}{p}}{\con{F}}$.
Desta forma, o produto contrativo com o tensor $\tnr{H}$ na
igualdade
\begin{eqnarray}
\grd{r}{\psi}{\tnr{X}}\odot_p\tnr{H}=\fua{\psi_1}{\tnr{X}}\odot_p\grd{r}{\psi_2}{\tnr{X}}\odot_p\tnr{H}+\nonumber\\
\lco\fua{\psi_2}{\tnr{X}}\rco_{(1,p+1)(p,2p)}\odot_p\lco\grd{r}{\psi_1}{\tnr{X}}\rco_{(1,2p+1)(p,3p)}\odot_p\tnr{H}\,,\forall\,\tnr{H}\in\crt{V}{p}\,.
 \nonumber
\end{eqnarray}
pode ser eliminado. A igualdade para o gradiente à esquerda é
obtida da mesma forma.

\item[iii.]
Pelas condições do ítem, pode-se verificar facilmente que
\begin{equation}
\lco\grd{e}{\psi_1}{\tnr{X}}\rco_{(1,2p+1)(p,3p)}= \grd{r}{\psi_1}{\tnr{X}}\,.\nonumber
\end{equation}
Tomando a igualdade do gradiente à direita no ítem ii, tem-se,
para um tensor identidade
$\tnr{I}\in{\con{T}}_{\con{\crt{V}{p}\times\crt{V}{p}}\mapsto\con{\con{F}}}$,
que
\begin{eqnarray}
\grd{r}{\psi}{\tnr{X}}\odot_{2p}\tnr{I}&=&\fua{\psi_1}{\tnr{X}}\odot_q\grd{r}{\psi_2}{\tnr{X}}\odot_{2p}\tnr{I}+\nonumber\\
&&\lco\fua{\psi_2}{\tnr{X}}\rco_{(1,p+1)(p,2p)}\odot_q\grd{r}{\psi_1}{\tnr{X}}\odot_{2p}\tnr{I}\,.
 \nonumber
\end{eqnarray}
Tal desenvolvimento também pode ser realizado para a igualdade do
gradiente à esquerda.
\end{itemize}
\end{prova}



\begin{prp}\label{teo:gradienteConstante}
Sejam os espaços tensoriais de Banach
$\ete{\crt{V}{p}}{\con{F}}$ e $\ete{\crt{W}{q}}{\con{F}}$. Seja o
subconjunto aberto ${\tilde{T}}_{\con{\crt{V}{p}}\mapsto\con{\con{F}}}\subset\cft{\crt{V}{p}}{\con{F}}$
e a função no mapeamento
\begin{equation}\nonumber
\map{\psi}{{\tilde{T}}_{\con{\crt{V}{p}}\mapsto\con{\con{F}}}}{\cft{\crt{W}{q}}{\con{F}}}
\end{equation}
G-diferenciável no seu domínio. Considerando os gradientes $\gqu{r}{\psi}$ e
$\gqu{e}{\psi}$ constantes em ${\tilde{T}}_{\con{\crt{V}{p}}\mapsto\con{\con{F}}}$ com regras respectivas  $\fua{\gqu{r}{\psi}}{\tnr{X}}=\tnr{G}_r$ e $\fua{\gqu{e}{\psi}}{\tnr{X}}=\tnr{G}_e$, então
\begin{equation}\nonumber
 \tnr{G}_r \odot_p \lpa \tnr{H}_1 - \tnr{H}_2 \rpa = \lpa \tnr{H}_1 - \tnr{H}_2 \rpa \odot_p \tnr{G}_e = \fua{\psi}{\tnr{H}_1} - \fua{\psi}{\tnr{H}_2}\,,
\end{equation}
onde $\tnr{H}_1$ e $\tnr{H}_2$ são tensores quaisquer de ${\tilde{T}}_{\con{\crt{V}{p}}\mapsto\con{\con{F}}}\,$.
\end{prp}
\begin{prova}
Consideremos, primeiramente, uma regra qualquer para $\psi$ tal que sua G-derivada, com regra $\fua{\dvg{\psi}}{\tnr{X}}=\tnr{G}$, é constante. Desta forma, seja a regra $\fua{\psi}{\tnr{X}}=\tnr{G}\odot_p \tnr{X} + \tnr{C}$, onde $\tnr{C}\in{\tilde{T}}_{\con{\crt{V}{p}}\mapsto\con{\con{F}}}$ é também constante. Aplicando (\ref{eq:calculoDerivadaDirecional}), tem-se que
\begin{equation}\nonumber
 \fua{\lco\fua{\dvg{\psi}}{\tnr{X}}\rco}{\tnr{H}}=\fua{\psi}{\tnr{H}}-\tnr{C}\,,\,\forall\,\tnr{H}\in{\tilde{T}}_{\con{\crt{V}{p}}\mapsto\con{\con{F}}}\,.
\end{equation}
Tomando os tensores $\tnr{H}_1$ e $\tnr{H}_2$ quaisquer, tem-se duas igualdades no formato na igualdade anterior. Quando tais igualdades são subtraídas uma da outra obtém-se
\begin{equation}\nonumber
 \fua{\lco\fua{\dvg{\psi}}{\tnr{X}}\rco}{\tnr{H}_1}-\fua{\lco\fua{\dvg{\psi}}{\tnr{X}}\rco}{\tnr{H}_2}=\fua{\psi}{\tnr{H}_1}-\fua{\psi}{\tnr{H}_2}\,.
\end{equation}
A partir daí, com base na definição de gradiente, obtém-se as últimas igualdades da proposição.
\end{prova}

\subsection{Contornos Regulares}\index{contorno!regular}\label{sec:contornoRegular}
Sejam os subespaços afins $\saf{U}{S}{\ele{a}}{F}$ e $\saf{
\partial U}{\partial S}{\ele{b}}{F}$ do espaço afim de Hilbert
$\eaf{V}{A}{F}$, cuja dimensão é maior que 2, onde $\partial U$ e
$\epo{\partial S}_\ele{b}$ são os contornos dos conjuntos
$\con{U}$ e $\epo{S}_\ele{a}$ respectivamente. Seja $\ehr{W}{F}$
subespaço bidimensional de $\ehr{V}{F}$. O contorno $\epo{\partial S}_\ele{b}$ é
dito regular se para cada
$\ele{u}\in\epo{\partial S}_\ele{b}$ existir um mapeamento bijetor
\begin{equation}
\map{\zeta}{\con{Z}}{\viz{\lch\ele{u}\rch}\cap\epo{\partial S}_\ele{b}}\,,
\end{equation}
onde
 \begin{itemize}
    \item[i.] o conjunto $\con{Z}\subset\con{W}$ é aberto;
    \item[ii.] o conjunto $\viz{\lch\ele{u}\rch}\subset\epo{A}$;
    \item[iii.] a bijeção $\zeta$ é G-suave de ordem 1 em $\con{Z}$;
    \item[iv.] a função $\zeta^{-1}$ é contínua em
    $\viz{\lch\ele{u}\rch}\cap\epo{\partial S}_\ele{b}$;
    \item[v.] a derivada $\fua{\dvg{\zeta}}{\fua{\zeta^{-1}}{\ele{u}}}$ é inversível.
\end{itemize}


\subsubsection{Função Normal Unitária}
Considerando as condições do ítem anterior, seja o campo característico
$\fac{F}_{\epo{\partial S}_\ele{b}}^{\partial\con{U}}$, tal que os pontos de
$\epo{\partial S}_\ele{b}$ são relacionados, de forma biunívoca, aos vetores de
$\partial\con{U}$. Seja $\lch \vun{w}_1 , \vun{w}_2\rch$ uma das bases ortonormais de
$\ehr{W}{F}$. Adotando
\begin{equation}
\overline{\zeta}:=\fac{F}_{\epo{\partial
S}_\ele{b}}^{\partial\con{U}}\circ\zeta\,,
\end{equation}
dado $\vto{z}_\vto{u}=\fua{\overline{\zeta}^{\,-1}}{\vto{u}}$, onde $\vto{u}=\fua{\fac{F}_{\epo{\partial
S}_\ele{b}}^{\partial\con{U}}}{\ele{u}}$, o vetor polar
\begin{equation}
\vto{n}_\vto{u}:=\frac{\fua{\lco\fua{\dvg{\overline{\zeta}}}{\vto{z}_\vto{u}}\rco}{\vun{w}_1}\barwedge
\fua{\lco\fua{\dvg{\overline{\zeta}}}{\vto{z}_\vto{u}}\rco}{\vun{w}_2}}{\|
\fua{\lco\fua{\dvg{\overline{\zeta}}}{\vto{z}_\vto{u}}\rco}{\vun{w}_1}\barwedge
\fua{\lco\fua{\dvg{\overline{\zeta}}}{\vto{z}_\vto{u}}\rco}{\vun{w}_2}\|_{\ehr{V}{F}}}
\end{equation}
é denominado \emph{vetor normal unitário}\index{vetor!normal
unitário} ao conjunto $\viz{\lch\vto{u}\rch}\cap\partial\con{U}$. Se
existir pelo menos mais uma função $\overline{\zeta}'\neq\overline{\zeta}$ definida a
partir de $\vto{u}$ e $\con{Z'}\subset\con{W}$ e
\begin{equation}
\vto{n}'_\vto{u}:=\frac{\fua{\lco\fua{\dvg{\overline{\zeta}'}}{\vto{z}'_\vto{u}}\rco}{\vun{w}_1}\barwedge
\fua{\lco\fua{\dvg{\overline{\zeta}'}}{\vto{z}'_\vto{u}}\rco}{\vun{w}_2}}{\|
\fua{\lco\fua{\dvg{\overline{\zeta}'}}{\vto{z}'_\vto{u}}\rco}{\vun{w}_1}\barwedge
\fua{\lco\fua{\dvg{\overline{\zeta}'}}{\vto{z}'_\vto{u}}\rco}{\vun{w}_2}\|_{\ehr{V}{F}}}=\vto{n}_\vto{u}\,,\,\forall\,\vto{u}\in\partial\con{U}\,,
\end{equation}
diz-se que o contorno regular $\partial\con{U}$ é
\emph{orientável}\index{contorno!orientável}. Neste contexto, é
possível definir o mapeamento
\begin{equation}
\map{\fac{N}_{\partial \con{U}}}{\partial \con{U}}{\con{V}}\,,
\end{equation}
com regra
$\fua{\fac{N}_{\partial\con{U}}}{\vto{x}}=\vto{n}_\vto{x}$. A
função $\fac{N}_{\partial \con{U}}$ é denominada normal unitária
ao contorno orientável $\partial \con{U}$. Se o vetor $\fua{\fac{N}_{\partial
U}}{\vto{u}}+\vto{u}\notin \con{U}$ para todo $\vto{u}\in\partial
\con{U}$, tem-se a representação $\fac{N}^+$  e o contorno orientável
$\partial \con{U}$ é dito estar \emph{positivamente
orientado}\index{contorno!positivamente orientado}. Caso
$\fua{\fac{N}_{\partial U}}{\vto{u}}+\vto{u}\in \con{U}$ para todo
$\vto{u}\in\partial \con{U}$, tem-se $\fac{N}^-$ e o contorno
$\partial \con{U}$ está \emph{negativamente
orientado}\index{contorno!negativamente orientado}.
