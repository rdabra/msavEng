
\chapter{Topics of Affine Geometry}

Although vector spaces are building blocks of the several important concepts presented so far on the two previous chapters, it is still not possible to recognize shapes in their definer sets. However, the study of mathematical shapes does not belong to Linear Algebra, but to the realms of Geometry, whose non empty sets we shall consider here to be constituted by ``shape'' elements called \textsb{points}\index{points}, which are geometric objects of primitive notion, devoid of dimensional features in order to morphologically represent physical locations. If a given set of such points is conveniently related to a vector space by a group action, other shapes can be obtained and thereby further morphological concepts can be developed, when we say that a certain geometry is defined. As physical spaces are usually abstracted by geometric sets, we can not prescind from studying at least the basic topics of the so called Affine Geometry, which includes the well known Euclidean Geometry and is sufficiently governed by the concept of parallelism.


\section{Affine Spaces}\index{affine!space}\index{space!affine}\label{sec:affine}

Recalling the concept of group action presented on chapter \ref{ch:Collect}, let the set of points $\epu{U}$ be the $G$-set of a vector space $U_\cam{F}$ through a simply transitive group action $\oplus$. Given two arbitrary points $a$ and $b$ of this $G$-set $\epu{U}$, now called a \textsb{point space}\index{space!point}, and a vector $\vto{u}\in U_\cam{F}$ where point $\fua{\oplus}{\vto{u}, a}=b$ or $\gloref{actVect}=b$, the axioms of simply transitive group actions can be rewritten the following way:
\begin{itemize}\label{ax:pointSpaces}
\setlength\itemsep{.1em}
\item[i.]  $\vto{0}\oplus a=a\,,\forall \, \ele{a} \in \epu{U}$;
\item[ii.] $\vto{u}\oplus(\vto{v}\oplus a)=(\vto{u}+\vto{v})\oplus a, \forall \, \ele{a}\in \epu{U},\,\forall\, \vto{u},\vto{v}\in U_\cam{F}$;
\item[iii.]  $\exists !\,\, \vto{u}\in U_\cam{F}\textrm{ such that } \vto{u}\oplus a=b$.	
\end{itemize}
In this context, the triple $(U_\cam{F},\epu{U},\oplus)$ is called an affine space, henceforth represented by $\eam{U}{F}$, which will also be used to refer to the point space $\epu{U}$, as we adopted similarly for other cases in order to simplify notation. Moreover, vector space $U_\cam{F}$ is usually called the \textsb{direction space}\index{space!direction} of affine space $\eam{U}{F}$. Therefore, we can say that an affine space is defined by a direction space, a point space and a group action. From axiom iii above, every double of points uniquely identifies a certain vector: if $\vto{u}\oplus a=b$, we define that double $(a,b)$ identifies vector $\vto{u}$, when this vector is represented by $\vv{ab}$ and then $\vv{ab}\oplus a=b$. \emph{The converse is not true: a certain vector can be identified by infinite doubles of points because $\vto{u}\oplus x=y$ for all $x\in\eam{U}{F}$ and then $\vv{ab}=\vv{xy}$}. Now, from axioms i and ii, considering point $\vto{v}\oplus a=c$, we conclude that $-\vto{v}\oplus c=a$, and then vector $\vto{v}=\vv{ac}=-\vv{ca}$. From this characteristic relationship between points and vectors, the classical pictorial representation of a vector identified by a double of points $(a,b)$ as an arrow whose tail ``starts'' on $a$ and head points to $b$ arises naturally (figure \ref{fg:vetorSeta}).
\begin{figure}[!ht]
\centering
\begin{center}
\scalebox{.72}{\input{partes/figs/vetorSeta.pstex_t}}
\end{center}
\titfigura{Vector as an arrow identified by points $a$ and $b$.}\label{fg:vetorSeta}
\end{figure}
As an arbitrary vector can be identified by infinite doubles of points, given arbitrary points $x,y,a,b\in\eam{U}{F}$, the second operand of $\vv{ab}+\vv{xy}$ can be described by a vector $\vv{bc}$, where point $c=\vv{xy}\oplus b$, and then equalities $(\vv{ab}+\vv{bc})\oplus a=(\vv{bc}+\vv{ab})\oplus a=\vv{bc}\oplus b=c$ enable us to conclude that if $(\vv{ab}+\vv{bc})\oplus a=c$ then $\vv{ab}+\vv{bc}=\vv{ac}$. From this algebraic property, it is possible to depict graphically a sum of vectors by concatenating each one of the their representative arrows in such a way that a head point to a tail, as shown in the following figure \ref{fg:vetorSoma}.
\begin{figure}[!ht]
\centering
\begin{center}
\scalebox{.72}{\input{partes/figs/vetorSoma.pstex_t}}
\end{center}
\titfigura{Sum of vectors $\protect\vv{ab}$ and $\protect\vv{xy}$ where $\protect\vv{xy}\oplus b=c$.}\label{fg:vetorSoma}
\end{figure}
It is then straightforward to conclude that a vector $\alpha\vv{ab}$, where $\alpha\in\real$, is represented by a ``compressed'' ($\alpha\leq 1$) or ``expanded'' ($\alpha >1$) vector $\vv{ab}$. Moreover, we can say that the coordinates of vector $\alpha\vv{ab}$ are the coordinates of $\vv{ab}$ equally multiplied, that is equally ``compressed'' or ``expanded'' by the real number $\alpha$. For the case of a complex affine space $\eam{U}{C}$, the polar complex coordinates of a vector when multiplied by a complex $\alpha$ are equally multiplied by the real scalar $|\alpha|$ and equally rotated by $\arg{(\alpha)}$ on the complex plane, according to the basic theory of complex number multiplication. Therefore, in affine geometric representation, a complex vector $\alpha\vv{ab}\in U_\cam{C}$ results also in a ``compressed'' or ``expanded'' complex vector $\vv{ab}$.

Considering previous conditions, given an arbitrary point $a\in \eam{U}{F}$ and a vector subspace $S_\cam{F}\subset U_\cam{F}$, the triple $(S_\cam{F},\epu{S}_a,\oplus)$ is called an \textsb{affine subspace}\index{affine!subspace} of $\eam{U}{F}$, represented by $\eams{S}{F}{a}$, if point space $\epu{S}_a:=\{\vto{x}\oplus a:\forall \vto{x}\in S_\cam{F}\}$, to which point $a$ also belongs since vector $\vto{0}\in S_\cam{F}$. Note that if a point $b\in\eams{S}{F}{a}$ defines the affine subspace $\eams{S}{F}{b}$, then 
\begin{equation}\label{eq:igualAff}
\eams{S}{F}{a}=\{\vto{x}\oplus a:\forall \vto{x}\in S_\cam{F}\}=\{\vto{x}+\vv{ab}\oplus b:\forall \vto{x}\in S_\cam{F}\}=\eams{S}{F}{b}\,.
\end{equation}
In the case of affine space $\eam{U}{F}$ defined by a $m$-dimensional vector space, an arbitrary affine  subspace $\eams{S}{F}{a}\subset\eam{U}{F}$ becomes another fundamental geometric object called a \textsb{hyperplane}\index{hyperplane} when its definer vector subspace $S_\cam{F}$ is $(m-1)$-dimensional: we specifically call the hyperplane a \textsb{line}\index{line} or a \textsb{plane}\index{plane} when $S_\cam{F}$ is one or two-dimensional respectively. Since it is the vector space which bears a dimensional feature, it is defined that $\dim{(\eam{U}{F})}=\dim{(U_\cam{F})}=m$ and then we use $\gloref{affSub}$ to represent a $m$-dimensional affine space. Now, from the definition of affine spaces using a group action approach, as we have done, angles and distances cannot be obtained, and then it results that requiring these two concepts means two additional features that turn affine geometry into Euclidean geometry, when we state that Euclidean is a more restricted form of affine geometry. On the other hand, our algebraic definition of affine spaces does not require to explicitly present the Playfair's Axiom as a restriction, as an axiom itself to be observed because this restriction becomes a natural property, a theorem. In order to present this very important property, we must first deal with the concept of parallelism using previous definitions. Considering our algebraic approach, two affine subspaces are considered to be \textsb{parallel}\index{parallelism} if the direction space of one is an improper subset of the other\footnote{This concept of parallelism admits that a hyperplane (a line or a plane, for example) is parallel to itself.}. In mathematical terms, we define two affine subspaces $\eamsd{S}{F}{a}{n}$ and $\eamsd{V}{F}{b}{r}$ of $\eamd{U}{F}{m}$ to be parallel, represented by $\gloref{affSubPar}\,$, if direction space $S_\cam{F}\subseteq V_\cam{F}$ or $V_\cam{F}\subseteq S_\cam{F}$. From this definition of parallelism, when $\eamsd{S}{F}{a}{n}$ and $\eamsd{V}{F}{b}{r}\,$ are indeed parallel, we obtain the following properties:
\begin{itemize}
	\setlength\itemsep{.1em}
	\item[i.] $n=r\iff S_\cam{F}=V_\cam{F}$;
	\item[ii.] $a\in \eamsd{V}{F}{b}{r}\,\wedge\, S_\cam{F}\subseteq V_\cam{F}\iff\eamsd{S}{F}{a}{n}\subseteq\eamsd{V}{F}{b}{r}$;
	\item[iii.] $a\notin \eamsd{V}{F}{b}{r}\,\vee\, b\notin \eamsd{S}{F}{a}{n} \iff \eamsd{S}{F}{a}{n}\cap\eamsd{V}{F}{b}{r}=\emptyset$.
\end{itemize}
Figure \ref{fg:paralelo} depicts a line and plane parallel to each other for the case of a fi\-ve-di\-men\-sio\-nal real affine space. Now, we say that subspaces $\eamsd{W}{F}{c}{r}$ and $\eamsd{S}{F}{a}{n}$ are \textsb{perpendicular}\index{perpendicularity}, represented by $\gloref{affSubPerp}\,$, if vectors of $W_\cam{F}$ and $V_\cam{F}$ do not have an incidence interrelationship, that is, if $W_\cam{F}\perp V_\cam{F}$. Geometrical representation of perpendicularity is more intuitive in the context of affine Euclidean spaces through the concept of angle, which we shall present later in this section.

{\footnotesize
\begin{proof}
For the first item of the previous properties, since $S_\cam{F}\subseteq V_\cam{F}$ or $V_\cam{F}\subseteq S_\cam{F}$, a direction space is a proper subset of the other only if they have different dimensions, otherwise they are equal. The inverse implication is trivial. For the second item, if $a\in \eamsd{V}{F}{b}{r}$ there is a vector $\vto{w}\in V_\cam{F}$ where $a=\vto{w}\oplus b$. Then, from axiom ii on page \pageref{ax:pointSpaces} and since $S_\cam{F}\subseteq V_\cam{F}$, we can write $\vto{x}\oplus a=(\vto{x}+\vto{w})\oplus b$, $\forall \vto{x}\in S_\cam{F}$, from which we conclude that $\eamsd{S}{F}{a}{n}\subseteq\eamsd{V}{F}{b}{r}$. Now, if $\eamsd{S}{F}{a}{n}\subseteq\eamsd{V}{F}{b}{r}$ is true we always have a vector $\vto{y}\in V_\cam{F}$ related to an arbitrary $\vto{x}\in S_\cam{F}$ through $\vto{x}\oplus a=\vto{y}\oplus b$ or $a=(\vto{y}-\vto{x})\oplus b$, from which we conclude that $a\in \eamsd{V}{F}{b}{r}$ and $S_\cam{F}\subseteq V_\cam{F}$. For  property iii, we prove only for $a\notin \eamsd{V}{F}{b}{r}$. In this case, there is no vector in $V_\cam{F}$ relating the pair of points $(b,a)$. Since $S_\cam{F}\subseteq V_\cam{F}$ or $V_\cam{F}\subseteq S_\cam{F}$, we can write for all $\vto{x}\in S_\cam{F}$ and $\vto{y}\in V_\cam{F}$ that
\begin{align*}
a &\neq \vto{y}\oplus b\\
\vto{x}\oplus a &\neq \vto{x}\oplus(\vto{y}\oplus b)\\
\vto{x}\oplus a &\neq (\vto{x}+\vto{y})\oplus b\,,
\end{align*}
from which we conclude that $\eamsd{S}{F}{a}{n}\cap\eamsd{V}{F}{b}{r}=\emptyset$. The inverse implication is verified considering $\eamsd{S}{F}{a}{n}\cap\eamsd{V}{F}{b}{r}=\emptyset$ valid and then, for all $\vto{x}\in S_\cam{F}$ and $\vto{y}\in V_\cam{F}$,
\begin{align*}
\vto{x}\oplus a &\neq \vto{y}\oplus b\\
-\vto{x}\oplus(\vto{x}\oplus a) &\neq -\vto{x}\oplus(\vto{y}\oplus b)\\
a &\neq (\vto{y}-\vto{x})\oplus b\,,
\end{align*}
from which we conclude that $a\notin \eamsd{V}{F}{b}{r}$.
\end{proof}}

\begin{figure}[!ht]
	\centering
	\begin{center}
		\scalebox{.72}{\input{partes/figs/paralelo.pstex_t}}
	\end{center}
	\titfigura{Line and plane parallel to each other.}\label{fg:paralelo}
\end{figure}

\begin{mteo}{Playfair's ``Axiom''}{playfair}
If $\eamd{U}{F}{n+1}$ is an affine space where $\eamsd{S}{F}{a}{n}$ is one of its hyperplanes and $k\notin \eamsd{S}{F}{a}{n}$ one of its points, there is a unique hyperplane $\eamsd{V}{F}{b}{n}\parallel\eamsd{S}{F}{a}{n}$ such that $k\in\eamsd{V}{F}{b}{n}$.
\end{mteo}

{\footnotesize
\begin{proof}
If $\eamsd{S}{F}{a}{n}$ exists, so does $\eamsd{S}{F}{b}{n}$, where $b\neq a$. Since these subspaces are obviously parallel, the existence of subspace $\eamsd{V}{F}{b}{n}=\eamsd{S}{F}{b}{n}$ is proved. Uniqueness is verified by the following rationale: since point $k\in\eamsd{V}{F}{b}{n}$ then $\eamsd{V}{F}{b}{n}=\eamsd{V}{F}{k}{n}$ according to \eqref{eq:igualAff}; and, supposing a third hyperplane $\eamsd{W}{F}{c}{n}\parallel \eamsd{S}{F}{a}{n}$ where $k\in\eamsd{W}{F}{c}{n}$, we also have $\eamsd{W}{F}{c}{n}=\eamsd{W}{F}{k}{n}$. Moreover, from the first property of parallelism, it is clear that $V_\cam{F}=S_\cam{F}=W_\cam{F}$ because the three hyperplanes have obviously the same dimension and then $\eamsd{W}{F}{k}{n}=\eamsd{V}{F}{k}{n}$. Since $\eamsd{V}{F}{b}{n}=\eamsd{V}{F}{k}{n}$ and $\eamsd{W}{F}{c}{n}=\eamsd{W}{F}{k}{n}$, then $\eamsd{V}{F}{b}{n}=\eamsd{W}{F}{c}{n}$. 
\end{proof}}

 Considering previous conditions, given a point $o\in\eamsd{S}{F}{a}{n}$ and a basis $B=\{\vto{v}_1,\cdots,\vto{v}_n\}$ of $S_\cam{F}$, where $n<m$, we call double $\gloref{coordSys}$ an \textsb{affine coordinate system}\index{affine!coordinate system} of $\eamsd{S}{F}{a}{n}$ because point space $\epu{S}_a$ can be obtained from $o$ and $B$ as follows:
\begin{equation}\label{eq:pointSpace}
\epu{S}_a=\{\vto{x}\oplus a:\forall \vto{x}\in S_\cam{F}\}= \{(\vto{x}+\vv{oa})\oplus o:\forall \vto{x}\in \spn{(B)}\}\,.
\end{equation}
In this context, point $o$ is called the \textsb{origin}\index{origin} of the coordinate system and the coordinates of an arbitrary vector $\vto{v}\in S_\cam{F}$ on $B$ are considered to be the coordinates of point $v:=\vto{v}\oplus o$ on $(o,B)$, that is, $\mav{v}{B}:=\mav{\vto{v}}{B}$. Thereby, it is trivial to obtain that the coordinates of point $\vto{u}\oplus v$ are the coordinates of vector $\vto{u}+\vto{v}$ or that $\mav{\vto{u}\oplus v}{B}=\mav{\vto{u}+\vto{v}}{B}$. Moreover, each line $((\mathcal{B}_i)_o)^1_\cam{F}$ defined by vector space $(B_i)_\cam{F}=\spn{(\{\vto{v}_i\})}$ is called an \textsb{axis}\index{axis} of $(o,B)$, which is depicted in figure \ref{fg:coordSystem}.
\begin{figure}[!ht]
	\centering
	\begin{center}
		\scalebox{.72}{\input{partes/figs/coordSystem.pstex_t}}
	\end{center}
	\titfigura{$n$-dimensional affine coordinate system $(o,B)$ and its axes.}\label{fg:coordSystem}
\end{figure}
On section \ref{sec:group}, we said that in a group action a biunivocal relationship between its definer set and its definer group is possible if an element of the set is fixed. For the case of affine spaces, this fixation is established by a coordinate system, which enables to define a point $v$ from a vector $\vto{v}$ through an origin $o$, as we have done. The association $(\eamsd{S}{F}{a}{n},o,B)$ of an affine subspace with a coordinate system can be structured like a vector space and then inherits the same classification, outlined in figure \ref{fig:esquemaEspacos}, from its direction space $S_\cam{F}$ if suitable expressions for metric, norm and inner product are defined. Thereby, for arbitrary points $v$ and $u$ of $\eamsd{S}{F}{a}{n}$, the triple $(\eamsd{S}{F}{a}{n},o,B)$ is called an affine metric space if $\fua{\varrho}{u,v}:=\fua{\varrho}{\vv{ou},\vv{ov}}$, an affine normed space if a norm $\|\vv{ov}\|$ is defined or an affine inner product space if there is the scalar $\vv{ov}\cdot\vv{ou}\,$. Since an affine subspace is itself an affine space, representation of triple $(\eamsd{S}{F}{a}{n},o,B)$ will be shortened by $\eamd{S}{F}{n}$, where point $a$ and the coordinate system are implicit. As we are dealing in this chapter with morphological matters, what are the geometric manifestations of metric, norm and inner product? We already said that metric refers to the concept of distance and thus scalar $\fua{\varrho}{u,v}$ informs the distance between arbitrary points $u$ and $v$ in the context of affine metric spaces. Norm however is not applicable to points, being geometrically represented by the size of a vector, which also depicts the concept of vector intensity: a ``bigger'' vector is considered to be a more ``intense'' vector. In the context of affine Banach spaces, where distance
\begin{equation}\label{eq:affDistance}
\fua{\varrho}{u,v} := \|\vv{ou} - \vv{ov}\|=\|\vv{vo} + \vv{ou}\|=\|\vv{vu}\|=\|\vv{uv}\|\,,
\end{equation}
norm enables distance measurement. Now, regarding the geometric effects of the inner product, we already said on section \ref{sec:espacoVet} that this product is closely related to vector incidence, being a scalar measure of this fundamental concept\footnote{See page \pageref{sec:incidence}.}. In the context of affine normed inner product spaces, the incidence interrelationship of non zero vectors $\vv{ov}$ and $\vv{ou}$ is expressed both by the \textsb{projection}\index{projection} $\fua{\zeta}{\vv{ov},\vv{ou}}$ of $\vv{ov}$ onto the line defined by the set $\spn{(\{\vv{ou}\})}$ and the projection $\fua{\zeta}{\vv{ou},\vv{ov}}$ of $\vv{ou}$ onto the line defined by $\spn{(\{\vv{ov}\})}$, where $\map{\zeta}{S_\cam{F}^2}{\cam{F}}$ whose function rule is
\begin{equation}\label{eq:projection}
\fua{\zeta}{\vv{ox},\vv{oy}}=\dfrac{\vv{ox}\cdot \vv{oy}}{\|\vv{oy}\|}
\end{equation}
and $S_\cam{F}$ defines an affine normed inner product space. Considering $\vv{ow}:=\fua{\zeta}{\vv{ox},\vv{oy}}\vv{oy}$, we can state that vector $\vv{ow}\in\spn{(\{\vv{oy}\})}$ where points $w$ and $y$ belong to the line defined by $\spn{(\{\vv{oy}\})}$. From the previous rule, if projection is known and not the inner product, we can write that $\vv{ox}\cdot \vv{oy}=\fua{\zeta}{\vv{ox},\vv{oy}}\|\vv{oy}\|$, which is usually and imprecisely regarded as the ``geometric definition'' of the inner product. It is important to note that if arguments $\vv{ox}$ and $\vv{oy}$ are orthogonal, then $\fua{\zeta}{\vv{ox},\vv{oy}}=0$.
 Moreover, in the context of affine Hilbert spaces, where $\|\vv{ox}\|^2=\vv{ox}\cdot \vv{ox}$, when $\vv{oy}=\alpha\vv{ox}$ then
\begin{equation}\label{eq:projColinear}
\fua{\zeta}{\vv{ox},\alpha\vv{ox}}=\dfrac{\overline{\alpha}(\vv{ox}\cdot \vv{ox})}{|\alpha|\|\vv{ox}\|}=\dfrac{\overline{\alpha}\|\vv{ox}\|}{|\alpha|}\,,\,\forall\alpha\in\{\cam{F}\setminus 0\}.
\end{equation}
For the case of affine Euclidean spaces, these equalities lead to
\begin{equation}\label{eq:projColinearEuclid}
\fua{\zeta}{\vv{ox},\alpha\vv{ox}}=\sgn{\alpha}\|\vv{ox}\|\,,\,\,\forall\,\alpha\in\{\real\setminus 0\}\,.
\end{equation}
Concerning projections on affine Hilbert spaces, there is a property, presented on the following theorem, that will be important for future geometric concepts.

\begin{mteo}{Modulus of Projection}{isoComp}
Given the arbitrary vectors $\vv{ox},\vv{oy}\in S_\cam{F}$, where $S_\cam{F}$ defines an affine Hilbert space $\eamd{S}{F}{n}$, the modulus of projection $\fua{\zeta}{\vv{ox},\vv{oy}}$ is never greater than the norm of the projected vector. In other words,
$|\fua{\zeta}{\vv{ox},\vv{oy}}|\leq \|\vv{ox}\|$.
\end{mteo}

{\footnotesize
\begin{proof}
From the Cauchy-Schwartz Inequality \eqref{eq:Cauchy-Schwartz} and the ``geometric definition'' of the inner product, the following development proves the theorem.
\begin{align*}
|\vv{ox}\cdot\vv{oy}|^2 &\leq (\vv{ox}\cdot\vv{ox})(\vv{oy}\cdot\vv{oy})\\
|\fua{\zeta}{\vv{ox},\vv{oy}}|^2\|\vv{oy}\|^2 &\leq \|\vv{ox}\|^2\|\vv{oy}\|^2\\
|\fua{\zeta}{\vv{ox},\vv{oy}}| &\leq \|\vv{ox}\|\,.
\end{align*}
\end{proof}}

Now, for an affine Euclidean space $\eamd{S}{R}{n}$, if $\theta$ is the smallest angle defined by vectors $\vv{ox},\vv{oy}\in S_\cam{R}$, the linear combination  $\vv{oz}:=\vartheta\vv{oy}/\|\vv{oy}\|$ and the scalar $\vartheta\in\real$ are called respectively the \textsb{vector and scalar geometric projections}\index{projection!vector}\index{projection!scalar} of $\vv{ox}$ onto $\vv{oy}$ when
\begin{equation}\label{eq:projgeo}
\vartheta:=\|\vv{ox}\|\cos\theta\,,
\end{equation}
which can be visualized in figure \ref{fg:geoProject}. In this context, the following corollary establishes a connection between geometric and algebraic projections. From this connection and rule \eqref{eq:projection}, the so called ``geometric definition'' of the Euclidean inner product results the classic expression
\begin{equation}\label{eq:geoProdInt}
\vv{ox}\cdot\vv{oy}=\|\vv{ox}\|\|\vv{oy}\|\cos\theta.
\end{equation}
From this equality and the scalar geometric projection $\vartheta$ of $\vv{ox}$ onto $\vv{oy}$, then
\begin{equation}
\fua{\zeta}{\vv{ox},\vv{oy}}=\dfrac{\vv{ox}\cdot \vv{oy}}{\|\vv{oy}\|} = \dfrac{\|\vv{ox}\|\|\vv{oy}\|\cos\theta }{\|\vv{oy}\|}= \|\vv{ox}\|\cos\theta=\vartheta\,.
\end{equation}

\begin{figure}[!ht]
	\centering
	\begin{center}
		\scalebox{.72}{\input{partes/figs/geoProject.pstex_t}}
	\end{center}
	\titfigura{Vector and scalar geometric projections.}\label{fg:geoProject}
\end{figure}

From the previous equality and definition \eqref{eq:projgeo}, if there is no geometric projection between $\vv{ox}$ and $\vv{oy}$, that is, $\vartheta=0$, we conclude that the angle $\theta=\pi(k+1/2)$, $k=0,1,..\,$. Thereby, if line $\eamsd{W}{\real}{b}{1}$ and plane $\eamsd{V}{\real}{a}{2}$, subspaces of affine Euclidean space $\eamd{S}{\real}{5}$, are perpendicular, we can represent them as depicted on figure \ref{fg:perpend}.
\begin{figure}[!ht]
	\centering
	\begin{center}
		\scalebox{.72}{\input{partes/figs/perpend.pstex_t}}
	\end{center}
	\titfigura{Line and plane perpendicular to each other.}\label{fg:perpend}
\end{figure}


\section{Affinities}
The subject of Geometry is to study not only morphological sets and their elements, as we have done on the previous section, but also the correlation of geometrical objects, that is, functions whose domain and image are morphological sets. Thereby, in the context of Affine Geometry, this section deals exclusively with affinities, here understood as ``geometrical'' functions that correlate points, defined by linear vector functions on Hilbert spaces. In more precise terms, given a Hilbert affine space $\eamd{U}{F}{m}$, a bijective mapping $\map{\mathit{g}}{\eamd{U}{F}{m}}{\eamd{U}{F}{m}}$ and a vector space $\evl{\cam{F}}{U}{U}$, we call  $\map{\mathit{f}}{\eamd{U}{F}{m}}{\eamd{U}{F}{m}}$ an \textsb{affine transformation}\index{affine!transformation} and its function an \textsb{affinity}\index{affinity} when there is a unique linear bijection $\vtf{f}\in\evl{\cam{F}}{U}{U}$ where
\begin{equation}\label{eq:affty}
\fua{\mathit{f}}{\vto{x}\oplus a}=\fua{\vtf{f}}{\vto{x}}\oplus\fua{\mathit{g}}{a}\,,\,\,\forall\,a\in\eamd{U}{F}{m}\,,\,\vto{x}\in U_\cam{F}\,.
\end{equation}
It is straightforward to note that since $\vtf{f}$ and $\mathit{g}$ are bijections, so results affinity $\mathit{f}$. Among other properties, it is possible to obtain that every affinity preserves parallelism and dimensionality; in other words, parallel subspaces on the  domain remain parallel on the image and $n$-dimensional subspaces on the domain, $n\leq m$, remain $n$-dimensional on the image: for example, lines are mapped to lines, planes are mapped to planes and so on. From corollary \ref{cor:soTens} and equalities \eqref{eq:Lin2secTens}, we know that operator $\vtf{f}$ is the representative function $\tnr{\Upsilon}_{\tnr{A}}$ of a tensor $\tnr{A}=\vto{a}_1\otimes\vto{a}_2\in\ete{\cam{F}}{U^2}$ in such a way that the previous definition can be rewritten as
\begin{equation}
\fua{\mathit{f}}{\vto{x}\oplus a}=\fua{\tnr{\Upsilon}_{\tnr{A}}}{\vto{x}}\oplus\fua{\mathit{g}}{a}=(\vto{x}\cdot\vto{a}_1)\vto{a}_2\oplus\fua{\mathit{g}}{a}\,,\,\,\forall\,a\in\eamd{U}{F}{m}\,,\,\vto{x}\in U_\cam{F}\,,
\end{equation}
when $\tnr{A}$ is called an \textsb{affinity tensor}\index{tensor!afinity}\index{affinity!tensor}.


{\footnotesize
\begin{proof}
Let's prove that an affinity preserves parallelism and dimension of subspaces. Considering $\eamsd{S}{F}{a}{n}$ and $\eamsd{V}{F}{b}{r}$ parallel subspaces of $\eamd{U}{F}{m}$, where $S_\cam{F}\subseteq V_\cam{F}$, we can say that if $\{\vto{u}_1,\cdots,\vto{u}_n,\cdots,\vto{u}_r,\cdots,\vto{u}_m\}$ is a basis of $U_\cam{F}$, so are $\{\vto{u}_1,\cdots,\vto{u}_n\}$ and $\{\vto{u}_1,\cdots,\vto{u}_n,\cdots,\vto{u}_r\}$ bases of $S_\cam{F}$ and $V_\cam{F}$ respectively. From definition \eqref{eq:affty}, function $\vtf{f}$ is the sole responsible for preserving parallelism or not. Thereby, since $\vtf{f}$ is a bijective linear unary operator, we already know that $\{\fua{\vtf{f}}{\vto{u}_1},\cdots,\fua{\vtf{f}}{\vto{u}_n}\}$ and $\{\fua{\vtf{f}}{\vto{u}_1},\cdots,\fua{\vtf{f}}{\vto{u}_n},\cdots,\fua{\vtf{f}}{\vto{u}_r}\}$ are bases of $S_\cam{F}$ and $V_\cam{F}$ (see p. \pageref{pg:bijecOpe}) and then $S_\cam{F}\subseteq V_\cam{F}$ is preserved. From these same arguments, it is straightforward to conclude that $\vtf{f}$ does not change vector space dimensions.
\end{proof}}


Considering previous conditions, an affinity $\mathit{f}$ is said to be centered at point $a$, represented by $\mathit{f}_a$, when $\mathit{g}$ is the identity function, that is, when point $\fua{\gloref{ctrAff}}{\vto{x}\oplus a}=\fua{\vtf{f}}{\vto{x}}\oplus a$. From this equality, when $\fua{\vtf{f}}{\vto{x}}=\vto{u}+\vto{x}$, we have $\fua{\mathit{f}_a}{\vto{x}\oplus a}=\vto{u}\oplus(\vto{x}\oplus a)$ and then affinity $\mathit{f}_a$ is called a \textsb{translation}\index{translation}, represented by $\gloref{transLat}$, described by the rule
\begin{equation}\label{eq:translt}
\fua{\mathit{t}_\vto{u}}{x}=\vto{u}\oplus x
\end{equation}
and depicted in figure \ref{fg:translat}, where $\mathit{t}_\vto{u}$ preserves the shape of the subspace $\eamd{S}{F}{n}$.
\begin{figure}[!ht]
\centering
\begin{center}
\scalebox{.72}{\input{partes/figs/translat.pstex_t}}
\end{center}
\titfigura{Hyperplane $\eamd{S}{F}{n}$ is mapped to $\eamd{V}{F}{n}$ by translation $\mathit{t}_\vto{u}$.}\label{fg:translat}
\end{figure}
Given an arbitrary vector $\vto{v}\in U_\cam{F}$, from equality $\fua{\mathit{t}_\vto{v}\circ\mathit{t}_\vto{u}}{x}=(\vto{v}+\vto{u})\oplus x$,
it is straightforward to conclude that $\mathit{t}_\vto{v}\circ\mathit{t}_\vto{u}=\mathit{t}_{\vto{v}+\vto{u}}$. Moreover, if $\vto{v}=-\vto{u}$ we can affirm that $\mathit{t}_{-\vto{u}}=\mathit{t}_\vto{u}^{-1}$ and translation $\mathit{t}_\vto{0}$ is the identity function. Previous equality $\mathit{t}_\vto{v}\circ\mathit{t}_\vto{u}=\mathit{t}_{\vto{v}+\vto{u}}$ also allows us to conclude that the set $\mathit{T}$ of all translations on $\eamd{U}{F}{m}$ defines an abelian group $(\mathit{T},\circ)$ because the composition of translations clearly observes the axioms of
\begin{itemize}
\setlength\itemsep{.1em}
\item[i.] associativity, where $\mathit{t}_\vto{u}\circ\lpa \mathit{t}_\vto{v} \circ \mathit{t}_\vto{w} \rpa =
\lpa\mathit{t}_\vto{u}\circ \mathit{t}_\vto{v} \rpa\circ \mathit{t}_\vto{w} \, , \forall \, \vto{u},\vto{v},\vto{w}\in U_\cam{F}$,
\item[ii.] identity element, where $\mathit{t}_\vto{u}\circ\mathit{t}_\vto{0}=\mathit{t}_\vto{0}\circ\mathit{t}_\vto{u}=\mathit{t}_\vto{u}\,,\forall \, \vto{u} \in U_\cam{F}$,
\item[iii.] inverse element, where
$\mathit{t}_\vto{u} \circ  \mathit{t}_{-\vto{u}}= \mathit{t}_{-\vto{u}} \circ \mathit{t}_{\vto{u}} = \mathit{t}_{\vto{0}}$, $\forall \, \vto{u} \in \{U_\cam{F}\setminus\vto{0}\}$, and of
\item[iv.] commutativity, where $\mathit{t}_\vto{u}\circ \mathit{t}_\vto{v}= \mathit{t}_\vto{v}\circ \mathit{t}_\vto{u}\,,\forall \, \vto{u},\vto{v} \in U_\cam{F}$.
\end{itemize}


Still considering previous conditions, the affinity in $\map{\mathit{h}}{\eamd{U}{F}{m}}{\eamd{U}{F}{m}}$ is called a \textsb{dilation}\index{dilation} if bijection $\vtf{h}\in\evl{\cam{F}}{U}{U}$ in
\begin{equation}\label{eq:dilat}
\fua{\mathit{h}}{\vto{x}\oplus a}=\fua{\vtf{h}}{\vto{x}}\oplus\fua{\mathit{g}}{a}\,,\,\,\forall\,a\in\eamd{U}{F}{m}\,,\,\vto{x}\in U_\cam{F}\,,
\end{equation}
is a positive-definite Hermitian operator usually called \textsb{stretch operator}\index{operator!stretch}, which represents, according to inequality \eqref{eq:nonnegTensor}, a positive-definite affinity tensor $\tnr{H}$, called \textsb{stretch tensor}\index{tensor!stretch}. Since the eigenvalues $\lambda_1,\cdots,\lambda_m$ of stretch operator $\vtf{h}$, or of stretch tensor $\tnr{H}$, are positive real scalars\footnote{See section \ref{sec:autoPares}.}, we call $\lambda_i$ a \textsb{stretch coefficient}\index{stretch coefficient}. If coefficient $\lambda_i>1$ or $\lambda_i\leqslant1$, it is called an \textsb{expansion}\index{expansion} or a \textsb{contraction}\index{contraction} respectively and in the case of proportional stretch, where $\lambda_1=\cdots=\lambda_m$, function $\vtf{h}$ (or tensor $\tnr{H}$) is equally called an expansion\index{operator!expansion} or contraction operator(tensor)\index{operator!contraction}. It is important to note that, from corollary \ref{cor:soSymTens}, in the specific context of Euclidean spaces, where $\vtf{h}^\dagger=\vtf{h}^\text{T}$, the stretch tensor $\tnr{H}$ is symmetric. As any other affinity, a dilation $\mathit{h}$ can be centered at point $a$, represented by $\gloref{dilat}$, where $\fua{\mathit{h}_a}{\vto{u}\oplus a}=\fua{\vtf{h}}{\vto{u}}\oplus a$. Figure \ref{fg:dilation} shows different types of dilations.
\begin{figure}[!ht]
\centering
\begin{center}
\scalebox{.72}{\input{partes/figs/dilation.pstex_t}}
\end{center}
\titfigura{Affinities $\mathit{h}$, $\mathit{h}_a$ and $\mathit{s}$ are ordinary, centered and proportional dilations.}\label{fg:dilation}
\end{figure}
Given two arbitrary dilations $\mathit{h}_1$ and $\mathit{h}_2$, equalities
\begin{equation*}
\fua{\mathit{h}_1\circ\mathit{h}_2}{\vto{x}\oplus a}=\fua{\mathit{h}_1}{\fua{\vtf{h}_2}{\vto{x}}\oplus\fua{\mathit{g}_2}{a}}=\fua{\vtf{h}_1\circ\vtf{h}_2}{\vto{x}}\oplus\fua{\mathit{g}_1\circ\mathit{g}_2}{a}
\end{equation*}
prove that $\mathit{h}_1\circ\mathit{h}_2$ is also a dilation and that $\fua{\mathit{h}^{-1}_1}{\vto{x}\oplus a}=\fua{\vtf{h}^{-1}_1}{\vto{x}}\oplus\fua{\mathit{g}^{-1}_1}{a}$. Considering the identity dilation $\mathit{i}$, similarly to the case of translations, it is straightforward to verify that dilations also obey the axioms of associativity, identity element, inverse element and commutativity on the operation of composition. Therefore, the set $H$ of all dilations on $\eamd{U}{F}{m}$ defines the abelian group $(H,\circ)$.

\begin{mteo}{Decompositions of Dilation}{dilaDecomp}
Given a dilation $\mathit{h}$ and a translation $\mathit{t}_\vto{x}$ on an affine space $\eamd{U}{F}{m}$, if vector $\fua{\kappa}{a}:=\vv{\fua{\mathit{g}}{a}a}$ then
\begin{equation*}
\mathit{h} = \mathit{t}_{\fua{\kappa}{a}}\circ\mathit{h}_a=\mathit{h}_{\fua{\mathit{g}}{a}}\circ\mathit{t}_{\fua{\kappa}{a}}\,,\,\forall a\in\eamd{U}{F}{m}\,.
\end{equation*}
\end{mteo}

{\footnotesize
\begin{proof}
Development
\begin{equation*}
\fua{\mathit{h}}{\vto{x}\oplus a} = \fua{\vtf{h}}{\vto{x}}\oplus (\fua{\kappa}{a}\oplus a) = \fua{\kappa}{a}\oplus (\fua{\vtf{h}}{\vto{x}}\oplus a) = \fua{\mathit{t}_{\fua{\kappa}{a}}\circ\mathit{h}_a}{\vto{x}\oplus a}
\end{equation*}
proves the first decomposition and
\begin{equation*}
\fua{\mathit{h}}{\vto{x}\oplus a} = \fua{\vtf{h}}{\vto{x}}\oplus \fua{\mathit{g}}{a} = \fua{\mathit{h}_{\fua{\mathit{g}}{a}}}{\vto{x}\oplus \fua{\mathit{g}}{a}} = \fua{\mathit{h}_{\fua{\mathit{g}}{a}}}{(\fua{\kappa}{a}+\vto{x})\oplus a} = \fua{\mathit{h}_{\fua{\mathit{g}}{a}}\circ\mathit{t}_{\fua{\kappa}{a}}}{\vto{x}\oplus a}
\end{equation*}
proves the second.
\end{proof}}


Still considering previous conditions, an affinity $\mathit{r}$ is called \textsb{isometric}\index{affinity!isometric} when its correspondent linear operator $\vtf{r}$ is isometric, that is, $\vtf{r}\in\grc{I}{\cam{F}}{U}$. In the context of Euclidean spaces, we already know that an isometric operator is also an orthogonal operator\footnote{See p. \pageref{prg:IsometOrth}.}, that is, the isometric group $\grc{I}{\real}{U}$ equals the orthogonal group $\grc{O}{\real}{U}$, whose elements preserve inner product and have determinant $\pm 1$. The centered isometric affinity $\mathit{r}_a$, described by $\fua{\mathit{r}_a}{\vto{x}\oplus a}=\fua{\vtf{r}}{\vto{x}}\oplus a$, is called a \textsb{rotation}\index{rotation} if isometric operator $\vtf{r}$ is proper orthogonal, or $\vtf{r}\in\grc{O^+}{\real}{U}$; otherwise, it is called a \textsb{rotoreflection}\index{rotoreflection}. The affinity tensor $\tnr{R}$ of a rotation is called a \textsb{rotation tensor}\index{rotation!tensor} and, similarly, tensor $\overline{\tnr{R}}$ of a rotoreflection is called a \textsb{rotoreflection tensor}\index{rotoreflection!tensor}. Now, recalling the concept of unimodular tensors presented in section \ref{sec:TensSpac}, let $(U_\real,\tnr{A}_B)$ be an $m$-dimensional Euclidean space oriented by the basis $B=\{\vun{u}_1,\cdots,\vun{u}_m\}$ where $\tnr{A}_B\in\etng{m}{\real}{U^m}$ is a unimodular tensor. Considering $\mathit{r}_a$ a rotation and $\overline{\mathit{r}}_a$ a rotoreflection, since every pair of Euclidean orthonormal basis can be related through an orthogonal operator\footnote{See p. \pageref{prg:orthBases}.}, we conclude that rotation  $C=\{\fua{\vtf{r}}{\vun{u}_1},\cdots,\fua{\vtf{r}}{\vun{u}_m}\}$ and rotoreflection $\overline{C}=\{\fua{\overline{\vtf{r}}}{\vun{u}_1},\cdots,\fua{\overline{\vtf{r}}}{\vun{u}_m}\}$ of basis $B$ are positively and negatively oriented bases because $\fua{\tnr{A}_C}{\vun{u}_1,\cdots,\vun{u}_m}=-\fua{\tnr{A}_{\overline{C}}}{\vun{u}_1,\cdots,\vun{u}_m}=1$ from definition \eqref{eq:basisAltern}. In order to geometrically represent in a simple way basis rotation and rotoreflection, we specify the affine Euclidean space in question to be two-dimensional, that is, $m=2$, and rotoreflection $\overline{\vtf{r}}=-\vtf{r}$. Thereby, for the basis vectors $\vun{u}_1$ and $\vun{u}_2$ the smallest angle $\theta$ between them leads to equalities $0=\vun{u}_1\cdot\vun{u}_2=\|\vun{u}_1\|\|\vun{u}_2\|\cos\theta=\cos\theta$, and since orthogonal functions preserve inner product, then $\theta$ is also the smallest angle between $\fua{\vtf{r}}{\vun{u}_1}$ and $\fua{\vtf{r}}{\vun{u}_2}$, or between $\fua{\overline{\vtf{r}}}{\vun{u}_1}$ and $\fua{\overline{\vtf{r}}}{\vun{u}_2}$, because the following equalities are all valid: $\vun{u}_1\cdot\vun{u}_2=\fua{\vtf{r}}{\vun{u}_1}\cdot\fua{\vtf{r}}{\vun{u}_2}=\fua{\overline{\vtf{r}}}{\vun{u}_1}\cdot\fua{\overline{\vtf{r}}}{\vun{u}_2}=0$. From these angle and size preserving features, we can represent an anticlockwise acute angle $\phi$ rotation and rotoreflection of basis $B$ according to figure \ref{fg:rotoReflex}.
\begin{figure}[!ht]
\centering
\begin{center}
\scalebox{.72}{\input{partes/figs/rotoReflex.pstex_t}}
\end{center}
\titfigura{Two-dimensional rotoreflexion and rotation.}\label{fg:rotoReflex}
\end{figure}
We can confirm that the function on the right of this figure is indeed a rotation and on the left a rotoreflection by proving that the rotated and reflected bases are positively and negatively oriented respectively in relation to $(U_\real,\tnr{A}_B)$, that is, by calculating $\fua{\tnr{A}_C}{\vun{u}_1,\vun{u}_2}$ and $\fua{\tnr{A}_{\overline{C}}}{\vun{u}_1,\vun{u}_2}$. From definition \eqref{eq:basisAltern} and equality \eqref{eq:geoProdInt}, development
\begin{align*}
\fua{\tnr{A}_{C}}{\vun{u}_1,\vun{u}_2}&=\sum_{i=1}^2\sum_{j=1}^2\fua{\vtf{f}^\con{C}_{i}}{\vun{u}_1}\fua{\vtf{f}^\con{C}_{j}}{\vun{u}_2}\epsilon_{ij}\\
&=(\vun{u}_1\cdot\fua{\vtf{r}}{\vun{u}_1})(\vun{u}_2\cdot\fua{\vtf{r}}{\vun{u}_2})-(\vun{u}_1\cdot\fua{\vtf{r}}{\vun{u}_2})(\vun{u}_2\cdot\fua{\vtf{r}}{\vun{u}_1})\\
&=(cos\phi cos\phi) - (-\sin\phi\sin\phi)=1
\end{align*}
proves that $C$ is positively oriented and, similarly, $\overline{C}$ results negatively oriented.


Considering  an oriented Euclidean space $(U_\real,\tnr{A}_B)$ where $B=\{\vun{u}_1,\cdots,\vun{u}_m\}$, the \textsb{simple bivector} $z$ defined by vectors $\vto{x},\vto{y}\in U_\real$ is here understood as the mathematical object, endowed with magnitude and orientation, used to express rotational physical quantities such as angular momentum and torque. In geometrical terms, the magnitude of this simple bivector $z$ is defined to be the area of the parallelogram constructed from line segments $\overline{a\,\vto{x}\oplus a}$ and $\overline{a\,\vto{y}\oplus a}$ while its orientation is related to $\tnr{A}_B$ and to rotating $\overline{a\,\vto{x}\oplus a}$ to $\overline{a\,\vto{y}\oplus a}$, which is the inverse of rotating $\overline{a\,\vto{y}\oplus a}$ to $\overline{a\,\vto{x}\oplus a}$ for the case of the simple bivector $-z$. It is important to say that this new entity is here loosely defined because it is a particular case of a more general concept called multivector, which is out of the scope of this elementary study. Nevertheless, to our good fortune, this simple bivector can be represented as a vector if $\dim(U_\real)=3$, when it is called an \textsb{axial vector}\index{vector!axial}. In this context, considering an axial vector $\vto{z}$ defined by the simple bivector $z$, where $\{\vto{x},\vto{y}\}$ is linearly independent, line $(\{\vto{z}\}_a)_\real^1$ is defined to be perpendicular to the plane $(\{\vto{x},\vto{y}\}_a)_\real^2$ and vector $\vv{a\,\vto{z}\oplus a}$ must be oriented in such a way that basis $\{\vto{z},\vto{x},\vto{y}\}$ is positively oriented, that is, $\fua{\tnr{A}_B}{\vto{z},\vto{x},\vto{y}}>0$.
\begin{figure}[!ht]
\centering
\begin{center}
\scalebox{.72}{\input{partes/figs/axial.pstex_t}}
\end{center}
\titfigura{Axial vector $\vto{z}$ defined by $\vto{x}$, $\vto{y}$ and oriented Euclidean space $(U_\real,\tnr{A}_B)$, where basis $B=\{\vun{u}_1,\vun{u}_2,\vun{u}_3\}$.}\label{fg:axial}
\end{figure}
There is a classic mnemonic rule for obtaining the direction of $\vv{a\,\vto{z}\oplus a}$ that works as follows: considering my right hand, I must point my index and middle fingers to the same directions of $\vv{o\,\vun{u}_1\oplus o}$ and $\vv{o\,\vun{u}_2\oplus o}$ respectively, and then if my thumb points to the same direction of $\vv{o\,\vun{u}_3\oplus o}$, then space $(U_\real,\tnr{A}_B)$ is oriented according to the \textsb{right-hand rule}\index{right-hand rule}; on the other hand literally, space $(U_\real,\tnr{A}_B)$ is oriented according to the \textsb{left-hand rule}\index{left-hand rule}. Once a hand is found, if I point my index and middle fingers to the same directions of $\vv{a\,\vto{x}\oplus a}$ and $\vv{a\,\vto{y}\oplus a}$ respectively, then the direction of my thumb defines the direction of $\vv{a\,\vto{z}\oplus a}$. For practical purposes, a three-dimensional Euclidean affine space is usually oriented according to the right-hand rule. An axial vector is always the result of a \textsb{cross product}\index{product!cross} of two vectors, that is, $\vto{z}=\gloref{crossProd}$ in the present case, where
\begin{equation}\label{eq:crossProduct}
\vto{x}\times\vto{y} := \tnr{A}_B\odot_2 (\vto{x}\otimes\vto{y})\,.
\end{equation}
For arbitrary vectors $\vto{u},\vto{v},\vto{w}\in U_\real$ and scalars $\alpha,\beta,\theta\in\real$, where $\theta$ is the smallest angle defined by vectors $\vv{ou},\vv{ov}$, the cross product has the following properties.
\begin{itemize}
	\setlength\itemsep{.1em}
	\item[i.] Anticommutativity: $\vto{u}\times\vto{v}=-(\vto{v}\times\vto{u})$;
	\item[ii.] Self cross product: $\vto{u}\times\vto{u}=\vto{0}$;
	\item[iii.] Distributivity over addition: $\vto{u}\times(\vto{v}+\vto{w})=(\vto{u}\times\vto{v})+(\vto{u}\times\vto{v})$;
	\item[iv.] Scalar multiplication: $\alpha\beta(\vto{u}\times\vto{v})=(\alpha\vto{u})\times(\beta\vto{v})=(\beta\vto{u})\times(\alpha\vto{v})$;
	\item[v.] Positive orientation: $\tnr{A}_B(\vto{u}\times\vto{v},\vto{u},\vto{v})>0$;
	\item[vi.] Orthogonality: $(\vto{u}\times\vto{v})\cdot\vto{u}=(\vto{u}\times\vto{v})\cdot\vto{v}=0$;
	\item[vii.] Triple cross product: $\vto{u}\times(\vto{v}\times\vto{w})=(\vto{u}\cdot\vto{w})\vto{v}-(\vto{u}\cdot\vto{v})\vto{w}$;
	\item[viii.] Area of parallelogram: $\|\vto{u}\times\vto{v}\|= \|\vto{u} \|\|\vto{v} \|\sin\theta$.
\end{itemize}

{\footnotesize
\begin{proof}
Given an arbitrary vector $\vto{x}\in U_\real$ and $\tnr{A}_B=\vto{a}\otimes\vto{b}\otimes\vto{c}$, item i is verified by the following equalities:
\begin{equation*}
\vto{x}\cdot(\vto{u}\times\vto{v})= (\vto{b}\cdot\vto{u})(\vto{c}\cdot\vto{v})(\vto{x}\cdot\vto{a})= \fua{\tnr{A}_B}{\vto{x},\vto{u},\vto{v}}=-\fua{\tnr{A}_B}{\vto{x},\vto{v},\vto{u}}=-\vto{x}\cdot(\vto{v}\times\vto{u})\,.
\end{equation*}
Item ii is a straightforward corollary of item i. We prove item iii similarly:
\begin{equation*}
\vto{x}\cdot[\vto{u}\times(\vto{v}+\vto{w})]= \fua{\tnr{A}_B}{\vto{x},\vto{u},\vto{v}+\vto{w}}=\fua{\tnr{A}_B}{\vto{x},\vto{v},\vto{u}}+\fua{\tnr{A}_B}{\vto{x},\vto{v},\vto{w}}=\vto{x}\cdot[(\vto{v}\times\vto{u})+(\vto{v}\times\vto{w})]\,.
\end{equation*}
By this same procedure, proof of property iv is trivial. Now, property v is verified by expression
\begin{equation*}
\tnr{A}_B(\vto{u}\times\vto{v},\vto{u},\vto{v})=\tnr{A}_B((\vto{b}\cdot\vto{u})(\vto{c}\cdot\vto{v})\vto{a},\vto{u},\vto{v})=(\vto{a}\cdot\vto{a})^2(\vto{b}\cdot\vto{u})^2(\vto{c}\cdot\vto{v})^2 > 0
\end{equation*}
and, since $\tnr{A}_B$ is antisymmetric, equalities $(\vto{u}\times\vto{v})\cdot\vto{u}=\tnr{A}_B(\vto{u},\vto{u},\vto{v})=0$ prove item vi. Considering an arbitrary orthonormal basis $X=\{\vun{x}_1,\vun{x}_2,\vun{x}_3\}$, equality \eqref{eq:coordAltern} and identity \eqref{eq:LeviKronecker}, development
\begin{align*}
\vto{u}\times(\vto{v}\times\vto{w})&=\tnr{A}_B\odot_2 [\vto{u}\otimes\tnr{A}_B\odot_2 (\vto{v}\otimes\vto{w})]\\
&=\sum_{i,j,k=1}^{m}\epsilon_{ijk}\vun{x}_i^*\otimes\vun{x}_j^*\otimes\vun{x}_k^*\odot_2 [\sum_{j=1}^{m}\fua{\vto{f}^X_j}{\vto{u}}\vun{x}_j\otimes\sum_{k,r,s=1}^{m}\epsilon_{krs}\fua{\vto{f}^X_r}{\vto{v}}\fua{\vto{f}^X_s}{\vto{w}}\vun{x}_k]\\
&=\sum_{i,j,k=1}^{m}\sum_{r,s=1}^{m}-\epsilon_{kji}\epsilon_{krs}\fua{\vto{f}^X_j}{\vto{u}}\fua{\vto{f}^X_r}{\vto{v}}\fua{\vto{f}^X_s}{\vto{w}}\vun{x}_i\\
&=\sum_{i,j,k,r,s=1}^{m}(\delta_{js}\delta_{ir}-\delta_{jr}\delta_{is})\fua{\vto{f}^X_j}{\vto{u}}\fua{\vto{f}^X_r}{\vto{v}}\fua{\vto{f}^X_s}{\vto{w}}\vun{x}_i\\
&=\sum_{i,j,k,r,s=1}^{m}\fua{\vto{f}^X_s}{\vto{u}}\fua{\vto{f}^X_i}{\vto{v}}\fua{\vto{f}^X_s}{\vto{w}}\vun{x}_i-\fua{\vto{f}^X_r}{\vto{u}}\fua{\vto{f}^X_r}{\vto{v}}\fua{\vto{f}^X_i}{\vto{w}}\vun{x}_i\\
&=(\vto{u}\cdot\vto{w})\vto{v}-(\vto{u}\cdot\vto{v})\vto{w}
\end{align*}
proves property vii. Now, in order to prove the last item, equality $\vto{u}\cdot\tnr{A}_B\odot_2\vto{v}\otimes\vto{w}=\tnr{A}_B(\vto{u},\vto{v},\vto{w})$, property vii and equality \eqref{eq:geoProdInt} are required. Then,
\begin{align*}
\|\vto{u}\times\vto{v}\|^2 &= (\vto{u}\times\vto{v})\cdot(\vto{u}\times\vto{v})\\
&= \tnr{A}_B(\vto{u}\times\vto{v},\vto{u},\vto{v})\\
&= -\tnr{A}_B(\vto{v},\vto{u},\vto{u}\times\vto{v})\\
&= -\vto{v}\cdot(\vto{u}\times(\vto{u}\times\vto{v}))\\
&= \vto{v}\cdot[(\vto{u}\cdot\vto{u})\vto{v}-(\vto{u}\cdot\vto{v})\vto{u}]\\
&= \|\vto{u}\|^2\|\vto{v}\|^2-(\vto{u}\cdot\vto{v})^2=\|\vto{u}\|^2\|\vto{v}\|^2-\|\vto{u}\|^2\|\vto{v}\|^2\cos^2\theta=\|\vto{u}\|^2\|\vto{v}\|^2\sin^2\theta\,.
\end{align*}
\end{proof}}

Considering the last property above, the geometrical definition of inner product \eqref{eq:geoProdInt} and arbitrary vectors $\vto{x},\vto{y},\vto{z}\in U$, if $\theta_1$ is the smallest angle defined by $\overline{a\vto{x}\oplus a}$ and $\overline{a(\vto{y}\times\vto{z})\oplus a}$, and angle $\theta_2$ defined by  $\overline{a\vto{y}\oplus a}$ and $\overline{a\vto{z}\oplus a}$, scalar
\begin{equation}
\fua{\tnr{A}_B}{\vto{x},\vto{y},\vto{z}}=\vto{x}\cdot(\vto{y}\times\vto{z})=\|\vto{x}\|(\|\vto{y} \|\|\vto{z} \|\sin\theta_2)\cos\theta_1=\underbrace{\|\vto{y} \|\|\vto{z} \|\sin\theta_2}_{\alpha} \underbrace{\|\vto{x}\|\cos\theta_1}_{\beta}\,,
\end{equation}
where $\alpha$ is the area of parallelogram defined by $\overline{a\vto{y}\oplus a}$ and $\overline{a\vto{z}\oplus a}$, as we already know. Moreover, it is possible to conclude that from vector $\vto{h}=\beta(\vto{y}\times\vto{z})/\|\vto{y}\times\vto{z}\|$, the line segments $\overline{a\vto{x}\oplus a}$, $\overline{a\vto{y}\oplus a}$ and $\overline{a\vto{z}\oplus a}$ define a parallelepiped of height $\overline{a\vto{h}\oplus a}$ and volume $\alpha\beta$, represented in figure \ref{fg:paralelogramo}.
\begin{figure}[!ht]
\centering
\begin{center}
\scalebox{.72}{\input{partes/figs/paralelogramo.pstex_t}}
\end{center}
\titfigura{Parallelogram of volume $\alpha\beta$ defined by $\vto{x},\vto{y},\vto{z}$.}\label{fg:paralelogramo}
\end{figure}
Considering definition \eqref{eq:detTensorSeg} in the present context of a three-di\-men\-sional Euclidean space and $\{\vto{x},\vto{y},\vto{z}\}\subset U$ a linearly independent set, for a second order tensor $\tnr{T}\in\ete{\cam{F}}{U^2}$ equalities
\begin{equation}
\hdet{\tnr{T}}=\dfrac{\fua{\tnr{A}_B}{\fua{\tnr{T}_1^*}{\vto{x}^*},{\fua{\tnr{T}_1^*}{\vto{y}^*}},{\fua{\tnr{T}_1^*}{\vto{z}^*}}}}{\fua{\tnr{A}_B}{\vto{x},\vto{y},\vto{z}}}=\dfrac{\fua{\tnr{T}_1^*}{\vto{x}^*}\cdot[\,\fua{\tnr{T}_1^*}{\vto{y}^*}\times\fua{\tnr{T}_1^*}{\vto{z}^*}\,]}{\vto{x}\cdot(\vto{y}\times\vto{z})}
\end{equation}
show that $\hdet{\tnr{T}}$ measures the change of volume of a parallelogram ``changed'' by $\tnr{T}$.

