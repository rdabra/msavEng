
\chapter{Basic Tensor Calculus}

Analysis is the branch of Mathematics that studies the behaviour of functions by using the idea of limit, considered fundamental, and its subordinate concepts. When such concepts involve  differentiation and integration, as well as their related definitions, Analysis is commonly known as Calculus. Intuitively speaking, up to this point, we have dealt with functions in a kind of ``static'' approach: on previous chapters, the sole definition of a mapping, with its associated function explicitly described or not, was sufficient for developing the concepts presented. Now, we are interested also in a ``dynamic'' approach for functions, where it is important not only to specify them but also to describe how their values evolve on their images, or how they behave, through derivatives and integrals. In this context, considering the study of tensors started on chapter \ref{cp:tensor}, the following pages introduce the Calculus of tensor functions, usually called Tensor Calculus. Since a tensor function is the most general type of function presented so far, we do not deviate from our main objective of being as abstract as possible on behalf of mathematical beauty.  



\section{Differentiation}

In the context of Banach tensor spaces, which are normed and complete, to differentiate a tensor function $\tnr{\psi}$ is to obtain its derivative, here understood as a linear tensor function that measures qualitatively and quantitatively the sensitivity of $\tnr{\psi}$ on each element of the domain. Intuitively speaking, sensitivity refers to the intrinsic ``susceptibility'' of a function, or how it ``reacts'' in terms of its values, to an \emph{eventual change} on its argument. However, not all tensor functions can be differentiable: the following mathematical conditions and definitions will describe which functions can. Considering $\ete{\cam{F}}{U^{\times m}}$ and $\ete{\cam{F}}{V^{\times s}}$ Banach tensor spaces, both defined by the same field $\cam{F}$, let $\pmb{\mathcal{U}}\subset\ete{\cam{F}}{U^{\times m}}$ be an open set and element  $\tnr{X}_0\in\pmb{\mathcal{U}}$ an arbitrary tensor. The functions in $\map{\tnr{\psi}}{\pmb{\mathcal{U}}}{\ete{\cam{F}}{V^{\times s}}}$ and $\map{\tnr{\kappa}}{\pmb{\mathcal{U}}}{\ete{\cam{F}}{V^{\times s}}}$ are said to be \textsb{tangent}\index{function!tangent} to each other at $\tnr{X}_0$ if
\begin{equation}\label{eq:tangent}
\lim_{\tnr{X}\to \tnr{0}} \dfrac{\fua{\tnr{\psi}}{\tnr{X}_0+\tnr{X}}-\fua{\tnr{\kappa}}{\tnr{X}_0+\tnr{X}}}{\|\tnr{X}\|}=\tnr{0}\,,
\end{equation}
In other words, as $\tnr{X}$ tends to the zero tensor, the numerator tends to the zero tensor faster than $\|\tnr{X}\|$ tends to zero and then, on the limit, equality $\fua{\tnr{\psi}}{\tnr{X}_0}=\fua{\tnr{\kappa}}{\tnr{X}_0}$ must hold. From this definition, it is possible to obtain that if two functions are tangent to $\tnr{\psi}$ at $\tnr{X}_0$, they are tangent to each other at this tensor.

{\footnotesize
\begin{proof}
Since the sum of limits is the limit of sums, the previous statement can be verified by considering functions $\tnr{\kappa}_1$ and $\tnr{\kappa}_2$ tangent to $\tnr{\psi}$ at $\tnr{X}_0$ and subtracting both limit expressions described by \eqref{eq:tangent} : 
\begin{align*}
\tnr{0}&=\lim_{\tnr{X}\to \tnr{0}} \dfrac{\fua{\tnr{\psi}}{\tnr{X}_0+\tnr{X}}-\fua{\tnr{\kappa}_1}{\tnr{X}_0+\tnr{X}}-\fua{\tnr{\psi}}{\tnr{X}_0+\tnr{X}}+\fua{\tnr{\kappa}_2}{\tnr{X}_0+\tnr{X}}}{\|\tnr{X}\|}\\
&=\lim_{\tnr{X}\to \tnr{0}} \dfrac{\fua{\tnr{\kappa}_2}{\tnr{X}_0+\tnr{X}}-\fua{\tnr{\kappa}_1}{\tnr{X}_0+\tnr{X}}}{\|\tnr{X}\|}\,.
\end{align*} 
\end{proof}}

Proceeding with the above conditions, if there is one and only one tensor function $\tnr{\psi_\mathit{d}}$, tangent to $\tnr{\psi}$ at $\tnr{X}_0$, described by the rule
\begin{equation}
\fua{\tnr{\psi_\mathit{d}}}{\tnr{X}}=\fua{\tnr{\psi}}{\tnr{X}_0}+\fua{[\fua{\tnr{\psi}'}{\tnr{X}_0}]}{\tnr{X}-\tnr{X}_0}\,,
\end{equation}
where function $\fua{\tnr{\psi}'}{\tnr{X}_0}\in\evl{\cam{F}}{\pmb{\mathcal{U}}}{\ete{\cam{F}}{V^{\times s}}}$ is bounded\footnote{Linearity and boundedness imply continuity. See definition \eqref{eq:funcaoLimitada}.}, then $\tnr{\psi}$ is called \textsb{Fr�chet differentiable}\index{function!Fr�chet differentiable}, \textsb{totally differentiable}\index{function!totally differentiable} or simply \textsb{differentiable}\index{function!differentiable} at $\tnr{X}_0$, while the linear function $\fua{\tnr{\psi}'}{\tnr{X}_0}$ is called the \textsb{derivative}\index{derivative} of $\tnr{\psi}$ at $\tnr{X}_0$. Moreover, the function in 
$\map{\tnr{\psi}'}{\pmb{\mathcal{U}}}{\evl{\cam{F}}{\pmb{\mathcal{U}}}{\ete{\cam{F}}{V^{\times s}}}}$ is said to be the \textsb{derivative function}\index{derivative!function} or simply the derivative of $\tnr{\psi}$. Rewriting expression \eqref{eq:tangent} for $\tnr{\psi}$ and $\tnr{\psi_\mathit{d}}$ tangent at $\tnr{X}_0$, the result is
\begin{equation}\label{eq:preDirect}
\lim_{\tnr{X}\to \tnr{0}} \dfrac{\fua{\tnr{\psi}}{\tnr{X}_0+\tnr{X}}-\fua{\tnr{\psi}}{\tnr{X}_0}-\fua{[\fua{\tnr{\psi}'}{\tnr{X}_0}]}{\tnr{X}}}{\|\tnr{X}\|}=\tnr{0}\,,
\end{equation} 
from which we can conclude that the numerator tends to the zero tensor faster than the denominator tends to zero, otherwise there would be no definite limit. If numerator tends to the zero tensor, the term $\fua{[\fua{\tnr{\psi}'}{\tnr{X}_0}]}{\tnr{X}}$ results an approximation for the difference $\fua{\tnr{\psi}}{\tnr{X}_0+\tnr{X}}-\fua{\tnr{\psi}}{\tnr{X}_0}$, and that is why the derivative is said to be a local linear approximation of a function and $\fua{[\fua{\tnr{\psi}'}{\tnr{X}_0}]}{\tnr{X}}$ is called the \textsb{differential}\index{differential} of $\tnr{\psi}$ at $\tnr{X}_0$. Moreover, since the previous equality is valid for $\tnr{X}$ tending to the zero tensor through any ``path'', let's choose a specific direction by considering an arbitrary tensor $\tnr{H}\in\pmb{\mathcal{U}}$ in such a way that $\tnr{X}=\alpha\tnr{H}$, where $\alpha\in\cam{F}$, and then 
\begin{equation*}
\lim_{\alpha\to 0} \dfrac{\fua{\tnr{\psi}}{\tnr{X}_0+\alpha\tnr{H}}-\fua{\tnr{\psi}}{\tnr{X}_0}-\fua{[\fua{\tnr{\psi}'}{\tnr{X}_0}]}{\alpha\tnr{H}}}{|\alpha|}=\tnr{0}\,.
\end{equation*} 
Multiplying both sides by $\sgn{\alpha}$, we have
\begin{equation*}
\lim_{\alpha\to 0} \dfrac{\fua{\tnr{\psi}}{\tnr{X}_0+\alpha\tnr{H}}-\fua{\tnr{\psi}}{\tnr{X}_0}}{\alpha}-\lim_{\alpha\to 0} \dfrac{\alpha\fua{[\fua{\tnr{\psi}'}{\tnr{X}_0}]}{\tnr{H}}}{\alpha}=\sgn{\alpha}\tnr{0}\,,
\end{equation*} 
which results
\begin{equation}
\fua{[\fua{\tnr{\psi}'}{\tnr{X}_0}]}{\tnr{H}}=\lim_{\alpha\to 0} \dfrac{\fua{\tnr{\psi}}{\tnr{X}_0+\alpha\tnr{H}}-\fua{\tnr{\psi}}{\tnr{X}_0}}{\alpha}\,,
\end{equation}
where $\fua{[\fua{\tnr{\psi}'}{\tnr{X}_0}]}{\tnr{H}}\in\ete{\cam{F}}{V^{\times s}}$ is called the \textsb{directional derivative}\index{derivative!directional} of $\tnr{\psi}$ along $\tnr{H}$ at $\tnr{X}_0$. From this definition, when there is a directional derivative of $\tnr{\psi}$ for any $\tnr{X}_0\in\pmb{\mathcal{U}}$, the function described by
\begin{equation}\label{eq:directDeriv}
\fua{[\fua{\tnr{\psi}'}{\tnr{X}}]}{\tnr{H}}=\lim_{\alpha\to 0} \dfrac{\fua{\tnr{\psi}}{\tnr{X}+\alpha\tnr{H}}-\fua{\tnr{\psi}}{\tnr{X}}}{\alpha}
\end{equation}
is said to be the directional derivative\footnote{It is important to say a few words about a less restricted or weaker type of derivative than the Fr�chet derivative just presented. It is not defined from the concept of tangency but from a directional derivative in which the derivative involved is not necessarily bounded and linear. Considering the above conditions, the tensor function in $\map{\tnr{\varphi}}{\pmb{\mathcal{U}}}{\ete{\cam{F}}{V^{\times s}}}$ is said to be \textsb{G�teaux differentiable}\index{function!G�teaux differentiable} at $\tnr{X}_0$ if there exists a mapping $\map{\fua{\tnr{\varphi}'_G}{\tnr{X}_0}}{\pmb{\mathcal{U}}}{\ete{\cam{F}}{V^{\times s}}}$ where 
	\begin{equation*}
	\fua{[\fua{\tnr{\varphi}'_G}{\tnr{X}_0}]}{\tnr{X}}=\lim_{\alpha\to 0} \dfrac{\fua{\tnr{\varphi}}{\tnr{X}_0+\alpha\tnr{X}}-\fua{\tnr{\varphi}}{\tnr{X}_0}}{\alpha}\,.
	\end{equation*}
	The tensor function $\fua{[\fua{\tnr{\varphi}'_G}{\tnr{X}_0}]}{\tnr{X}}$ is called the \textsb{G�teaux differential}\index{differential!G�teaux} of $\tnr{\varphi}$ at $\tnr{X}_0$ and $\tnr{\varphi}'_G$ is the \textsb{G�teaux derivative}\index{derivative!G�teaux} of $\tnr{\varphi}$. Every Fr�chet differentiable function is also G�teaux differential, but the converse is obviously not true. What is less obvious is the following: even when G�teaux derivative $\fua{\tnr{\varphi}'_G}{\tnr{X}_0}$ is bounded and linear, G�teaux differentiability, which means differentiability along all directions at a point, does not assure total differentiability. This present study will require only the stronger restrictions from the Fr�chet differentiation. For further developments of the G�teaux derivative, see \aut{Wouk}\cite{wouk_1979_1}, chapter 12.} of $\tnr{\psi}$ along $\tnr{H}$. 
 

{\footnotesize
\begin{proof}
The above theory would be useless if there were no tangent function $\tnr{\delta_\mathit{d}}$ for any given function $\tnr{\delta}$. Considering the rules $\fua{\tnr{\delta}}{\tnr{X}}=\alpha\tnr{X}$ and $\fua{\tnr{\delta_\mathit{d}}}{\tnr{X}}=\fua{\tnr{\delta}}{\tnr{X}_0}+\alpha(\tnr{X}-\tnr{X}_0)=\alpha\tnr{X}$, function $\tnr{\delta_\mathit{d}}$ tangent to $\tnr{\delta}$ on $\tnr{X}_0$ can be easily verified from \eqref{eq:preDirect}. Now, let's prove that there is only one function $\tnr{\psi_\mathit{d}}$ tangent to $\tnr{\psi}$ by supposing functions $\tnr{\psi_\mathit{d1}}$ and $\tnr{\psi_\mathit{d2}}$ tangent to $\tnr{\psi}$. One equality \eqref{eq:preDirect} can be obtained for each function and subtracting them results
\begin{equation*}
\lim_{\tnr{X}\to \tnr{0}} \dfrac{\fua{[\fua{\tnr{\psi}'_2}{\tnr{X}_0}]}{\tnr{X}}-\fua{[\fua{\tnr{\psi}'_1}{\tnr{X}_0}]}{\tnr{X}}}{\|\tnr{X}\|}=\lim_{\tnr{X}\to \tnr{0}} \dfrac{\{[\tnr{\psi}'_2-\tnr{\psi}'_1](\tnr{X}_0)\}(\tnr{X})}{\|\tnr{X}\|}=\tnr{0}\,.
\end{equation*}
Since these equalities are valid for any $\tnr{X}\in\pmb{\mathcal{U}}$, let's admit $\tnr{X}=\alpha\tnr{H}$, where $\tnr{H}\in\pmb{\mathcal{U}}$ is a non zero given tensor and $\alpha$ is a positive real number. Thereby, previous equalities lead to
\begin{equation*}
	\lim_{\alpha\to 0} \dfrac{\{[\tnr{\psi}'_2-\tnr{\psi}'_1](\tnr{X}_0)\}(\alpha\tnr{H})}{\|\alpha\tnr{H}\|}=\dfrac{\{[\tnr{\psi}'_2-\tnr{\psi}'_1](\tnr{X}_0)\}(\tnr{H})}{\|\tnr{H}\|}=\tnr{0}\,
\end{equation*}
which are valid for any chosen $\tnr{X}_0$ and then $\tnr{\psi}'_2=\tnr{\psi}'_1$.
\end{proof}}

Considering previous conditions, it is possible to obtain the following important properties for derivatives by considering the function in mapping $\map{\tnr{\psi}}{\pmb{\mathcal{U}}}{\ete{\cam{F}}{V^{\times s}}}$ differentiable in its domain, an arbitrary tensor $\tnr{H}\in\pmb{\mathcal{U}}$ and a scalar $\alpha\in\cam{F}$.
\begin{itemize}
	\setlength\itemsep{.1em}
	\item[ii.] If $\fua{\tnr{\psi}}{\tnr{X}}=\alpha\tnr{X}$ then $\tnr{\psi}'=????$\,
	\item[ii.] If $\fua{\tnr{\psi}}{\tnr{X}}=\tnr{H}$ then $\tnr{\psi}'=\tnr{0}\,$;
	\item[iii.] Sum rule: if $\tnr{\psi}=\tnr{\psi}_1+\tnr{\psi}_2$ then $\tnr{\psi}'=\tnr{\psi}_1'+\tnr{\psi}_2'\,$;
    \item[iv.] Product rule: if $\fua{\tnr{\psi}}{\tnr{X}}=\fua{\tnr{\psi}_1}{\tnr{X}}\odot_q\fua{\tnr{\psi}_2}{\tnr{X}}$, where mappings  $\map{\tnr{\psi}_1}{\pmb{\mathcal{U}}}{\ete{\cam{F}}{V^{\times r}\times W^{\times q}}}$ and $\map{\tnr{\psi}_2}{\pmb{\mathcal{U}}}{\ete{\cam{F}}{W^{\times q}\times V^{\times p}}}$ are defined for $s=r+p$, then
	\begin{equation*}
		\fua{[\fua{\tnr{\psi}'}{\tnr{X}}]}{\tnr{H}}=\fua{\tnr{\psi}_1}{\tnr{X}}\odot_{q}\fua{[\fua{\tnr{\psi}_2'}{\tnr{X}}]}{\tnr{H}} + \fua{[\fua{\tnr{\psi}_1'}{\tnr{X}}]}{\tnr{H}} \odot_{q} \fua{\tnr{\psi}_2}{\tnr{X}}\,;
	\end{equation*}
	\item[vi.] Chain rule:
\end{itemize}  

In the context of Hilbert spaces, we already know that there is a biunivocal relationship between a linear tensor function space and a tensor space described by the theorem \ref{teo:repRieszTensor}. For the present case, considering the terms of this theorem and the previous conditions restricted to Hilbert spaces, the derivative $\fua{\tnr{\psi}'}{\tnr{X}_0}\in\evl{\cam{F}}{\pmb{\mathcal{U}}}{\ete{\cam{F}}{V^{\times s}}}$ has its correspondent tensor $\fua{\nabla\tnr{\psi}}{\tnr{X}_0}\in\ete{\cam{F}}{U^{\times m}\times V^{\times s}}$, called the \textsb{gradient}\index{gradient} of $\tnr{\psi}$ at $\tnr{X}_0$, if the following rule is defined:
\begin{equation}
[\fua{\tnr{\psi}'}{\tnr{X}_0}](\tnr{X}) = \tnr{X}\odot_m\fua{\nabla\tnr{\psi}}{\tnr{X}_0}\,.
\end{equation} 
Moreover, the tensor function in $\map{\nabla\tnr{\psi}}{\pmb{\mathcal{U}}}{\ete{\cam{F}}{U^{\times m}\times V^{\times s}}}$, called the gradient of $\tnr{\psi}$, assigns to a tensor of order $m$ a tensor of order $m+s$. Therefore, for the particular case of a differentiable tensor operator, its gradient assigns to a $m$-th order tensor a $2m$-th order tensor. Now, from this definition of gradient and rule \eqref{eq:directDeriv}, the directional derivative of $\tnr{\psi}$ along $\tnr{H}\in\pmb{\mathcal{U}}$ is described by 
\begin{equation}
\fua{[\fua{\tnr{\psi}'}{\tnr{X}}]}{\tnr{H}}=\tnr{H}\odot_m\fua{\nabla\tnr{\psi}}{\tnr{X}}=\lim_{\alpha\to 0} \dfrac{\fua{\tnr{\psi}}{\tnr{X}+\alpha\tnr{H}}-\fua{\tnr{\psi}}{\tnr{X}}}{\alpha}\,.
\end{equation}
REESCREVER It is then possible to obtain the following important properties by considering the function in $\map{\tnr{\psi}}{\pmb{\mathcal{U}}}{\ete{\cam{F}}{V^{\times s}}}$ differentiable in its domain and direction $\tnr{H}\in\pmb{\mathcal{U}}$.
\begin{itemize}
	\setlength\itemsep{.1em}
	\item[i.] If $\tnr{\psi}=\tnr{\psi}_1+\tnr{\psi}_2$ then $\nabla\tnr{\psi}=\nabla\tnr{\psi}_1+\nabla\tnr{\psi}_2\,$;
	\item[ii.] If $\fua{\tnr{\psi}}{\tnr{X}}=\fua{\tnr{\psi}_1}{\tnr{X}}\cdot\fua{\tnr{\psi}_2}{\tnr{X}}$, where mappings  $\map{\tnr{\psi}_1}{\pmb{\mathcal{U}}}{\ete{\cam{F}}{W^{\times q}}}$ and $\map{\tnr{\psi}_2}{\pmb{\mathcal{U}}}{\ete{\cam{F}}{W^{\times q}}}$ are defined for $s=0$, then
	\begin{equation*}
	\fua{\nabla\tnr{\psi}}{\tnr{X}} = \fua{\nabla\tnr{\psi}_1}{\tnr{X}}\odot_q\fua{\tnr{\psi}_2}{\tnr{X}}+\fua{\nabla\tnr{\psi}_2}{\tnr{X}}\odot_q\fua{\tnr{\psi}_1}{\tnr{X}}\,;
	\end{equation*}	
	\item[iii.] Chain rule: NO CADERNO.
\end{itemize}


% ver se 3.43 consegue provar chain rule
% falar do divergente
% falar das propriedades do gradiente










It is interesting to note that derivatives may be differentiable themselves, and also derivatives of derivatives, and so on. Considering previous conditions, tensor function $\tnr{\psi}$ is defined to be the derivative of order zero of itself, function $\tnr{\psi}'$ is the derivative of order one of $\tnr{\psi}$, $\tnr{\psi}''$ of order two and $\tnr{\psi}^{(k)}$ of order $k\geqslant 0$. Thereby, a tensor function $\tnr{\psi}$ is said to be differentiable of order $k+1$ at $\tnr{X}_0$ if there is a function $\tnr{\psi_\mathit{d}}^{(k)}$ tangent to $\tnr{\psi}^{(k)}$ at $\tnr{X}_0$ where   
\begin{equation}
\fua{\tnr{\psi_\mathit{d}}^{(k)}}{\tnr{X}}=\fua{\tnr{\psi}}{\tnr{X}_0}+\fua{[\fua{\tnr{\psi}^{(k+1)}}{\tnr{X}_0}]}{\tnr{X}-\tnr{X}_0}\,.
\end{equation}
. 




%LEMBRAR DE DEMONSTRAR A UNICIDADE DA FUNCAO 

% como todo espa�o tensorial � finitamente dimensional ent�o a diferen�a entre F-derivada e G-derivada reside somente na lineridade da derivada. A G-derivada adimite operador n�o linear, enquanto que a F-derivada exige.


% ap�s definir frechet derivative, mostra a fracao cl�ssica que obtem o frechet derivative. a partir dessa fracao, defina gateaux derivative. Colocar nota de rodap� explicando que se os espacos envolvidos n�o fossem dimensionalmente finitos as diferencas entre Gateaux e Frechet seriam maiores. 
    