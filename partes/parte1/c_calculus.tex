
\chapter{Calculus of Tensor Functions}

Although vector spaces are building blocks of the several important concepts presented so far on the two previous chapters, it is still not possible to recognize shapes in their definer sets. However, the study of mathematical shapes does not belong to Linear Algebra, but to the realms of Geometry, whose non empty sets we shall consider here to be constituted by ``shape'' elements called \textsb{points}\index{points}, which are geometric objects of primitive notion, devoid of dimensional features in order to morphologically represent physical locations. If a given set of such points is conveniently related to a vector space by a group action, other shapes can be obtained and thereby further morphological concepts can be developed, when we say that a certain geometry is defined. As physical spaces are usually abstracted by geometric sets, we can not prescind from studying at least the basic topics of the so called Affine Geometry, which includes the well known Euclidean Geometry and is sufficiently governed by the concept of parallelism.


\section{Affine Spaces}