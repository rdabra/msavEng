
\chapter{Basic Tensor Calculus}

Analysis is the branch of Mathematics that studies the behaviour of functions by using the idea of limit, considered fundamental, and its subordinate concepts. When such concepts involve  differentiation and integration, as well as their related definitions, Analysis is commonly known as Calculus. Intuitively speaking, up to this point, we have dealt with functions in a kind of ``static'' approach: on previous chapters, the sole definition of a mapping, with its associated function explicitly described or not, was sufficient for developing the concepts presented. Now, we are also interested in a ``dynamic'' approach for functions, where it is important not only to specify them but also to describe how their values evolve on their images, or how they behave, through derivatives and integrals. In this context, considering the study of tensors started on chapter \ref{cp:tensor}, the following pages introduce the Calculus of tensor functions, usually called Tensor Calculus. Since a tensor function is the most general type of function presented so far, we do not deviate from our main objective of being as abstract as possible on behalf of mathematical beauty.



\section{Differentiation}

In the context of Banach tensor spaces, which are normed and complete, to differentiate a tensor function $\tnr{\psi}$ is to obtain its derivative, here understood as a linear tensor function that measures qualitatively and quantitatively the sensitivity of $\tnr{\psi}$ on each element of the domain. Intuitively speaking, sensitivity refers to the intrinsic ``susceptibility'' of a function, or how it would ``react'', in terms of its values, to an \emph{eventual change} on its argument. In the specific case of derivatives, this function ``reaction'' is \emph{linearly estimated} on an element of its domain. However, not all tensor functions can be differentiable: the following mathematical conditions and definitions will describe which functions admit differentiation. Considering $\ete{\cam{F}}{U^{\times m}}$ and $\ete{\cam{F}}{V^{\times s}}$ Banach tensor spaces, both defined by the same field $\cam{F}$, let $\pmb{\mathcal{U}}\subset\ete{\cam{F}}{U^{\times m}}$ be an open set and element  $\tnr{X}_0\in\pmb{\mathcal{U}}$ an arbitrary tensor. The functions in $\map{\tnr{\psi}}{\pmb{\mathcal{U}}}{\ete{\cam{F}}{V^{\times s}}}$ and $\map{\tnr{\kappa}}{\pmb{\mathcal{U}}}{\ete{\cam{F}}{V^{\times s}}}$ are said to be \textsb{tangent}\index{function!tangent} to each other at $\tnr{X}_0$ if
\begin{equation}\label{eq:tangent}
\lim_{\tnr{X}\to \tnr{0}} \dfrac{\fua{\tnr{\psi}}{\tnr{X}_0+\tnr{X}}-\fua{\tnr{\kappa}}{\tnr{X}_0+\tnr{X}}}{\|\tnr{X}\|}=\tnr{0}\,,
\end{equation}
In other words, as $\tnr{X}$ tends to the zero tensor, the numerator tends to the zero tensor faster than $\|\tnr{X}\|$ tends to zero and then, on the limit, equality $\fua{\tnr{\psi}}{\tnr{X}_0}=\fua{\tnr{\kappa}}{\tnr{X}_0}$ must hold. From this definition, it is possible to obtain that if two functions are tangent to $\tnr{\psi}$ at $\tnr{X}_0$, they are tangent to each other at this tensor.

{\footnotesize
\begin{proof}
Since the sum of limits is the limit of sums, the previous statement can be verified by considering functions $\tnr{\kappa}_1$ and $\tnr{\kappa}_2$ tangent to $\tnr{\psi}$ at $\tnr{X}_0$ and subtracting both limit expressions described by \eqref{eq:tangent} :
\begin{align*}
\tnr{0}&=\lim_{\tnr{X}\to \tnr{0}} \dfrac{\fua{\tnr{\psi}}{\tnr{X}_0+\tnr{X}}-\fua{\tnr{\kappa}_1}{\tnr{X}_0+\tnr{X}}-\fua{\tnr{\psi}}{\tnr{X}_0+\tnr{X}}+\fua{\tnr{\kappa}_2}{\tnr{X}_0+\tnr{X}}}{\|\tnr{X}\|}\\
&=\lim_{\tnr{X}\to \tnr{0}} \dfrac{\fua{\tnr{\kappa}_2}{\tnr{X}_0+\tnr{X}}-\fua{\tnr{\kappa}_1}{\tnr{X}_0+\tnr{X}}}{\|\tnr{X}\|}\,.
\end{align*}
\end{proof}}

Proceeding with the above conditions, if there is one and only one tensor function \gloref{tangent}, tangent to $\tnr{\psi}$ at $\tnr{X}_0$, described by the rule
\begin{equation}\label{eq:tangentFunct}
\fua{\tnr{\psi_\mathit{d}}}{\tnr{X}}=\fua{\tnr{\psi}}{\tnr{X}_0}+\fua{[\fua{\tnr{\psi}'}{\tnr{X}_0}]}{\tnr{X}-\tnr{X}_0}\,,
\end{equation}
where function $\fua{\tnr{\psi}'}{\tnr{X}_0}\in\evl{\cam{F}}{\pmb{\mathcal{U}}}{\ete{\cam{F}}{V^{\times s}}}$ is bounded, then $\tnr{\psi}$ is called \textsb{Fr�chet differentiable}\index{function!Fr�chet differentiable}, \textsb{totally differentiable}\index{function!totally differentiable} or simply \textsb{differentiable}\index{function!differentiable} at $\tnr{X}_0$, while the linear function $\fua{\tnr{\psi}'}{\tnr{X}_0}$ is called the \textsb{derivative}\index{derivative} of $\tnr{\psi}$ at $\tnr{X}_0$. According to \eqref{eq:funcaoLimitada}, linearity and boundedness imply continuity, and then $\fua{\tnr{\psi}'}{\tnr{X}_0}$ belongs to function space $\evlc{\cam{F}}{\pmb{\mathcal{U}}}{\ete{\cam{F}}{V^{\times s}}}$, which we specify to be also a Banach space of continuous linear functions. Moreover, the function in
$\map{\gloref{derivative}}{\pmb{\mathcal{U}}}{\evlc{\cam{F}}{\pmb{\mathcal{U}}}{\ete{\cam{F}}{V^{\times s}}}}$ is said to be the \textsb{derivative function}\index{derivative!function} or simply the derivative of $\tnr{\psi}$. Rewriting expression \eqref{eq:tangent} for $\tnr{\psi}$ and $\tnr{\psi_\mathit{d}}$ tangent at $\tnr{X}_0$, the result is
\begin{equation}\label{eq:preDirect}
\lim_{\tnr{X}\to \tnr{0}} \dfrac{\fua{\tnr{\psi}}{\tnr{X}_0+\tnr{X}}-\fua{\tnr{\psi}}{\tnr{X}_0}-\fua{[\fua{\tnr{\psi}'}{\tnr{X}_0}]}{\tnr{X}}}{\|\tnr{X}\|}=\tnr{0}\,,
\end{equation}
from which we can conclude that the numerator tends to the zero tensor faster than the denominator tends to zero, otherwise there would be no definite limit. If numerator tends to the zero tensor, the term $\fua{[\fua{\tnr{\psi}'}{\tnr{X}_0}]}{\tnr{X}}$ results an approximation for the difference $\fua{\tnr{\psi}}{\tnr{X}_0+\tnr{X}}-\fua{\tnr{\psi}}{\tnr{X}_0}$, and that is why the derivative is said to be a local linear approximation of a function and $\fua{[\fua{\tnr{\psi}'}{\tnr{X}_0}]}{\tnr{X}}$ is called the \textsb{differential}\index{differential} of $\tnr{\psi}$ at $\tnr{X}_0$. Moreover, since the previous equality is valid for $\tnr{X}$ tending to the zero tensor through any ``path'', let's choose a specific direction by considering an arbitrary tensor $\tnr{H}\in\pmb{\mathcal{U}}$ in such a way that $\tnr{X}=\alpha\tnr{H}$, where $\alpha\in\cam{F}$, and then
\begin{equation*}
\lim_{\alpha\to 0} \dfrac{\fua{\tnr{\psi}}{\tnr{X}_0+\alpha\tnr{H}}-\fua{\tnr{\psi}}{\tnr{X}_0}-\fua{[\fua{\tnr{\psi}'}{\tnr{X}_0}]}{\alpha\tnr{H}}}{|\alpha|}=\tnr{0}\,.
\end{equation*}
Multiplying both sides by $\sgn{\alpha}$, we have
\begin{equation*}
\lim_{\alpha\to 0} \dfrac{\fua{\tnr{\psi}}{\tnr{X}_0+\alpha\tnr{H}}-\fua{\tnr{\psi}}{\tnr{X}_0}}{\alpha}-\lim_{\alpha\to 0} \dfrac{\alpha\fua{[\fua{\tnr{\psi}'}{\tnr{X}_0}]}{\tnr{H}}}{\alpha}=\sgn{\alpha}\tnr{0}\,,
\end{equation*}
which results
\begin{equation}
\fua{[\fua{\tnr{\psi}'}{\tnr{X}_0}]}{\tnr{H}}=\lim_{\alpha\to 0} \dfrac{\fua{\tnr{\psi}}{\tnr{X}_0+\alpha\tnr{H}}-\fua{\tnr{\psi}}{\tnr{X}_0}}{\alpha}\,,
\end{equation}
where $\fua{[\fua{\tnr{\psi}'}{\tnr{X}_0}]}{\tnr{H}}\in\ete{\cam{F}}{V^{\times s}}$ is called the \textsb{directional derivative}\index{derivative!directional} of $\tnr{\psi}$ along $\tnr{H}$ at $\tnr{X}_0$. From this definition, when there is a directional derivative of $\tnr{\psi}$ for any $\tnr{X}_0\in\pmb{\mathcal{U}}$, the function described by
\begin{equation}\label{eq:directDeriv}
\fua{[\fua{\tnr{\psi}'}{\tnr{X}}]}{\tnr{H}}=\lim_{\alpha\to 0} \dfrac{\fua{\tnr{\psi}}{\tnr{X}+\alpha\tnr{H}}-\fua{\tnr{\psi}}{\tnr{X}}}{\alpha}
\end{equation}
is said to be the directional derivative\footnote{It is important to say a few words about a less restricted or weaker type of derivative than the Fr�chet derivative just presented. It is not defined from the concept of tangency but from a directional derivative in which the derivative involved is not necessarily bounded and linear. Considering the above conditions, the tensor function in $\map{\tnr{\varphi}}{\pmb{\mathcal{U}}}{\ete{\cam{F}}{V^{\times s}}}$ is said to be \textsb{G�teaux differentiable}\index{function!G�teaux differentiable} at $\tnr{X}_0$ if there exists a mapping $\map{\fua{\tnr{\varphi}'_G}{\tnr{X}_0}}{\pmb{\mathcal{U}}}{\ete{\cam{F}}{V^{\times s}}}$ where
\begin{equation*}
\fua{[\fua{\tnr{\varphi}'_G}{\tnr{X}_0}]}{\tnr{X}}=\lim_{\alpha\to 0} \dfrac{\fua{\tnr{\varphi}}{\tnr{X}_0+\alpha\tnr{X}}-\fua{\tnr{\varphi}}{\tnr{X}_0}}{\alpha}\,.
\end{equation*}
The tensor function $\fua{[\fua{\tnr{\varphi}'_G}{\tnr{X}_0}]}{\tnr{X}}$ is called the \textsb{G�teaux differential}\index{differential!G�teaux} of $\tnr{\varphi}$ at $\tnr{X}_0$ and $\tnr{\varphi}'_G$ is the \textsb{G�teaux derivative}\index{derivative!G�teaux} of $\tnr{\varphi}$. Every Fr�chet differentiable function is also G�teaux differential, but the converse is obviously not true. What is less obvious is the following: even when G�teaux derivative $\fua{\tnr{\varphi}'_G}{\tnr{X}_0}$ is bounded and linear, G�teaux differentiability, which means differentiability along all directions at a point, does not assure total differentiability. This present study will require only the stronger restrictions from the Fr�chet differentiation. For further developments of the G�teaux derivative, see \aut{Wouk}\cite{wouk_1979_1}, chapter 12.} of $\tnr{\psi}$ along $\tnr{H}$.


{\footnotesize
\begin{proof}
The above theory would be useless if there were no tangent function $\tnr{\delta_\mathit{d}}$ for any given function $\tnr{\delta}$. Considering the rules $\fua{\tnr{\delta}}{\tnr{X}}=\alpha\tnr{X}$ and $\fua{\tnr{\delta_\mathit{d}}}{\tnr{X}}=\fua{\tnr{\delta}}{\tnr{X}_0}+\alpha(\tnr{X}-\tnr{X}_0)=\alpha\tnr{X}$, function $\tnr{\delta_\mathit{d}}$ tangent to $\tnr{\delta}$ on $\tnr{X}_0$ can be easily verified from \eqref{eq:preDirect}. Now, let's prove that there is only one function $\tnr{\psi_\mathit{d}}$ tangent to $\tnr{\psi}$ by supposing functions $\tnr{\psi_\mathit{d1}}$ and $\tnr{\psi_\mathit{d2}}$ tangent to $\tnr{\psi}$. One equality \eqref{eq:preDirect} can be obtained for each function and subtracting them results
\begin{equation*}
\lim_{\tnr{X}\to \tnr{0}} \dfrac{\fua{[\fua{\tnr{\psi}'_2}{\tnr{X}_0}]}{\tnr{X}}-\fua{[\fua{\tnr{\psi}'_1}{\tnr{X}_0}]}{\tnr{X}}}{\|\tnr{X}\|}=\lim_{\tnr{X}\to \tnr{0}} \dfrac{\{[\tnr{\psi}'_2-\tnr{\psi}'_1](\tnr{X}_0)\}(\tnr{X})}{\|\tnr{X}\|}=\tnr{0}\,.
\end{equation*}
Since these equalities are valid for any $\tnr{X}\in\pmb{\mathcal{U}}$, let's admit $\tnr{X}=\alpha\tnr{H}$, where $\tnr{H}\in\pmb{\mathcal{U}}$ is a non zero given tensor and $\alpha$ is a positive real number. Thereby, previous equalities lead to
\begin{equation*}
	\lim_{\alpha\to 0} \dfrac{\{[\tnr{\psi}'_2-\tnr{\psi}'_1](\tnr{X}_0)\}(\alpha\tnr{H})}{\|\alpha\tnr{H}\|}=\dfrac{\{[\tnr{\psi}'_2-\tnr{\psi}'_1](\tnr{X}_0)\}(\tnr{H})}{\|\tnr{H}\|}=\tnr{0}\,
\end{equation*}
which are valid for any chosen $\tnr{X}_0$ and then $\tnr{\psi}'_2=\tnr{\psi}'_1$.
\end{proof}}

Considering previous conditions, it is possible to obtain the following important properties for derivatives by considering the function in mapping $\map{\tnr{\psi}}{\pmb{\mathcal{U}}}{\ete{\cam{F}}{V^{\times s}}}$ differentiable in its domain, an arbitrary tensor $\tnr{H}\in\pmb{\mathcal{U}}$ and a scalar $\beta\in\cam{F}$.
\begin{itemize}
	\setlength\itemsep{.1em}
	\item[i.] If $\fua{\tnr{\psi}}{\tnr{X}}=\tnr{H}$ then $\tnr{\psi}'=\tnr{0}\,$;
	\item[ii.] If $\fua{\tnr{\psi}}{\tnr{X}}=\beta\tnr{X}$ then $\fua{[\fua{\tnr{\psi}'}{\tnr{X}}]}{\tnr{H}}=\beta\tnr{H}\,$;
	\item[iii.] Sum rule: if $\tnr{\psi}=\tnr{\psi}_1+\tnr{\psi}_2$ then $\tnr{\psi}'=\tnr{\psi}_1'+\tnr{\psi}_2'\,$;
	\item[iv.] Chain rule: if $\tnr{\psi}=\tnr{\psi}_1\circ\tnr{\psi}_2$, where $\map{\tnr{\psi}_1}{\ete{\cam{F}}{W^{\times r}}}{\ete{\cam{F}}{V^{\times s}}}$ and $\map{\tnr{\psi}_2}{\pmb{\mathcal{U}}}{\ete{\cam{F}}{W^{\times r}}}$, then
	\begin{equation}
		\fua{\tnr{\psi}'}{\tnr{X}}=[\fua{\tnr{\psi}_1'\circ\tnr{\psi}_2}{\tnr{X}}]\circ\fua{\tnr{\psi}_2'}{\tnr{X}}\,;
	\end{equation}
    \item[v.] Product rule: if $\fua{\tnr{\psi}}{\tnr{X}}=\fua{\tnr{\psi}_1}{\tnr{X}}\odot_q\fua{\tnr{\psi}_2}{\tnr{X}}$, where mappings  $\map{\tnr{\psi}_1}{\pmb{\mathcal{U}}}{\ete{\cam{F}}{V^{\times r}\times W^{\times q}}}$ and $\map{\tnr{\psi}_2}{\pmb{\mathcal{U}}}{\ete{\cam{F}}{W^{\times q}\times V^{\times p}}}$ are defined for $s=r+p$, then
	\begin{equation}
		\fua{[\fua{\tnr{\psi}'}{\tnr{X}}]}{\tnr{H}}=\fua{\tnr{\psi}_1}{\tnr{X}}\odot_{q}\fua{[\fua{\tnr{\psi}_2'}{\tnr{X}}]}{\tnr{H}} + \fua{[\fua{\tnr{\psi}_1'}{\tnr{X}}]}{\tnr{H}} \odot_{q} \fua{\tnr{\psi}_2}{\tnr{X}}\,;
	\end{equation}
	\item[vi.] Leibniz's rule: if function in $\map{\tnr{\psi}}{\pmb{\mathcal{U}}}{\cam{F}}$ is described by $\fua{\tnr{\psi}}{\tnr{X}}=\fua{\tnr{\psi}_1}{\tnr{X}}\cdot\fua{\tnr{\psi}_2}{\tnr{X}}$, where $\map{\tnr{\psi}_1}{\pmb{\mathcal{U}}}{\ete{\cam{F}}{V^{\times s}}}$ and $\map{\tnr{\psi}_2}{\pmb{\mathcal{U}}}{\ete{\cam{F}}{V^{\times s}}}$, then
	\begin{equation}
	\fua{[\fua{\tnr{\psi}'}{\tnr{X}}]}{\tnr{H}}=\fua{\tnr{\psi}_1}{\tnr{X}}\cdot\fua{[\fua{\tnr{\psi}_2'}{\tnr{X}}]}{\tnr{H}} + \fua{[\fua{\tnr{\psi}_1'}{\tnr{X}}]}{\tnr{H}} \cdot \fua{\tnr{\psi}_2}{\tnr{X}}\,;
	\end{equation}
	\item[vii.] If $\fua{\tnr{\psi}}{\tnr{X}}=\fua{\tnr{\psi}_1}{\tnr{X}}\otimes\fua{\tnr{\psi}_2}{\tnr{X}}$, where $\map{\tnr{\psi}_1}{\pmb{\mathcal{U}}}{\ete{\cam{F}}{V^{\times s-r}}}$ and $\map{\tnr{\psi}_2}{\pmb{\mathcal{U}}}{\ete{\cam{F}}{V^{\times r}}}$, then
\begin{equation}
\fua{[\fua{\tnr{\psi}'}{\tnr{X}}]}{\tnr{H}}=\fua{\tnr{\psi}_1}{\tnr{X}}\otimes\fua{[\fua{\tnr{\psi}_2'}{\tnr{X}}]}{\tnr{H}} + \fua{[\fua{\tnr{\psi}_1'}{\tnr{X}}]}{\tnr{H}} \otimes \fua{\tnr{\psi}_2}{\tnr{X}}\,;
\end{equation}
	\item[viii.] If $\fua{\tnr{\psi}}{\vto{x}}=\fua{\tnr{\psi}_1}{\vto{x}}\times\fua{\tnr{\psi}_2}{\vto{x}}$, where $U_\real$ in mappings  $\map{\tnr{\psi}_1}{\pmb{\mathcal{U}}}{U_\cam{R}}$ and $\map{\tnr{\psi}_2}{\pmb{\mathcal{U}}}{U_\cam{R}}$ defines a three dimensional oriented Euclidean space, then
\begin{equation}\label{eq:crossRule}
\fua{[\fua{\tnr{\psi}'}{\vto{x}}]}{\vto{h}} = \fua{\tnr{\psi}_1}{\vto{x}}\times\fua{[\fua{\tnr{\psi}_2'}{\vto{x}}]}{\vto{h}} + \fua{[\fua{\tnr{\psi}_1'}{\vto{x}}]}{\vto{h}} \times \fua{\tnr{\psi}_2}{\vto{x}}\,.
\end{equation}
\end{itemize}

{\footnotesize
\begin{proof}
Verification of items i and ii are straightforward results after specifying their respective rules of $\tnr{\psi}$ on definition \eqref{eq:directDeriv}. For the case of item iii, the use of \eqref{eq:directDeriv} leads to
\begin{equation*}
\fua{[\fua{\tnr{\psi}'}{\tnr{X}}]}{\tnr{H}} = \fua{[\fua{\tnr{\psi}_1'}{\tnr{X}}]}{\tnr{H}} + \fua{[\fua{\tnr{\psi}'_2}{\tnr{X}}]}{\tnr{H}} = \fua{[\fua{\tnr{\psi}'_1}{\tnr{X}}+\fua{\tnr{\psi}'_2}{\tnr{X}}]}{\tnr{H}}=\fua{[\fua{[\tnr{\psi}'_1+\tnr{\psi}'_2]}{\tnr{X}}]}{\tnr{H}}\,.
\end{equation*}
Now, in order to prove the chain rule, we adapt the strategy of \aut{Zeidler}\cite{zeidler_1995_1}, p. 248, as follows. Considering the concept of derivative as an approximation, from \eqref{eq:directDeriv} it is correct to say that there is a residue $\fua{r_1}{\beta\tnr{Z}}$ that tends to the zero tensor when non-zero $\beta\in\cam{F}$ tends to zero, where
\begin{equation*}
\beta\fua{[\fua{\tnr{\psi}'_1}{\tnr{Y}}]}{\tnr{Z}}=\fua{\tnr{\psi}_1}{\tnr{Y}+\beta\tnr{Z}}-\fua{\tnr{\psi}_1}{\tnr{Y}}+\fua{r_1}{\beta\tnr{Z}}
\end{equation*}
and $\tnr{Y},\tnr{Z}\in\ete{\cam{F}}{W^{\times r}}$ are arbitrary. For the case of $\tnr{\psi}_2$, we write
\begin{equation*}
\fua{\tnr{\psi}_2}{\tnr{X}}=\fua{\tnr{\psi}_2}{\tnr{X}+\alpha\tnr{H}}-\alpha\fua{[\fua{\tnr{\psi}'_2}{\tnr{X}}]}{\tnr{H}}+\fua{r_2}{\alpha\tnr{H}}.
\end{equation*}
Specifying $\tnr{Y}=\fua{\tnr{\psi}_2}{\tnr{X}}$ and $\tnr{Z}=\alpha/\beta\fua{[\fua{\tnr{\psi}'_2}{\tnr{X}}]}{\tnr{H}}$ on the first equality, then
\begin{align*}
	\alpha\fua{[\fua{\tnr{\psi}'_1\circ\tnr{\psi}_2}{\tnr{X}}]}{\fua{[\fua{\tnr{\psi}'_2}{\tnr{X}}]}{\tnr{H}}}&=\fua{\tnr{\psi}_1}{\fua{\tnr{\psi}_2}{\tnr{X}}+\alpha\fua{[\fua{\tnr{\psi}'_2}{\tnr{X}}]}{\tnr{H}}}-\fua{\tnr{\psi}_1}{\tnr{Y}}+\fua{r_1}{\alpha\fua{[\fua{\tnr{\psi}'_2}{\tnr{X}}]}{\tnr{H}}}\\
	\alpha[[\fua{\tnr{\psi}_1'\circ\tnr{\psi}_2}{\tnr{X}}]\circ\fua{\tnr{\psi}_2'}{\tnr{X}}](\tnr{H})&=\fua{\tnr{\psi}_1}{\fua{\tnr{\psi}_2}{\tnr{X}+\alpha\tnr{H}}+\fua{r_2}{\alpha\tnr{H}}}-\fua{\tnr{\psi}_1\circ\tnr{\psi}_2}{\tnr{X}}+\fua{r_1}{\alpha\fua{[\fua{\tnr{\psi}'_2}{\tnr{X}}]}{\tnr{H}}}
\end{align*}
Dividing both sides by $\alpha$ and taking them to the limit of $\alpha$ tending to zero, item iv is verified. The product rule is proved by the following development: adding and subtracting the right hand side of
\begin{equation*}
\fua{[\fua{\tnr{\psi}'}{\tnr{X}}]}{\tnr{H}}=\lim_{\alpha\to 0}1/\alpha[\fua{\tnr{\psi}_1}{\tnr{X}+\alpha\tnr{H}}\odot_q\fua{\tnr{\psi}_2}{\tnr{X}+\alpha\tnr{H}}-\fua{\tnr{\psi}_1}{\tnr{X}}\odot_q\fua{\tnr{\psi}_2}{\tnr{X}}]
\end{equation*}
by $\fua{\tnr{\psi}_1}{\tnr{X}+\alpha\tnr{H}}\odot_{q}\fua{\tnr{\psi}_2}{\tnr{X}}$, we arrive at
\begin{align*}
\fua{[\fua{\tnr{\psi}'}{\tnr{X}}]}{\tnr{H}}&=\lim_{\alpha\to 0}1/\alpha\{ \fua{\tnr{\psi}_1}{\tnr{X}+\alpha\tnr{H}}\odot_{q}[\fua{\tnr{\psi}_2}{\tnr{X}+\alpha\tnr{H}}-\fua{\tnr{\psi}_2}{\tnr{X}}] +\\
& \qquad\qquad\qquad+[\fua{\tnr{\psi}_1}{\tnr{X}+\alpha\tnr{H}}-\fua{\tnr{\psi}_1}{\tnr{X}}]\odot_q\fua{\tnr{\psi}_2}{\tnr{X}}\}\\
&= \fua{\tnr{\psi}_1}{\tnr{X}}\odot_{q}\lim_{\alpha\to 0}1/\alpha[\fua{\tnr{\psi}_2}{\tnr{X}+\alpha\tnr{H}}-\fua{\tnr{\psi}_2}{\tnr{X}}] +\\
& \qquad\qquad\qquad+\lim_{\alpha\to 0}1/\alpha[\fua{\tnr{\psi}_1}{\tnr{X}+\alpha\tnr{H}}-\fua{\tnr{\psi}_1}{\tnr{X}}]\odot_q\fua{\tnr{\psi}_2}{\tnr{X}}\,.
\end{align*}
Since item vi is a corollary of the product rule, its verification is trivial. Item vii is verified by a similar procedure we used for the product rule. Considering $(U_\real,\tnr{A}_B)$ a three dimensional oriented Euclidean space, we prove the item viii first by the development
\begin{align*}
\fua{[\fua{\tnr{\psi}'}{\vto{x}}]}{\vto{h}}&=\lim_{\alpha\to 0}1/\alpha\{\tnr{A}_B\odot_{2}[\fua{\tnr{\psi}_1}{\vto{x}+\alpha\vto{h}}\otimes\fua{\tnr{\psi}_2}{\vto{x}+\alpha\vto{h}}]-\tnr{A}_B\odot_{2}[\fua{\tnr{\psi}_1}{\vto{x}}\otimes\fua{\tnr{\psi}_2}{\vto{x}}]\}\\
&=\tnr{A}_B\odot_{2}\lim_{\alpha\to 0}1/\alpha\{\fua{\tnr{\psi}_1}{\vto{x}+\alpha\vto{h}}\otimes\fua{\tnr{\psi}_2}{\vto{x}+\alpha\vto{h}}-\fua{\tnr{\psi}_1}{\vto{x}}\otimes\fua{\tnr{\psi}_2}{\vto{x}}\}
\end{align*}
and then by adding and subtracting the limit on the right hand side by $\fua{\tnr{\psi}_1}{\vto{x}+\alpha\vto{h}}\otimes\fua{\tnr{\psi}_2}{\vto{x}}$, as we similarly did for the product rule. The result is
\begin{equation*}
\fua{[\fua{\tnr{\psi}'}{\vto{x}}]}{\vto{h}}= \tnr{A}_B\odot_{2}[\fua{\tnr{\psi}_1}{\vto{x}}\otimes\fua{[\fua{\tnr{\psi}_2'}{\vto{x}}]}{\vto{h}}]+\tnr{A}_B\odot_{2}[ \fua{[\fua{\tnr{\psi}_1'}{\vto{x}}]}{\vto{h}} \otimes \fua{\tnr{\psi}_2}{\vto{x}}]\,,
\end{equation*}
from which we easily verify the item.
\end{proof}}

Considering a mapping $\map{\tnr{\varphi}}{\pmb{\mathcal{U}}_1\times\cdots\times\pmb{\mathcal{U}}_q}{\ete{\cam{F}}{V^{\times s}}}$, where $\pmb{\mathcal{U}}_i\subset\ete{\cam{F}}{U^{\times m}}$ is open and the values of $\tnr{\varphi}$ are represented by the multivariate notation $\fua{\tnr{\varphi}}{\tnr{X}_1,\cdots,\tnr{X}_q}$, if there is a tangent function $\tnr{\varphi_\mathit{d}}$ to $\tnr{\varphi}$ at $(\tnr{Y}_1,\cdots,\tnr{Y}_q)\in\pmb{\mathcal{U}}_1\times\cdots\times\pmb{\mathcal{U}}_q$ where
\begin{equation*}
\fua{\tnr{\varphi_\mathit{d}}}{\tnr{X}_1,\cdots,\tnr{X}_q}=\fua{\tnr{\varphi}}{\tnr{Y}_1,\cdots,\tnr{Y}_q}+\fua{[\fua{\tnr{\varphi}'}{\tnr{Y}_1,\cdots,\tnr{Y}_q}]}{\tnr{X}_1-\tnr{Y}_1,\cdots,\tnr{X}_q-\tnr{Y}_q}
\end{equation*}
and $\fua{\tnr{\varphi}'}{\tnr{Y}_1,\cdots,\tnr{Y}_q}$ is multilinear, then $\tnr{\varphi}$ is said to be totally differentiable and, by a similar development to the case of univariate tensor functions, we arrive at the multivariate definition for directional derivatives:
\begin{equation}\label{eq:totalDerivative}
\fua{[\fua{\tnr{\varphi}'}{\tnr{X}_1,\cdots,\tnr{X}_q}]}{\tnr{H}_1,\cdots,\tnr{H}_q}=\lim_{\alpha\to 0} \dfrac{\fua{\tnr{\varphi}}{\tnr{X}_1+\alpha\tnr{H}_1,\cdots,\tnr{X}_q+\alpha\tnr{H}_q}-\fua{\tnr{\varphi}}{\tnr{X}_1,\cdots,\tnr{X}_q}}{\alpha}\,,
\end{equation}
where $\tnr{H}_i\in\pmb{\mathcal{U}}_i$ is an arbitrary direction. When the strategy is to study each variable $\tnr{X}_i$ independently, \emph{caeteris paribus}, development from the concept of tangency is the same as the univariate case, but now the derivative of $\tnr{\varphi}$  is called \textsb{partial derivative}\index{derivative!partial} of $\tnr{\varphi}$ with respect to $\tnr{X}_r$, $1\leqslant r\leqslant q$, represented by \gloref{partialDer}, where
\begin{equation}\label{eq:partialDerivative}
[\fua{\tnr{\varphi}_{\tnr{X}_r}^{'}}{\tnr{X}_1,\cdots,\tnr{X}_q}]{(\tnr{H}_r)}=\lim_{\alpha\to 0} \dfrac{\fua{\tnr{\varphi}}{\tnr{X}_1,\cdots,\tnr{X}_r+\alpha\tnr{H}_r,\cdots,\tnr{X}_q}-\fua{\tnr{\varphi}}{\tnr{X}_1,\cdots,\tnr{X}_q}}{\alpha}\,.
\end{equation}
From the above conditions, an important property can be obtained which relates total and partial derivatives of multivariate tensor functions, described as follows:
\begin{equation}
\fua{[\fua{\tnr{\varphi}'}{\tnr{X}_1,\cdots,\tnr{X}_q}]}{\tnr{H}_1,\cdots,\tnr{H}_q}=\sum_{i=1}^{q}[\fua{\tnr{\varphi}_{\tnr{X}_i}^{'}}{\tnr{X}_1,\cdots,\tnr{X}_q}]{(\tnr{H}_i)}\,.
\end{equation}

{\footnotesize
\begin{proof}
Let's prove this property. From \eqref{eq:totalDerivative} and \eqref{eq:partialDerivative} it is clear that, for each direction $\tnr{H}_i$, we have $q$ equalities of the form
\begin{equation*}
\fua{[\fua{\tnr{\varphi}'}{\tnr{X}_1,\cdots,\tnr{X}_q}]}{\tnr{0},\cdots,\tnr{H}_i,\cdots,\tnr{0}}=[\fua{\tnr{\varphi}_{\tnr{X}_r}^{'}}{\tnr{X}_1,\cdots,\tnr{X}_q}]{(\tnr{H}_i)}\,.
\end{equation*}
Summing these $q$ equalities, we prove the property because function $\fua{\tnr{\varphi}'}{\tnr{X}_1,\cdots,\tnr{X}_q}$ is multilinear.
\end{proof}}



It is interesting to note that derivatives may be differentiable themselves, and also derivatives of derivatives, and so on. Considering previous conditions, the tensor function $\tnr{\psi}$ is defined to be the zero order derivative of itself, function $\tnr{\psi}'$ is the order one derivative of $\tnr{\psi}$, $\tnr{\psi}''$ the order two and $\gloref{derivHigh}$ the order $k\geqslant 0$. Thereby, from definition \eqref{eq:tangentFunct}, a tensor function $\tnr{\psi}$ is said to be differentiable of order two at $\tnr{X}_0$ if it is differentiable at $\tnr{X}_0$ and there is a unique function $\tnr{\psi_\mathit{d}}^{(1)}$ tangent to the function in $\map{\tnr{\psi}'}{\pmb{\mathcal{U}}}{\evlc{\cam{F}}{\pmb{\mathcal{U}}}{\ete{\cam{F}}{V^{\times s}}}}$ at $\tnr{X}_0$ where
\begin{equation*}
\fua{\tnr{\psi_\mathit{d}}^{(1)}}{\tnr{X}}=\fua{\tnr{\psi}'}{\tnr{X}_0}+\fua{[\fua{\tnr{\psi}''}{\tnr{X}_0}]}{\tnr{X}-\tnr{X}_0}\,.
\end{equation*}
Since function $\fua{\tnr{\psi}'}{\tnr{X}_0}\in\evlc{\cam{F}}{\pmb{\mathcal{U}}}{\ete{\cam{F}}{V^{\times s}}}$ then  $\map{\tnr{\psi}''}{\pmb{\mathcal{U}}}{\evlc{\cam{F}}{\pmb{\mathcal{U}}}{\evlc{\cam{F}}{\pmb{\mathcal{U}}}{\ete{\cam{F}}{V^{\times s}}}}}$. Similarly, if $\tnr{\psi}$ is differentiable of order three at $\tnr{X}_0$, then it is differentiable of order two at $\tnr{X}_0$ and there is a unique $\tnr{\psi_\mathit{d}}^{(2)}$ tangent to $\tnr{\psi}''$ at $\tnr{X}_0$ where
\begin{equation*}
\fua{\tnr{\psi_\mathit{d}}^{(2)}}{\tnr{X}}=\fua{\tnr{\psi}''}{\tnr{X}_0}+\fua{[\fua{\tnr{\psi}'''}{\tnr{X}_0}]}{\tnr{X}-\tnr{X}_0}
\end{equation*}
and $\map{\tnr{\psi}'''}{\pmb{\mathcal{U}}}{\evlc{\cam{F}}{\pmb{\mathcal{U}}}{\evlc{\cam{F}}{\pmb{\mathcal{U}}}{\evlc{\cam{F}}{\pmb{\mathcal{U}}}{\ete{\cam{F}}{V^{\times s}}}}}}$. Generically, tensor function $\tnr{\psi}$ is said to be differentiable of order $k+1$ at $\tnr{X}_0$ if it is differentiable of order $k$ at $\tnr{X}_0$ and there is a unique function $\tnr{\psi_\mathit{d}}^{(k)}$ tangent to function $\tnr{\psi}^{(k)}$ at $\tnr{X}_0$ described by
\begin{equation}\label{eq:tangentOrderK1}
\fua{\tnr{\psi_\mathit{d}}^{(k)}}{\tnr{X}}=\fua{\tnr{\psi}^{(k)}}{\tnr{X}_0}+\fua{[\fua{\tnr{\psi}^{(k+1)}}{\tnr{X}_0}]}{\tnr{X}-\tnr{X}_0}\,,
\end{equation}
where
\begin{equation*}
\map{\tnr{\psi}^{(k+1)}}{\pmb{\mathcal{U}}}{{\evlc{\cam{F}}{\pmb{\mathcal{U}}_{k+1}}{\evlc{\cam{F}}{\pmb{\mathcal{U}}_k}{\cdots\evlc{\cam{F}}{\pmb{\mathcal{U}}_{1}}{\ete{\cam{F}}{V^{\times s}}}}}}}
\end{equation*}
and $\pmb{\mathcal{U}}_i=\pmb{\mathcal{U}}$. In this context, proceeding with a similar development we performed for one order derivative, rule \eqref{eq:tangentOrderK1} leads to equality
\begin{equation}\label{eq:directDerivK1}
\fua{[\fua{\tnr{\psi}^{(k+1)}}{\tnr{X}}]}{\tnr{H}}=\lim_{\alpha\to 0} \dfrac{\fua{\tnr{\psi}^{(k)}}{\tnr{X}+\alpha\tnr{H}}-\fua{\tnr{\psi}^{(k)}}{\tnr{X}}}{\alpha}\,,
\end{equation}
which is the $k+1$ order directional derivative of $\tnr{\psi}$ along $\tnr{H}\in\pmb{\mathcal{U}}$. From previous equality, if $k>0$ the directional derivative of order $k+1$ along a given tensor is always a function and then by choosing arbitrary tensors $\tnr{H}_1,\cdots,\tnr{H}_{k+1}\in\pmb{\mathcal{U}}$, we conclude that
\begin{equation}\label{eq:toOrderZero}
[[\fua{[\fua{\tnr{\psi}^{(k+1)}}{\tnr{X}}]}{\tnr{H}_{k+1}}](\tnr{H}_k)\cdots](\tnr{H}_{1})\in\ete{\cam{F}}{V^{\times s}}\,.
\end{equation}

Now, let's restrict our attention to tensor functions that are continuous, that is, to elements of function space $\evc{\cam{F}}{\pmb{\mathcal{U}}}{\ete{\cam{F}}{V^{\times s}}}$ in this present case, when they are said to be of \textsb{differentiability class}\index{differentiability class} $\mathcal{C}^0$. Function $\tnr{\psi}$ is of class $\mathcal{C}^1$ or \textsb{continuously differentiable}\index{function!continuously differentiable} if it is of class $\mathcal{C}^0$ and differentiable of order one with derivative $\tnr{\psi}^{'}$ also of differentiability class $\mathcal{C}^0$. In general terms, $\tnr{\psi}$ is of class $\mathcal{C}^{k+1}$ if it is of class $\mathcal{C}^{k}$ and differentiable of order $k+1$ with $\tnr{\psi}^{(k+1)}$ of class $\mathcal{C}^k$. Stating that function $\tnr{\psi}$ is of class $\mathcal{C}^{k+1}$ is equal to stating that $\tnr{\psi}$ is \textsb{smooth}\index{function!smooth} of order $k+1$ and for the case where $\tnr{\psi}$ is smooth of any order, we say that it is of class $\mathcal{C}^\infty$ or simply call it smooth. A differentiable tensor function is called a \textsb{diffeomorphism}\index{diffeomorphism} if it is a bijection and its inverse is also differentiable. When a diffeomorphism is smooth of order $k+1$, then it is called a $\mathcal{C}^{k+1}$-diffeomorphism.




% ---------------

\section{The Gradient}

The concept of gradient we shall detail in this section is restricted to the context of Hilbert tensor spaces because it relies on the Riesz-Fr�chet Representation of Tensors, described on theorem \ref{teo:repRieszTensor}. Similarly to what we already presented in this chapter, an open set $\pmb{\mathcal{U}}\subset\ete{\cam{F}}{U^{\times m}}$ and a differentiable function in $\map{\tnr{\psi}}{\pmb{\mathcal{U}}}{\ete{\cam{F}}{V^{\times s}}}$ are also considered, but $\ete{\cam{F}}{U^{\times m}}$ and $\ete{\cam{F}}{V^{\times s}}$ are Hilbert tensor spaces here. Considering the conditions of the aforementioned theorem, there is a unique tensor $\fua{\nabla\tnr{\psi}}{\tnr{X}_0}\in\ete{\cam{F}}{U^{\times m}\times V^{\times s}}$, called the \textsb{gradient}\index{gradient} of $\tnr{\psi}$ at $\tnr{X}_0$, which the derivative $\fua{\tnr{\psi}'}{\tnr{X}_0}\in\evlc{\cam{F}}{\pmb{\mathcal{U}}}{\ete{\cam{F}}{V^{\times s}}}$ is the $m$-cotensor of and where
\begin{equation}\label{eq:defGradient}
[\fua{\tnr{\psi}'}{\tnr{X}_0}](\tnr{X}) = \tnr{X}\odot_m\fua{\nabla\tnr{\psi}}{\tnr{X}_0}\,.
\end{equation}
Moreover, the tensor function in $\map{\gloref{gradi}}{\pmb{\mathcal{U}}}{\ete{\cam{F}}{U^{\times m}\times V^{\times s}}}$, called the gradient of $\tnr{\psi}$, assigns to a tensor of order $m$ a tensor of order $m+s$. Therefore, \emph{for the particular case of a differentiable unary tensor operator on $\pmb{\mathcal{U}}$, its gradient assigns to a $m$-th order tensor a $2m$-th order tensor}. Now, from this definition of gradient and from rule \eqref{eq:directDeriv}, the directional derivative of $\tnr{\psi}$ along $\tnr{H}\in\pmb{\mathcal{U}}$ is described by
\begin{equation}
\fua{[\fua{\tnr{\psi}'}{\tnr{X}}]}{\tnr{H}}=\tnr{H}\odot_m\fua{\nabla\tnr{\psi}}{\tnr{X}}=\lim_{\alpha\to 0} \dfrac{\fua{\tnr{\psi}}{\tnr{X}+\alpha\tnr{H}}-\fua{\tnr{\psi}}{\tnr{X}}}{\alpha}\,.
\end{equation}
From the properties of derivatives presented on previous section, it is then possible to obtain the following important properties of gradients.
\begin{itemize}
	\setlength\itemsep{.1em}
	\item[i.] If $\tnr{\psi}=\tnr{\psi}_1+\tnr{\psi}_2$ then $\nabla\tnr{\psi}=\nabla\tnr{\psi}_1+\nabla\tnr{\psi}_2\,$;
	\item[ii.] If $\tnr{\psi}=\tnr{\psi}_1\circ\tnr{\psi}_2$, where $\map{\tnr{\psi}_1}{\ete{\cam{F}}{W^{\times r}}}{\ete{\cam{F}}{V^{\times s}}}$ and $\map{\tnr{\psi}_2}{\pmb{\mathcal{U}}}{\ete{\cam{F}}{W^{\times r}}}$, then
	\begin{equation}\label{eq:prop1Grad}
	\fua{\nabla\tnr{\psi}}{\tnr{X}} = \fua{\nabla\tnr{\psi}_2}{\tnr{X}} \odot_r \fua{\nabla\tnr{\psi}_1\circ\tnr{\psi}_2}{\tnr{X}}\,;
	\end{equation}	
	\item[iii.] If $\fua{\tnr{\psi}}{\tnr{X}}=\fua{\tnr{\psi}_1}{\tnr{X}}\cdot\fua{\tnr{\psi}_2}{\tnr{X}}$, where   $\map{\tnr{\psi}_1}{\pmb{\mathcal{U}}}{\ete{\real}{V^{\times s}}}$, $\map{\tnr{\psi}_2}{\pmb{\mathcal{U}}}{\ete{\real}{V^{\times s}}}$ and domain $\pmb{\mathcal{U}}\subset\ete{\real}{U^{\times m}}$, then
	\begin{equation}\label{eq:prop2Grad}
	\fua{\nabla\tnr{\psi}}{\tnr{X}} = \fua{\nabla\tnr{\psi}_1}{\tnr{X}}\odot_s\fua{\tnr{\psi}_2}{\tnr{X}}+\fua{\nabla\tnr{\psi}_2}{\tnr{X}}\odot_s\fua{\tnr{\psi}_1}{\tnr{X}}\,;
	\end{equation}
	\item[iv.] If $\fua{\tnr{\psi}}{\vto{x}}=\fua{\tnr{\psi}_1}{\vto{x}}\times\fua{\tnr{\psi}_2}{\vto{x}}$, where $U_\real$ in mappings  $\map{\tnr{\psi}_1}{\pmb{\mathcal{U}}}{U_\cam{R}}$ and $\map{\tnr{\psi}_2}{\pmb{\mathcal{U}}}{U_\cam{R}}$ is properly defined for enabling cross products, then
	\begin{equation}\label{eq:prop3Grad}
	\fua{\nabla\tnr{\psi}}{\vto{x}} = \fua{\nabla\tnr{\psi}_1}{\vto{x}}\times\fua{\tnr{\psi}_2}{\vto{x}}-\fua{\nabla\tnr{\psi}_2}{\vto{x}}\times\fua{\tnr{\psi}_1}{\vto{x}}\,;
	\end{equation}
\end{itemize}

{\footnotesize
\begin{proof}
The first item is verified through the following development:
\begin{align*}
\fua{[\fua{[\tnr{\psi}']}{\tnr{X}}}{\tnr{H}}&=\fua{\{[\fua{\tnr{\psi}'_1+\tnr{\psi}'_2]}{\tnr{X}}\}}{\tnr{H}}\\
\tnr{H}\odot_m\fua{\nabla\tnr{\psi}}{\tnr{X}}&=\fua{\fua{\tnr{\psi}'_1}{\tnr{X}}}{\tnr{H}}+\fua{\fua{\tnr{\psi}'_2}{\tnr{X}}}{\tnr{H}}\\
&=\tnr{H}\odot_m\fua{\nabla\tnr{\psi}_1}{\tnr{X}}+\tnr{H}\odot_m\fua{\nabla\tnr{\psi}_2}{\tnr{X}}\\
&=\tnr{H}\odot_m\fua{[\nabla\tnr{\psi}_1+\nabla\tnr{\psi}_2]}{\tnr{X}}\,.
\end{align*}	
Proof of item ii is obtained from the chain rule, where $\fua{[\fua{\tnr{\psi}'}{\tnr{X}}]}{\tnr{H}}=\fua{\{[\fua{\tnr{\psi}_1'\circ\tnr{\psi}_2}{\tnr{X}}]\circ\fua{\tnr{\psi}_2'}{\tnr{X}}\}}{\tnr{H}}$. We can develop the right hand side of this equality using equalities \eqref{eq:compoCotens} because
$\fua{\tnr{\psi}_1'\circ\tnr{\psi}_2}{\tnr{X}}$ and $\fua{\tnr{\psi}_2'}{\tnr{X}}$ are the $r$-cotensor and $m$-cotensor of $\fua{\nabla\tnr{\psi}_1\circ\tnr{\psi}_2}{\tnr{X}}$ and $\fua{\nabla\tnr{\psi}_2}{\tnr{X}}$ respectively. Item ii is proved by
\begin{align*}
\fua{[\fua{\tnr{\psi}'}{\tnr{X}}]}{\tnr{H}}&=\fua{\{[\fua{\tnr{\psi}_1'\circ\tnr{\psi}_2}{\tnr{X}}]\circ\fua{\tnr{\psi}_2'}{\tnr{X}}\}}{\tnr{H}}\\
\tnr{H}\odot_m\fua{\nabla\tnr{\psi}}{\tnr{X}}&=\tnr{H}\odot_m[\fua{\nabla\tnr{\psi}_2}{\tnr{X}}\odot_r\fua{\nabla\tnr{\psi}_1\circ\tnr{\psi}_2}{\tnr{X}}]\,.
\end{align*}
Now, from the Leibniz's rule we have
\begin{align*}
\fua{[\fua{\tnr{\psi}'}{\tnr{X}}]}{\tnr{H}}&=\fua{[\fua{\tnr{\psi}_1'}{\tnr{X}}]}{\tnr{H}} \cdot \fua{\tnr{\psi}_2}{\tnr{X}}+\fua{[\fua{\tnr{\psi}_2'}{\tnr{X}}]}{\tnr{H}}\cdot\fua{\tnr{\psi}_1}{\tnr{X}}\\
\tnr{H}\odot_m\fua{[\nabla\tnr{\psi}]}{\tnr{X}}&=\tnr{H}\odot_m\fua{[\nabla\tnr{\psi}_1]}{\tnr{X}}\odot_s\fua{\tnr{\psi}_2}{\tnr{X}}+\tnr{H}\odot_m\fua{[\nabla\tnr{\psi}_2]}{\tnr{X}}\odot_s\fua{\tnr{\psi}_1}{\tnr{X}}\\
&=\tnr{H}\odot_m\{\fua{[\nabla\tnr{\psi}_1]}{\tnr{X}}\odot_s\fua{\tnr{\psi}_2}{\tnr{X}}+\fua{[\nabla\tnr{\psi}_2]}{\tnr{X}}\odot_s\fua{\tnr{\psi}_1}{\tnr{X}}\}\,,
\end{align*}
which proves item iii. From \eqref{eq:crossRule}, development
\begin{align*}
\fua{[\fua{\tnr{\psi}'}{\vto{x}}]}{\vto{h}} &= \fua{[\fua{\tnr{\psi}_1'}{\vto{x}}]}{\vto{h}} \times \fua{\tnr{\psi}_2}{\vto{x}}-\fua{[\fua{\tnr{\psi}_2'}{\vto{x}}]}{\vto{h}}\times\fua{\tnr{\psi}_1}{\vto{x}}\\
\vto{h}\odot_{1}\fua{\nabla\tnr{\psi}}{\vto{x}}&=\vto{h}\odot_{1}[\fua{\nabla\tnr{\psi}_1}{\vto{x}}\times\fua{\tnr{\psi}_2}{\vto{x}}-\fua{\nabla\tnr{\psi}_2}{\vto{x}}\times\fua{\tnr{\psi}_1}{\vto{x}}]\,,
\end{align*}
proves item iv.
\end{proof}}


Still in the context of Hilbert spaces, we define from the concept of gradient three important and very useful tensor functions, namely divergence, Laplacian and curl. Considering the mapping $\map{\tnr{\varphi}}{\pmb{\mathcal{U}}}{\ete{\cam{F}}{U^{\times m}\times V^{\times s}}}$ where function $\tnr{\varphi}$ is differentiable on its domain, the function in $\map{\gloref{diverg}}{\pmb{\mathcal{U}}}{\ete{\cam{F}}{V^{\times s}}}$ is called the \textsb{divergence}\index{divergence} of $\tnr{\varphi}$ if its rule is described by
\begin{equation}\label{eq:divergence}
\fua{\dvt{\tnr{\varphi}}}{\tnr{X}}=\tnr{I}\odot_{2m}\fua{\nabla\tnr{\varphi}}{\tnr{X}}\,,
\end{equation}
where identity tensor $\tnr{I}\in\ete{\cam{F}}{U^{\times m}\times U^{\times m}}$ and $\fua{\nabla\tnr{\varphi}}{\tnr{X}}\in\ete{\cam{F}}{U^{\times m}\times U^{\times m}\times V^{\times s}}$. While the gradient always assigns to a tensor $\tnr{X}$ a tensor of higher order, the divergence assigns to $\tnr{X}$ a tensor of lower order only when $s<m$. Moreover, if $\tnr{\varphi}$ is a unary tensor operator on $\pmb{\mathcal{U}}$, that is $s=0$, then its divergence is a scalar function for all $m>0$.
From this concept of divergence and from mapping $\map{\nabla\tnr{\psi}}{\pmb{\mathcal{U}}}{\ete{\cam{F}}{U^{\times m}\times V^{\times s}}}$, it is valid and convenient to define a function
\begin{equation}\label{eq:laplace}
\Delta\tnr{\psi}=\dvt{\nabla\tnr{\psi}}\,,
\end{equation}
called the \textsb{Laplacian}\index{Laplacian} of $\tnr{\psi}$, where $\fua{\gloref{laplac}}{\tnr{X}}\in \ete{\cam{F}}{V^{\times s}}$. Now, we restrict the context of function $\tnr{\varphi}$ to tensor spaces of order $m=1$ and $s=1$, specifying that $V_\real$ is an Euclidean space and $U_\real$ defines a three dimensional oriented Euclidean space $(U_\real,\tnr{A}_B)$, where $B$ is an orthonormal basis of $U_\cam{R}$. In this context, it is possible to define a mapping  $\map{\gloref{curl}}{\pmb{\mathcal{U}}}{V_\real}$, whose function is called the \textsb{curl}\index{curl} of $\tnr{\varphi}$, with rule
\begin{equation}\label{eq:curl}
\fua{\crl{\tnr{\varphi}}}{\vto{x}}=\tnr{A}_B\odot_{2}\fua{\nabla\tnr{\varphi}}{\vto{x}}\,,
\end{equation}
where third order tensor $\fua{\nabla\tnr{\varphi}}{\vto{x}}\in\ete{\cam{F}}{U^2\times V}$. Therefore, it is clear that  $\fua{\dvt{\tnr{\varphi}}}{\vto{x}}$ and $\fua{\crl{\tnr{\varphi}}}{\vto{x}}$ are vectors of $V_\real$. Similarly to the case of gradients, there are some notable properties of divergence and curl, presented as follows, considering the appropriate previous conditions.
\begin{itemize}
\setlength\itemsep{.1em}
\item[i.] If $\tnr{\varphi}=\tnr{\varphi}_1+\tnr{\varphi}_2$ then $\dvt{\tnr{\varphi}}=\dvt{\tnr{\varphi}_1}+\dvt{\tnr{\varphi}_2}$ and $\crl{\tnr{\varphi}}=\crl{\tnr{\varphi}_1}+\crl{\tnr{\varphi}_2}\,$;
\item[ii.] If $\tnr{\varphi}=\tnr{\varphi}_1\circ\tnr{\varphi}_2$, where mapping $\map{\tnr{\varphi}_1}{\ete{\cam{F}}{W^{\times r}}}{\ete{\cam{F}}{U^{\times m}\times V^{\times s}}}$ and mapping $\map{\tnr{\varphi}_2}{\pmb{\mathcal{U}}}{\ete{\cam{F}}{W^{\times r}}}$ are defined, then
\begin{equation}
\fua{\dvt{\tnr{\varphi}}}{\tnr{X}} =\fua{\dvt{\tnr{\varphi}_2}}{\tnr{X}}\odot_{r}\fua{\nabla\tnr{\varphi}_1\circ\tnr{\varphi}_2}{\tnr{X}}\,,
\end{equation}	
and for $r=m=s=1$,
\begin{equation}
\fua{\crl{\tnr{\varphi}}}{\vto{x}} = \fua{\crl{\tnr{\varphi}_2}}{\vto{x}}\odot_{1}\fua{\nabla\tnr{\varphi}_1\circ\tnr{\varphi}_2}{\vto{x}}\,;
\end{equation}	

\item[iii.] If $\fua{\tnr{\varphi}}{\tnr{X}}=\fua{\tnr{\varphi}_1}{\tnr{X}}\cdot\fua{\tnr{\varphi}_2}{\tnr{X}}$, where mapping  $\map{\tnr{\varphi}_1}{\pmb{\mathcal{U}}}{\ete{\cam{F}}{U^{\times m}\times V^{\times s}}}$ and mapping $\map{\tnr{\varphi}_2}{\pmb{\mathcal{U}}}{\ete{\cam{F}}{U^{\times m}\times V^{\times s}}}$ are defined, then
\begin{equation}
\fua{\dvt{\tnr{\varphi}}}{\tnr{X}} = \fua{\dvt{\tnr{\varphi}_1}}{\tnr{X}}\odot_s\fua{\tnr{\varphi}_2}{\tnr{X}}+\fua{\dvt{\tnr{\varphi}_2}}{\tnr{X}}\odot_s\fua{\tnr{\varphi}_1}{\tnr{X}}\,,
\end{equation}	
and for $m=s=1$,
\begin{equation}
\fua{\crl{\tnr{\varphi}}}{\vto{x}} = \fua{\crl{\tnr{\varphi}_1}}{\vto{x}}\odot_s\fua{\tnr{\varphi}_2}{\vto{x}}+\fua{\crl{\tnr{\varphi}_2}}{\vto{x}}\odot_s\fua{\tnr{\varphi}_1}{\vto{x}}\,;
\end{equation}

\item[iv.] If $\fua{\tnr{\varphi}}{\vto{x}}=\fua{\tnr{\varphi}_1}{\vto{x}}\times\fua{\tnr{\varphi}_2}{\vto{x}}$, where mappings  $\map{\tnr{\varphi}_1}{\pmb{\mathcal{U}}}{U_\cam{R}}$ and $\map{\tnr{\varphi}_2}{\pmb{\mathcal{U}}}{U_\cam{R}}$ are defined, then
\begin{equation}
\fua{\crl{\tnr{\varphi}}}{\vto{x}} = \fua{\crl{\tnr{\varphi}_1}}{\vto{x}}\times\fua{\tnr{\varphi}_2}{\vto{x}}-\fua{\crl{\tnr{\varphi}_2}}{\vto{x}}\times\fua{\tnr{\psi}_1}{\vto{x}}\,;
\end{equation}
	
\item[v.]	If $\map{\tnr{\varphi}}{\pmb{\mathcal{U}}}{\real}$, where $\pmb{\mathcal{U}}\subset U_\real$, then $\fua{\crl{\nabla \tnr{\varphi}}}{\vto{x}}=\vto{0}\,$.
\end{itemize}

{\footnotesize
\begin{proof}
Considering definitions \eqref{eq:divergence} and \eqref{eq:curl}, since partial inner products are left (right) distributive, proof of item i is trivial. Pre multiplying properties \eqref{eq:prop1Grad}, \eqref{eq:prop2Grad} and \eqref{eq:prop3Grad} adequately by $\tnr{I}$ or $\tnr{A}_B$, proofs of items ii, iii and iv are straightforward. For the item v, since we already know that $\vto{u}\cdot\tnr{A}_B\odot_2\vto{v}\otimes\vto{w}=\tnr{A}_B(\vto{u},\vto{v},\vto{w})$, if we consider $\fua{{\nabla \tnr{\varphi}}}{\vto{x}}=\vto{v}\otimes\vto{w}$, then $\vto{u}\cdot\fua{\crl{\nabla \tnr{\varphi}}}{\vto{x}}=\tnr{A}_B(\vto{u},\vto{v},\vto{w})$ for all $\vto{u}\in\pmb{\mathcal{U}}$. Thereby, choosing $\vto{u}=\vto{v}$, we have $\vto{v}\cdot\fua{\crl{\nabla \tnr{\varphi}}}{\vto{x}}=\tnr{A}_B(\vto{v},\vto{v},\vto{w})=0$ because $\tnr{A}_B$ is antisymmetric, and since $\vto{v}\neq\vto{0}$, we prove the item.
\end{proof}}


Recalling the concept of higher order differentiation presented on the previous section and the conditions of theorem \ref{teo:repRieszTensor}, we can state that if $\tnr{\psi}$ is $k+1$ differentiable at $\tnr{X}_0$ then there is a unique tensor $\fua{\nabla\tnr{\psi}^{(k)}}{\tnr{X}_0}\in\ete{\cam{F}}{(U^{\times m})^{k+1}\times V^{\times s}}$ which the derivative
\begin{equation*}
\fua{\tnr{\psi}^{(k+1)}}{\tnr{X}_0}\in{\evlc{\cam{F}}{\pmb{\mathcal{U}}_{k+1}}{\evlc{\cam{F}}{\pmb{\mathcal{U}}_k}{\cdots\evlc{\cam{F}}{\pmb{\mathcal{U}}_{1}}{\ete{\cam{F}}{V^{\times s}}}}}}
\end{equation*}
is the $m$-cotensor of and each $\pmb{\mathcal{U}}_i=\pmb{\mathcal{U}}$ defines a Hilbert space of continuous functions. Moreover, choosing arbitrary directions $\tnr{H}_1,\cdots,\tnr{H}_{k+1}\in\pmb{\mathcal{U}}$ and specifying $r=m\cdot(k+1)$, statement \eqref{eq:toOrderZero} can be rewritten as
\begin{equation}
\tnr{H}_{1}\otimes\cdots\otimes\tnr{H}_{k+1}\odot_r\fua{\nabla\tnr{\psi}^{(k)}}{\tnr{X}}\in\ete{\cam{F}}{V^{\times s}}\,.
\end{equation}


\section{Integration}\index{integration}

The integration approach we shall deal with in this section relies fundamentally on Lebesgue's theory, which is built from the generalization of the concept of size, that is, length, area, volume or \textsb{hypervolume}\index{hypervolume} of respectively one, two, three or $n$ dimensional affine Hilbert spaces. In simple terms, this generalization is strong, based on assigning a nonnegative real number to a set, whatever measure this real number expresses. But this general measure must obey certain rules that require and lead to new important definitions. In this context, from specific scalar functions with measurable domains, integrable tensor functions can be defined.

From the concept of class and its dependent definitions presented on section \ref{sec:Sets}, the double $(M,\css{M})$ is said to be a \textsb{measurable space}\index{space!measurable} when $\sigma$-ring $\css{M}$ is a class of the set $M$; and when this set defines a group or a field we also qualify them with ``measurable''. Now, considering the mapping $\map{\mu}{\css{M}}{\real}$ and an arbitrary disjoint countable class $\{A_1,\cdots,A_n\}\subseteq\css{M}$, set function $\mu$ is called additive if
\begin{equation}\label{eq:addMeasure}
\fua{\mu}{\bigcup_{i=1}^nA_i}=\sum_{{i}=1}^n\fua{\mu}{A_i}\,.
\end{equation}
From this definition, it is clear that $\fua{\mu}{\emptyset}=\fua{\mu}{\emptyset\cup\emptyset}=2\fua{\mu}{\emptyset}=0$. For the special case of a mapping $\map{\mu}{\css{M}}{\real^+}$ where $\css{M}$ is a ring, additive set function $\mu$ is said to be a \textsb{continent}\index{ring!continent} of ring $\css{M}$. But if $\css{M}$ is a $\sigma$-ring, its continent $\mu$ is called a \textsb{measure}\index{measure} when
\begin{equation}\label{eq:measuSum}
\fua{\mu}{\bigcup_{i=1}^\infty A_i}=\sum_{{i}=1}^\infty\fua{\mu}{A_i}\,,
\end{equation}
for an arbitrary countable class $\{A_1,A_2,\cdots\}\subseteq\css{M}$. Thereby, the double constituted by a measurable space and a measure, that is $((M,\css{M}),\mu)$, is called a \textsb{measure space}\index{space!measure}, which we shall hereafter represent by the triple $(M,\css{M},\mu)$. 
Considering measure space $(M,\css{M},\mu)$, a measurable field $(\real,\css{F})$ and a mapping  $\map{f}{M}{\real}$, a sequence of functions $f_1,f_2,\cdots$ that also map $M$ to $\real$ \textsb{converges in measure}\index{convergence!in measure} $\mu$ to function $f$ when, given an arbitrary real number $\epsilon>0$,
\begin{equation}
\lim_{i\to\infty}\fua{\mu}{\{x\in M : |\fua{f}{x}-\fua{f_i}{x}|\geqslant\epsilon\}}=0\,.
\end{equation}

Now, let $(M,\css{M})$ be a measurable space and $(\real,\css{F})$ a measurable field. The function in $\map{s}{M}{\real}$ is called \textsb{measurable}\index{function!measurable} if, for every element $\con{B}\in\css{F}$, preimage $\con{R}^{-1}_{\con{B}} \in \css{M}$. Considering a disjoint countable class $\{A_1,\cdots,A_n\}\subseteq\css{M}$ and non zero scalars $\alpha_1,\cdots,\alpha_n\in\real$, where $\css{A}=\{A_1,\cdots,A_n\}$ is called a \textsb{partition}\index{set!partition of} of an arbitrary set $A\in\css{M}$, that is $A=\bigcup_{i=1}^nA_i$, the measurable function in $\map{s_A}{M}{\real}$ is called \textsb{step}\index{function!step} or \textsb{simple}\index{function!simple} with respect to $\css{A}$ if its rule is described by 
\begin{equation}
\fua{s_\css{A}}{x} = \sum_{i=1}^n\alpha_i\fua{\tnr{1}_{A_i}}{x}\,,
\end{equation}
where the function in $\map{\tnr{1}_{A_i}}{A_i}{\real}$ is commonly known as the \textsb{indicator function}\index{function!indicator} of $A_i$, defined by
\begin{equation*}
\fua{\tnr{1}_{A_i}}{x}=
\begin{dcases}
0 & \text{if } x\notin A_i\\
1 & \text{if } x\in A_i
\end{dcases}\,.
\end{equation*}
In the particular case of a measure space $(M,\css{M},\mu)$, a nonnegative step function $s_\css{A}$, where $\alpha_i\geqslant 0$, is defined to be \textsb{integrable}\index{function!integrable step} on set $E\in\css{M}$ when $\fua{\mu}{E\cap A_i} < \infty$. If this is the case, real number
\begin{equation}\label{eq:defInteSimp}
\int_E s_\css{A} := \sum_{i=1}^n\alpha_i\fua{\mu}{E\cap A_i}
\end{equation}
is called the \textsb{integral}\index{integral!of step function} of $s_\css{A}$ on $E$. In this context, if $\css{B}=\{B_1,\cdots,B_m\}\subseteq\css{M}$ is another partition of $A$, it is clear that class $\css{C}=\{A_1\cap B_1,A_1\cap B_2,\cdots,A_n\cap B_m\}$ is also a partition of $A$ and then, by using previous definition, we can write 
\begin{equation*}
\int_E s_\css{C} = \sum_{i=1}^n\sum_{j=1}^m\gamma_{ij}\fua{\mu}{E\cap A_i\cap B_j}\,,
\end{equation*}
which means that, when $A_i$ and $B_j$ are not disjoint,
\begin{equation*}
\fua{s_\css{C}}{x} = \sum_{i=1}^n\sum_{j=1}^m\gamma_{ij}\fua{\tnr{1}_{A_i\cap B_j}}{x} = \sum_{i=1}^n\underbrace{\sum_{j=1}^m\gamma_{ij}}_{\alpha_i}\fua{\tnr{1}_{A_i}}{x}=\sum_{j=1}^m\underbrace{\sum_{i=1}^n\gamma_{ij}}_{\beta_j}\fua{\tnr{1}_{B_j}}{x}\,.
\end{equation*}
From these equalities, when $x\in A_i\cap B_j$ it is clear that $\alpha_i=\beta_j$ and therefore
\begin{align*}
\int_E s_\css{A} &= \sum_{i=1}^n\alpha_i\fua{\mu}{E\cap A_i}\\
&= \sum_{i=1}^n\alpha_i\fua{\mu}{E\cap \bigcup_{j=1}^mA_i\cap B_j}\\
&= \sum_{j=1}^m\sum_{i=1}^n\alpha_i\fua{\mu}{E\cap A_i\cap B_j}\\
&= \sum_{j=1}^m\beta_j\fua{\mu}{E\cap \bigcup_{i=1}^nB_j\cap A_i}\\
&= \sum_{j=1}^m\beta_j\fua{\mu}{E\cap B_j}\\
&=\int_E s_\css{B}\,,
\end{align*}
since intersection is distributive on union and $\mu$ is additive according to \eqref{eq:addMeasure}. For integration purposes, this equality enables us to remove the identification of the partition of $A$ from the representation of the function in a step mapping $\map{s}{M}{\real}$, since the integral of $s$ is the same for any partition of arbitrary $A\subseteq M$.  

Let $(S,\varrho)$ be a normed metric space of integrable step functions on $E$ with norm and metric respectively defined by
\begin{alignat*} {3}
\|s\|=\int_E |s|&\qquad \text{ and } \qquad&\fua{\varrho}{g_1,g_2}=\|g_1-g_2\|\,,
\end{alignat*}
where $|s|(x):=|\fua{s}{x}|$ and $|s|,g_1,g_2\in S$. In this context, if a Cauchy sequence of integrable step functions $s_1,s_2,\cdots\in S$ exists and converges in measure $\mu$ to the measurable nonnegative function in $\map{g}{M}{\real}$, we say that $g$ is integrable or Labesgue integrable on $E$ and scalar 
\begin{equation}\label{eq:integ}
\int_E g := \lim_{i\to \infty}\int_E s_i 
\end{equation} 
is the integral or Labesgue integral of $g$ on $E$. Considering a mapping $\map{f}{M}{\real}$ where $f=f^{+}-f^{-}$ such that 
\begin{alignat*} {3}
\fua{f^+}{x} = \begin{dcases}
\fua{f}{x} & \text{if } \fua{f}{x}> 0\\
0 & \text{if } \fua{f}{x} \leqslant 0
\end{dcases} &\qquad \text{ and } \qquad&\fua{f^-}{x} = \begin{dcases}
-\fua{f}{x} & \text{if } \fua{f}{x}< 0\\
0 & \text{if } \fua{f}{x} \geqslant 0
\end{dcases}\,.
\end{alignat*}
Since functions $f^{+}$ and $f^{-}$ are nonnegative, we can define that $f$ is integrable on $E$ if and only if both $f^{+}$ and $f^{-}$ are integrable on $E$ and also that 
\begin{equation}\label{eq:decompInt}
\int_E f = \int_E f^{+} + \int_E f^{-}\,. 
\end{equation}
It is important to study some fundamental properties of integrals and we shall presented them as follows by firstly specifying functions in $\map{f}{M}{\real}$ and $\map{g}{M}{\real}$ to be measurable, and scalars $\alpha,\beta\in\real$, where $(M,\css{M})$ and $(\real,\css{F})$ are also measurable.

\begin{itemize}
	\setlength\itemsep{.1em}
	\item[i.] If $f$ is integrable on every element of disjoint class $\{E_1,\cdots,E_m\}\subseteq \css{M}$, then
	\begin{equation}
	\int_{\bigcup_{i=1}^mE_i}f=\sum_{{i}=1}^{m} \int_{E_i}f\,;
	\end{equation}
	\item[ii.] If $f$ is integrable on $E\in\css{M}$ and $\fua{\mu}{E}=0$, then
	\begin{equation}
	\int_{E}f =0\,;
	\end{equation}
	 \item[iii.] If $f$ and $g$ are integrable on $E\in\css{M}$, then
	 \begin{equation}
	 \int_{E}(\alpha f +\beta g) = \alpha\int_{E}f+\beta\int_{E}g \,;
	 \end{equation}
\end{itemize}

{\footnotesize
\begin{proof}
Considering equality \eqref{eq:measuSum}, definition \eqref{eq:integ} and mapping $\map{h}{M}{\real^+}$, we have
\begin{align*}
\int_{\bigcup_{i=1}^mE_i}h &= \lim_{k\to \infty}\int_{\bigcup_{i=1}^mE_i} s_k\\
&=\lim_{k\to \infty} \sum_{j=1}^n\alpha_{j}^{(k)}\fua{\mu}{A_j^{(k)}\cap\bigcup_{i=1}^m E_i}\\
&=\sum_{i=1}^m\lim_{k\to \infty} \sum_{j=1}^n\alpha_{j}^{(k)}\fua{\mu}{ E_i\cap A_j^{(k)}}\\
&=\sum_{i=1}^m\lim_{k\to \infty} \int_{E_i} s_k = \sum_{i=1}^m \int_{E_i} h\,.
\end{align*}
Therefore, we prove item i by 
\begin{equation*}
\int_{\bigcup_{i=1}^mE_i}f = \int_{\bigcup_{i=1}^mE_i}f^+ -\int_{\bigcup_{i=1}^mE_i}f^- = \sum_{i=1}^m \int_{E_i} f^+ - \sum_{i=1}^m \int_{E_i} f^- = \sum_{i=1}^m \int_{E_i} f\,.
\end{equation*}
Now, since $\mu$ is nonnegative and additive, that is $\fua{\mu}{X\cup Y}=0\Longleftrightarrow\fua{\mu}{X}=\fua{\mu}{Y}=0$, and partitioned set $A$ on \eqref{eq:defInteSimp} is arbitrary, then choosing $E$ to be partitioned leads to equalities
\begin{equation*}
\int_E f^+ = \lim_{k\to \infty} \sum_{i=1}^n\alpha_{i}^{(k)}\fua{\mu}{ E\cap E_i^{(k)}} = \lim_{k\to \infty} \sum_{i=1}^n\alpha_{i}^{(k)}\fua{\mu}{E_i^{(k)}} = 0
\end{equation*}
and, by the same development, $\int_E f^-=0$, which prove item ii. Now, we prove item iii. It is easy to obtain equality
\begin{equation*}
\alpha\int_{E}f+\beta\int_{E}g = (\alpha\int_{E}f^++\beta\int_{E}g^+) - (\alpha\int_{E}f^-+\beta\int_{E}g^-)\,.
\end{equation*}
Therefore, it suffices to prove only one of the nonnegative terms. Let's specify two partitions $\{A_1^{(k)},\cdots,A_n^{(k)}\}$ and $\{B_1^{(k)},\cdots,B_m^{(k)}\}$ of a set $A^{(k)}\subset M$ which define two step functions $s_k=\sum_{{i}=1}^n\alpha^{(k)}_i\tnr{1}_{A_i^{(k)}}$ and $t_k=\sum_{{j}=1}^m\beta^{(k)}_j\tnr{1}_{B_j^{(k)}}$ in such a way that 
\begin{align*}
\alpha\int_{E}f^++\beta\int_{E}g^+&=\lim_{k\to\infty}\alpha\int_{E}s_k + \lim_{k\to\infty}\beta\int_{E}t_k\\
&=\lim_{k\to\infty}\sum_{i=1}^n\alpha\alpha_i^{(k)}\fua{\mu}{E\cap A_i^{(k)}}+\lim_{k\to\infty}\sum_{j=1}^m\beta\beta_j^{(k)}\fua{\mu}{E\cap B_j^{(k)}}\\
&=\lim_{k\to\infty}\sum_{i=1}^n\sum_{j=1}^m\alpha\alpha_i^{(k)}\fua{\mu}{E\cap A_i^{(k)}\cap B_j^{(k)}} + \sum_{i=1}^n\sum_{j=1}^m\beta\beta_j^{(k)}\fua{\mu}{E\cap A_i^{(k)}\cap B_j^{(k)}}\\
&=\lim_{k\to\infty}\sum_{i=1}^n\sum_{j=1}^m(\alpha\alpha_i^{(k)}+\beta\beta_j^{(k)})\fua{\mu}{E\cap A_i^{(k)}\cap B_j^{(k)}}.
\end{align*}
Let's label this result with (a) and proceed with the following development:
\begin{align*}
\alpha s_k+\beta t_k&=\sum_{{i}=1}^n\alpha\alpha^{(k)}_i\tnr{1}_{A_i^{(k)}}+\sum_{{j}=1}^m\beta\beta^{(k)}_j\tnr{1}_{B_j^{(k)}}\\
&=\sum_{{i}=1}^n\sum_{{j}=1}^m\alpha\alpha^{(k)}_i\tnr{1}_{A_i^{(k)}\cap B_j^{(k)}}+\sum_{{i}=1}^n\sum_{{j}=1}^m\beta\beta^{(k)}_j\tnr{1}_{A_i^{(k)}\cap B_j^{(k)}}\\
&=\sum_{{i}=1}^n\sum_{{j}=1}^m(\alpha\alpha_i^{(k)}+\beta\beta_j^{(k)})\tnr{1}_{A_i^{(k)}\cap B_j^{(k)}}
\end{align*}
from which we can write that 
\begin{align*}
\int_E\alpha s_k+\beta t_k&=
\end{align*}

\end{proof}}

