\chapter{Continuum Kinematics}

bla, bla, bla, bla


\section{The Continuum Model}


In a restricted and objective way, the study of the geometry of motion, or Kinematics, follows the subject-object philosophical approach, where the subject is merely an observer and the object the thing observed. In the act of kinematically describing a mechanical event, the observer makes use of two fundamental concepts: position and time. In this sense, \textsb{space}\index{space!physical} is the context or the framework that enables the subject to perform its positional identifications, when either a certain region of this space or a specific physical body are the observed objects. In Newtonian Mechanics, which is our concern here, the space in which the whole universe belongs is considered to be \textsb{absolute}\index{space!absolute}, that is, it has a fixed size, it is completely at rest and not affected by its objects, being a totally independent entity. Any observer attached to the absolute space is also called absolute\index{observer!absolute}. 

When performing a certain measurement on the object under observation, the observer notes that changes may occur: given a point in space or in a physical body, there may be a set of different measurement values on this point that develop one after another, sequentially. \textsb{Time}\index{time} is the concept that enables the observer to describe these changes by identifying which measurement value follows the other in such a way as to build a set of successive measurement values for every observed point. There is always a specific instance or value of time, called \textsb{instant}\index{instant!of time}, that can label a certain measurement value; in other words, time is infinite and infinitely divisible, or an infinite unlimited complete set, in mathematical terminology. When a limited portion of this set is selected to describe a certain mechanical event, we call this portion a \textsb{period}. If the observer measures positions of the points of a physical body, the set of points labeled by the same instant is called a \textsb{configuration}\index{configuration} of this body and a set of these configurations in a given period of time is called a \textsb{motion}\index{motion} of the body. Still in the context of Newtonian Mechanics, time is considered to be unstoppable, always progressing indefinitely, and the rate of this progress is the same for any observer; for this reason, time is also considered absolute. Therefore, we can say that every observer's clock is synchronized with a unique reference clock. As a consequence of these features of Newtonian time, an observer describing a certain mechanical event can never label two distinct measurement values on the same point of an object by the same instant: simultaneity makes sense only for different points. 

An observer becomes an object when another observer can describe its Kinematics. That said, an observer at rest or moving with constant velocity relative to an absolute observer is called \textsb{inertial}\index{observer!inertial}: it is only from the point of view of an inertial observer that Newton's Law of Inertia is valid.  








% Para poder fazer calculo diferencial a partir de um corpo físico, a estrutura do corpo precisa atender ao requisito de completude de um espaço de Banach
% O problema é que um corpo físico não é intrinsecamente completo. Dependendo da escala que se observe, a estrutura da matéria se revelará não é completa como um espaço de Banach pois apresenta mais vazios que conteúdo.