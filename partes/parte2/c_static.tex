\chapter{Continuum Kinematics}
% explicar que daqui para a frente os conceitos matemáticos utilizados estão todos explicados na primeira parte. qualquer 
% outro conceito matemático não explicado na primeira parte, será explicado aqui.
bla, bla, bla, bla


\section{Continuum Mechanics: a Mathematical Model}

In classical terms, Mechanics is the physical science of \textsb{motion}\index{motion} concerned with its descriptive and predictive aspects. The descriptive study of motion, or Kinematics, deals mainly with its geometric measurements while the predictive study, Dynamics, considers its causes. For both of these aspects, the notion of \textsb{time}\index{time} is fundamental, enabling the very idea of motion and most of its related concepts, particularly in Kinematics. For the purposes of Dynamics, \textsb{force}\index{force} is the main concept, being the motive of geometric changes in time, a causality relation, called the \textsb{Laws of Motion}\index{motion! Laws of}, which aims to be as generic as possible and independent of material specificities. But when the influence of these specificities can not be disregarded for the mechanical phenomena under study, particular \textsb{constitutive}\index{description!constitutive} relations have to be considered. 


In a restricted and objective way, Kinematics adopts the common subject-object philosophical approach, where the subject is merely an \textsb{observer}\index{observer} and the object the thing observed. In the act of kinematically describing a mechanical event, the observer makes use of two fundamental concepts: position and time. In Newtonian or Classical Mechanics, which is our concern here, \textsb{space}\index{space!physical}, mathematically modeled as a three dimensional Euclidean space, is the context or the framework that enables the subject to perform its positional identifications, when either a certain region of this space or a specific physical entity are the observed objects. The biggest physical space, which includes the whole universe, is defined to be \textsb{absolute}\index{space!absolute}, that is, it has a fixed size, it is completely at rest and not affected by its elements, being a totally independent entity. Any observer attached to the absolute space is also called absolute\index{observer!absolute}. 

When performing a certain measurement on the object under observation, the observer notes that changes may occur: given a point in space or a material element of a physical entity, there may be a set of different measurement values on this point or element that develop one after another, sequentially. Time is the concept that enables the observer to describe these changes by identifying which measurement value follows the other in such a way as to build a set of successive measurement values for every observed point or element.  There is always a specific instance or value of time, called \textsb{instant}\index{instant!of time}, that can label a certain measurement value; in other words, time is infinite and infinitely divisible\footnote{On this subject, we cannot help but quoting the great writer Leo Tolstoy: \textit{``For human reason, absolute continuity of movement is incomprehensible. Man begins to understand the laws of any kind of movement only when he examines the arbitrarily chosen units of that movement. But at the same time it is from this arbitrary division of continuous movement into discrete units that the greater part of human errors proceeds... A new branch of mathematics, having attained to the art of dealing with infinitesimal quantities in other, more complex problems of movement as well, now gives answers to questions that used to seem insoluble. This new branch of mathematics, unknown to the ancients, in examining questions of movement, allows for infinitesimal quantities, that is, such as restore the main condition of movement (absolute continuity), and thereby corrects the inevitable error that human reason cannot help committing when it examines discrete units of movement instead of continuous movement.''} (\aut{Tolstoy}\cite{tolstoi_2014_1}, p. 955)}, or an infinite unlimited complete set, conveniently modeled as a real field. When a limited portion of this set is selected to describe a certain mechanical event, we call this portion a \textsb{period}. If the observer measures the positions of the material elements of a limited physical entity, a biunivocal relation of these elements with space points in a given instant is called
a \textsb{configuration}\index{configuration} of this physical entity and the set of these space points is said to be a \textsb{shape}\index{shape} of this entity. Moreover, a set of these shapes on a certain period of time is defined to be a motion of the physical entity. Still in the context of Newtonian Mechanics, time is considered to be unstoppable, always progressing indefinitely, and the rate of this progress is the same for any observer: time is also considered absolute. Therefore, we can say that every observer's clock is synchronized with a unique reference clock. As a consequence of the ever increasing Newtonian time, an observer describing a certain mechanical event can never label two distinct measurement values relative to the same point of an object by the same instant: simultaneity here makes sense only for different points. Moreover, an observer may eventually become an object when another observer can describe its motion. That said, an observer at rest or moving with constant velocity relative to an absolute observer is called \textsb{inertial}\index{observer!inertial}: it is only from the point of view of an inertial observer that Newton's Law of Inertia is valid.   

Following Hermann Weyl, ``\emph{space and time are commonly regarded as the forms of existence of the real world, \textsb{matter}\index{matter} as its substance.}\footnote{See \aut{Weyl}\cite{weyl_1952_2}, p.1.}'' The purpose of the Newtonian concept of \textsb{mass}\index{mass} is to measure the quantity of this substance, as a tangible entity. Then, we impose that every limited portion of matter, called a body, must have a certain mass. Moreover, it is observed that what constitutes matter, the specific material it is made of, influences its mechanical behavior. In this sense, a mechanical description concerned with material peculiarities is called constitutive. For the purposes of this book, matter is considered to be impenetrable, that is, two simultaneous configurations can never define shapes occupying the same portion of space, either totally or partially. Another topic on the study of matter is its structure, that is, its building blocks, commonly called particles, and the way they are organized. For some reason, the popularity of this topic promoted the preconception, even in academic circles, that a physical theory which seeks to describe the overall behavior of matter becomes more reliable, or is much closer to reality, when it considers structural phenomena. In our opinion, this idea tries to attach what it is not attachable. A mechanical theory, expressed as a mathematical model, is good if its results are sufficiently close to experimental data, when available, and if it describes an ample set of physical phenomena, regardless whether structural variables are considered or not. In studying the mechanical behavior of matter, the approach called Continuum Mechanics, which considers a body a continuum, presents both of these virtues, notwithstanding it totally disregards structural phenomena. 

In the context of Continuum Mechanics, where the physical space is a three-dimensional Euclidean affine space, time is a real field and body is a continuum, we conclude that a configuration of a given body is a bijection -- labeled by a real scalar representing an instant of time -- that maps the body to a shape, which is a bounded subset of the three-dimensional Euclidean affine space. In this sense, the body is here considered to be a formless portion of matter, related through configuration to an Euclidean geometric shape in a certain instant of time. Considering the biunivocal property of the configuration and since every portion of matter has mass, we can define on an arbitrary shape of a given body a mass density functional, whose volume integral results the mass of the body relative to the shape in question. For the purposes of our study, we impose that the mass of the body is configuration independent, that is, the mass calculated on any arbitrary shape is always the mass of the body\footnote{In motions where a portion of the mass of a body is burned in order to propel another portion, the above restriction is still valid because any shape of this body as a whole must include not only the propelled portion of its mass, but also that which is burned.}.


\section{Deformation}\index{deformation}

Concerning the subject of this chapter, the intuitive concepts and definitions introduced in the previous section will now be detailed from the mathematical material presented in the first part of this book. In our study of Mechanics, the absolute space is a three-dimensional Euclidean affine space $\eamd{U}{R}{3}$ whose affine coordinate system is the absolute observer. A complete metric space $(B,\varrho)$ is called a body \gloref{bdy} if it is connected and totally bounded; in other words, if it is a continuum. The kinematical description of a body is performed in a geometric space and is enabled by the bijective mapping $\map{\gloref{confi}}{\mathfrak{B}\times\gloref{prd}}{\mathcal{B}_t}$, where $\tempo\subseteq\real$ is the set constituted by instants of time, bijection $\chi$ is a configuration of the body and its shape $\gloref{shp}:=(\eamd{B}{R}{3})_t$ is a subset, at instant $t\in\tempo$, of the absolute universe $\eamd{U}{R}{3}$, defined by an oriented Euclidean vector space $(U_\real,\tnr{A}_O)$, where $O$ is its natural basis. Among other reasons, since surface integrals will be a fundamental tool in this study, we shall deal only with configurations that define shapes bounded by Lipschitz surfaces. In order to identify the particles of body $\mathfrak{B}$ by geometric points, a certain shape $\overline{\mathcal{B}}$ of this body is chosen to be a reference, in such a way that an arbitrary particle $\text{x}\in\mathfrak{B}$ is uniquely identified by the point $\fua{\overline{\chi}}{\text{x}}:=\fua{\chi}{\text{x},\overline{t}}$ of $\overline{\mathcal{B}}$, where $\overline{t}$ is a fixed instant of time taken as reference. This one-to-one identification allows us to call the reference shape $\overline{\mathcal{B}}$ also a body, but now, a body with affine Euclidean features. In order to make shapes viable for mathematical calculations, we still specify that
\begin{equation}
\tnr{\mathcal{B}}_t:=\{\fua{\upsilon_o}{x}:\forall x\in\mathcal{B}_t\}\,,
\end{equation}
where $\upsilon_o$ ``vectorizes'' the points of $\mathcal{B}_t$ relative to an origin $o\in\eamd{U}{R}{3}$, according to definition \eqref{eq:funVect}. Since $\upsilon_o$ is a bijection, the sets of vectors $\tnr{\overline{\mathcal{B}}}$ and $\tnr{\mathcal{B}}_t$ will also be called body and shape respectively.

Given the mapping $\map{\tnr{\chi}}{\tnr{\overline{\mathcal{B}}}\times\tempo}{\tnr{\mathcal{B}}_t}$, bijection $\tnr{\chi}$ is called a deformation of $\tnr{\overline{\mathcal{B}}}$ when it is a $\mathcal{C}^{2}$-diffeomorphism that observes $\fua{\tnr{\chi}}{\overline{\vto{x}},\overline{t}}=\overline{\vto{x}}$, $\forall\,\overline{\vto{x}}\in\tnr{\overline{\mathcal{B}}}$. For most of the forthcoming concepts to be presented, it is convenient to work with an arbitrary instant of time $t\in\tempo$ and a bijective mapping $\map{\gloref{defInst}}{\tnr{\overline{\mathcal{B}}}}{\tnr{\mathcal{B}}_t}$ with rule $\fua{\tnr{\chi}_t}{\vto{x}}=\fua{\tnr{\chi}}{\vto{x},t}$, from which we conclude that $\tnr{\chi}_t$ is also a $\mathcal{C}^{2}$-diffeomorphism that observes $\tnr{\chi}_{\overline{t}}=\tnr{i}_{\tnr{\overline{\mathcal{B}}}}$. Given an arbitrary point $\overline{\vto{u}}\in\tnr{\overline{\mathcal{B}}}$ and conveniently chosen non zero vectors $\overline{\vto{x}}_i\in\tnr{\overline{\mathcal{B}}}$, from which $\overline{\vto{u}}_i=\overline{\vto{x}}_i-\overline{\vto{u}}$ constitute a linearly independent set $\{ \overline{\vto{u}}_1,\overline{\vto{u}}_2,\overline{\vto{u}}_3\}$, since body points cannot collapse during deformation, according to \eqref{eq:volumeCross} volume
\begin{equation}
{\tnr{A}_O}[\fua{\tnr{\chi}_t}{\overline{\vto{u}}_1},\fua{\tnr{\chi}_t}{\overline{\vto{u}}_2},\fua{\tnr{\chi}_t}{\overline{\vto{u}}_3}]\neq 0\,.
\end{equation}
The continuity and bijectivity of $\tnr{\chi}_t$ respectively ensures that Lipschitz surfaces are preserved\footnote{See \aut{Ciarlet}\cite{ciarlet_1988_2_2}, theorem 1.2-8, p.16.} and no distinct elements of the body become indistinct or collapsed in the shape; while the differentiability of class two provides the minimum level of regularity required by future concepts. The entities and their relationships presented so far are depicted in figure \ref{fg:deformacao}. 
\begin{figure}[!ht]
\centering
\begin{center}
\scalebox{.72}{\input{partes/figs/deformacao.pstex_t}}
\end{center}
\titfigura{Body configurations, vectorizations and deformations.}\label{fg:deformacao}
\end{figure}
From this figure, it is possible to conclude that if both configurations $\overline{\chi}$ and $\chi_t$ are specified, a deformation $\tnr{\chi}_t$ can be defined by 
\begin{equation}
\tnr{\chi}_t = \upsilon_o\circ\chi_t\circ\overline{\chi}^{\,-1}\circ\upsilon_o^{-1}\,.
\end{equation}
Moreover, concerning the elements of the domains represented in the figure, we say that a material particle $\text{x}\in\mathfrak{B}$, represented by the reference point $\overline{x}=\fua{\overline{\chi}}{\text{x}}$ or by the reference vector $\vto{\overline{x}}:=\fua{\upsilon_o\circ\overline{\chi}}{\text{x}}$, occupies the place $x:=\fua{\chi_t}{\text{x}}$ or $\vto{x}=\fua{\upsilon_o\circ\chi_t}{\text{x}}$. In order to simplify our study, since deformation was defined as a vector function, we shall mostly use vectors to denote body and shape elements.  

Considering the previous regularity conditions and an open set $\pmb{\mathcal{U}}\subset\tnr{\overline{\mathcal{B}}}$, the local sensitivity of $\tnr{\chi}_t$ measured by the vector function in $\map{\nabla\tnr{\chi}_t}{\pmb{\mathcal{U}}}{\ete{\real}{U^{2}}}$, called \textsb{deformation gradient}\index{deformation!gradient}, has fundamental importance in Continuum Kinematics because it bases the concept of strain. Given an arbitrary reference point $\overline{\vto{u}}\in\pmb{\mathcal{U}}$, the second order tensor $\gloref{defGrad}$ is said to be the deformation gradient at $\overline{\vto{u}}$ and then, from equality \eqref{eq:gradFuncRep}, we have scalar
\begin{equation}
[\fua{\nabla\tnr{\chi}_t}{\overline{\vto{u}}}](\overline{\vto{x}},\overline{\vto{y}})=\overline{\vto{x}}\cdot\fua{{\tnr{\chi}'_t}_{\overline{\vto{u}}}}{\overline{\vto{y}}}\,,\,\,\forall\, \overline{\vto{x}},\overline{\vto{y}}\in\pmb{\mathcal{U}}\,.
\end{equation}
Now, at reference instant $t=\overline{t}$, we know that $\tnr{\chi}_{\overline{t}}=\tnr{i}_{\tnr{\overline{\mathcal{B}}}}$ and therefore, from a basic property of directional derivatives, equality ${\tnr{\chi}'_{\overline{t}} }_{\overline{\vto{u}}}=\tnr{i}_{\tnr{\overline{\mathcal{B}}}}$ holds. But since ${\tnr{\chi}'_{\overline{t}} }_{\overline{\vto{u}}}$ is the representative function of second order Euclidean tensor $\fua{\nabla\tnr{\chi}_{\overline{t}}}{\overline{\vto{u}}}$, from \eqref{eq:det3D} we conclude that volume change $\textnormal{Det}[\fua{\nabla\tnr{\chi}_{\overline{t}}}{\overline{\vto{u}}}]=1$. Moreover, since deformation $\tnr{\chi}_{t}$ is a diffeomorphism by definition, the property relative to equality \eqref{eq:indentDiffeo} is valid and then $\textnormal{Det}[\fua{\nabla\tnr{\chi}_{t}}{\overline{\vto{u}}}]\neq 0$ because $\fua{\nabla\tnr{\chi}_{t}}{\overline{\vto{u}}}\neq \tnr{0}$. Therefore, considering $\tnr{\overline{\mathcal{B}}}$ the initial undeformed shape of a given body motion, we can state that the volume change       
\begin{equation}\label{eq:detPositive}
\textnormal{Det}[\fua{\nabla\tnr{\chi}_t}{\overline{\vto{u}}}]>0\,,
\end{equation}
since deformation $\tnr{\chi}$ and its derivative are continuous: during motion, volume change $\textnormal{Det}[\fua{\nabla\tnr{\chi}_{t}}{\overline{\vto{u}}}]$, which is one at $t=\overline{t}$ and different from zero otherwise, cannot ``jump'' zero and be negative. From \eqref{eq:det3D} and this previous inequality, we can conclude that vector function ${\tnr{\chi}'_t}_{\overline{\vto{u}}}$ is orientation-preserving\footnote{About this topic, professor Spivak wrote: ``\emph{The non-singular linear maps $\map{f}{V}{V}$ from a finite dimensional vector space to itself fall into two groups, those with $\det f>0$ and those with $\det f<0$; linear transformations in the first group are called \textsb{orientation preserving} and the others are called \textsb{orientation reversing}...There is no way to pass continuously between these two groups...}'' (\aut{Spivak}\cite{spivak_2005_1}, p. 84)}, a consequence that is used by some authors as a local necessary condition for the vector function $\tnr{\chi}_t$ to be considered a deformation.  


A deformation $\tnr{\chi}_t$ is called \textsb{homogeneous}\index{deformation!homogeneous} when its gradient is constant on $\tnr{\overline{\mathcal{B}}}$, that is, tensor $\fua{\nabla\tnr{\chi}_t}{\overline{\vto{u}}}=\tnr{F}$ for all $\overline{\vto{u}}\in\tnr{\overline{\mathcal{B}}}$,  and then from \eqref{eq:derivSameArg} we have
\begin{equation}
(\overline{\vto{x}}-\overline{\vto{y}})^*\hat{\odot}_1\tnr{F}=\fua{D\tnr{\chi}_t}{\overline{\vto{x}}-\overline{\vto{y}}} = \fua{D\tnr{\chi}_t}{\overline{\vto{x}}} - \fua{D\tnr{\chi}_t}{\overline{\vto{y}}}\,,\,\forall\,\overline{\vto{x}},\overline{\vto{y}}\in\tnr{\overline{\mathcal{B}}}\,, 
\end{equation}
since $D\tnr{\chi}_t$ is linear.
 

