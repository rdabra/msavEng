\chapter{Continuum Kinematics}
% explicar que daqui para a frente os conceitos matemáticos utilizados estão todos explicados na primeira parte. qualquer 
% outro conceito matemático não explicado na primeira parte, será explicado aqui.
bla, bla, bla, bla


\section{Continuum Mechanics: a Mathematical Model}

In classical terms, Mechanics is the physical science of \textsb{motion}\index{motion} concerned with its descriptive and predictive aspects. The descriptive study of motion, or Kinematics, deals mainly with its geometric measurements while the predictive study, Dynamics, considers its causes. For both of these aspects, the notion of \textsb{time}\index{time} is fundamental, enabling the very idea of motion and most of its related concepts, particularly in Kinematics. For the purposes of Dynamics, \textsb{force}\index{force} is the main concept, being the motive of geometric changes in time, a causality relation, called the \textsb{Laws of Motion}\index{motion! Laws of}, which aims to be as generic as possible and independent of material specificities. But when the influence of these specificities can not be disregarded for the mechanical phenomena under study, particular \textsb{constitutive}\index{description!constitutive} relations have to be considered. 


In a restricted and objective way, Kinematics adopts the common subject-object philosophical approach, where the subject is merely an \textsb{observer}\index{observer} and the object the thing observed. In the act of kinematically describing a mechanical event, the observer makes use of two fundamental concepts: position and time. In Newtonian or Classical Mechanics, which is our concern here, \textsb{space}\index{space!physical}, mathematically modeled as a three dimensional Euclidean space, is the context or the framework that enables the subject to perform its positional identifications, when either a certain region of this space or a specific physical entity are the observed objects. The biggest physical space, which includes the whole universe, is defined to be \textsb{absolute}\index{space!absolute}, that is, it has a fixed size, it is completely at rest and not affected by its elements, being a totally independent entity. Any observer attached to the absolute space is also called absolute\index{observer!absolute}. 

When performing a certain measurement on the object under observation, the observer notes that changes may occur: given a point in space or a material element of a physical entity, there may be a set of different measurement values on this point or element that develop one after another, sequentially. Time is the concept that enables the observer to describe these changes by identifying which measurement value follows the other in such a way as to build a set of successive measurement values for every observed point or element.  There is always a specific instance or value of time, called \textsb{instant}\index{instant!of time}, that can label a certain measurement value; in other words, time is infinite and infinitely divisible\footnote{On this subject, we cannot help but quoting the great writer Leo Tolstoy: \textit{``For human reason, absolute continuity of movement is incomprehensible. Man begins to understand the laws of any kind of movement only when he examines the arbitrarily chosen units of that movement. But at the same time it is from this arbitrary division of continuous movement into discrete units that the greater part of human errors proceeds... A new branch of mathematics, having attained to the art of dealing with infinitesimal quantities in other, more complex problems of movement as well, now gives answers to questions that used to seem insoluble. This new branch of mathematics, unknown to the ancients, in examining questions of movement, allows for infinitesimal quantities, that is, such as restore the main condition of movement (absolute continuity), and thereby corrects the inevitable error that human reason cannot help committing when it examines discrete units of movement instead of continuous movement.''} (\aut{Tolstoy}\cite{tolstoi_2014_1}, p. 955)}, or an infinite unlimited complete set, conveniently modeled as a real field. When a limited portion of this set is selected to describe a certain mechanical event, we call this portion a \textsb{period}. If the observer measures the positions of the material elements of a limited physical entity, a biunivocal relation of these elements with space points in a given instant is called
a \textsb{configuration}\index{configuration} or \textsb{deformation}\index{deformation} of this physical entity and the set of these space points is said to be a \textsb{shape}\index{shape} of this entity. Moreover, a set of distinct shapes on a certain period of time is defined to be the image of a motion of the physical entity. Still in the context of Newtonian Mechanics, time is considered to be unstoppable, always progressing indefinitely, and the rate of this progress is the same for any observer: time is also considered absolute. Therefore, we can say that every observer's clock is synchronized with a unique reference clock. As a consequence of the ever increasing Newtonian time, an observer describing a certain mechanical event can never label two distinct measurement values relative to the same point of an object by the same instant: simultaneity here makes sense only for different points. Moreover, an observer may eventually become an object when another observer can describe its motion. That said, an observer at rest or moving with constant velocity relative to an absolute observer is called \textsb{inertial}\index{observer!inertial}: it is only from the point of view of an inertial observer that Newton's Law of Inertia is valid.   

Following Hermann Weyl, ``\emph{space and time are commonly regarded as the forms of existence of the real world, \textsb{matter}\index{matter} as its substance.}\footnote{See \aut{Weyl}\cite{weyl_1952_2}, p.1.}'' The purpose of the Newtonian concept of \textsb{mass}\index{mass} is to measure the quantity of this substance, as a tangible entity. Then, we impose that every limited portion of matter, called a body, must have a certain mass. Moreover, it is observed that what constitutes matter, the specific material it is made of, influences its mechanical behavior. In this sense, a mechanical description concerned with material peculiarities is called constitutive. For the purposes of this book, matter is considered to be impenetrable, that is, two simultaneous deformations can never define shapes occupying the same portion of space, either totally or partially. Another topic on the study of matter is its structure, that is, its building blocks, commonly called particles, and the way they are organized. For some reason, the popularity of this topic promoted the preconception, even in academic circles, that a physical theory which seeks to describe the overall behavior of matter becomes more reliable, or is much closer to reality, when it considers structural phenomena. In our opinion, this idea tries to attach what it is not attachable. A mechanical theory, expressed as a mathematical model, is good if its results are sufficiently close to experimental data, when available, and if it describes an ample set of physical phenomena, regardless whether structural variables are considered or not. In studying the mechanical behavior of matter, the approach called Continuum Mechanics, which considers a body a continuum, presents both of these virtues, notwithstanding it totally disregards structural phenomena. 

In the context of Continuum Mechanics, where the physical space is a three-dimensional Euclidean affine space, time is a real field and body is a continuum, we conclude from the impenetrability of matter that a deformation of a given body is always a bijection -- labeled by a real scalar representing an instant of time -- that maps the body to a shape, which is a bounded subset of the three-dimensional Euclidean affine space. In this sense, the body is here considered to be a formless portion of matter, related through deformation to an Euclidean geometric shape in a certain instant of time. Considering the biunivocal property of the deformation and since every portion of matter has mass, we can define on an arbitrary shape of a given body a mass density functional, whose volume integral results the mass of the body relative to the shape in question. For the purposes of our study, we impose that the mass of the body is deformation independent, that is, the mass calculated on any arbitrary shape is always the mass of the body\footnote{In motions where a portion of the mass of a body is burned in order to propel another portion, the above restriction is still valid because any shape of this body as a whole must include not only the propelled portion of its mass, but also that which is burned.}.


\section{Deformation}\index{deformation}


Concerning the subject of this chapter, the intuitive concepts and definitions introduced in the previous section will now be detailed from the mathematical material presented in the first part of this book. In our study of Mechanics, the absolute space is a three-dimensional Euclidean affine space $\eamd{U}{R}{3}$ whose affine coordinate system is the absolute observer. A complete metric space $(B,\varrho)$ is called a body \gloref{bdy} if it is connected and totally bounded; in other words, if it is a continuum. During a motion of this body, its shape at instant $t$ is $\mathcal{B}_t:=(\eamd{B}{R}{3})_t$, which is a subset of the absolute universe $\eamd{U}{R}{3}$, defined by an oriented Euclidean vector space $(U_\real,\tnr{A}_O)$, where $O$ is its natural basis. Formally, a \textsb{motion}\index{motion} of $\mathfrak{B}$ is the function in the surjective mapping 
\begin{equation}
\map{\gloref{confi}}{\mathfrak{B}\times\gloref{prd}}{\mathcal{S}}\,,
\end{equation}
where $\tempo\subseteq\real$ is the set constituted by instants of time and image 
\begin{equation}
\mathcal{S}=\bigcup_{t\in\tempo}\mathcal{B}_t\,.
\end{equation}
Since time is an independent variable, for most of the forthcoming concepts to be presented, it is convenient to fix an arbitrary instant $t\in\tempo$ and define a bijective mapping $\map{\chi_t}{\mathfrak{B}}{\mathcal{B}_t}$ where $\fua{{\chi_t}}{\text{x}}=\fua{\chi}{\text{x},{t}}$. Among other reasons, since surface integrals will be a fundamental tool in this study, we shall deal only with deformations that define shapes bounded by Lipschitz surfaces. In order to identify the particles of body $\mathfrak{B}$ by geometric points, we choose a reference instant of time $\overline{t}$ in such a way that an arbitrary particle $\text{x}\in\mathfrak{B}$ is uniquely identified by the point $\fua{\overline{\chi}}{\text{x}}:=\fua{\chi_{\overline{t}}}{\text{x}}$, from which set $\overline{\mathcal{B}}:=\{\fua{\overline{\chi}}{\text{x}}:\text{x}\in\mathfrak{B}\}$ is called a reference shape. It is important to note that the reference instant is not necessarily the initial instant $t_0$ of the period $[t_0,t_f]$ under study, that is, $t_f\geqslant t_0\geqslant\overline{t}\,$; a condition that enables body $\mathfrak{B}$ to have three notable shapes: $\overline{\mathcal{B}}$, $\mathcal{B}_{t_0}$ and $\mathcal{B}_{t_f}$. The one-to-one identification of particles by points allows us to call the reference shape also a body, but now, a body with affine Euclidean features. In order to make shapes viable for mathematical calculations, we still specify that
\begin{alignat}{3} 
\tnr{\overline{\mathcal{B}}}:=\{\fua{\upsilon_o}{x}:\forall x\in\overline{\mathcal{B}}\}&\qquad \text{and}\qquad & \tnr{\mathcal{B}}_t:=\{\fua{\upsilon_o}{x}:\forall x\in\mathcal{B}_t\}\,,
\end{alignat}
where $\upsilon_o$ ``vectorizes'' the points of $\overline{\mathcal{B}}$ and $\mathcal{B}_t$ relative to an origin $o\in\eamd{U}{R}{3}$, according to definition \eqref{eq:funVect}. Since $\upsilon_o$ is a bijection, the sets of vectors $\tnr{\overline{\mathcal{B}}}$ and $\tnr{\mathcal{B}}_t$ will also be called body and shape respectively. In this vector framework where $\tnr{\mathcal{S}}$ is the vector image of the motion, we impose that the vector motion $\tnr{\chi}$ in mapping 
\begin{equation}\label{eq:vecMotion}
\map{\tnr{\chi}}{\tnr{\overline{\mathcal{B}}}\times\tempo}{\tnr{\mathcal{S}}}
\end{equation}
is a $\mathcal{C}^{3}$ surjection that observes $\fua{\tnr{\chi}}{\overline{\vto{x}},\overline{t}}=\overline{\vto{x}}$. By fixing time, we can define the bijective mapping $\map{\gloref{defInst}}{\tnr{\overline{\mathcal{B}}}}{\tnr{\mathcal{B}}_t}$ where function is called the deformation\index{deformation} or deformation\index{deformation} of $\tnr{\overline{\mathcal{B}}}$ at $t$ if it obeys equality 
\begin{equation}
\fua{\tnr{\chi}_t}{\overline{\vto{x}}}=\fua{\tnr{\chi}}{\overline{\vto{x}},t}\,,\,\forall\, \overline{\vto{x}}\in\tnr{\overline{\mathcal{B}}}\,,
\end{equation}
from which we conclude that the univariate motion $\tnr{\chi}_t$ observes restriction $\tnr{\chi}_{\overline{t}}(\overline{\vto{x}})=\overline{\vto{x}}$ and results a $\mathcal{C}^{3}$-diffeomorphism from the impenetrability of matter. Given an arbitrary point $\overline{\vto{u}}\in\tnr{\overline{\mathcal{B}}}$ and conveniently chosen non zero vectors $\overline{\vto{v}}_i\in\tnr{\overline{\mathcal{B}}}$, from which $\overline{\vto{u}}_i=\overline{\vto{v}}_i-\overline{\vto{u}}$ constitute a linearly independent set $\{ \overline{\vto{u}}_1,\overline{\vto{u}}_2,\overline{\vto{u}}_3\}$, since body points cannot collapse during deformation, according to \eqref{eq:volumeCross} volume
\begin{equation}
{\tnr{A}_O}[\fua{\tnr{\chi}_t}{\overline{\vto{u}}_1},\fua{\tnr{\chi}_t}{\overline{\vto{u}}_2},\fua{\tnr{\chi}_t}{\overline{\vto{u}}_3}]\neq 0\,.
\end{equation}
The continuity and bijectivity of $\tnr{\chi}_t$ respectively ensures that Lipschitz surfaces are preserved\footnote{See \aut{Ciarlet}\cite{ciarlet_1988_2_2}, theorem 1.2-8, p.16.} and no distinct elements of the body become indistinct or collapsed in the shape; while the differentiability of class two provides the minimum level of regularity required by future concepts. The entities and their relationships presented so far are depicted in figure \ref{fg:deformacao}. 
\begin{figure}[!ht]
\centering
\begin{center}
\scalebox{.72}{\input{partes/figs/deformacao.pstex_t}}
\end{center}
\titfigura{Body shapes, vectorizations and deformations.}\label{fg:deformacao}
\end{figure}
From this figure, it is possible to conclude that if both deformations $\overline{\chi}$ and $\chi_t$ are specified, a deformation $\tnr{\chi}_t$ can be defined by 
\begin{equation}
\tnr{\chi}_t = \upsilon_o\circ\chi_t\circ\overline{\chi}^{\,-1}\circ\upsilon_o^{-1}\,.
\end{equation}
Moreover, concerning the elements of the domains represented in the figure, we say that a material particle $\text{x}\in\mathfrak{B}$, represented by the \textsb{material point}\index{point!material} $\overline{x}=\fua{\overline{\chi}}{\text{x}}$ or by the \textsb{material vector}\index{vector!material} $\vto{\overline{x}}:=\fua{\upsilon_o\circ\overline{\chi}}{\text{x}}$, occupies the \textsb{spatial place}\index{spatial place} $x:=\fua{\chi_t}{\text{x}}$ or $\vto{x}=\fua{\upsilon_o\circ\chi_t}{\text{x}}$. In order to simplify our study, since deformation was defined as a vector function, we shall mostly use vectors to denote body and shape elements. In this context, when a vector function is defined on a material domain $\tnr{\overline{\mathcal{B}}}$ or on a spatial domain $\tnr{\mathcal{B}}_t$, it is called \textsb{Lagrangian}\index{function!Lagrangian}\index{Lagrangian function} or \textsb{Eulerian}\index{function!Eulerian}\index{Eulerian function} respectively. 

Considering the previous regularity conditions and an open set $\pmb{\mathcal{W}}\subset\tnr{\overline{\mathcal{B}}}$, the derivative of $\tnr{\chi}_t$ at an arbitrary particle $\overline{\vto{u}}\in\pmb{\mathcal{W}}$ is the vector function $(\tnr{\chi}_t)'_{\overline{\vto{u}}}$ such that, from equality \eqref{eq:derivVecGradVec}, 
\begin{equation}
\fua{(\tnr{\chi}_t)'_{\overline{\vto{u}}}}{\vto{x}}= \fua{ \nabla\tnr{\chi}_t}{\overline{\vto{u}}}\hat{\odot}_1\vto{x}^*\,,
\end{equation}
where the function in $\map{\nabla\tnr{\chi}_t}{\pmb{\mathcal{W}}}{\ete{\real}{U^{2}}}$, called \textsb{deformation gradient}\index{deformation!gradient}, has fundamental importance in Continuum Kinematics because it bases the concept of strain. The second order tensor $\fua{ \nabla\tnr{\chi}_t}{\overline{\vto{u}}}$ is said to be the deformation gradient at $\overline{\vto{u}}$ and, since deformation $\tnr{\chi}_t$ is a diffeomorphism, we can recall property \eqref{eq:gradDiffeo2}, from which tensor
\begin{equation}\label{eq:invGradDef}
\fua{\nabla\tnr{\chi}_t^{-1}}{\vto{u}} = \fua{\nabla\tnr{\chi}_t}{\overline{\vto{u}}}^{-1}\,.
\end{equation}
Now, at reference instant $t=\overline{t}$, we know that $\tnr{\chi}_{\overline{t}}(\overline{\vto{x}})=\overline{\vto{x}}=\tnr{i}_{\tnr{\overline{\mathcal{B}}}}(\overline{\vto{x}})$ and therefore, from a basic property of directional derivatives, equality ${\tnr{\chi}'_{\overline{t}} }_{\overline{\vto{u}}}=\tnr{i}_{\tnr{\overline{\mathcal{B}}}}$ holds. But since ${\tnr{\chi}'_{\overline{t}} }_{\overline{\vto{u}}}$ is the representative function of second order Euclidean tensor $\fua{\nabla\tnr{\chi}_{\overline{t}}}{\overline{\vto{u}}}$, from \eqref{eq:det3D} we conclude that the infinitesimal or local volume change $\textnormal{Det}[\fua{\nabla\tnr{\chi}_{\overline{t}}}{\overline{\vto{u}}}]=1$. Moreover, since deformation $\tnr{\chi}_{t}$ is a diffeomorphism by definition, the property relative to equality \eqref{eq:indentDiffeo} is valid and then $\textnormal{Det}[\fua{\nabla\tnr{\chi}_{t}}{\overline{\vto{u}}}]\neq 0$ because $\fua{\nabla\tnr{\chi}_{t}}{\overline{\vto{u}}}\neq \tnr{0}$. Therefore, we can state that the local volume change       
\begin{equation}\label{eq:detPositive}
\textnormal{Det}[\fua{\nabla\tnr{\chi}_t}{\overline{\vto{u}}}]>0
\end{equation}
because motion $\tnr{\chi}$ and its derivative are continuous: during a motion where $\overline{t}=t_0$, the local volume change $\textnormal{Det}[\fua{\nabla\tnr{\chi}_{t}}{\overline{\vto{u}}}]$, which is one at $t=t_0$ and different from zero otherwise, cannot ``jump'' zero and be negative. From \eqref{eq:det3D} and this previous inequality, we can conclude that vector function ${\tnr{\chi}'_t}_{\overline{\vto{u}}}$ is orientation-preserving\footnote{About this topic, professor Spivak wrote: ``\emph{The non-singular linear maps $\map{f}{V}{V}$ from a finite dimensional vector space to itself fall into two groups, those with $\det f>0$ and those with $\det f<0$; linear transformations in the first group are called \textsb{orientation preserving} and the others are called \textsb{orientation reversing}...There is no way to pass continuously between these two groups...}'' (\aut{Spivak}\cite{spivak_2005_1}, p. 84)}, a consequence that is used by some authors as a local necessary condition for the vector function $\tnr{\chi}_t$ to be considered a deformation. A deformation  $\tnr{\chi}_t$ where the local volume change
\begin{equation}
\textnormal{Det}[\fua{\nabla\tnr{\chi}_t}{\overline{\vto{u}}}]=1
\end{equation}
is said to be \textsb{isochoric}\index{deformation!isochoric}\index{isochoric deformation} on $\overline{\vto{u}}$ or just isochoric when this equality is valid for all $\overline{\vto{u}}\in\tnr{\overline{\mathcal{B}}}$. 


Recalling affinities and their related concepts presented in section \ref{sec:aff}, a motion $\tnr{\varphi}$ is said to be \textsb{affine}\index{affine!motion}\index{motion!affine} if it is described by the rule 
\begin{equation}\label{eq:affDeform}
\fua{\tnr{\varphi}}{\overline{\vto{x}},t} =[\fua{\nabla\tnr{\varphi}_t}{\overline{\vto{x}}}^\text{T}\,\hat{\odot}_1\,\overline{\vto{x}}^*] + \vto{c}\,, 
\end{equation}
where second order tensor 
\begin{equation}
\gloref{defGrad}:=\fua{\nabla\tnr{\varphi}_t}{\overline{\vto{u}}}^\text{T}
\end{equation}
is called the \textsb{affine deformation gradient} of vector affinity $\tnr{\varphi}_t$ at $\overline{\vto{u}}$ and $\vto{c}\in U_\real$ is a constant vector. Therefore, if $\tnr{F}_{\overline{\vto{x}}}$ is the identity tensor $\tnr{I}\in\ete{\real}{U^{2}}$ or a stretch tensor for all $\overline{\vto{x}}\in\tnr{\overline{\mathcal{B}}}$, then affine deformation $\tnr{\varphi}_t$ is a vector translation or a centered vector dilation respectively, according to \eqref{eq:transltVec} and \eqref{eq:dilatVec}. It is important to observe that an affine deformation can be decomposed in a rotodilation, according to theorem \ref{teo:rotoDilaF} and corollary \ref{cor:rotoDilaT}, in such a way that 
\begin{equation}\label{eq:decompDefGrad}
\tnr{\varphi}_t = \tnr{\mathit{s}}_2\circ\tnr{\mathit{r}}_{\vto{c}} = \tnr{\mathit{r}}_{\vto{c}}\circ\tnr{\mathit{s}}_1
\end{equation}
and
\begin{equation}
\tnr{F}_{\overline{\vto{u}}}=\gloref{leftStretch}\odot_1\tnr{R}_{\overline{\vto{u}}}=\tnr{R}_{\overline{\vto{u}}}\odot_1\gloref{rightStretch}\,,\,\forall\, \overline{\vto{u}}\in\tnr{\overline{\mathcal{B}}}\,, 
\end{equation}
where $\tnr{V}_{\overline{\vto{u}}}$ and $\tnr{U}_{\overline{\vto{u}}}$, usually called \textsb{left}\index{tensor!left stretch} and \textsb{right stretch tensors}\index{tensor!right stretch}, are the affinity tensors of $\tnr{\mathit{s}}_2$ and $\tnr{\mathit{s}}_1$, both having the same stretch coefficients $\lambda_1,\lambda_2,\lambda_3$. Since the local volume change is always positive, as we already know, $\tnr{R}_{\overline{\vto{u}}}$ can never be a rotoreflection tensor. Moreover, from the polar decomposition \eqref{eq:decompDefGrad}, we can conclude that stretch operators $\vtf{s}_1^{\nicefrac{1}{2}}$ is Lagrangian and $\vtf{s}_2^{-\nicefrac{1}{2}}$ is Eulerian, representing $\tnr{V}_{\overline{\vto{u}}}^{-1}$, whose stretch coefficients are $\lambda_1^{-1},\lambda_2^{-1},\lambda_3^{-1}$. The stretch coefficients $\lambda_1,\lambda_2,\lambda_3$ and $\lambda_1^{-1},\lambda_2^{-1},\lambda_3^{-1}$ are also said to be the \textsb{principal stretches}\index{principal!stretches}, at instant $t$, on material point $\overline{\vto{u}}$ and on spatial point $\vto{u}$ respectively. In practice, it is convenient not to deal with square root operators $\vtf{s}_1^{\nicefrac{1}{2}}$ and $\vtf{s}_2^{\nicefrac{1}{2}}$, which represent the above stretch tensors and are troublesome to compute, but instead to specify 
\begin{alignat}{3} 
\gloref{leftCauchy}:=\tnr{V}_{\overline{\vto{u}}}^2&\qquad \text{and}\qquad & \gloref{rightCauchy}:=\tnr{U}_{\overline{\vto{u}}}^2\,,
\end{alignat}
known as the \textsb{left}\index{tensor!left Cauchy-Green} and \textsb{right Cauchy-Green deformation tensors}\index{tensor!right Cauchy-Green}, represented by operators $\vtf{s}_2$ and $\vtf{s}_1$ respectively, according to \eqref{eq:compoFuncRepFunc}, whose stretch coefficients are $\lambda_1^2,\lambda_2^2,\lambda_3^2$. Since $\vtf{s}_2^{-1}$ and $\vtf{s}_1$ are respectively Eulerian and Lagrangian, their represented tensors $\tnr{B}_{\overline{\vto{u}}}^{-1}$ and $\tnr{C}_{\overline{\vto{u}}}$ are also called  Eulerian and Lagrangian. In terms of the affine deformation gradient,
\begin{alignat}{3}\label{eq:CauchyGreenGradDef}
\tnr{B}_{\overline{\vto{u}}}=\tnr{F}_{\overline{\vto{u}}}\odot_1\tnr{F}_{\overline{\vto{u}}}^\text{T}&\qquad \text{and}\qquad & \tnr{C}_{\overline{\vto{u}}}=\tnr{F}_{\overline{\vto{u}}}^\text{T}\odot_1\tnr{F}_{\overline{\vto{u}}}\,.
\end{alignat}


{\footnotesize
\begin{proof}
Firstly, the following development 
\begin{align*}
0&<\textnormal{Det}[\fua{\nabla\tnr{\varphi}_t}{\overline{\vto{u}}}]\\ 
&< \textnormal{Det}(\tnr{F}_{\overline{\vto{u}}}^\text{T})\\
&< \textnormal{Det}(\tnr{R}_{\overline{\vto{u}}}^{-1}\odot_1\tnr{V}_{\overline{\vto{u}}})\\
&< [\textnormal{Det}(\tnr{R}_{\overline{\vto{u}}})]^{-1}\textnormal{Det}(\tnr{V}_{\overline{\vto{u}}})
\end{align*}
proves that $\tnr{R}_{\overline{\vto{u}}}$ is not a rotoreflection tensor, that is $\textnormal{Det}(\tnr{R}_{\overline{\vto{u}}})>0$, because $\textnormal{Det}(\tnr{V}_{\overline{\vto{u}}})>0$ as a consequence of $\tnr{V}_{\overline{\vto{u}}}$ being symmetric positive-definite. Now, we'll verify one of the properties \eqref{eq:CauchyGreenGradDef}. 
\begin{align*}
\tnr{B}_{\overline{\vto{u}}}&=\tnr{F}_{\overline{\vto{u}}}\odot_1\tnr{F}_{\overline{\vto{u}}}^\text{T}\\
&=\tnr{V}_{\overline{\vto{u}}}\odot_1\tnr{R}_{\overline{\vto{u}}}\odot_1(\tnr{V}_{\overline{\vto{u}}}\odot_1\tnr{R}_{\overline{\vto{u}}})^\text{T}\\
&=\tnr{V}_{\overline{\vto{u}}}\odot_1\tnr{R}_{\overline{\vto{u}}}\odot_1\tnr{R}_{\overline{\vto{u}}}^\text{T}\odot_1\tnr{V}_{\overline{\vto{u}}}^\text{T}\\
&=\tnr{V}_{\overline{\vto{u}}}\odot_1\tnr{R}_{\overline{\vto{u}}}\odot_1\tnr{R}_{\overline{\vto{u}}}^\text{-1}\odot_1\tnr{V}_{\overline{\vto{u}}}\\
&=\tnr{V}_{\overline{\vto{u}}}\odot_1\tnr{V}_{\overline{\vto{u}}}\,.
\end{align*}
\end{proof}
}

\noindent When the affine deformation gradient is constant on $\tnr{\overline{\mathcal{B}}}$, the deformation is called \textsb{homogeneous}\index{deformation!homogeneous}, represented by $\tnr{\tilde{\varphi}}_t$. In other words, tensor $\tnr{F}_{\overline{\vto{u}}}=\tnr{F}$ for all $\overline{\vto{u}}\in\tnr{\overline{\mathcal{B}}}$ and then 
\begin{equation}\label{eq:homogDef}
\fua{\tnr{\tilde{\varphi}}_t}{\overline{\vto{x}}} =(\tnr{F}\,\hat{\odot}_1\,\overline{\vto{x}}^*) + \vto{c}\,.
\end{equation}
It is interesting to note that if $\tnr{F}=\tnr{I}$, deformation $\tnr{\tilde{\varphi}}_t$ results a vector translation. In the rotodilation decomposition of the homogeneous affinity tensor $\tnr{F}$, the subscripts identifying point dependence can be removed, resulting  
\begin{alignat}{3}
\tnr{F}=\tnr{V}\odot_1\tnr{R}=\tnr{R}\odot_1\tnr{U}\,,&\qquad \qquad &\tnr{B}=\tnr{F}\odot_1\tnr{F}^\text{T}&\qquad \text{and}\qquad & \tnr{C}=\tnr{F}^\text{T}\odot_1\tnr{F}\,.
\end{alignat}

\begin{example}
Before examples of notable homogeneous deformations are presented, we specify eight body points that constitute the vertices of a cube $\tnr{\overline{\mathcal{C}}}$: an arbitrarily chosen $\overline{\vto{u}}$, three orthonormal vectors $\overline{\vto{u}}_i=\vun{e}_i+\overline{\vto{u}}$, $i=1,2,3$, that constitute basis $U$, where $\vun{e}_i$ constitute the natural basis $O$; points $\overline{\vto{u}}_4=\vun{e}_1 + \vun{e}_2 + \overline{\vto{u}}$, $\overline{\vto{u}}_5=\vun{e}_1 + \vun{e}_3 + \overline{\vto{u}}$, $\overline{\vto{u}}_6=\vun{e}_2 + \vun{e}_3 + \overline{\vto{u}}$ and $\overline{\vto{u}}_7=\vun{e}_1+\vun{e}_2 + \vun{e}_3 + \overline{\vto{u}}$. If we consider $\vtf{f}$ the representative function of deformation gradient $\tnr{F}$, rule \eqref{eq:homogDef} can be rewritten as $\fua{\tnr{\tilde{\varphi}}_t}{\overline{\vto{x}}} =\fua{\vto{f}}{\overline{\vto{x}}} + \vto{c}$. An homogeneous deformation where $\tnr{F}=\tnr{R}$ is called \textbf{rigid}\index{deformation!rigid} and then $\vtf{f}$ results proper orthogonal. Figure \ref{fg:defRigida} depicts a rigid deformation of $\tnr{\overline{\mathcal{C}}}$ where it translates and rotates a counterclockwise angle $\theta$ around the axis defined by $\overline{\vto{u}}$ and $\overline{\vto{u}}_3$. In matrix terms, the rule of this deformation is  
\begin{equation*}
[\fua{\tnr{\tilde{\varphi}}_t}{\overline{\vto{x}}}]^U=
\underbrace{\begin{bmatrix}
\cos\theta & -\sin\theta & 0\\
\sin\theta & \cos\theta & 0\\
0 & 0 & 1
\end{bmatrix}}_{[\vtf{f}_U]^U}
[\overline{\vto{x}}]^U + [\vto{c}]^U\,.
\end{equation*}


\begin{center}
\scalebox{.72}{\input{partes/figs/defHomogRigida.pstex_t}}
\vspace{9pt}
\captionof{figure}{Rigid deformation.}\label{fg:defRigida}
\end{center}


In the case of $\tnr{R}=\tnr{I}$, $\vto{c}=\vto{0}$ and $\tnr{U}=\tnr{V}=\sum_{i=1}^3\lambda_{i}\overline{\vto{u}}_i\otimes\overline{\vto{u}}_i$, where $\lambda_i$ is a stretch coefficient, then $\tnr{\tilde{\varphi}}_t$ is called a \textsb{pure stretch}\index{stretch!pure}. Figure \ref{fg:alongPuro} shows $\tnr{\overline{\mathcal{C}}}$ subjected to a pure stretch described by      
\begin{equation*}
[\fua{\tnr{\tilde{\varphi}}_t}{\overline{\vto{x}}}]^U=
\begin{bmatrix}
\lambda_1 & 0 & 0\\
0 & \lambda_2 & 0\\
0 & 0 & \lambda_3
\end{bmatrix}
[\overline{\vto{x}}]^U\,.
\end{equation*}

\begin{center}
\scalebox{.72}{\input{partes/figs/defHomogStretch.pstex_t}}
\vspace{9pt}
\captionof{figure}{Pure stretch.}\label{fg:alongPuro}
\end{center}


We recall that when stretch coefficients are identical, pure stretch is said to be proportional. Now, if $\vto{c}=\vto{0}$ and deformation gradient $\tnr{F}=\tnr{I}+\gamma\overline{\vto{u}}_i\otimes\overline{\vto{u}}_j$ , where $i\neq j$, then $\tnr{\tilde{\varphi}}_t$ is called a \textsb{simple shear}\index{shear!simple}. Considering $i=2$ and $j=3$, we have $\tnr{\overline{\mathcal{C}}}$ subjected to a simple shear represented on figure \ref{fg:cisaSimples} and described by
\begin{equation*}
[\fua{\tnr{\tilde{\varphi}}_t}{\overline{\vto{x}}}]^U=
\begin{bmatrix}
1 & 0 & 0\\
0 & 1 & \gamma\\
0 & 0 & 1
\end{bmatrix}
[\overline{\vto{x}}]^U\,.
\end{equation*}

\begin{center}
\scalebox{.72}{\input{partes/figs/defHomogCisa.pstex_t}}
\vspace{9pt}
\captionof{figure}{Simple shear.}\label{fg:cisaSimples}
\end{center}

\end{example}

The quantity or level of local deformation can be obtained by measuring the change in an infinitesimal length of the body. This local length change, called \textsb{strain}\index{strain}, can be measured from the stretch tensors related to an arbitrary material point $\overline{\vto{u}}$. Thereby, in the context of the normalized eigenbases $X$ and $Y$ of the stretch operators that represent $\tnr{U}_{\overline{\vto{u}}}$ and $\tnr{V}_{\overline{\vto{u}}}$ respectively, most of the strain measures evaluate the deviation of a given stretch from the stretch of the rigid deformation where $\lambda_1=\lambda_2=\lambda_3=1$. The simplest type of this deviation is an algebraic difference called \textsb{extension}\index{extension}, whose Lagrangian and Eulerian forms are respectively described by $\delta_i=\lambda_i-1$ and $\Delta_i=1-\lambda_i^{-1}$, which are zero for rigid deformations. But since principal stretches $\lambda_i$ are related to annoying square root operators, convenience led to the definition of extensions described by 
\begin{alignat}{3}
\tilde{\delta}_i=\lambda_i^2-1&\qquad \text{and}\qquad & \tilde{\Delta}_i=1-\lambda_i^{-2}\,,
\end{alignat}
which are valid because $\tilde{\delta}_i=\tilde{\Delta}_i=0$ in the context of rigid deformations. Such extensions can respectively be written in matrix form as $[\tnr{U}_{\overline{\vto{u}}}^2]^X-I$ and $I-[\tnr{V}_{\overline{\vto{u}}}^{-2}]^Y$, which are matrix representations of the classical definitions 
\begin{alignat}{3}
2\tnr{E}_{\overline{\vto{u}}}=\tnr{C}_{\overline{\vto{u}}}-\tnr{I}&\qquad \text{and}\qquad & 2\tnr{e}_{\overline{\vto{u}}}=\tnr{I}-\tnr{B}_{\overline{\vto{u}}}^{-1}\,,
\end{alignat}
where Lagrangian tensor $\tnr{E}_{\overline{\vto{u}}}$ is called the \textsb{Green-St Venant}\index{strain!Green-St Venant} strain tensor and Eulerian tensor $\tnr{e}_{\overline{\vto{u}}}$ the \textsb{Almansi-Hamel}\index{strain!Almansi-Hamel} strain tensor. From both of these tensors, \aut{Doyle \& Ericksen}\cite{doyle_1956} and \aut{Seth}\cite{seth_1961} independently proposed a generalization that include most of the different strain measures available. In more complicated terms, they specified the extension
\begin{alignat}{3}
\delta_i^{(k)}:=\begin{dcases}
(\lambda_i^k-1)/k &  \text{if } k\neq 0\\
\ln{\lambda_i}&  \text{if } k=0
\end{dcases}\,,\, \forall \,\, k\in\real\,,
\end{alignat}
usually called the \textsb{Doyle-Ericksen extension}\index{Doyle-Ericksen extension}\index{extension!Doyle-Ericksen}, which is valid because $\delta_i^{(k)}=0$ for rigid deformations. Note that $2\delta_i^{(2)}=\tilde{\delta}_i$ and $2\delta_i^{(-2)}=\tilde{\Delta}_i$, leading to the Green-St Venant and Almansi-Hamel strain tensors when $\tnr{U}_{\overline{\vto{u}}}$ and $\tnr{V}_{\overline{\vto{u}}}$ are respectively considered, as we have done above. Table \ref{tb:strains} lists the most common strain tensors available, called the \textsb{Doyle-Ericksen tensors}\index{Doyle-Ericksen tensor}\index{Doyle-Ericksen tensor}. For $k\neq 0$, they are based on the generic strain tensor 
\begin{equation}
\gloref{doyle}:=\dfrac{1}{k}(\tnr{S}^k-\tnr{I})
\end{equation}
where tensor $\tnr{S}$ is a positive-definite symmetric tensor equal to or defined from the stretch tensors. Professor \aut{Hill}\cite{hill_1968_1} generalized the proposition of \aut{Seth}\cite{seth_1961} even further by defining an extension $\fua{f}{\lambda_i}$, where mapping $\map{f}{\real^+_*}{\real}$ is smooth and monotonically increasing, that is, $\fua{f'}{x}>0$ for all $x>0$. Moreover, function $f$ must obviously observe $\fua{f}{1}=0$ and, in order to make all strain measures equivalent in deformations almost rigid or small, he imposed $\fua{f'}{1}=1$.

\begin{table}[!htt]
\small
\centering
\begin{tabular}{|c|c|c|c|}
\hline
& & &\\
\textbf{   $k$   } & $\tnr{S}$  &\textbf{  Doyle-Ericksen Tensor    } & \textbf{   Name   } \\
& & & \\
\hline
& & & \\
-2 & $\tnr{V}_{\overline{\vto{u}}}$   & $\tnr{E}_{\tnr{V}_{\overline{\vto{u}}}}^{(-2)}=\tnr{e}_{\overline{\vto{u}}}=(\tnr{I} - \tnr{B}^{-1}_{\overline{\vto{u}}})/2$   & Almansi-Hamel\\
& & & \\
\hline
& & & \\
-2 & $\tnr{V}^{-1}_{\overline{\vto{u}}}$  & $\tnr{E}_{\tnr{V}^{-1}_{\overline{\vto{u}}}}^{(-2)}=(\tnr{I} - \tnr{B}_{\overline{\vto{u}}})/2$   & Finger\index{strain!Finger} \\
& & & \\
\hline
& & & \\
-1 & $\tnr{V}_{\overline{\vto{u}}}$  & $\tnr{E}_{\tnr{V}_{\overline{\vto{u}}}}^{(-1)}=\tnr{I} - \tnr{V}^{-1}_{\overline{\vto{u}}}$   & Swainger\index{strain!Swainger} \\
& & & \\
\hline
& & &\\
$0$ & - & $\tnr{E}^{(0)}=\sum_{i=1}^3\ln\lambda_i\vun{x}_{i}\otimes\vun{x}_{i}$   & Henky\index{strain!Henky} \\
& & &\\
\hline
& & &\\
$1$ &  $\tnr{U}_{\overline{\vto{u}}}$ &$\tnr{E}_{\tnr{U}_{\overline{\vto{u}}}}^{(1)}=\tnr{U}_{\overline{\vto{u}}}-\tnr{I}$ & Biot\index{Biot!tensor medida de} \\
& & &\\
\hline
& & & \\
$2$ & $\tnr{U}^{-1}_{\overline{\vto{u}}}$ & $\tnr{E}_{\tnr{U}^{-1}_{\overline{\vto{u}}}}^{(2)}=(\tnr{C}^{-1}_{\overline{\vto{u}}}-\tnr{I})/2$ & Piola\index{strain!Piola} \\
& & &\\
\hline
& & &\\
$2$ & $\tnr{U}_{\overline{\vto{u}}}$ & $\tnr{E}_{\tnr{U}_{\overline{\vto{u}}}}^{(2)}=\tnr{E}_{\overline{\vto{u}}}=(\tnr{C}_{\overline{\vto{u}}}-\tnr{I})/2$ & Green-St Venant\\
& & & \\
\hline
\end{tabular}
\vspace{9pt}
\captionof{table}{Examples of Doyle-Ericksen Tensors.}\label{tb:strains}
\end{table}

Now, in order to describe mathematically the context of \textsb{small or infinitesimal deformations}\index{deformations!small} cited above, we need first to define a bijective Lagrangian mapping $\map{\gloref{displa}}{\tnr{\overline{\mathcal{B}}}}{U_\real}$ where 
\begin{equation}\label{eq:displac}
 \fua{\tnr{\mathit{u}}_t}{\overline{\vto{x}}}=\fua{\tnr{\varphi}_t}{\overline{\vto{x}}} - \overline{\vto{x}}\,.
 \end{equation} 
 The function $\tnr{\mathit{u}}_t$ is called \textsb{displacement function}\index{displacement} whose value $\fua{\tnr{\mathit{u}}_t}{\overline{\vto{u}}}$ is said to be the \textsb{displacement vector}\index{vector!displacement} of an arbitrary material point $\overline{\vto{u}}$ at instant $t$. From the previous rule, it is straightforward to obtain that tensor       
\begin{equation}\label{eq:gradDispla}
\fua{\nabla\tnr{\mathit{u}}_t}{\overline{\vto{u}}}=\tnr{F}_{\overline{\vto{u}}}^{\text{T}} - \tnr{I}\,,
\end{equation} 
called the \textsb{displacement gradient}\index{gradient!displacement} of point $\overline{\vto{u}}$ at instant $t$. Thereby, the rule of the inverse of the displacement function and its corresponding gradient are respectively
\begin{alignat}{3} \label{eq:invDisplace}
\fua{\tnr{\mathit{u}}_t^{-1}}{\vto{x}}=\vto{x}-\fua{\tnr{\varphi}_t^{-1}}{\vto{x}}&\qquad \text{and}\qquad & \fua{\nabla\tnr{\mathit{u}}_t^{-1}}{\vto{u}}=\tnr{I}-\tnr{F}_{\overline{\vto{u}}}^{-\text{T}}\,.
\end{alignat}
As a consequence, the following equalities can be obtained:
\begin{equation}
\begin{aligned} 
\tnr{C}_{\overline{\vto{u}}}&=\fua{\nabla\tnr{\mathit{u}}_t}{\overline{\vto{u}}}\odot_1\fua{\nabla\tnr{\mathit{u}}_t}{\overline{\vto{u}}}^{\text{T}}+\fua{\nabla\tnr{\mathit{u}}_t}{\overline{\vto{u}}}+\fua{\nabla\tnr{\mathit{u}}_t}{\overline{\vto{u}}}^{\text{T}}+\tnr{I}\,;\\
\tnr{B}_{\overline{\vto{u}}}&=\fua{\nabla\tnr{\mathit{u}}_t}{\overline{\vto{u}}}^{\text{T}}\odot_1\fua{\nabla\tnr{\mathit{u}}_t}{\overline{\vto{u}}}+\fua{\nabla\tnr{\mathit{u}}_t}{\overline{\vto{u}}}+\fua{\nabla\tnr{\mathit{u}}_t}{\overline{\vto{u}}}^{\text{T}}+\tnr{I}\,;\\
\tnr{C}_{\overline{\vto{u}}}^{-1}&=\fua{\nabla\tnr{\mathit{u}}_t^{-1}}{\vto{u}}^\text{T}\odot_1\fua{\nabla\tnr{\mathit{u}}_t^{-1}}{\vto{u}}-\fua{\nabla\tnr{\mathit{u}}_t^{-1}}{\vto{u}}-\fua{\nabla\tnr{\mathit{u}}_t^{-1}}{\vto{u}}^{\text{T}}+\tnr{I}\,;\\
\tnr{B}_{\overline{\vto{u}}}^{-1}&=\fua{\nabla\tnr{\mathit{u}}_t^{-1}}{\vto{u}}\odot_1\fua{\nabla\tnr{\mathit{u}}_t^{-1}}{\vto{u}}^\text{T}-\fua{\nabla\tnr{\mathit{u}}_t^{-1}}{\vto{u}}-\fua{\nabla\tnr{\mathit{u}}_t^{-1}}{\vto{u}}^{\text{T}}+\tnr{I}\,.
\end{aligned}
\end{equation}

{\footnotesize
\begin{proof}
First, we verify the second expression of \eqref{eq:invDisplace} from the definition of affine deformation gradient and equality \eqref{eq:invGradDef}, from which we conclude that $\tnr{F}_{\overline{\vto{u}}}^{\text{-T}}=\fua{\nabla\tnr{\varphi}_t^{-1}}{\vto{u}}$. Now, considering \eqref{eq:CauchyGreenGradDef}, proof of the above equalities is trivial by isolating $\tnr{F}_{\overline{\vto{u}}}^{\text{T}}$, $\tnr{F}_{\overline{\vto{u}}}^{-1}$ and $\tnr{F}_{\overline{\vto{u}}}^{-\text{T}}$ in \eqref{eq:gradDispla} and \eqref{eq:invDisplace}.
\end{proof}
}

\noindent A deformation is called small when $\|\fua{\nabla\tnr{\mathit{u}}_t}{\overline{\vto{u}}}\|=\|\tnr{F}_{\overline{\vto{u}}}^{\text{T}} - \tnr{I}\|\ll 1$, which leads to $\tnr{F}_{\overline{\vto{u}}}\approx\tnr{I}$ and then, from \eqref{eq:invDisplace}, we can conclude that $\fua{\nabla\tnr{\mathit{u}}_t^{-1}}{\vto{u}}\approx\fua{\nabla\tnr{\mathit{u}}_t}{\overline{\vto{u}}}$. Moreover, concerning the equalities above, the four nonlinear terms involving partial inner products of two small displacement gradients become negligible, from which equality $\tnr{C}_{\overline{\vto{u}}}=\tnr{B}_{\overline{\vto{u}}}$ can be considered valid. But since $\fua{\nabla\tnr{\mathit{u}}_t^{-1}}{\vto{u}}\approx\fua{\nabla\tnr{\mathit{u}}_t}{\overline{\vto{u}}}$, we can also conclude that  $\tnr{C}_{\overline{\vto{u}}}\approx\tnr{C}_{\overline{\vto{u}}}^{-1}$ and $\tnr{B}_{\overline{\vto{u}}}\approx\tnr{B}_{\overline{\vto{u}}}^{-1}$. Therefore, concerning the Doyle-Ericksen tensors of table \ref{tb:strains}, we have
\begin{equation*}
\tnr{E}_{\overline{\vto{u}}} \approx \tnr{e}_{\overline{\vto{u}}} \approx  \tnr{E}_{\tnr{U}^{-1}_{\overline{\vto{u}}}}^{(2)} \approx \tnr{E}_{\tnr{V}_{\overline{\vto{u}}}}^{(-2)}
\end{equation*}
in the context of small deformations. If translations are disregarded or simply not present in the affine deformation rule, another consequence of $\tnr{F}_{\overline{\vto{u}}}\approx\tnr{I}$ is that the shape $\tnr{\overline{\mathcal{B}}}\approx\tnr{\mathcal{B}}_t$. Therefore, we can assume that $\tnr{\overline{\mathcal{B}}}$ is the locus where small deformations take place and tensor 
\begin{equation}\label{eq:infinitStrain}
\gloref{inftyDef}:=(\fua{\nabla\tnr{\mathit{u}}_t}{\overline{\vto{u}}}+\fua{\nabla\tnr{\mathit{u}}_t}{\overline{\vto{u}}}^{\text{T}})/2
\end{equation}
is their strain measure, called \textsb{infinitesimal strain tensor}\index{strain!infinitesimal}, which is clearly taken from the Green-St Venant tensor written in terms of displacement gradients after neglecting the very small nonlinear term. In those cases where these nonlinear terms cannot be neglected, that is, cases where small deformations are not valid, the context under study is said to be of \textsb{large or finite deformations}\index{deformations!finite}. From the above definition, it is evident that the matrix representation of $\tnr{\epsilon}_{\overline{\vto{u}}}$, which measures length changes in infinitesimal deformations, is the symmetric part of matrix $[\fua{\nabla\tnr{\mathit{u}}_t}{\overline{\vto{u}}}]$. But if the symmetric part of $[\fua{\nabla\tnr{\mathit{u}}_t}{\overline{\vto{u}}}]$ refers to the length changing portion of infinitesimal deformation, we can state that its antisymmetric part $\gloref{antiSymPart}$, where
\begin{equation}\label{eq:infinitRotation}
\tnr{W}_{\overline{\vto{u}}}:=(\fua{\nabla\tnr{\mathit{u}}_t}{\overline{\vto{u}}}-\fua{\nabla\tnr{\mathit{u}}_t}{\overline{\vto{u}}}^{\text{T}})/2\,,
\end{equation}
refers to the rotation portion\footnote{In Continuum Mechanics literature, $\tnr{W}_{\overline{\vto{u}}}$ is commonly called the infinitesimal rotation tensor, but we shall not use this nomenclature here in order to avoid confusion.}. Speaking of antisymmetric tensors of second order, we recall that theorem \ref{teo:antSymVectors} establishes a biunivocal relationship between them and three dimensional Euclidean vectors. Therefore, considering the conditions of this theorem, the axial vector defined by $\tnr{W}_{\overline{\vto{u}}}$ is  
\begin{equation}\label{eq:vectinfinitRotation}
\tnr{\mathit{w}}_{\overline{\vto{u}}}:=\tnr{A}_B\hat{\odot}_2 \tnr{W}_{\overline{\vto{u}}}\,.
\end{equation}


\section{Motions}

In this section, time is also considered variable in the mathematical study of motion, when new fundamental concepts arise. But firstly, as we did in the previous section for a certain time $t$, it is convenient to fix an arbitrary particle $\overline{\vto{u}}\in\tnr{\overline{\mathcal{B}}}$ and then define a surjective mapping $\map{\tnr{\chi}_{\overline{\vto{u}}}}{\tempo}{\tnr{\mathcal{T}}_{\overline{\vto{u}}}}$ where 
\begin{equation}
\fua{\tnr{\chi}_{\overline{\vto{u}}}}{t}=\fua{\tnr{\chi}}{\overline{\vto{u}},t}\,,\,\forall t\in\tempo\,.
\end{equation}
Moreover, the univariate motion $\tnr{\chi}_{\overline{\vto{u}}}$, which results a $\mathcal{C}^{3}$ surjection, must also  observe restriction 
$\fua{\tnr{\chi}_{\overline{\vto{u}}}}{\overline{t}}=\overline{\vto{u}}$. Therefore, $\tnr{\chi}_{\overline{\vto{u}}}$ describes the temporal deformation of particle $\overline{\vto{u}}$ and, given a period $\mathbb{P}\subset\tempo$, the collection of places 
\begin{equation}
\gloref{traject}:=\{\,\fua{\tnr{\chi}_{\overline{\vto{u}}}}{t}\,:\,t\in\mathbb{P}\,\}
\end{equation}
is called the \textsb{trajectory}\index{trajectory} or \textsb{pathline}\index{pathline} of $\overline{\vto{u}}$ during $\mathbb{P}$. Additionally to this collection of places, it is also important to define the collection of particles that ``occupy'' a fixed place during a period of time. In order to do this, from the codomain of mapping \eqref{eq:vecMotion}, we define the surjective mapping $\map{\tnr{p}}{\tnr{\mathcal{S}}\times\tempo}{\tnr{\overline{\mathcal{B}}}}$, where $\tnr{p}$ is called \textsb{streak function}\index{function!streak} if it is a $\mathcal{C}^3$ surjection whith rule
\begin{equation}
\gloref{streak}=\tnr{\chi}(\fua{\tnr{\chi}_{t}^{-1}}{\vto{x}},t)\,.
\end{equation}
Following the same procedure adopted for deformations, by fixing a position $\vto{u}$ we define univariate surjection $\tnr{p}_\vto{u}$, where $\fua{\tnr{p}_\vto{u}}{t}=\fua{\tnr{p}}{\vto{u},t}$. In this context, the collection of particles
\begin{equation}
\gloref{streakline}:=\{\,\fua{\tnr{p}_{\vto{u}}}{t}\,:\,t\in\mathbb{P}\,\}
\end{equation}
is said to be the \textsb{streakline}\index{streakline} of place $\vto{u}$ during $\mathbb{P}$. 


The functions\footnote{In Continuum Mechanics classical literature, functions like $\tnr{\omega}$ and $\tnr{\Omega}$ are usually called tensor, vector or scalar fields, but we shall not use this nomenclature here in order to avoid confusion.} in  mappings $\map{\tnr{\omega}}{\tnr{\overline{\mathcal{B}}}\times\tempo}{W}$ and $\map{\tnr{\Omega}}{\tnr{\mathcal{B}}_t\times\tempo}{W}$, where codomain $W\in\{\ete{\real}{U^m},U_\real,\real\}$, are specified to be continuously differentiable. In order to simplify notation, it is sometimes very useful to also define  
\begin{alignat}{3}\label{eq:LagEulDesc} 
\tnr{\Omega}_L(\overline{\vto{x}},t)=\tnr{\Omega}(\fua{\tnr{\chi}_t}{\overline{\vto{x}}},t)&\qquad \text{and}\qquad & \tnr{\omega}_E(\vto{x},t)=\tnr{\omega}(\fua{\tnr{\chi}_t^\text{-1}}{{\vto{x}}},t)\,,
\end{alignat}
commonly called the Lagrangian and Eulerian descriptions of $\tnr{\Omega}$ and $\tnr{\omega}$. In genereal terms, notations \gloref{lagDesc} and \gloref{eulerDesc} can be used on any Eulerian and Lagrangian function. It is then straightforward to conclude that that streak function  is the Eulerian description of the deformation, that is, $\tnr{p}=\tnr{\chi}_E$. Generically, since each of the functions $\tnr{\Omega}$ and $\tnr{\omega}$ can be described by two domains, the differentiation chain rules force us to clearly specify the domain of derivation, namely, whether these functions will be derived on the material or spatial domain. Therefore, derivatives here are called Lagrangian or Eulerian when performed on particles or places respectively. In this context, \emph{it is specified that the time derivative of $\bullet$ is said to be Lagrangian, represented by \gloref{lagTimeDeriv}, or Eulerian, represented by \gloref{eulerTimeDeriv}, when a particle or a place are fixed}. Whenever possible, notations ${\bar{\partial}}_t^k\bullet$ and ${\tilde{\partial}}_t^k\bullet$ define Lagrangian and Eulerian time derivatives of order $k\geqslant 2$, while time derivatives $d\bullet/dt$ and $d^2\bullet/dt^2$ of univariate functions of time are simplified by overdotted symbols \gloref{timeDeriv} and \gloref{timeDerivTwo} respectively. Concerning functions $\tnr{\omega}$ and $\tnr{\Omega}$, we have the Lagrangian time derivatives at particle $\overline{\vto{u}}$
\begin{alignat}{3}\label{eq:LagDesc} 
\bar{\partial}_t\tnr{\omega}(\overline{\vto{u}},t):=\dfrac{d\,\tnr{\omega}_{\overline{\vto{u}}}}{dt}(t)={\dot{\tnr{\omega}}}_{\overline{\vto{u}}}(t)&\qquad \text{and}\qquad & \bar{\partial}_t\tnr{\Omega}(\vto{u},t):=\tnr{\Omega}'(\fua{\tnr{\chi}_{\overline{\vto{u}}}}{t},t)
\end{alignat}
as well as the Eulerian time derivatives at place ${\vto{u}}$
\begin{alignat}{3}\label{eq:EulDesc} 
\tilde{\partial}_t\tnr{\Omega}({\vto{u}},t):=\dfrac{d\,\tnr{\Omega}_{{\vto{u}}}}{dt}(t)={\dot{\tnr{\Omega}}}_{{\vto{u}}}(t)&\qquad \text{and}\qquad & \tilde{\partial}_t\tnr{\omega}(\overline{\vto{u}},t):=\tnr{\omega}'(\fua{\tnr{p}_{\vto{u}}}{t},t)\,,
\end{alignat}
Considering these previous definitions and the Chain Rule for Partial Derivatives \eqref{eq:chainRulePartial}, by developing the Lagrangian time derivative 
\begin{equation}\label{eq:LagrTimeDerivative}
\bar{\partial}_t\tnr{\Omega}({\vto{x}},t)=\dot{\tnr{\Omega}}_{\vto{x}}(t)+\dot{\tnr{\Omega}}_{t}(\tnr{\chi}_{\overline{\vto{x}}}^{-1}(t))
\end{equation}
and Eulerian time derivative 
\begin{equation}\label{eq:EulerTimeDerivative}
\tilde{\partial}_t\tnr{\omega}(\overline{\vto{x}},t)=\dot{\tnr{\omega}}_{\overline{\vto{x}}}(t)+\dot{\tnr{\omega}}_{t}(\tnr{\chi}_{{\vto{x}}}(t))\,,
\end{equation}
it is possible to obtain that
\begin{equation}\label{eq:derivPartConst}
\bar{\partial}_t\tnr{\Omega}({\vto{x}},t)=\tilde{\partial}_t\tnr{\Omega}({\vto{x}},t)+\tnr{\Omega}_t'(\vto{x})\circ\dot{\tnr{\chi}}_{\overline{\vto{x}}}(t)
\end{equation}
and
\begin{equation}\label{eq:derivPlaceConst}
\tilde{\partial}_t\tnr{\omega}(\overline{\vto{x}},t)=\bar{\partial}_t\tnr{\omega}(\overline{\vto{u}},t)+\tnr{\omega}_t'(\overline{\vto{x}})\circ\dot{\tnr{p}}_{{\vto{x}}}(t)\,.
\end{equation}
Moreover, in the particular cases of $\tnr{\omega}=\tnr{\Omega}_L$ and $\tnr{\Omega}=\tnr{\omega}_E$,
\begin{alignat}{3}\label{eq:zerosDescript} 
[\dot{\tnr{\Omega}}_{t}]_L+\dot{[\tnr{\Omega}_L]}_{t}=\vto{0}&\qquad \text{and}\qquad & [\dot{\tnr{\omega}}_{t}]_E+\dot{[\tnr{\omega}_E]}_{t}=\vto{0}\,.
\end{alignat}
The specification of the codomains of functions $\tnr{\omega}$ and $\tnr{\Omega}$ leads to the following important conclusions, where Lagrangian and Eulerian gradients, related to material and spatial derivatives, are represented by symbols $\bar{\bullet}$ and $\tilde{\bullet}$ respectively.
\begin{itemize}
    \setlength\itemsep{.1em}
    \item[i.] If $W=\ete{\real}{U^m}$, functions are tensor valued and then, from \eqref{eq:defGradient}, 
\begin{equation}
\begin{aligned}
\tnr{\Omega}_t'(\vto{x})\circ\dot{\tnr{\chi}}_{\overline{\vto{x}}}(t)&=[\dot{\tnr{\chi}}_{\overline{\vto{x}}}(t)]^*\odot_1\fua{\tilde{\nabla}\tnr{\Omega}_t}{{\vto{x}}}\,,\\
\tnr{\omega}_t'(\overline{\vto{x}})\circ\dot{\tnr{p}}_{{\vto{x}}}(t)&=[\dot{\tnr{p}}_{{\vto{x}}}(t)]^*\odot_1\fua{\bar{\nabla}\tnr{\omega}_t}{\overline{\vto{x}}}\,;
\end{aligned} 
\end{equation}   
    \item[ii.] If $W=U_\real$, functions are vector valued and then, from \eqref{eq:derivVecGradVec},
\begin{equation}\label{eq:compoVector}
\begin{aligned}
\tnr{\Omega}_t'(\vto{x})\circ\dot{\tnr{\chi}}_{\overline{\vto{x}}}(t)&=\fua{\tilde{\nabla}\tnr{\Omega}_t}{{\vto{x}}}^{\text{T}}\hat{\odot}_1[\dot{\tnr{\chi}}_{\overline{\vto{x}}}(t)]^*\,,\\
\tnr{\omega}_t'(\overline{\vto{x}})\circ\dot{\tnr{p}}_{{\vto{x}}}(t)&=\fua{\bar{\nabla}\tnr{\omega}_t}{\overline{\vto{x}}}^{\text{T}}\hat{\odot}_1[\dot{\tnr{p}}_{{\vto{x}}}(t)]^*\,;
\end{aligned} 
\end{equation} 
\item[iii.] If $W=\real$, functions are scalar valued and then, from \eqref{eq:gradScalVectFunc},
\begin{equation}
\begin{aligned}
\tnr{\Omega}_t'(\vto{x})\circ\dot{\tnr{\chi}}_{\overline{\vto{x}}}(t)&=\fua{\tilde{\operatorname{grad}}\,{\tnr{\Omega}_t}}{{\vto{x}}}\cdot\dot{\tnr{\chi}}_{\overline{\vto{x}}}(t)\,,\\
\tnr{\omega}_t'(\overline{\vto{x}})\circ\dot{\tnr{p}}_{{\vto{x}}}(t)&=\fua{\bar{\operatorname{grad}}\,{\tnr{\omega}_t}}{\overline{\vto{x}}}\cdot\dot{\tnr{p}}_{{\vto{x}}}(t)\,.
\end{aligned} 
\end{equation} 
\end{itemize}
For future purposes, it is now important to present properties involving Lagrangian and Eulerian gradients of respectively Eulerian and Lagrangian functions. The following equalities are straightforward consequences of property \eqref{eq:prop1Grad}, valid for tensor valued composite functions, which can be easily extended for vector and scalar valued composite functions:
\begin{equation}\label{eq:gradRelFuncs}
\begin{aligned}
\bar{\nabla}[{\tnr{\Omega}}_{t}]_L(\overline{\vto{x}})&=\bar{\nabla}{\tnr{\chi}}_{t}(\overline{\vto{x}})\odot_1\tilde{\nabla}[{\tnr{\Omega}}_{t}]_L(\overline{\vto{x}})\,;\\
\tilde{\nabla}[{\tnr{\omega}}_{t}]_E({\vto{x}})&=\tilde{\nabla}{\tnr{p}}_{t}({\vto{x}})\odot_1\bar{\nabla}[{\tnr{\omega}}_{t}]_E({\vto{x}})\,.
\end{aligned} 
\end{equation} 

{\footnotesize
\begin{proof}
From the chain rules \eqref{eq:chainRule} and \eqref{eq:chainRulePartial}, the following development proves \eqref{eq:derivPartConst}; equality \eqref{eq:derivPlaceConst} is similarly verified.
\begin{align*}
\bar{\partial}_t\tnr{\Omega}({\vto{x}},t)&=\tnr{\Omega}'(\fua{\tnr{\chi}_{\overline{\vto{x}}}}{t},t)\\
&=\dot{\tnr{\Omega}}_\vto{x}(t)+\dot{\tnr{\Omega}}_\vto{t}(\fua{\tnr{\chi}_{\overline{\vto{x}}}}{t}) \\
&=\tilde{\partial}_t\tnr{\Omega}({\vto{x}},t)+[\tnr{\Omega}_t'\circ\fua{\tnr{\chi}_{\overline{\vto{x}}}}{t}]\circ\fua{\dot{\tnr{\chi}}_{\overline{\vto{x}}}}{t}\\
&=\tilde{\partial}_t\tnr{\Omega}({\vto{x}},t)+\tnr{\Omega}_t'(\vto{x})\circ\fua{\dot{\tnr{\chi}}_{\overline{\vto{x}}}}{t}\,.
\end{align*}
Now, we prove the first of properties \eqref{eq:zerosDescript}; the other is similarly verified. From \eqref{eq:EulerTimeDerivative}, the sum of the last equality of development  
\begin{align*}
\tilde{\partial}_t\tnr{\Omega}_L(\overline{\vto{x}},t)&=\dot{[\tnr{\Omega}_L]}_{\overline{\vto{x}}}(t)+\dot{[\tnr{\Omega}_L}]_{t}(\tnr{\chi}_{{\vto{x}}}(t))\\
\dot{\tnr{\Omega}}_{\vto{x}}(t)&=\bar{\partial}_t\tnr{\Omega}({\vto{x}},t)+\dot{[\tnr{\Omega}_L]}_{t}(\overline{\vto{x}})
\end{align*}
with \eqref{eq:LagrTimeDerivative} proves the property.  
\end{proof}
}

The vector $\dot{\tnr{\chi}}_{\overline{\vto{u}}}(t)$, which measures the temporal sensitivity of the deformation at a fixed particle $\overline{\vto{u}}$, is called the \textsb{velocity}\index{velocity} of $\overline{\vto{u}}$ at $t$. If the particle is variable, the Lagrangian function in mapping $\map{{\tnr{\nu}}}{\overline{\tnr{\mathcal{B}}}\times\tempo}{U_\real}$ is called velocity if it is $\mathcal{C}^{2}$ and vector
\begin{equation}
\gloref{velocity}=\bar{\partial}_t\tnr{\chi}(\overline{\vto{x}},t)\,.
\end{equation}
In the context of affine deformations, if instant $t$ is variable for displacements described by \eqref{eq:displac}, that is, $\fua{\tnr{\mathit{u}}}{\overline{\vto{x}},t}=\fua{\tnr{\varphi}}{\overline{\vto{x}},t} - \overline{\vto{x}}$, then it is straightforward to obtain 
\begin{equation}
\bar{\partial}_t\fua{\tnr{\mathit{u}}}{\overline{\vto{x}},t} =  {\tnr{\nu}}(\overline{\vto{x}},t)\,,
\end{equation}
which is obviously a consequence, not a definition, of velocity. Now, the temporal sensitivity of the velocity of particle $\overline{\vto{u}}$ at instant $t$ is the vector $\dot{\tnr{\nu}}_{\overline{\vto{u}}}(t)=\ddot{\tnr{\chi}}_{\overline{\vto{u}}}(t)$, called the \textsb{acceleration}\index{acceleration} of $\overline{\vto{u}}$ at $t$. The function in mapping $\map{{\tnr{a}}}{\overline{\tnr{\mathcal{B}}}\times\tempo}{U_\real}$ is called acceleration if it is continuously differentiable and
\begin{equation}
\gloref{accel}=\bar{\partial}_t\tnr{\nu}(\overline{\vto{x}},t)\,,
\end{equation}
from which the following equality is evident:
\begin{equation}
\bar{\partial}_t^2\fua{\tnr{\mathit{u}}}{\overline{\vto{x}},t} =  {\tnr{a}}(\overline{\vto{x}},t)\,.
\end{equation}
Considering property \eqref{eq:derivPlaceConst}, we can conclude that vector $\dot{\tnr{p}}_{{\vto{u}}}(t)$ measures the temporal sensitivity of the Eulerian description $\tnr{\chi}_E$ of place $\vto{u}$ at instant $t$. In this context, equalities
\begin{alignat*}{3} 
\dot{\tnr{p}}_{{\vto{x}}}(t) = \tilde{\partial}_t \tnr{p}(\vto{x},t) = \tilde{\partial}_t \tnr{p}_L(\overline{\vto{x}},t)&\qquad \text{and}\qquad & \dot{\tnr{\chi}}_{\overline{\vto{x}}}(t) = \tnr{\nu}(\overline{\vto{x}},t) = \tnr{\nu}_E({\vto{x}},t)\,
\end{alignat*}
enable us to rewrite each of expressions \eqref{eq:derivPartConst} and \eqref{eq:derivPlaceConst} solely described by places and particles respectively, namely,   
\begin{equation}
\bar{\partial}_t\tnr{\Omega}({\vto{x}},t)=\tilde{\partial}_t\tnr{\Omega}({\vto{x}},t)+\tnr{\Omega}_t'(\vto{x})\circ\tnr{\nu}_E(\vto{x},t)
\end{equation}
and
\begin{equation}
\tilde{\partial}_t\tnr{\omega}(\overline{\vto{x}},t)=\bar{\partial}_t\tnr{\omega}(\overline{\vto{x}},t)+\tnr{\omega}_t'(\overline{\vto{x}})\circ\tilde{\partial}_t \tnr{p}_L(\overline{\vto{x}},t)\,.
\end{equation}
By definition, acceleration is the Lagrangian time derivative of the velocity $\tnr{\nu}$, but what about its Eulerian time derivative? From the second of the previous qualities and expressions \eqref{eq:compoVector}, it is clear that the vector 
\begin{equation}
\tilde{\partial}_t\tnr{\nu}(\overline{\vto{x}},t)=\tnr{a}(\overline{\vto{x}},t)+\fua{\bar{\nabla}\tnr{\nu}_t}{\overline{\vto{x}}}^{\text{T}}\hat{\odot}_1[\tilde{\partial}_t \tnr{p}_L(\overline{\vto{x}},t)]^*\,.
\end{equation}
Similarly, the Lagrangian time derivative of the Eulerian description $\tnr{\nu}_E$ of the velocity results the Eulerian description  
\begin{equation}
\tnr{a}_E(\vto{x},t)=\tilde{\partial}_t\tnr{\nu}_E({\vto{x}},t)+\fua{\tilde{\nabla}[\tnr{\nu}_t]_E}{{\vto{x}}}^{\text{T}}\hat{\odot}_1[{\tnr{\nu}_E}(\vto{x},t)]^*\,,
\end{equation}
from which second order tensor 
 \begin{equation}\label{eq:eulerVelocityGrad}
\gloref{eulerVelGrad}:=\fua{\tilde{\nabla}[\tnr{\nu}_t]_E}{{\vto{u}}}^{\text{T}}
 \end{equation}
is said to be the \textsb{Eulerian velocity gradient}\index{velocity!Eulerian gradient} of $\tnr{\nu}_E$ at place $\vto{u}$, while
 \begin{equation}\label{eq:lagrangVelocityGrad}
\gloref{lagraVelGrad}:=\fua{\bar{\nabla}\tnr{\nu}_t}{\overline{\vto{u}}}^{\text{T}}
 \end{equation}
is the \textsb{Lagrangian velocity gradient}\index{velocity!Lagrangian gradient} of $\tnr{\nu}$ at particle $\overline{\vto{u}}$. From the Eulerian description of the velocity, a motion of the body $\tnr{\overline{\mathcal{B}}}$ is said to be \textsb{steady}\index{motion!steady} in a period $[t_0,t_f]$ if both $\tnr{\nu}_E$ and $\tnr{\mathcal{B}}_t$  are time independent, that is, if
\begin{alignat}{3} 
\tilde{\partial}_t\tnr{\nu}_E(\vto{x},t)=\vto{0} &\qquad \text{and}\qquad & \tnr{\mathcal{B}}_t= \tnr{\mathcal{B}}_{t_0}\,,
\end{alignat}
for all $\vto{x}\in\tnr{\mathcal{B}}_{t_0}$ and $t\in [t_0,t_f]$. In this context, we can conclude that every streakline $S_{\vto{x}}$ is constituted by particles that have the same velocity and the distinct places of every trajectory $T_{\overline{\vto{x}}}$ constitute a subset of shape $\tnr{\mathcal{B}}_{t_0}$.

Considering these previous definitions in the context of affine motions, the following important properties can be obtained:
\begin{equation}\label{eq:timeDerivGrad}
\begin{aligned}
\tnr{M}_{\overline{\vto{x}}}&= \tnr{L}_{\tnr{\chi}_t(\overline{\vto{x}})}\odot_1\tnr{F}_{\overline{\vto{x}}}\,;\\
\fua{\bar{\nabla}\tnr{a}_t}{\overline{\vto{x}}}&=\fua{\tilde{\nabla}\tnr{a}_t}{\overline{\vto{x}}}\odot_1\tnr{F}_{\overline{\vto{x}}}\,.
\end{aligned} 
\end{equation}
Based on the tensors $\tnr{\epsilon}_{\overline{\vto{u}}}$ and $\tnr{W}_{\overline{\vto{u}}}$ defined by \eqref{eq:infinitStrain} and \eqref{eq:infinitRotation} in the context of infinitesimal deformations, whose locus was chosen to be $\overline{\tnr{\mathcal{B}}}$, since time derivative $\bar{\partial}_t\tnr{\mathit{u}}=\tnr{\nu}$, symmetric tensor 
\begin{equation}\label{eq:infinitTimeStrain}
\gloref{strainRate}:=(\tnr{M}_{\overline{\vto{u}}}^{\text{T}}+\tnr{M}_{\overline{\vto{u}}})/2
\end{equation}
is called the \textsb{strain-rate tensor}\index{strain-rate!tensor} at $\overline{\vto{u}}$, which measures the rate of local changes in length,  and antisymmetric tensor
\begin{equation}\label{eq:infinitTimeSpin}
\gloref{spin}:=(\tnr{M}_{\overline{\vto{u}}}^{\text{T}}-\tnr{M}_{\overline{\vto{u}}})/2
\end{equation}
is the \textsb{spin tensor}\index{spin!tensor} at $\overline{\vto{u}}$, which measures the rate of local rotation. Moreover, from \eqref{eq:timeDerivGrad}, the context of infinitesimal deformations, where $\tnr{F}_{\overline{\vto{x}}}\approx\tnr{I}$, leads to the conclusion that $\tnr{M}_{\overline{\vto{x}}}\approx\tnr{L}_{\vto{x}}$. The neglecting conditions for the above definitions, which are the same for  \eqref{eq:infinitStrain} and \eqref{eq:infinitRotation}, allow us to assume that $\tnr{M}_{\overline{\vto{x}}}=\tnr{L}_{\vto{x}}$, from which the following rate tensors can be defined:
\begin{alignat}{3} \label{eq:equalGradVel}
\tnr{\dot{\epsilon}}_{{\vto{u}}}:=(\tnr{L}_{{\vto{u}}}^{\text{T}}+\tnr{L}_{{\vto{u}}})/2=\tnr{\dot{\epsilon}}_{\overline{\vto{u}}} &\qquad \text{and}\qquad & \tnr{\dot{W}}_{{\vto{u}}}:=(\tnr{L}_{{\vto{u}}}^{\text{T}}-\tnr{L}_{{\vto{u}}})/2=\tnr{\dot{W}}_{\overline{\vto{u}}}\,,
\end{alignat}
from which matrices $[\tnr{\dot{\epsilon}}_{{\vto{u}}}]$ and $[\tnr{\dot{W}}_{{\vto{u}}}]$ are respectively the symmetric and antisymmetric parts of $[\tnr{L}_{{\vto{u}}}]$. In the classical Continuum Mechanics literature, both of these Eulerian measures are more commonly used than their Lagrangian counterparts. In this sense, according to theorem \ref{teo:antSymVectors}, since $\tnr{\dot{W}}_{{\vto{u}}}$ is antisymmetric, its correspondent axial vector 
\begin{equation}\label{eq:vectSpin}
\gloref{vorticity}:=\tnr{A}_B\hat{\odot}_2 \tnr{\dot{W}}_{{\vto{u}}}
\end{equation}
is called \textsb{vorticity vector}\index{vorticity!vector}, from which it is possible to obtain that
\begin{equation}\label{eq:curlVelocities}
\tnr{\dot{\mathit{w}}}_{{\vto{u}}}=\tilde{\nabla}\times[\tnr{\nu}_t]_E({\vto{u}})=\bar{\nabla}\times\tnr{\nu}_t(\overline{\vto{u}})\,.
\end{equation}
If the spin tensor is zero at $\vto{u}$, the vorticity vector is consequently zero at this place, when the motion is said to be \textsb{irrotational}\index{motion!irrotational} or \textsb{non-vortical} at $\vto{u}$. In this context, from the definition of $\tnr{\dot{W}}_{{\vto{u}}}$, it is obvious that the Eulerian velocity gradient $\tnr{L}_{{\vto{u}}}$ results symmetric and therefore no rotation is present since the antisymmetric part of matrix $[\tnr{L}_{{\vto{u}}}]$ is zero. In the case of zero spin tensor at every place $\vto{u}$, then the motion as a whole is called irrotational or non-vortical.

{\footnotesize
\begin{proof}
The following development proves only the first of equalities \eqref{eq:timeDerivGrad} because the second is similarly verified. Making $\tnr{\Omega}=\tnr{\nu}_E$ in property \eqref{eq:gradRelFuncs}, 
\begin{align*}
\bar{\nabla}\{[{\tnr{\nu}_E}]_{t}\}_L(\overline{\vto{x}})&=\bar{\nabla}{\tnr{\varphi}}_{t}(\overline{\vto{x}})\odot_1\tilde{\nabla}\{[{\tnr{\nu}_E}]_{t}\}_L(\overline{\vto{x}})\\
\bar{\nabla}{\tnr{\nu}}_{t}(\overline{\vto{x}})^\text{T}&=\tilde{\nabla}[\tnr{\nu}_E]_{t}({\vto{x}})^\text{T}\odot_1\bar{\nabla}{\tnr{\varphi}}_{t}(\overline{\vto{x}})^\text{T}\\
\tnr{M}_{\overline{\vto{x}}}&=\tnr{L}_{{\vto{x}}}\odot_1\tnr{F}_{\overline{\vto{x}}}\,.
\end{align*}
Now, considering an arbitrary non zero vector $\vto{x}\in U_\real$ and tensor ${\tnr{L}}_{{\vto{u}}}^{\text{T}}=\vto{g}_1^*\otimes\vto{g}_2^*$, as well as definitions \eqref{eq:curlVect} and \eqref{eq:equalGradVel}, we prove the first of equalities \eqref{eq:curlVelocities} from \eqref{eq:vectSpin} through the following development:
\begin{align*}
2\tnr{\dot{\mathit{w}}}_{{\vto{u}}} &=\tnr{A}_B\hat{\odot}_2{\tnr{L}}_{{\vto{u}}}^{\text{T}}-\tnr{A}_B\hat{\odot}_2{\tnr{L}}_{{\vto{u}}}\\
2\vto{x}\cdot\tnr{\dot{\mathit{w}}}_{{\vto{u}}}&=\vto{x}\cdot\tnr{A}_B\hat{\odot}_2(\vto{g}_1^*\otimes\vto{g}_2^*) - \vto{x}\cdot\tnr{A}_B\hat{\odot}_2(\vto{g}_2^*\otimes\vto{g}_1^*)\\
&=\tnr{A}_B(\vto{x},\vto{g}_1,\vto{g}_2) + \tnr{A}_B(\vto{x},\vto{g}_1,\vto{g}_2)\\
&=2\vto{x}\cdot\tnr{A}_B\hat{\odot}_2{\tnr{L}}_{{\vto{u}}}^{\text{T}}\\
&=2\vto{x}\cdot\tilde{\nabla}\times\fua{[\tnr{\nu}_t]_E}{{\vto{u}}}\,.
\end{align*}
The second equality of \eqref{eq:curlVelocities} is verified similarly.
\end{proof}
}


