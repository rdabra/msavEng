\chapter{Continuum Kinematics}

bla, bla, bla, bla


\section{The Continuum Model}


In a restricted and objective way, the study of the geometry of motion, or Kinematics, adopts the subject-object philosophical approach where the subject is merely an \textsb{observer}\index{observer} and the object the thing observed. In the act of kinematically describing a mechanical event, the observer makes use of two fundamental concepts: position and time. In this sense, \textsb{space}\index{space!physical} is the context or the framework that enables the subject to perform its positional identifications, when either a certain region of this space or a specific physical entity are the observed objects. In Newtonian Mechanics, which is our concern here, the space to which the whole universe belongs is considered to be \textsb{absolute}\index{space!absolute}, that is, it has a fixed size, it is completely at rest and not affected by its objects, being a totally independent concept. Any observer attached to the absolute space is also called absolute\index{observer!absolute}. 

When performing a certain measurement on the object under observation, the observer notes that changes may occur: given a point in space or in a physical entity, there may be a set of different measurement values on this point that develop one after another, sequentially. \textsb{Time}\index{time} is the concept that enables the observer to describe these changes by identifying which measurement value follows the other in such a way as to build a set of successive measurement values for every observed point.  There is always a specific instance or value of time, called \textsb{instant}\index{instant!of time}, that can label a certain measurement value; in other words, time is infinite and infinitely divisible, or an infinite unlimited complete set in mathematical terminology. When a limited portion of this set is selected to describe a certain mechanical event, we call this portion a \textsb{period}. If the observer measures positions of the points of a physical entity, the set of points labeled by the same instant is called a \textsb{configuration}\index{configuration} of this physical entity and a set of these configurations in a given period of time is called a \textsb{motion}\index{motion} of this entity. Still in the context of Newtonian Mechanics, time is considered to be unstoppable, always progressing indefinitely, and the rate of this progress is the same for any observer: time is also considered absolute. Therefore, we can say that every observer's clock is synchronized with a unique reference clock. As a consequence of the ever increasing Newtonian time, an observer describing a certain mechanical event can never label two distinct measurement values on the same point of an object by the same instant: simultaneity here makes sense only for different points. Moreover, an observer may eventually become an object when another observer can describe its kinematics. That said, an observer at rest or moving with constant velocity relative to an absolute observer is called \textsb{inertial}\index{observer!inertial}: it is only from the point of view of an inertial observer that Newton's Law of Inertia is valid. The absolute time is mathematically modeled as a real field and the absolute space as a three dimensional Euclidean space.   


Following Hermann Weyl, ``\emph{space and time are commonly regarded as the forms of existence of the real world, \textsb{matter}\index{matter} as its substance.}\footnote{See \aut{Weyl}\cite{weyl_1952_2}, p.1.}'' The purpose of the Newtonian concept of \textsb{mass}\index{mass} is to measure the quantity of this substance, as a tangible entity that occupies a \textsb{volume} in a given portion of the three dimensional space. Then, a limited portion of matter, called a \textsb{body}\index{body}, is said to have a certain mass. Moreover, it is observed that what constitutes matter, the specific material it is made of, influences its mechanical behavior. In this sense, a mechanical description concerned with material peculiarities is called \textsb{constitutive}\index{description!constitutive}. For the purposes of this book, matter is considered to be impenetrable, that is, two simultaneous configurations can never occupy the same portion of space, either totally or partially. A popular topic on the study of matter is its structure, that is, its building blocks, commonly called particles, and the way they are organized. For some reason, this popularity promoted the preconception, even in academic circles, that a physical theory which seeks to describe the overall behavior of matter becomes more reliable, is much closer to reality, when it considers structural phenomena. In our humble opinion, this idea attaches what is not attachable: a mechanical theory, expressed as a mathematical model, is better if it is closer to experimental data, when available, and if it describes an ampler broad of physical phenomena, regardless whether structural variables are considered or not. The approach, called Continuum Mechanics, that models a body as a mathematical continuum in order to study is mechanical behavior presents both of these virtues notwithstanding it totally disregards structural phenomena. Now, with the body modeled as a continuum, it is possible to consider any of its configurations as an affine space that can be biunivocally related to a suitable domain for differentiable and integrable functions.




% Para poder fazer calculo diferencial a partir de um corpo físico, a estrutura do corpo precisa atender ao requisito de completude de um espaço de Banach
% O problema é que um corpo físico não é intrinsecamente completo. Dependendo da escala que se observe, a estrutura da matéria se revelará não é completa como um espaço de Banach pois apresenta mais vazios que conteúdo.