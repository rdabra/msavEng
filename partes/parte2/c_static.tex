\chapter{Continuum Kinematics}

bla, bla, bla, bla


\section{The Continuum Model}


In a restricted and objective way, the morphological study of natural phenomena by the science of Mechanics follows the subject-object philosophical approach, where the subject is merely an observer and the object the thing observed. In this sense, \textsb{space}\index{space!physical} is the context in which this morphological study is performed, or the framework in which geometric aspects of mechanical objects are observed, namely, their positions and shapes, or their configurations. In the act of observing the morphological behavior of a mechanical object, the subject notes that the position and shape of this object may eventually change: there may be a set of different configurations that develop one after another, sequentially. \textsb{Time}\index{time} is then the concept that enables the observer to describe morphological changes by identifying which object geometry follows the other in such a way as to build different collections of successive configurations. There is always a specific instance or value of time, called \textsb{instant}\index{instant!of time}, that can label a certain object configuration; in other words, time is infinite and infinitely divisible, or an infinite unlimited complete set, in mathematical terminology. When a limited portion of this set is selected to describe a certain mechanical event, we call this portion a \textsb{period}. In the context of Classical Mechanics, which is our concern here, time is unstoppable, always progressing indefinitely, and the rate of this progress is the same for any observer: time is absolute. As a consequence of these features of classical time, an observer can never label two distinct configurations of the same object by the same instant. 



and can never describe the same configuration for different objects at the same instant.   


description of changeThe concept of time is then  


 that succeeds a be   


In describing the behavior of mechanical objects, the observer notes 

needs to register not only static positions and shapes but their evolution,  






% Para poder fazer calculo diferencial a partir de um corpo físico, a estrutura do corpo precisa atender ao requisito de completude de um espaço de Banach
% O problema é que um corpo físico não é intrinsecamente completo. Dependendo da escala que se observe, a estrutura da matéria se revelará não é completa como um espaço de Banach pois apresenta mais vazios que conteúdo.