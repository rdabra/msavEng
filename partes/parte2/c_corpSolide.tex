\documentclass[leqno,openright,smallroyalvopaper,8pt,twoside,showtrims]{memoir}
\medievalpage[15]
\setlrmargins{*}{1.4cm}{*}
\setulmargins{*}{*}{1}
\checkandfixthelayout
\usepackage[english]{babel}
\usepackage{slantsc}
\usepackage{array}
\usepackage{amsmath}
\usepackage{adforn}
\pagestyle{empty}


\renewcommand*\rmdefault{obn}
\usepackage[LCYW]{fontenc}
\usepackage{mathtools}

\newtheorem{thm}{Theorem}
\renewcommand\thethm{\Roman{thm}}



\begin{document}
\vspace*{0.5cm}
\noindent
\rule{\textwidth}{0.5pt}\vspace*{-\baselineskip}\vspace*{2pt} 
%\rule{\textwidth}{0.5pt} 
\vspace*{.2cm}
\begin{center}
{\LARGE \textbf{ON THE PRESSURE OR TENSION}}\vspace*{3pt} 

{\Large \textbf{IN A SOLID BODY.}}\vspace*{10pt} 

{\large BY Mr. AUGUSTIN-LOUIS CAUCHY} \vspace*{2pt} 

{CHIEF ENGINEER AT PONTS ET CHAUSS\'EES, PROFESSOR AT \'ECOLE POLYTECHNIQUE, ADJOINT PROFESSOR AT FACULT\'E DES SCIENCES, MEMBER OF THE ACADEMY OF SCIENCES, KNIGHT OF THE LEGION OF HONOUR}\vspace*{2pt} 


{\small Translated by Mr. R.D. {\normalfont\scshape Algarte}}.\vspace*{.5cm}

\adforn{21}\quad\adforn{11}\quad\adforn{49}\vspace*{.5cm}


\end{center}
%\linespread{1.0}
\normalfont

Geometers who have researched the equations of equilibrium or motion of plates or of elastic or non-elastic surfaces have distinguished two types of forces produced, some by expansion or contraction, others by the flexion of these same surfaces. Moreover, they generally assumed, in their calculations, that the forces of the first type, called tensions, remain perpendicular to the lines against which they are exerted. It seemed to me that these two types of forces could be reduced by a single one, which must constantly be called tension or pressure, acting on each element of a section chosen at will, not only in a flexible surface, but also in an elastic or non-elastic solid, and which is of the same nature of the hydrostatic pressure exerted by a fluid at rest against the exterior surface of a body. However, the new pressure does not always remain perpendicular to the faces which are subjected to it, nor the same in all directions at a given point. In developing this idea, I came to recognize that the pressure or tension exerted against any plane at a given point of a solid body is very easily deduced, both in magnitude and in direction, from the pressures or tensions exerted against three rectangular planes defined through this same point. This proposition, which I have already addressed in the January 1823 edition of the \emph{Bulletin Des Sciences de La Soci\'et\'e Philomatique de Paris}, can be established using the following considerations.

If in an elastic or non-elastic solid body, a small element of volume defined by arbitrary faces is made rigid and invariable, this small element will experience on its different faces and at each point of each of them a certain pressure or tension. This pressure or tension will be similar to the pressure that a fluid exerts against an element of the surface of a solid body, with the only difference that the pressure exerted by a fluid at rest, against the surface of a solid body, is directed perpendicular to this surface from outside to inside, and independent at each point of the inclination of the surface with respect to the coordinate planes, while the pressure or tension exerted at a given point of a solid body, through which a very small surface element passes can be directed perpendicular or obliquely to this surface, sometimes from outside to inside, if there is contraction, sometimes from inside to outside, if there is expansion, and may depend on the inclination of the surface in relation to the planes in question. That being said, let $\upsilon$ be the volume of a portion of the body that has become rigid, $s$, $s'$, $s''$, ... the areas of the plane or curved surfaces which cover the volume $\upsilon$; $x$, $y$, $z$ the rectangular coordinates of a point taken at random in the surface $s$; $p$ the pressure or tension exerted at this point against the surface; $\alpha$, $\beta$, $\gamma$ the angles that the perpendicular to the surface forms with the semi-axes of the positive coordinates; finally $\lambda$, $\mu$, $\nu$ the angles formed with the same semi-axes by the direction of the force $p$. If we project onto the axes $x$, $y$ and $z$ the various pressures or tensions to which the surface will be subjected, the sums of their algebraic projections on these three axes will be represented by the integrals\footnote{All the subsequent terms $\sec\gamma\, dydx$ correct the term  $\cos\gamma\, dydx$ of the original text. (TN)}
 \begin{equation}\label{eq:forcas}
   \begin{dcases}
     \textstyle\iint   p\cos\lambda\sec\gamma\, dydx,\\
     \textstyle\iint   p\cos\mu\sec\gamma\, dydx,\\
     \textstyle\iint   p\cos\nu\sec\gamma\, dydx,
   \end{dcases}
 \end{equation}
while the sums of the algebraic projections of their linear moments will be respectively
 \begin{equation}
   \begin{dcases}
     \textstyle\iint   p(y\cos\nu-z\cos\mu)\sec\gamma\, dydx,\\
     \textstyle\iint   p(z\cos\lambda-x\cos\nu)\sec\gamma\, dydx,\\
     \textstyle\iint   p(x\cos\mu-y\cos\lambda)\sec\gamma\, dydx,
   \end{dcases}
 \end{equation}
if we take the origin of the coordinates as the center of moments, or, if we transport it to a point with coordinates $x_0$, $y_0$, $z_0$, 
 \begin{equation}\label{eq:momentos}
   \begin{dcases}
     \textstyle\iint   p[(y-y_0)\cos\nu-(z-z_0)\cos\mu]\sec\gamma\, dydx,\\
     \textstyle\iint   p[(z-z_0)\cos\lambda-(x-x_0)\cos\nu]\sec\gamma\, dydx,\\
     \textstyle\iint   p[(x-x_0)\cos\mu-(y-y_0)\cos\lambda]\sec\gamma\, dydx.
   \end{dcases}
 \end{equation}
 In all these integrals, the limits of the integrations relating to variables $x$, $y$ must be determined from the shape of the contour of the surface $s$, so that inside these limits 
\begin{equation}
\textstyle\iint\sec\gamma\, dydx = s\,.
\end{equation}

If the surface $s$ becomes plane and the volume $\upsilon$ very small, so that each of its dimensions can be considered an infinitely small quantity of the first order, then the variations that the three products
\begin{equation}
p\cos\lambda,\quad p\cos\mu,\quad p\cos\nu
\end{equation}
will experience, in the passage from one point to another in the surface $s$, will still be infinitely small of the first order; and, by neglecting the infinitely small third order values in the integrals \eqref{eq:forcas}, we shall reduce these integrals to the quantities
\begin{equation}\label{eq:ps}
ps\cos\lambda,\quad ps\cos\mu,\quad ps\cos\nu\,.
\end{equation}
Moreover, if we make the center of moments coincide with a point in volume $\upsilon$, the integrals \eqref{eq:momentos} will be infinitely small quantities of the third order, and it will suffice to neglect, in these integrals, the infinitely small term of the fourth order, so that they are reduced to products
 \begin{equation}\label{eq:ps2}
   \begin{dcases}
        ps[(\eta-y_0)\cos\nu-(\zeta-z_0)\cos\mu],\\
       ps[(\zeta-z_0)\cos\lambda-(\xi-x_0)\cos\nu],\\
        ps[(\xi-x_0)\cos\mu-(\eta-y_0)\cos\lambda],
   \end{dcases}
 \end{equation}
 $\xi$, $\eta$, $\zeta$ designating the fractions
 \begin{equation}
 \dfrac{\textstyle\iint x\sec\gamma\, dydx}{s},\quad  \dfrac{\textstyle\iint y\sec\gamma\, dydx}{s},\quad  \dfrac{\textstyle\iint z\sec\gamma\, dydx}{s},
 \end{equation}
that is, the coordinates of the center of gravity of the surface $s$.

Now, let $m$ be the infinitely small mass related to the volume $\upsilon$. Moreover, let us consider that the letter $\varphi$ represents the accelerating force applied to this mass, if the solid body is in equilibrium, and, on the contrary case, the excess of the accelerating force applied on it which would be able to produce the observed motion of mass $m$. Finally, let us call X, Y, Z the algebraic projections of the force $\varphi$, and $\xi_0$, $\eta_0$, $\zeta_0$ the coordinates of the center of gravity of the mass $m$. If we suppose that the accelerating force $\varphi$ remains the same in magnitude and direction in all points of the mass $m$, there must be equilibrium between the driving force $m\varphi$ applied at the point $(\xi_0,\eta_0,\zeta_0)$ and the forces to which the pressures or tensions exerted on the surfaces $s$, $s'$, ... are reduced. So the sums of the algebraic projections of all these forces and their linear moments on the axes $x$, $y$, $z$ will have to be reduced to zero. So, if we want to place one or more accents after the letters $p$, $\lambda$, $\mu$, $\nu$, $\xi$, $\eta$, $\zeta$, presented in expressions \eqref{eq:ps} and \eqref{eq:ps2}, to indicate the new values that these expressions take when one passes from the surface $s$ to the surface $s'$, or $s''$, or $s''',$ ... one will find, neglecting, in the sums of the projected forces, the infinitely small of the third order, and in the sums of the projected linear moments, the infinitely small of the fourth order,
 \begin{equation}
   \begin{dcases}
        ps\cos\lambda+p's'\cos\lambda'+\cdots+m\text{X}=0,\\
       ps\cos\mu+p's'\cos\mu'+\cdots+m\text{Y}=0,\\
        ps\cos\nu+p's'\cos\nu'+\cdots+m\text{Z}=0;
   \end{dcases}
 \end{equation}
 \begin{equation}
   \begin{dcases}
   \begin{aligned}
        ps[(\eta-y_0)\cos\nu-(\zeta-z_0)\cos\mu&]+p's'[(\eta'-y_0)\cos\nu'-(\zeta'-z_0)\cos\mu']+\\
        &+\cdots+m[(\eta_0-y_0)\text{Z}-(\zeta_0-z_0)\text{Y}]=0,
   \end{aligned}\\
         \begin{aligned}
        ps[(\zeta-z_0)\cos\lambda-(\xi-x_0)\cos\nu&]+p's'[(\zeta'-z_0)\cos\lambda'-(\xi'-x_0)\cos\nu']+\\
        &+\cdots+m[(\zeta_0-z_0)\text{X}-(\xi_0-x_0)\text{Z}]=0
   \end{aligned}\\
         \begin{aligned}
        ps[(\xi-x_0)\cos\mu-(\eta-y_0)\cos\lambda&]+p's'[(\xi'-x_0)\cos\mu'-(\eta'-y_0)\cos\lambda']+\\
        &+\cdots+m[(\xi_0-x_0)\text{Y}-(\eta_0-y_0)\text{X}]=0.
   \end{aligned}
   \end{dcases}
 \end{equation}
Now, the mass $m$ being itself infinitely small of the third order, the terms which contain it will be of the third order in the formulas \eqref{eq:forcas} and of the fourth order in the formulas \eqref{eq:momentos}. We can therefore neglect these terms, and replace the formulas in question by the following
\begin{equation}\label{eq:forcas2}
   \begin{dcases}
        ps\cos\lambda+p's'\cos\lambda'+p''s''\cos\lambda''+p'''s'''\cos\lambda'''+\cdots=0,\\
       ps\cos\mu+p's'\cos\mu'+p''s''\cos\mu''+p'''s'''\cos\mu'''+\cdots=0,\\
        ps\cos\nu+p's'\cos\nu'+p''s''\cos\nu''+p'''s'''\cos\nu'''+\cdots=0;
   \end{dcases}
 \end{equation}
 \begin{equation}\label{eq:momentos2}
   \begin{dcases}
   \begin{aligned}
        ps[(\eta-y_0)\cos\nu-(\zeta-z_0)\cos\mu]+p's'[(\eta'-y_0)\cos\nu'-&(\zeta'-z_0)\cos\mu']+\\
        &+\cdots=0,
   \end{aligned}\\
         \begin{aligned}
        ps[(\zeta-z_0)\cos\lambda-(\xi-x_0)\cos\nu]+p's'[(\zeta'-z_0)\cos\lambda'-&(\xi'-x_0)\cos\nu']+\\
        &+\cdots=0
   \end{aligned}\\
         \begin{aligned}
        ps[(\xi-x_0)\cos\mu-(\eta-y_0)\cos\lambda]+p's'[(\xi'-x_0)\cos\mu'-&(\eta'-y_0)\cos\lambda']+\\
        &+\cdots=0.
   \end{aligned}
   \end{dcases}
 \end{equation}

 If we wanted to take into account the variations that the accelerating force $\varphi$ and its projections X, Y, Z can experience, when we go from one point to another in the mass $m$, we would have to replace, in equations \eqref{eq:forcas} and \eqref{eq:momentos}, the six quantities
\begin{align*}
m\text{X},\quad m&\text{Y},\quad m\text{Z};\\[5pt]
m[(\eta_0-y_0)\text{Z}&-(\zeta_0-z_0)\text{Y}],\\
m[(\zeta_0-z_0)\text{X}&-(\xi_0-x_0)\text{Z}],\\
m[(\xi_0-x_0)\text{Y}&-(\eta_0-y_0)\text{X}]
\end{align*}
by six integrals of the form
\begin{align*}
\textstyle\iiint \rho \text{X}\,dzdydx,\quad \textstyle\iiint \rho &\text{Y}\,dzdydx,\quad \textstyle\iiint \rho \text{Z}\,dzdydx;\\[5pt]
\textstyle\iiint\rho[(y-y_0)&\text{Z}-(z-z_0)\text{Y}]\,dzdydx,\\
\textstyle\iiint\rho[(z-z_0)&\text{X}-(x-x_0)\text{Z}]\,dzdydx,\\
\textstyle\iiint\rho[(x-x_0)&\text{Y}-(y-y_0)\text{X}]\,dzdydx;
\end{align*}
$\rho$ denoting the density of the solid body at the point $(x, y, z)$, and the integration limits being relative to the limits of the volume $\upsilon$. But, since the first three integrals would be infinitely small of the third order, and the last three infinitely small of the fourth order, we would still find ourselves brought back to formulas \eqref{eq:forcas2} and \eqref{eq:momentos2}. It remains to show how, with the help of these formulas, one can discover the relations that exist between the pressures or tensions exerted in a given point of a solid body against various planes carried out successively by the same point.

Let us first consider that the volume $\upsilon$ takes the form of a right prism, the two bases of which are represented by $s$ and by $s'$. We shall have $s'=s$; and if, the dimensions of each base being considered as infinitely small of the first order, the height of the prism becomes an infinitely small quantity of an order greater than the first, then, neglecting, in formulas \eqref{eq:forcas2}, the infinitely small order greater than the second, we shall find
\begin{equation*}
(p\cos\lambda+p'\cos\lambda')s=0, \quad(p\cos\mu+p'\cos\mu')s=0,\quad(p\cos\nu+p'\cos\nu')s=0,
\end{equation*}
or equally,
\begin{equation*}
p'\cos\lambda'=-p\cos\lambda, \quad p'\cos\mu'=-p\cos\mu,\quad p'\cos\nu'=-p\cos\nu,
\end{equation*}
and we conclude that
\begin{align*}
p&=p',\\
\cos\lambda'=-\cos\lambda,\quad \cos\mu'&=-\cos\mu,\quad \cos\nu'=-\cos\nu.
\end{align*}
These last equations, which take place only in the case where the height of the prism vanishes, comprise a theorem easy to predict, the statement of which is the following:
\begin{thm}
The pressures or tensions, at a given point of a solid body, exerted against the two faces of an arbitrary plane through this point, are equal and directly opposite forces.
\end{thm} 

Now, let 
\begin{equation}
p',\quad p'',\quad p'''
\end{equation}
be the pressures or tensions exerted at the point $(x,y,z)$ and on the side of the positive coordinates against three planes through this point parallel to the coordinate planes of $y$, $z$, of $z$, $x$ and of $x$, $y$. Moreover, $\lambda'$, $\mu'$, $\nu'$; $\lambda''$, $\mu''$, $\nu''$; $\lambda'''$, $\mu'''$, $\nu'''$ are the angles formed by the directions
of the forces $p', p'',p'''$ with the semi-axes of the positive coordinates. Finally, let us consider that the volume $\upsilon$, taking the form of a rectangular parallelepiped, is enclosed between the three planes through the point $(x, y, z)$, and three parallel planes trough a very close point $(x+\Delta x, y+\Delta y, z+\Delta z)$. The pressures or tensions, supported by the faces of the parallelepiped which will end at this last point, will be approximately
\begin{equation}
p'\Delta y\Delta z,\quad p''\Delta z\Delta x,\quad p''''\Delta x\Delta y,
\end{equation}
while their algebraic projections on the axes $(x,y,z)$ will clearly reduce to the quantities
\begin{equation}
   \begin{dcases}
       p's'\cos\lambda'\Delta y\Delta z,\quad p''s''\cos\lambda''\Delta z\Delta x,\quad p'''s'''\cos\lambda'''\Delta x\Delta y,\\
       p's'\cos\mu'\Delta y\Delta z,\quad p''s''\cos\mu''\Delta z\Delta x,\quad p'''s'''\cos\mu'''\Delta x\Delta y,\\
       p's'\cos\nu'\Delta y\Delta z,\quad p''s''\cos\nu''\Delta z\Delta x,\quad p'''s'''\cos\nu'''\Delta x\Delta y.
   \end{dcases}
 \end{equation}
Concerning the pressures or tensions supported by the faces which end at the point $(x, y, z)$, they will be, by virtue of Theorem I, respectively equal, but directly opposite, to those which act on the parallel faces through the point $(x+\Delta x, y+\Delta y, z+\Delta z)$. So the algebraic projections of these new tensions will be numerically equal to the algebraic projections of the other three, but affected by opposite signs, so that each of the formulas \eqref{eq:forcas2} will become an identity. Let us add that the centers of gravity of the six faces of the parallelepiped will merge with their geometric centers, and will be located on three lines carried out parallel to the axes $(x, y, z)$ through the center of the parallelepiped, that is to say, through the point which has coordinates
\begin{equation*}
x+\dfrac{1}{2} \Delta x,\quad y+\dfrac{1}{2} \Delta y,\quad z+\dfrac{1}{2} \Delta z.
\end{equation*}
That said, it is clear that, if we take this last point as the center of the moments, the first of the formulas \eqref{eq:momentos2} will give
\begin{align*}
p''\cos\nu''\Delta z\Delta x\dfrac{\Delta y}{2}- p'''\cos\mu'''\Delta x\Delta y&\dfrac{\Delta z}{2}-(-p''\cos\nu'')\Delta z\Delta x\dfrac{\Delta y}{2}+\\
&+(-p'''\cos\mu''')\Delta x\Delta y\dfrac{\Delta z}{2}=0
\end{align*}
and consequently 
\begin{equation}\label{eq:momentos3}
   \begin{dcases}
   p''\cos\nu'' = p'''\cos\mu'''.\\
\text{Similarly, one will find that}\\
p'''\cos\lambda''' = p'\cos\nu',\\
p'\cos\mu'=p''\cos\lambda''.
    \end{dcases}
 \end{equation}
Since the axes $x$, $y$, $z$ are entirely arbitrary, the equations \eqref{eq:momentos3} obviously comprise the theorem we are going to state:
\begin{thm}
If through any point of a solid body we define two axes which intersect at right angles, and if we project onto one of these axes the pressure or tension supported by a plane perpendicular to the other at the point where it acts, the projection thus obtained will not vary when these same axes are exchanged between them.
\end{thm} 

Let us now consider that the volume takes the form of a tetrahedron of which three edges coincide with three infinitely small lengths carried from the point $(x, y, z)$ on lines parallel to the coordinate axes. Consider the point $(x, y, z)$ to be the vertex of this tetrahedron; denote its base by $s$, and let $\alpha$, $\beta$, $\gamma$ be the angles formed, with its semi-axes of positive coordinates, by a perpendicular raised through a point of this base, but extended outside the tetrahedron. The three faces which end at the vertex of the tetrahedron will be measured by the numerical values of the products
\begin{equation}
s\cos\alpha,\quad s\cos\beta,\quad s\cos\gamma.
\end{equation}
Thereby, if one calls $p$ the pressure or tension supported by the base of the tetrahedron, and if one continues to attribute to the quantities $p'$, $p''$, $p'''$ the values which they received in equations \eqref{eq:momentos3}, the first of the formulas \eqref{eq:forcas2} will obviously give
\begin{equation*}
ps\cos\lambda-p'\cos\lambda's\cos\alpha-p''\cos\lambda''s\cos\beta-p'''\cos\lambda'''s\cos\gamma = 0
\end{equation*}
and consequently
\begin{equation}
   \begin{dcases}
   p\cos\lambda = p'\cos\lambda'\cos\alpha+p''\cos\lambda''\cos\beta+p'''\cos\lambda'''\cos\gamma.\\
\text{Similarly, one will find that}\\
p\cos\mu = p'\cos\mu'\cos\alpha+p''\cos\mu''\cos\beta+p'''\cos\mu'''\cos\gamma,\\
p\cos\nu = p'\cos\nu'\cos\alpha+p''\cos\nu''\cos\beta+p'''\cos\nu'''\cos\gamma.
    \end{dcases}
 \end{equation}
 So, in order to shorten expressions, if we make
 \begin{equation}
   \begin{dcases}
   \text{A} = p'\cos\lambda',\\
   \text{B} = p''\cos\mu'',\\
   \text{C} = p'''\cos\nu''',\\
    \text{D} = p''\cos\nu''=p'''\cos\mu''',\\  
    \text{E} = p'''\cos\lambda'''=p'\cos\nu',\\ 
    \text{F} = p'\cos\mu'=p''\cos\lambda'',\\ 
    \end{dcases}
 \end{equation}
 we shall simply have 
\begin{equation}\label{eq:matriz}
   \begin{dcases}
p\cos\lambda = \text{A}\cos\alpha+\text{F}\cos\beta+\text{E}\cos\gamma,\\
p\cos\mu = \text{F}\cos\alpha+\text{B}\cos\beta+\text{D}\cos\gamma,\\
p\cos\nu = \text{E}\cos\alpha+\text{D}\cos\beta+\text{C}\cos\gamma.
    \end{dcases}
 \end{equation}
These last equations show the relations which subsist, for the point $(x, y, z)$, between the algebraic projections
\begin{equation}\left\{
   \begin{array}{ccc}
\text{A}, & \text{F}, & \text{E};\\
\text{F}, & \text{B}, & \text{D};\\
\text{E}, & \text{D}, & \text{C}
    \end{array}\right.
 \end{equation}
 of pressures $p'$, $p''$, $p'''$ exerted at this point, on the side of the positive coordinates, against three planes parallel to the coordinate planes, and the algebraic projections
\begin{equation*}
p\cos\lambda,\quad p\cos\mu,\quad p\cos\nu
\end{equation*}
of the pressure or tension $p$ exerted at this same point against any plane perpendicular to a straight line which, extended from the side where the force $p$ occurs, forms, with the semi-axes of positive coordinates, the angles $\alpha$, $\beta$, $\gamma$. 

From equations \eqref{eq:matriz}, it is easy to recognize that, if the volume $\upsilon$, instead of having the shape of a tetrahedron, is defined by any number of plane faces, the formulas \eqref{eq:forcas2} and \eqref{eq:momentos2} will always be verified. Indeed, these different faces being represented by $s$, $s'$, ..., let us call $\alpha$, $\beta$, $\gamma$; $\alpha'$, $\beta'$, $\gamma'$; ... the angles that straight lines perpendicular to the planes of these same faces, and extended outside the volume $\upsilon$, form with the semi-axes of positive coordinates. In order to obtain the first of formulas \eqref{eq:forcas2}, it will suffice to add the equations (1) on page 55\footnote{This reference concerns the article \textit{Sur Quelques Propri\'et\'es des Poly\`edres}, in the same volume of the current article, from which we extract the following excerpt. (TN).

\noindent\textbf{Theorem I} - \emph{The sum of the algebraic projections of the faces of any polyhedron on the coordinate planes results zero.} ...
\begin{equation*}
   \begin{dcases}
s\cos\alpha+s'\cos\alpha'+s''\cos\alpha''+\cdots = 0,\\
s\cos\beta+s'\cos\beta'+s''\cos\beta''+\cdots = 0,\\
s\cos\gamma+s'\cos\gamma'+s''\cos\gamma''+\cdots = 0.
    \end{dcases}\tag{1}
 \end{equation*}
}, after having respectively multiplied them by A, F, E, taking into account the first of formulas \eqref{eq:matriz}, as well as similar formulas. We establish, in the same way, the second and the third of the formulas \eqref{eq:forcas2}, by adding the equations (1) (p.55), after having respectively multiplied them by the coefficients F, B, D, or by the coefficients E, D, C. Finally, if we combine the formulas \eqref{eq:matriz} and others from the same type, not only with the equations (1) on page 55, but also with the equations (5) on page 56\footnote{Again referring to the same article of the previous footnote, this equation (5) can be expressed by the equality
\begin{equation}\left[
   \begin{array}{ccc}
\xi & \xi' & \cdots\\
\eta & \eta' & \cdots\\
\zeta & \zeta' & \cdots
    \end{array}\right]\left[
   \begin{array}{ccc}
s\cos\alpha & s\cos\beta  & s\cos\gamma\\
s'\cos\alpha' & s'\cos\beta' & s'\cos\gamma'\\
\vdots & \vdots & \vdots
    \end{array}\right]=\left[
   \begin{array}{ccc}
\upsilon & 0 & 0\\
0 & \upsilon & 0\\
0 & 0 & \upsilon
    \end{array}\right]\,,\tag{5}
 \end{equation}
according to what is developed in the article. (TN)
}, we shall easily arrive at formulas \eqref{eq:momentos2}. 

One can easily deduce from formulas \eqref{eq:matriz}:  1\textsuperscript{o} the intensity of the force $p$; 2\textsuperscript{o} the angle between the direction of this force and the perpendicular to the plane against which it is exerted. Thereby, if we add these formulas, after having squared each of their members, we shall find
\begin{align}\label{eq:p2}
\begin{dcases}
p^2 = &(\text{A}\cos\alpha+\text{F}\cos\beta+\text{E}\cos\gamma)^2+\\
&(\text{F}\cos\alpha+\text{B}\cos\beta+\text{D}\cos\gamma)^2+\\
&(\text{E}\cos\alpha+\text{D}\cos\beta+\text{C}\cos\gamma)^2.
\end{dcases}
\end{align}
Moreover, if we call $\delta$ the angle that we have just spoken, we will obviously have
\begin{equation}
\cos\delta = \cos\alpha\cos\lambda + \cos\beta\cos\mu+\cos\gamma\cos\nu,
\end{equation}
and consequently,
\begin{align}
\cos\delta = \dfrac{1}{p}(\text{A}\cos^2\alpha+\text{B}\cos^2\beta+\text{C}\cos^2\gamma+2\text{D}\cos\beta\cos\gamma &+2\text{E}\cos\gamma\cos\alpha+\\
&+2\text{F}\cos\alpha\cos\beta).\nonumber
\end{align}
Let us add that, if we describe the force $p$ by two components, one of which is included in the plane under consideration, and the other perpendicular to this plane, the second component will be represented, except for the sign, by the product
\begin{equation}\label{eq:componente}
\begin{dcases}
\begin{aligned}
p\cos\delta &= \text{A}\cos^2\alpha+\text{B}\cos^2\beta+\text{C}\cos^2\gamma+\\
&+2\text{D}\cos\beta\cos\gamma +2\text{E}\cos\gamma\cos\alpha+2\text{F}\cos\alpha\cos\beta.
\end{aligned}
\end{dcases}
\end{equation}
Let us observe finally that this second component will be a tension or a pressure according to whether the formula \eqref{eq:componente} will present a positive or negative value in the second member.

Suppose now that from the point $(x, y, z)$ we define, on the perpendicular to the plane against which the force $p$ acts, a length $r$ whose square represents the numerical value of the ratio
\begin{equation}
\dfrac{1}{p\cos\delta}
\end{equation}
and denote by $x + \text{x}$, $y + \text{y}$, $z + \text{z}$ the coordinates of the extremity of that same length. We shall have
\begin{align}
\dfrac{\text{x}}{\cos\alpha}=\dfrac{\text{y}}{\cos\beta}=\dfrac{\text{z}}{\cos\gamma}=&\pm\sqrt{\text{x}^2+\text{y}^2+\text{z}^2}=\pm r\,,\\
\dfrac{1}{p\cos\delta}=&\pm r^2;
\end{align}
and, therefore, formula \eqref{eq:componente} will give
\begin{equation}\label{eq:quadr}
\text{A}\text{x}^2+\text{B}\text{y}^2+\text{C}\text{z}^2+2\text{Dyz}+2\text{Ezx}+2\text{Fxy} = \pm 1\,.
\end{equation}
The variables x, y, z, included in equation \eqref{eq:quadr}, are the coordinates of the end of the length $r$, computed from the point $(x, y, z)$ on three rectangular axes; and this equation itself defines a quadratic surface whose center is the point $(x, y, z)$. When the polynomial
\begin{equation}\label{eq:ellip}
\begin{dcases}
\begin{aligned}
\text{A}\cos^2\alpha &+\text{B}\cos^2\beta+\text{C}\cos^2\gamma+\\
&+2\text{D}\cos\beta\cos\gamma +2\text{E}\cos\gamma\cos\alpha+2\text{F}\cos\alpha\cos\beta.
\end{aligned}
\end{dcases}
\end{equation}
preserves the same sign, for whatever values assigned to the angles $\alpha$, $\beta$, $\gamma$, then equation \eqref{eq:componente}, reduced to one of the following
\begin{align}
\text{A}\text{x}^2+\text{B}\text{y}^2+\text{C}\text{z}^2+2\text{Dyz}+2\text{Ezx}+2\text{Fxy} =& 1\,,\label{eq:ellip1}\\
\text{A}\text{x}^2+\text{B}\text{y}^2+\text{C}\text{z}^2+2\text{Dyz}+2\text{Ezx}+2\text{Fxy} =& -1\,,\label{eq:ellip2}
\end{align}
represents an ellipsoid. But, if the polynomial \eqref{eq:ellip} changes in sign while the angles $\alpha$, $\beta$, $\gamma$ vary, the ellipsoid in question will give way to the system of two hyperboloids, one of which will be represented by the equation \eqref{eq:ellip1}, the other by equation \eqref{eq:ellip2}; and these two hyperboloids, one of which will have a single sheet, the other two distinct sheets, will be conjugated\footnote{Regarding the properties of conjugated hyperboloids, see \emph{Le\c{c}ons sur les applications du Calcul infinit\'esimal \`a la G\'eom\'etrie}, p. 275, (\emph{OEuvres de Cauchy}, S. II, T.V)} between them, so that they will have the same center with the same axes, and will be touched at infinity by the same conical surface of the second degree. Let us add that, in the first case, the force
\begin{equation}
 \pm p\cos\delta = \dfrac{1}{r^2}
 \end{equation}
 will always be a tension, if the polynomial \eqref{eq:ellip} is positive, a pressure if it is negative. In the second case, on the contrary, the force in question will be sometimes a pressure, sometimes a tension, according to whether the end of the vector ray $r$ is located on the surface of one or the other hyperboloid; and the same force will vanish whenever this vector ray is directed along a generatrix of the above mentioned conical surface.

It can be easily demonstrated that the normal, defined at the end of the radius vector $r$ on the surface \eqref{eq:ellip1} or \eqref{eq:ellip2}, forms, with the semi-axes of positive coordinates, angles whose cosines are proportional to the three polynomials
\begin{align*}
&\text{A}\cos\alpha+\text{F}\cos\beta+\text{E}\cos\gamma\,,\\
&\text{F}\cos\alpha+\text{B}\cos\beta+\text{D}\cos\gamma\,,\\ 
&\text{E}\cos\alpha+\text{D}\cos\beta+\text{C}\cos\gamma\,.
\end{align*}
So this normal will be directed along the same line as the radius vector if we have
\begin{equation}\label{eq:princ}
\begin{dcases}
\begin{aligned}
&\dfrac{\text{A}\cos\alpha+\text{F}\cos\beta+\text{E}\cos\gamma}{\cos\alpha}\\
=&\dfrac{\text{F}\cos\alpha+\text{B}\cos\beta+\text{D}\cos\gamma}{\cos\beta}\\
=&\dfrac{\text{E}\cos\alpha+\text{D}\cos\beta+\text{C}\cos\gamma}{\cos\gamma}\,.
\end{aligned}
\end{dcases}
\end{equation}
Formula \eqref{eq:princ} is verified, in fact, when the radius vector coincides with one of the axes of the surface \eqref{eq:ellip1} or \eqref{eq:ellip2}. Then, we write equations \eqref{eq:matriz}, combined with formula \eqref{eq:princ}, as
\begin{equation}\label{eq:princ1}
\dfrac{\cos\lambda}{\cos\alpha}=\dfrac{\cos\mu}{\cos\beta}=\dfrac{\cos\nu}{\cos\gamma}=\pm\dfrac{\sqrt{\cos^2\lambda+\cos^2\mu+\cos^2\nu}}{\sqrt{\cos^2\alpha+\cos^2\beta+\cos^2\gamma}}=\pm 1\,,
\end{equation}
and it follows that the force $p$ is itself directed along the radius vector $r$, or along its extension. Consequently, to the three axes of the surface \eqref{eq:ellip1} or \eqref{eq:ellip2} correspond three pressures or tensions, each of which is perpendicular to the plane against which it is exerted. We shall call them \emph{principal pressures} or \emph{tensions}. It is then easy to ensure that one finds among them the \emph{maximum} pressure or tension and the \emph{minimum} pressure or tension; because, if we equal to zero the value of $p$ taken from formula \eqref{eq:p2}, and if we consider the equation
\begin{equation}\label{eq:cos1}
\cos^2\alpha+\cos^2\beta+\cos^2\gamma = 1\,,
\end{equation}
from which one of the three variables $\alpha$, $\beta$, $\gamma$ becomes a function of the two others considered as independent, we shall immediately be brought back to formula \eqref{eq:princ}.

If, starting from the point $(x, y, z)$, we had, on the perpendicular to the plane against which the force $p$ acts, a length equivalent, no longer to the square root of the ratio $\pm \frac{1}{p\cos\delta}$, but to the fraction $\frac{1}{p}$, by designating $x + \text{x}$, $y + \text{y}$, $z+ \text{z}$ the coordinates of the end of this length, we would find
\begin{equation}\label{eq:ellip4}
\begin{aligned}
\dfrac{\text{x}}{\cos\alpha}=\dfrac{\text{y}}{\cos\beta}=\dfrac{\text{z}}{\cos\gamma}=&\pm\sqrt{\text{x}^2+\text{y}^2+\text{z}^2}=\pm r\,,\\
\dfrac{1}{p}=&r;
\end{aligned}
\end{equation}
and, consequently, the formula \eqref{eq:p2} would give
\begin{equation}\label{eq:ellip3}
(\text{Ax}+\text{Fy}+\text{Ez})^2+(\text{Fx}+\text{By}+\text{Dz})^2+(\text{Ex}+\text{Dy}+\text{Cz})^2 = 1\,.
\end{equation}
Equation \eqref{eq:ellip3} describes an ellipsoid whose axes correspond to the values of $\alpha,\beta,\gamma$ determined by the formula

\begin{scriptsize}
\begin{equation}
\begin{dcases}
\begin{aligned}
&\dfrac{\text{A}(\text{A}\cos\alpha+\text{F}\cos\beta+\text{E}\cos\gamma)+\text{F}(\text{F}\cos\alpha+\text{B}\cos\beta+\text{D}\cos\gamma)+\text{E}(\text{E}\cos\alpha+\text{D}\cos\beta+\text{C}\cos\gamma)}{\cos\alpha}\\
=&\dfrac{\text{F}(\text{A}\cos\alpha+\text{F}\cos\beta+\text{E}\cos\gamma)+\text{B}(\text{F}\cos\alpha+\text{B}\cos\beta+\text{D}\cos\gamma)+\text{D}(\text{E}\cos\alpha+\text{D}\cos\beta+\text{C}\cos\gamma)}{\cos\beta}\\
=&\dfrac{\text{E}(\text{A}\cos\alpha+\text{F}\cos\beta+\text{E}\cos\gamma)+\text{D}(\text{F}\cos\alpha+\text{B}\cos\beta+\text{D}\cos\gamma)+\text{C}(\text{E}\cos\alpha+\text{D}\cos\beta+\text{C}\cos\gamma)}{\cos\gamma}\,.
\end{aligned}
\end{dcases}
\end{equation}
\end{scriptsize}

\noindent However, since this formula is obviously verified by the values of $\alpha,\beta,\gamma$ which satisfy the formula \eqref{eq:princ}, one can affirm that the axes of the new ellipsoid are directed along the same lines of the principal pressures or tensions. We would come to the same conclusion by observing that the \emph{maximum} and \emph{minimum} values of the vector radius, that is to say the major axis and the minor axis of the ellipsoid, necessarily correspond, by virtue of equation \eqref{eq:ellip4}, the first, to the \emph{maximum} pressure or tension, the second, to the \emph{minimum} pressure or tension.

By summarizing the various propositions that we have just established, we shall obtain the following theorem:
\begin{thm}
If, after having made an arbitrary plane pass through a given point of a solid body, we define, from this point and on each of the half-axes perpendicular to the plane, two equivalent lengths, the first, of the unit divided by the pressure or tension exerted against the plane, the second, of the unit divided by the square root of this force projected on one of the semi-axes which one considers, these two lengths will be the vector rays of two ellipsoids, whose axes will be directed along the same lines. To these axes there will correspond the principal pressures or tensions, each of which will be normal to the plane which will support it, and on which we will always find the \emph{maximum} pressure or tension, as well as the \emph{minimum} pressure or tension. Concerning the other pressures or tensions, they will be distributed symmetrically around the axes of the two ellipsoids. Let us add that, in certain cases, the second ellipsoid will be replaced by two conjugated hyperboloids. These cases are those in which the system of principal pressures or tensions consists of one tension and two pressures or one pressure and two tensions. Then, if we substitute the force which acts against each plane by two rectangular components, one of which is normal to the plane, this component will be a tension or a pressure, depending on whether the vector ray perpendicular to the plane belongs to one or the other of the two hyperboloids,
and it will vanish when the vector ray is directed along one of the generatrices of the conical surface of the second degree which touches the two hyperboloids at infinity.
\end{thm} 


Let us now consider that from the center of the first ellipsoid one defines arbitrarily three vector rays which intersect at right angles. We can easily prove that, if we divide the unit by each of these vector rays, the sum of the squares of the quotients will be a constant value, equal to the sum that would be obtained by making the three vector rays coincide with the three semi-axes of the ellipsoid (see \emph{Le\c{c}ons sur les applications du Calcul infinit\'esimal \`a la G\'eom\'etrie}, pp. 274-275)\footnote{\emph{OEuvres de Cauchy}, S. II, T.V.}. From this remark, together with the third theorem, we immediately deduce the following proposition:
\begin{thm}
If through a given point of a solid body we pass three rectangular planes between them, the sum of the squares of the pressures or tensions supported by these same planes will be a constant value, equal to the sum of the squares of the principal pressures or tensions.
\end{thm}

It may happen that the three principal pressures or tensions, or at least two of them, become equivalent. When these forces reduce to three equal pressures, or three equal tensions, the two ellipsoids we have spoken of reduce to two spheres. Then there is equality of pressure or tension in all directions, and each pressure or tension is perpendicular to the plane which supports it. Moreover, it is important to observe that, from these last two conditions, the second can only be attained as far as the first is equally attained. Indeed, if one supposes the force $p$ constantly directed along the line which forms, with the half-axes of the positive coordinates, the angles $\alpha,\beta,\gamma$, the formula \eqref{eq:princ} or \eqref{eq:princ1} will remain for any position in this line, and, consequently, for all the values of $\alpha,\beta,\gamma$ suitable for verifying equation \eqref{eq:cos1}. Now we derive from formula \eqref{eq:princ}: 1\textsuperscript{o} assuming two of the values
\begin{equation*}
\cos\alpha,\quad \cos\beta,\quad \cos\gamma
\end{equation*}
to be zero, and the third, to be unity, 
\begin{equation}\label{eq:dzero}
\text{D}=0,\quad \text{E}=0,\quad \text{F}=0;
\end{equation}
2\textsuperscript{o} considering equations \eqref{eq:dzero},
\begin{equation}
\text{A}=\text{B}=\text{C}\,.
\end{equation}
Consequently, under the accepted hypothesis, the formulas \eqref{eq:matriz} become
\begin{equation}
p\cos\alpha=\text{A}\cos\alpha,\quad p\cos\mu=\text{A}\cos\beta,\quad p\cos\nu=\text{A}\cos\gamma\,;
\end{equation}
and, since we conclude from these last formulas that
\begin{align}
\dfrac{\cos\alpha}{\cos\lambda}=\dfrac{\cos\beta}{\cos\mu}=\dfrac{\cos\gamma}{\cos\nu}=&\pm\dfrac{\sqrt{\cos^2\alpha+\cos^2\beta+\cos^2\gamma}}{\sqrt{\cos^2\lambda+\cos^2\mu+\cos^2\nu}}=\pm 1\,,\\
p=&\pm \text{A}\,,
\end{align}
it is clear that the pressure or tension, denoted by $p$, will remain the same in all directions. This is precisely what takes place when we consider a fluid mass in equilibrium. If two principal pressures or tensions become equal, the two ellipsoids mentioned in Theorem III are reduced to two ellipsoids of revolution, the second of which is replaced, in certain cases, by a system of two hyperboloids of revolution conjugated to one another. Then all the planes including the axis of revolution of these ellipsoids or hyperboloids support equivalent pressures or tensions, each of which, being perpendicular to the plane which is be subjected to it, can be considered as a principal pressure or tension.

The supposition we have just made includes the case where the three forces composing the system of the principal pressures or tensions are equivalent but reduced to a pressure and to two tensions, or to two pressures and a tension. It is important only to observe that, in this case, the first ellipsoid would be replaced by a sphere, and consequently all the planes carried out by the point $(x, y, z)$ would support equivalent pressures or tensions, but directed, some by perpendicular lines, others by straight lines oblique to these same planes.


Generally, whenever a principal tension becomes equivalent to a principal pressure, the planes driven by the axis perpendicular to the directions of these two forces will bear equivalent pressures or tensions, but which will remain oblique to the planes in question, as long as they are distinct from these same forces.

It can still be assumed that one or two of the principal tensions or pressures reduce to zero, or that they all disappear. In the first case, the ellipsoids or hyperboloids, mentioned in the third theorem, will turn into right cylinders which will have conjugate ellipses or hyperbolas as their bases. In the second case, each of these cylinders will be replaced by two parallel planes. In the third case, the pressure or tension, exerted against any plane led by the point $(x, y, z)$, will always reduce to zero.

The formulas previously obtained are simplified when we take for coordinate axes lines parallel to the directions of the principal pressures or tensions corresponding to the point $(x, y, z)$. Then, in fact, the surface, represented by equation \eqref{eq:quadr}, must be a quadratic surface correspondent to this coordinate axes, not only through its center, but also through its axes; and one must therefore have
\begin{equation*}
\text{D}=0,\quad \text{E}=0,\quad \text{F}=0.
\end{equation*}
That said, the numerical values of A, B, C will obviously represent the principal pressures or tensions and the formulas \eqref{eq:matriz}, \eqref{eq:p2}, \eqref{eq:componente} will be reduced to
\begin{align}
p\cos\lambda=\text{A}\cos\alpha,\quad p\cos\mu=&\text{B}\cos\beta,\quad p\cos\nu=\text{C}\cos\gamma\label{eq:45}\,,\\
p^2=\text{A}^2\cos^2\alpha+\text{B}^2&\cos^2\beta+\text{C}^2\cos^2\gamma\,,\label{eq:46}\\
p\cos\delta=\text{A}\cos^2\alpha+\text{B}&\cos^2\beta+\text{C}\cos^2\gamma\,\label{eq:47},
\end{align}
while equations \eqref{eq:quadr} and \eqref{eq:ellip3} will become
\begin{align}
\text{A}\text{x}^2+\text{B}\text{y}^2+\text{C}\text{z}^2=&\pm 1\label{eq:48}\,,\\
\text{A}^2\text{x}^2+\text{B}^2\text{y}^2+\text{C}^2\text{z}^2=&1\label{eq:49}\,,
\end{align}
Equations \eqref{eq:46} and \eqref{eq:47} show the relations which exist: 1\textsuperscript{o} between the main pressures or tensions, and the pressure or tension $p$ supported by an arbitrary plane; 2\textsuperscript{o} between the first three
forces and projections of the latter on a line perpendicular to the plane in question. The angles $\alpha,\beta,\gamma$ included in these same equations are precisely the angles formed by the perpendicular to the plane with the axes along which the principal pressures or tensions are directed.

In the particular case where we consider only points located in the x, y plane, and where we disregard one of the dimensions of the solid body, the formulas \eqref{eq:45}, \eqref{eq:46}, \eqref{eq:47}, \eqref{eq:48}, \eqref{eq:49} can be replaced by the following
\begin{align}
p\cos\lambda=\text{A}\cos\alpha,\quad &p\cos\mu=\text{B}\cos\beta\,,\\
p^2=\text{A}^2\cos^2\alpha&+\text{B}^2\cos^2\beta\,,\\
p\cos\delta=\text{A}\cos^2&\alpha+\text{B}\cos^2\beta\,\\
\text{A}\text{x}^2+\text{B}\text{y}^2&=\pm 1\,,\label{eq:53}\\
\text{A}^2\text{x}^2+\text{B}^2\text{y}^2&=1\,,\label{eq:54}
\end{align}
Then the ellipsoids or hyperboloids, mentioned in Theorems II and III, are also reduced to ellipses or conjugate hyperbolas, represented by equations \eqref{eq:53} and \eqref{eq:54}.

\vspace*{.5cm}
In other articles, I will show how one can deduce from the principles established above the equations which express the state of equilibrium or the internal motion of an elastic or non-elastic solid body.

% alfa, beta e gama são angulos das normais
\vspace*{.5cm}
\noindent 
\begin{center}
 \adforn{21}\quad\adforn{11}\quad\adforn{49}\vspace*{.5cm}
 \end{center}

\noindent\rule{\textwidth}{0.5pt}\vspace*{-\baselineskip}\vspace*{2pt} 

\setcounter{equation}{0}
\vspace*{1cm}
\begin{center}
{\LARGE \textbf{ADITION TO THE PREVIOUS ARTICLE}}\vspace*{3pt} 
\end{center}

The values of $p\cos\lambda$, $p\cos\mu$, $p\cos\nu$ given by the formulas (20) of the previous article, are entirely similar to the values of the rectangular components of the force which would solicit a material point placed in the presence of several fixed centers of attraction or repulsion, and very little displaced from a position in which it remained in equilibrium in the middle of the centers in question. Indeed, let us consider that the material point, after having coincided, in the equilibrium position, with the origin of the coordinates, has been transported to a very small distance and designated by $\varrho$. Moreover, let

\noindent $r$, $r'$, ... be the vector rays defined from the origin to the various centers fixed;

\noindent R, R', ... be the forces of attraction or repulsion which, emanating from the same centers, solicit the material point in the position of equilibrium;

\noindent P be the resultant force of those to which this point is subjected after its displacement.

Finally, let
\begin{equation*}
\alpha,\beta,\gamma\,; \quad a,b,c\,;\quad a',b',c'\,;\quad \cdots\,;\quad \lambda,\mu,\nu
\end{equation*}
be the angles the semi-axes of the positive coordinates form with: 1\textsuperscript{o} the radius vector $\varrho$; 2\textsuperscript{o} the vector rays $r$, $r'$, ...; the direction of the force $P$. Assuming the material point brought back to the equilibrium position, we will easily establish the equations
\begin{equation}
\textstyle\sum(\pm\text{R}\cos a)=0,\quad \textstyle\sum(\pm\text{R}\cos b)=0,\quad\textstyle\sum(\pm\text{R}\cos c)=0,
\end{equation}
where the symbol $\sum$ indicates a sum of similar terms, but relative to the various fixed centers and the sign $\pm$ reduces, sometimes to the sign $-$, sometimes to the sign $+$, depending on whether the force $R$ is repulsive or attractive. Let $f(r)$ now be the function of the distance $r$ which measures the force $R$. While the material point is transported from the origin to the end of the radius $\varrho$, the values $r$, $\text{R}$, $a$, $b$, $c$  will suffer correspondent increments that we will designate with the aid of the characteristic $\Delta$, and we will obviously have
\begin{equation}
\begin{dcases}
\begin{aligned}
(r+\Delta r)\cos(a+\Delta a)&=r\cos a-\varrho\cos\alpha,\\
(r+\Delta r)\cos(b+\Delta b)&=r\cos b-\varrho\cos\beta,\\
(r+\Delta r)\cos(a+\Delta a)&=r\cos a-\varrho\cos\gamma;
\end{aligned}
\end{dcases}
\end{equation}

\begin{equation}
\text{R}+\Delta \text{R} = f(r+\Delta r);
\end{equation}

\begin{equation}
\begin{dcases}
\begin{aligned}
\text{P}\cos\lambda&=\textstyle\sum[\pm (\text{R}+\Delta\text{R})\cos(a+\Delta a)]\,,\\
\text{P}\cos\mu&=\textstyle\sum[\pm (\text{R}+\Delta\text{R})\cos(b+\Delta b)]\,,\\
\text{P}\cos\nu&=\textstyle\sum[\pm (\text{R}+\Delta\text{R})\cos(c+\Delta c)]\,.
\end{aligned}
\end{dcases}
\end{equation}
Thereby, we will find that
\begin{equation}
\begin{dcases}
\begin{aligned}
(r+\Delta r)^2&=(r\cos a-\varrho\cos\alpha)^2+(r\cos b-\varrho\cos\beta)^2+(r\cos c-\varrho\cos\gamma)^2\\
&=r^2-2r\varrho(\cos a\cos\alpha+\cos b\cos\beta+\cos c\cos\gamma)+\varrho^2\,;
\end{aligned}
\end{dcases}
\end{equation}
and then, by considering the quantity $p$ as infinitely small of the first order and neglecting the infinitely small of the second order, we will conclude from formulas (2), (3), (5) that
\begin{equation}
\Delta r = - \varrho (\cos a\cos\alpha+\cos b\cos\beta+\cos c\cos\gamma)\,;
\end{equation}

\begin{align}
\text{R}+\Delta \text{R} = f(r) + f'(r)\Delta r = \text{R} + f'(r)\Delta r\,;
\end{align}

\begin{equation}
\begin{dcases}
\begin{aligned}
\cos(a+\Delta a)=\dfrac{r\cos a-\varrho\cos\alpha}{r+\Delta r}&=\cos a - \varrho\dfrac{\cos \alpha}{r}-\dfrac{\cos a}{r}\Delta r\,,\\
\cos(b+\Delta b)=\dfrac{r\cos b-\varrho\cos\beta}{r+\Delta r}&=\cos b - \varrho\dfrac{\cos \beta}{r}-\dfrac{\cos b}{r}\Delta r\,,\\
\cos(c+\Delta c)=\dfrac{r\cos c-\varrho\cos\beta}{r+\Delta r}&=\cos c - \varrho\dfrac{\cos \beta}{r}-\dfrac{\cos c}{r}\Delta r\,.
\end{aligned}
\end{dcases}
\end{equation}
That said, considering equations (1), we will recognize that formulas (4) can be reduced to
\begin{equation}
\begin{dcases}
\begin{aligned}
\text{P}\cos\lambda&=\textstyle\sum\left \{\pm \left[f'(r)-\dfrac{f(r)}{r}\right ]\cos a \Delta r\mp\dfrac{f(r)}{r}\varrho\cos\alpha\right \}\,,\\
\text{P}\cos\mu&=\textstyle\sum\left \{\pm \left[f'(r)-\dfrac{f(r)}{r}\right ]\cos b \Delta r\mp\dfrac{f(r)}{r}\varrho\cos\beta\right \}\,,\\
\text{P}\cos\nu&=\textstyle\sum\left \{\pm \left[f'(r)-\dfrac{f(r)}{r}\right ]\cos c \Delta r\mp\dfrac{f(r)}{r}\varrho\cos\gamma\right \}\,;
\end{aligned}
\end{dcases}
\end{equation}
then, recalling the value of $\Delta r$ in formula (6), and defining
\begin{equation}
\begin{dcases}
\begin{aligned}
\text{A}&=\varrho\textstyle\sum\left \{\mp \left[f'(r)-\dfrac{f(r)}{r}\right ]\cos^2 a \mp\dfrac{f(r)}{r}\right \}\,,\\
\text{B}&=\varrho\textstyle\sum\left \{\mp \left[f'(r)-\dfrac{f(r)}{r}\right ]\cos^2 b \mp\dfrac{f(r)}{r}\right \}\,,\\
\text{C}&=\varrho\textstyle\sum\left \{\mp \left[f'(r)-\dfrac{f(r)}{r}\right ]\cos^2 c \mp\dfrac{f(r)}{r}\right \}\,;
\end{aligned}
\end{dcases}
\end{equation}
\begin{equation}
\begin{dcases}
\begin{aligned}
\text{D}&=\varrho\textstyle\sum\left \{\mp \left[f'(r)+\dfrac{f(r)}{r}\right ]\cos b \cos c\right \}\,,\\
\text{E}&=\varrho\textstyle\sum\left \{\mp \left[f'(r)+\dfrac{f(r)}{r}\right ]\cos c \cos a\right \}\,,\\
\text{F}&=\varrho\textstyle\sum\left \{\mp \left[f'(r)+\dfrac{f(r)}{r}\right ]\cos a \cos b\right \}\,,
\end{aligned}
\end{dcases}
\end{equation}
in order to shorten expressions, we will definitely find
\begin{equation}
\begin{dcases}
\begin{aligned}
\text{P}\cos\lambda&=\text{A}\cos\alpha+\text{F}\cos\beta+\text{E}\cos\gamma\,,\\
\text{P}\cos\mu&=\text{F}\cos\alpha+\text{B}\cos\beta+\text{D}\cos\gamma\,,\\
\text{P}\cos\nu&=\text{E}\cos\alpha+\text{D}\cos\beta+\text{C}\cos\gamma\,.
\end{aligned}
\end{dcases}
\end{equation}
The formulas (12), as well as several propositions which are deduced from them and which are analogous to theorems I, II, III of the previous article, are due to Mr. Fresnel, who presented them in his \emph{Recherches sur la double r\'efraction}.

\vspace*{.5cm}
\noindent 
\begin{center}
 \adforn{21}\quad\adforn{11}\quad\adforn{49}\vspace*{.5cm}
 \end{center}
 
\end{document}
