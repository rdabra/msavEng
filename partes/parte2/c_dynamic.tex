\chapter{Continuum Dynamics}

bla bla bla        

bla bla bla 

bla bla bla 

bla bla bla 

bla bla bla 

\section{Inertia and Force}\index{inertia}\index{force} 

In simple terms, Dynamics is here understood as the descriptive study of the causes of motion. In the very first chapter of his \emph{Principia}, Newton inadvertently presents, in the third definition, the point of departure of Dynamics: the concept of inertia. Obscurely defined as an ``inherent force'' of matter, he states that this ``force'' is \emph{``the power of resisting by which every body, so far as it is able, perseveres in its state either of resting or of moving uniformly straight forward.\footnote{\aut{Newton}\cite{newton_1999_1}, p.50.}''} By this inherent resistance feature of matter, or inertia, presented by every body -- which is abstracted devoid of volume\footnote{Usually called a point mass.} -- Newton meant resistance to a change of its velocity in time, that is, resistance to acceleration. Therefore, the Newtonian concept of inertia makes the old Aristotelian idea of \emph{``a thing will either be at rest or must be moved \emph{ad infinitum}, unless something more powerful gets in its way\footnote{\aut{Aristotle}\cite{aristotle_1984_1}, p.366.}''} one of its corollaries: if a volumeless body is subjected to a condition where its inertia does not manifest itself, then this body will perform a non-accelerated motion, that is, it will be either at rest or move uniformly straight forward. One of the measures of the inertia of a volumeless body is a proportional scalar quantity called \textsb{mass}\index{mass}, or inertial mass: the greater or lesser this scalar, the greater or lesser the inertia. For the case of bodies with volume, which is our concern here, we impose that every body must have an overall regularly distributed mass, which is also deformation independent, that is, the mass calculated on any arbitrary shape is always the mass of the body\footnote{In motions where a portion of the mass of a body is burned in order to propel another portion, the above restriction is still valid because any shape of this body as a whole must include not only the propelled portion of its mass, but also that which is burned.}. This last condition, valid only in the context of Newtonian Mechanics, is a manifestation of the \textsb{Principle of Mass Conservation}\index{mass!principle of conservation of}, which states that when there are no mass exchanges with the surroundings of a certain physical system, the total mass of this system remains constant in time.

Following the first definition of the \emph{Principia}, ``quantity of matter'', Newton presented a measure called ``quantity of motion'', which is one of the many concepts he borrowed from Galileo. Dealing with the impact of solid volumeless bodies, a subject of great interest in his time, Galileo wrote in the sixth day of \emph{Dialogs Concerning Two New Sciences} the following reasoning exposed by the character Salviati\footnote{See section \ref{sec:leoGal} for details.}: \emph{``It is evident that the property of force in the mover and of resistance [inertia] in the moved is not single and simple, but is compounded from two actions, by which their energy must be measured. One of these is the weight \emph{[mass]}, of the mover as well as of the resistent; the other is the speed \emph{[velocity]} with which the one must move and the other be moved... if with a lesser weight we wish to raise a greater, it will be necessary to arrange the machine in such a way that the smaller moving weight goes in the same time through a greater space \emph{[distance]} than does the other weight; that is to say, the former is moved more swiftly than the latter... Let us say in general, then, that the momentum of the less heavy body balances the momentum of the more heavy when the speed of the lesser has the same ratio to the speed of the greater as the heaviness of the greater has to that of the lesser...\footnote{\aut{Galileo}\cite{galileo_1989_1}, pp.289-290.}}'' In modern terms, Galileo's \textsb{linear momentum}\index{momentum!linear}, or Newton's ``quantity of motion'', is defined to be a vector multiple of the velocity vector through inertial mass, that is, linear momentum is mass multiplied by velocity. Therefore, if a volumeless body is subjected to a condition where its inertia does manifest itself, the momentum of this body will change in time, a quantity usually called \textsb{impulse}\index{impulse}. Since inertial mass is constant in time, impulse is a vector multiple of the acceleration vector through inertial mass, or impulse is mass times acceleration. The notorious Newton's Second Law, as it is currently stated, specifies that the impulse of a volumeless body corresponds to the action on this body of an external physical entity called applied force. In other words, whenever a force is applied to a volumeless body, this body will present an impulse identical to the force applied. But now, what is the definition of force?  Unfortunately, there is not a precise answer to this question

     
