\chapter{Continuum Dynamics}

bla bla bla        


\section{Inertia and Force}\index{inertia}\index{force} 

In simple terms, Dynamics is here understood as the descriptive study of the causes of motion. In the very first chapter of his most famous work, Newton inadvertently presents, in the third definition, the point of departure of Dynamics: the concept of inertia. Defined as an ``inherent force'' of matter, he states that this ``force'' is \emph{``the power of resisting by which every body, so far as it is able, perseveres in its state either of resting or of moving uniformly straight forward.\footnote{\aut{Newton}\cite{newton_1999_1}, p.50.}''} By this resistance inherent feature of matter, or inertia, Newton meant resistance to any change in velocity. Therefore, the Newtonian concept of inertia makes the old Aristotelian idea of \emph{``a thing will either be at rest or must be moved \emph{ad infinitum}, unless something more powerful gets in its way\footnote{\aut{Aristotle}\cite{aristotle_1984_1}, p.366.}''} one of its corollaries: if no change is performed in the velocity of a body, its non manifested inertia will keep its velocity constant, that is, the body will be either at rest or moving uniformly straight forward.



 is not resisting to a change in its velocity, this body is necessarily in a state of constant velocity. 

whenever the inertia of a body does not manifest itself, this body is necessarily       


In simple terms, the purpose of the concept of mass is to measure the ``quantity of matter'', as Newton called it. For the purposes of our study, we impose that every limited portion of matter, called a body as we already know, must have a certain mass, which is also deformation independent, that is, the mass calculated on any arbitrary shape is always the mass of the body\footnote{In motions where a portion of the mass of a body is burned in order to propel another portion, the above restriction is still valid because any shape of this body as a whole must include not only the propelled portion of its mass, but also that which is burned.}.

In the very first chapter of his most famous work, \aut{Newton}\cite{newton_1999_1} presents, as the second definition, following ``quantity of matter'', a measure he called ``quantity of motion'', which is one of the many concepts he borrowed from Galileo. Dealing with the impact of solid bodies, a subject of great interest in his time, Galileo wrote in the sixth day of \emph{Dialogs Concerning Two New Sciences} the following reasoning explained by the character called Salviati\footnote{See section \ref{sec:leoGal} for details.}: \emph{``It is evident that the property of force in the mover and of resistance [inertia] in the moved is not single and simple, but is compounded from two actions, by which their energy must be measured. One of these is the weight \emph{[mass]}, of the mover as well as of the resistent; the other is the speed \emph{[velocity]} with which the one must move and the other be moved... if with a lesser weight we wish to raise a greater, it will be necessary to arrange the machine in such a way that the smaller moving weight goes in the same time through a greater space \emph{[distance]} than does the other weight; that is to say, the former is moved more swiftly than the latter... Let us say in general, then, that the momentum of the less heavy body balances the momentum of the more heavy when the speed of the lesser has the same ratio to the speed of the greater as the heaviness of the greater has to that of the lesser...\footnote{\aut{Galileo}\cite{galileo_1989_1}, pp.289-290.}}''
