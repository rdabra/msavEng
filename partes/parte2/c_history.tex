\chapter{Introduction to Continuum Mechanics}\label{ch:Collect}


This chapter starts with some historical aspects of Continuum Mechanics in order to situate the reader on the subject, which is a fundamental part of the science of Mechanics as a whole. By history of Continuum Mechanics we mean the direct contributions to the study of motion, strength and deformation of non rigid bodies; in other words, contributions to the mechanics of deformable bodies. Here, it is sufficient to rely on intuitive notions of all these concepts, which will be rigorously defined in the following chapters. Since our description is focused on non rigid bodies, works that contributed solely to the mechanics of material points will not be covered; otherwise, our history would be very, very long. Bla, bla, bla... 


\section{History}


An adequate historical account of the main contributors to the subject of Continuum Mechanics must start with the painter Leonardo da Vinci (1452-1519). Born in the city of Vinci, a comune of Florence, in the italian region of Tuscany, da Vinci became an apprentice, at the age of fourteen, in Andrea del Verrocchio's workshop, the most famous florentine artist of the time. Already informally educated on Latin, Geometry and Mathematics, at the workshop, he was taught, besides artistic abilities, a wide range of technical skills, including engineering, architecture, metallurgy and chemistry. Among his notable works, he left not only famous paintings like \emph{Mona Lisa} and \emph{The Last Supper}, but also notebooks, whose descriptive annotations and sketches cover a great variety of themes, from prosaic issues of his everyday life to exercises on Mathematics, architectural drawings, studies on painting and sculpture, engineering, cartography, astronomy, optics, botany, human anatomy, motions of fluids and solids, machine design and so on. Although it is not our purpose to criticize or profoundly analyze da Vinci's work, we must inform that in his notebooks there can be found the earliest known records on the mechanics of deformable bodies: a study on tensile test in notebook Codex Atlanticus f.222r; studies on beams and columns in notebooks Codices Atlanticus f.562r, 908r, Forster I  f.88v, 89r, Madrid I f.84v, 85v, 135v, 136r, 139r, 177v and Paris Manuscript A f.45v. 
\begin{figure}[!ht]
	\centering
	\begin{center}
		\scalebox{.80}{\includegraphics{partes/figs/Madrid84vAtl908r.jpg}}
	\end{center}
	\titfigura{Paris A f.45v, Atlanticus f.908r, Madrid I f.84v (right), 177v (bottom).}\label{fg:Madrid84vAtl908r}
\end{figure}
On figure \ref{fg:Madrid84vAtl908r}, some of these records are shown: sketches Paris A f.45v and Madrid I f.177v are studies on buckling of columns and the others are studies on bending of beams. In particular, the text on Madrid I f.84v, translated by \aut{Zammattio}\cite{zammattio_1980}, reads: \emph{``If a straight spring is bent, it is necessary that its convex part become thinner and its concave part, thicker. This modification is pyramidal, and consequently, there will never be a change in the middle of the spring. You shall discover, if you consider all of the aforementioned modifications, that by taking part `ab' in the middle of its length and then bending the spring in a way that the two parallel lines, `a' and `b' touch a the bottom, the distance between the parallel lines has grown as much at the top as it has diminished at the bottom. Therefore, the center of its height has become much like a balance for the sides. And the ends of those lines draw as close at the bottom as much as they draw away at the top. From this you will understand why the center of the height of the parallels never increases in `ab' nor diminishes in the bent spring at `co'.''} Moreover, the striking sketch on Codex Atlanticus f.222r, entitled \emph{Testing The Strength of Iron wires of Various Lengths}, shown in figure \ref{fg:Atl222r}, is the first known record on strength of materials. 
\begin{figure}[!ht]
	\centering
	\begin{center}
		\scalebox{.70}{\includegraphics{partes/figs/Atl222r.jpg}}
	\end{center}
	\titfigura{Codex Atlanticus f.222r.}\label{fg:Atl222r}
\end{figure}
The drawing on the figure represents a test scheme of an iron wire with length `ab' and a given thickness. The wire suspends an initially empty basket `q' which is slowly loaded with sand by a hopper `c'. When the wire breaks, a spring closes the hopper and the basket falls a short distance into a hole, so as to not drop the sand. The sand in the basket is then weighted to obtain the strength of the wire. Still on this figure, an excerpt of the text reads: \emph{``The object of this test is to find the load an iron wire can carry. Attach an iron wire 2 braccia long to something which will firmly support it, then attach a basket or similar container to the wire and feed into the basket some fine sand through a small hole placed at the end of the hopper. A spring is fixed so that it will close the hole as soon as the wire breaks. The basket is not upset while falling, since it falls through a very short distance. The weight of sand and the location of the fracture of the wire are to be recorded. The test is repeated several times to check the results. Then a wire of 1/2 the previous length is tested and the additional weight it carries is recorded; then a wire of 1/4 length is tested and so forth, noting the ultimate strength and the location of the fracture.\footnote{See \aut{Lund \& Byrne}\cite{lund_2000_1}, p.3.}''}       


Concerning the subject of Mechanics, which is our interest here, some scholars on the field contest the alleged scientific value of da Vinci's works: the most aggressive criticism is given by \aut{Truesdell}\cite{truesdell_1968}, which doubts, in his typical verbosity, whether da Vinci's sketches and annotations on engineering are really his creation or merely reproduce common technical knowledge of his time. Moreover, Truesdell argues that da Vinci's proposed physical laws, all of them linear, conceived intuitively from simple rules of three, are mostly wrong and the right ones just happened to be correct, since the laws in Physics are either linear or nonlinear. According to \aut{Dugas}\cite{dugas_1988_1}, da Vinci \emph{``...cuts the figure of a gifted amateur... He tackled all kinds of problem, often with more faith than success. Frequently he returned to the same problem by very different paths, and did not scruple to contradict himself.''} Therefore, from today's perspective, it is not possible to state that da Vinci's work on Mechanics directly anticipated relevant definitions or concepts on the field, despite his undeniable outstanding efforts as a curious and creative individual. But even if da Vinci's contributions were scientifically substantial, they would not serve as reference for the study of subsequent professional scholars because none of his notebooks was published in his time or shortly after his death.  


Unlike Leonardo da Vinci, his fellow countryman Galileo Galilei (1564-1642) received a formal education and, at the age of sixteen, enrolled the university of his hometown, Pisa, to study Medicine. Among other disciplines, the course required knowledge on Philosophy, provided by professors Francesco Buonamici and Girolamo Borro, as well as Mathematics and Astronomy, given by Father Filippo Fantoni, a monk. Both Buonamici and Borro were strict aristotelians, which meant that their teachings conflicted with the Christian dogmas of creationism, afterlife, immortality of the soul and others. Some biographers believe that this teachings intensified the impetuous spirit of Galileo, which would cause him trouble with fellow catholic academicians and also with the Inquisition years later. At a certain moment in 1583, Galileo attended lectures on Euclid's \emph{Elements} given by the former mathematician Ostilio Ricci, instructor at the Medici court. Amazed by the performance of the student, Ricci managed to convince Vincenzo Galilei, Galileo's father, to allow his son to switch from Medicine, which would assure a promising career, to Mathematics and Natural Philosophy, areas that captivated the young Galileo during his studies at the faculty. In 1585, following a not unusual practice among noble youngsters of his time, Galileu dropped out of the university, without a degree, in order to get a job: at Florence and Siena, he started private teaching his most interested subjects as a preparation for an eventual post of Mathematics professor on some important university. Meanwhile he kept attending the lectures of Ricci and also studying the work of other eminent mathematicians such as Giovanni Battista Benedetti, Christoph Clavius and Guidobaldo del Monte. In 1588, Galileo was invited to lecture at the Florentine Academy about the location and dimensions of the hell in Dante's \emph{Inferno}\footnote{See \aut{Wallace}\cite{wallace_1998_1}.}. The favorable impressions that these lectures caused on the tuscan nobility and the quitting of Father Fantoni, as well as recommendations from professor Clavius and other mathematicians, enable Galileo to get a chair on Mathematics at the University of Pisa in 1589. After three years of professorship in Pisa, Galileo was convinced by the fellow Medicine professor Girolamo Mercuriale to apply for a vacant chair on Mathematics at the University of Padua, which paid three times more than Pisa. Starting in December 1592, he worked at Padua for eighteen years and then returned to Florence as a mathematician of the Medici court.    

During his Padua years, which he considered the happiest of his life, Galileo developed his most notable works, particularly on Kinematics and Astronomy. He also endeavored to develop studies on strength of materials, which would be published in \emph{Dialogs Concerning Two New Sciences} forty years later. From this book, whose dialogs between characters Salviati, Sagredo and Simplicio occur during four days, it is noteworthy for our purposes to present the  attempt of Galileo to calculate the strength of cantilever beams. On the second day, Salviati proposes a solution to the problem depicted on the right in figure \ref{fg:galileo}. 
\begin{figure}[!ht]
	\centering
	\begin{center}
		\scalebox{.72}{\includegraphics{partes/figs/galileo.jpg}}
	\end{center}
	\titfigura{Beams in tension and bending (\aut{Galileo}\cite{galileo_1954_2}).}\label{fg:galileo}
\end{figure}
He says to his interlocutors that ``\emph{the \emph{momento} of the force applied at C bears to the \emph{momento} of the resistance, found in the thickness of the prism, i. e., in the attachment of the base BA to its contiguous parts, the same ratio which the length CB bears to half the length BA; if we now define absolute resistance to fracture as that offered to a longitudinal pull -- \emph{[drawing on the left in previous figure]}... then it follows that the absolute resistance of the prism BD is to the breaking load placed at the end of the lever BC in the same radio as the length BC is to the half of AB, in the case of a prism, or the semidiameter in the case of a cylinder}.'' In other words, if we call $h$ the thickness of the prism, $\tau.h.AB$ his ``absolute resistance to fracture'' and $E_u$ his ``breaking load'', then Salviati states that 
\begin{eqnarray*}
\dfrac{\tau.h.AB}{E_u}=\dfrac{BC}{AB/2}&\text{or}&E_u=\dfrac{\tau.h.AB^2}{2.BC}\,.
\end{eqnarray*}
But if Simplicio and Sagredo were more argumentative and cautious, they would check this categoric statement and would see that Salviati's proposal is dangerously wrong for the case of beams made of steel: the ultimate load for rectangular cross sectional cantilever steel beams on pure bending is actually three times less\footnote{See \aut{Popov}\cite{popov_1990_1}, p.286.} than $E_u$. When Salviati specifies a \emph{momento} from a load with an arm $AB/2$, he inadvertently defines that the load distribution on section $AB$ is uniform. The reasons that made Galileo arrive at this conclusion are subject of debate: \aut{Higdon et al}\cite{higdon_1981_3} speculate that he might have observed the failure of bending beams made of stone. On brittle materials like this, the shape of the fracture on axial loads and pure bending are very similar, which probably induced him to consider an axial resistance on the bending beam. This problem of the strength of a cantilever beam subjected to pure bending is commonly known as \textsb{Galileo's Problem}\index{Galileo!Problem}\footnote{See \aut{Benvenuto}\cite{benvenuto_1991}, p.177.}. 

    