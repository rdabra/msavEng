\chapter{Introduction to Continuum Mechanics}\label{ch:Collect}


This chapter starts with some early historical aspects of Continuum Mechanics in order to situate the reader on the subject, which is a fundamental part of the science of Mechanics as a whole. By history of Continuum Mechanics we mean the direct contributions to the study of motion, strength and deformation of non rigid bodies; in other words, contributions to the mechanics of deformable bodies. Here, it is sufficient to rely on intuitive notions of all these concepts, which will be rigorously defined in the following chapters. Since our description is focused on non rigid bodies, works that contributed chiefly to the mechanics of material points will not be covered; otherwise, our history would be very, very long. Bla, bla, bla... 


\section{Early History}


An adequate historical account of the main contributors to the subject of Continuum Mechanics must start with the painter Leonardo da Vinci (1452-1519). Born in the city of Vinci, a comune of Florence, in the italian region of Tuscany, da Vinci became an apprentice, at the age of fourteen, in Andrea del Verrocchio's workshop, the most famous florentine artist of the time. Already informally educated on Latin, Geometry and Mathematics, at the workshop, he was taught, besides artistic abilities, a wide range of technical skills, including engineering, architecture, metallurgy and chemistry. Among his notable works, he left not only famous paintings like \emph{Mona Lisa} and \emph{The Last Supper}, but also notebooks, whose descriptive annotations and sketches cover a great variety of themes, from prosaic issues of his everyday life to exercises on Mathematics, architectural drawings, studies on painting and sculpture, engineering, cartography, astronomy, optics, botany, human anatomy, motions of fluids and solids, machine design and so on. Although it is not our purpose to criticize or profoundly analyze da Vinci's work, we must inform that in his notebooks there can be found the earliest known records on the mechanics of deformable bodies: a study on tensile test in notebook Codex Atlanticus f.222r; studies on beams and columns in notebooks Codices Atlanticus f.562r, 908r, Forster I  f.88v, 89r, Madrid I f.84v, 85v, 135v, 136r, 139r, 177v and Paris Manuscript A f.45v. 
\begin{figure}[!ht]
	\centering
	\begin{center}
		\scalebox{.80}{\includegraphics{partes/figs/Madrid84vAtl908r.jpg}}
	\end{center}
	\titfigura{Paris A f.45v, Atlanticus f.908r, Madrid I f.84v (right), 177v (bottom).}\label{fg:Madrid84vAtl908r}
\end{figure}
On figure \ref{fg:Madrid84vAtl908r}, some of these records are shown: sketches Paris A f.45v and Madrid I f.177v are studies on buckling of columns and the others are studies on bending of beams. In particular, the text on Madrid I f.84v, translated by \aut{Zammattio}\cite{zammattio_1980}, reads: \emph{``If a straight spring is bent, it is necessary that its convex part become thinner and its concave part, thicker. This modification is pyramidal, and consequently, there will never be a change in the middle of the spring. You shall discover, if you consider all of the aforementioned modifications, that by taking part `ab' in the middle of its length and then bending the spring in a way that the two parallel lines, `a' and `b' touch a the bottom, the distance between the parallel lines has grown as much at the top as it has diminished at the bottom. Therefore, the center of its height has become much like a balance for the sides. And the ends of those lines draw as close at the bottom as much as they draw away at the top. From this you will understand why the center of the height of the parallels never increases in `ab' nor diminishes in the bent spring at `co'.''} Moreover, the striking sketch on Codex Atlanticus f.222r, entitled \emph{Testing The Strength of Iron wires of Various Lengths}, shown in figure \ref{fg:Atl222r}, is the first known record on strength of materials. 
\begin{figure}[!ht]
	\centering
	\begin{center}
		\scalebox{.70}{\includegraphics{partes/figs/Atl222r.jpg}}
	\end{center}
	\titfigura{Codex Atlanticus f.222r.}\label{fg:Atl222r}
\end{figure}
The drawing on the figure represents a test scheme of an iron wire with length `ab' and a given thickness. The wire suspends an initially empty basket `q' which is slowly loaded with sand by a hopper `c'. When the wire breaks, a spring closes the hopper and the basket falls a short distance into a hole, so as to not drop the sand. The sand in the basket is then weighted to obtain the strength of the wire. Still on this figure, an excerpt of the text reads: \emph{``The object of this test is to find the load an iron wire can carry. Attach an iron wire 2 braccia long to something which will firmly support it, then attach a basket or similar container to the wire and feed into the basket some fine sand through a small hole placed at the end of the hopper. A spring is fixed so that it will close the hole as soon as the wire breaks. The basket is not upset while falling, since it falls through a very short distance. The weight of sand and the location of the fracture of the wire are to be recorded. The test is repeated several times to check the results. Then a wire of 1/2 the previous length is tested and the additional weight it carries is recorded; then a wire of 1/4 length is tested and so forth, noting the ultimate strength and the location of the fracture.\footnote{\aut{Lund \& Byrne}\cite{lund_2000_1}, p.3.}''}       


Concerning the subject of Mechanics, which is our interest here, some scholars on the field contest the alleged scientific value of da Vinci's works: the most aggressive criticism is given by \aut{Truesdell}\cite{truesdell_1968}, which doubts, in his typical verbosity, whether da Vinci's sketches and annotations on engineering are really his creation or merely reproduce common technical knowledge of his time. Moreover, Truesdell argues that da Vinci's proposed physical laws, all of them linear, conceived intuitively from simple rules of three, are mostly wrong and the right ones just happened to be correct, since the laws in Physics are either linear or nonlinear. According to \aut{Dugas}\cite{dugas_1988_1}, da Vinci \emph{``...cuts the figure of a gifted amateur... He tackled all kinds of problem, often with more faith than success. Frequently he returned to the same problem by very different paths, and did not scruple to contradict himself.''} Therefore, from today's perspective, it is not possible to state that da Vinci's work on Mechanics directly anticipated relevant definitions or concepts on the field, despite his undeniable outstanding efforts as a curious and creative individual. But even if da Vinci's contributions were scientifically substantial, they would not serve as reference for the study of subsequent professional scholars because none of his notebooks was published in his time or shortly after his death.  


Unlike Leonardo da Vinci, his fellow countryman Galileo Galilei (1564-1642) received a formal education and, at the age of sixteen, enrolled the university of his hometown, Pisa, to study Medicine. Among other disciplines, the course required knowledge on Philosophy, provided by professors Francesco Buonamici and Girolamo Borro, as well as Mathematics and Astronomy, given by Father Filippo Fantoni, a monk. Both Buonamici and Borro were strict aristotelians, which meant that their teachings conflicted with the Christian dogmas of creationism, afterlife, immortality of the soul and others. Some biographers believe that this teachings intensified the impetuous spirit of Galileo, which would cause him trouble with fellow catholic academicians and also with the Inquisition years later. At a certain moment in 1583, Galileo attended lectures on Euclid's \emph{Elements} given by the former mathematician Ostilio Ricci, instructor at the Medici court. Amazed by the performance of the student, Ricci managed to convince Vincenzo Galilei, Galileo's father, to allow his son to switch from Medicine, which would assure a promising career, to Mathematics and Natural Philosophy, areas that captivated the young Galileo during his studies at the faculty. In 1585, following a not unusual practice among noble youngsters of his time, Galileu dropped out of the university, without a degree, in order to get a job: at Florence and Siena, he started private teaching his most interested subjects as a preparation for an eventual post of Mathematics professor on some important university. Meanwhile he kept attending the lectures of Ricci and also studying the work of other eminent mathematicians such as Giovanni Battista Benedetti, Christoph Clavius and Guidobaldo del Monte. In 1588, Galileo was invited to lecture at the Florentine Academy about the location and dimensions of the hell in Dante's \emph{Inferno}\footnote{See \aut{Wallace}\cite{wallace_1998_1}.}. The favorable impressions that these lectures caused on the tuscan nobility and the quitting of Father Fantoni, as well as recommendations from professor Clavius and other mathematicians, enable Galileo to get a chair on Mathematics at the University of Pisa in 1589. After three years of professorship in Pisa, Galileo was convinced by the fellow Medicine professor Girolamo Mercuriale to apply for a vacant chair on Mathematics at the University of Padua, which paid three times more than Pisa. Starting in December 1592, he worked at Padua for eighteen years and then returned to Florence as a mathematician of the Medici court.    

During his Padua years, which he considered the happiest of his life, Galileo developed his most notable works, particularly on Kinematics and Astronomy. He also endeavored to develop studies on strength of materials, which would be published in \emph{Dialogs Concerning Two New Sciences} forty years later. From this book, whose dialogs between characters Salviati, Sagredo and Simplicio occur during four days, it is noteworthy for our purposes to present the  attempt of Galileo to calculate the strength of cantilever beams. On the second day, Salviati proposes a solution to the problem depicted on the right in figure \ref{fg:galileo}. 
\begin{figure}[!ht]
	\centering
	\begin{center}
		\scalebox{.72}{\includegraphics{partes/figs/galileo.jpg}}
	\end{center}
	\titfigura{Beams in tension and bending (\aut{Galileo}\cite{galileo_1954_2}).}\label{fg:galileo}
\end{figure}
He says to his interlocutors that ``\emph{the \emph{momento} of the force applied at C bears to the \emph{momento} of the resistance, found in the thickness of the prism, i. e., in the attachment of the base BA to its contiguous parts, the same ratio which the length CB bears to half the length BA; if we now define absolute resistance to fracture as that offered to a longitudinal pull -- \emph{[drawing on the left in previous figure]}... then it follows that the absolute resistance of the prism BD is to the breaking load placed at the end of the lever BC in the same radio as the length BC is to the half of AB, in the case of a prism, or the semidiameter in the case of a cylinder}.\footnote{\aut{Galileo}\cite{galileo_1954_2}, p.115.}'' In other words, if we call $h$ the thickness of the prism, $\tau_u.h.AB$ his ``absolute resistance to fracture'' and $E_u$ his ``breaking load'', then Salviati states that 
\begin{eqnarray*}
\dfrac{\tau_u.h.AB}{E_u}=\dfrac{BC}{AB/2}&\text{or}&E_u=\dfrac{\tau_u.h.AB^2}{2.BC}\,.
\end{eqnarray*}
But if Simplicio and Sagredo were more argumentative and cautious, they would verify this categoric statement and would see that Salviati's proposal is dangerously wrong for the case of beams made of steel: the ultimate load for rectangular cross sectional cantilever steel beams on pure bending is actually three times less\footnote{See \aut{Popov}\cite{popov_1990_1}, p.286.} than $E_u$. When Salviati specifies a \emph{momento} from a load with an arm $AB/2$ that balances the \emph{momento} of the lever BC caused by $E$, he inadvertently defines that the tensile load distribution on section $AB$ is uniform and that there is a longitudinal compressive load concentrated at fulcrum $B$ in order to balance this tensile load distribution. The reasons that made Galileo arrive at this conclusion are subject of debate: \aut{Higdon et al}\cite{higdon_1981_3} speculate that he might have observed the failure of bending beams made of stone. On brittle materials like this, the shape of the fracture caused by axial loads and pure bending are very similar, which probably induced him to wrongly consider an axial resistance on the bending beam. Since Galileo was the first to formally address the problem of the strength of a cantilever beam subjected to pure bending, it is commonly known as \textsb{Galileo's Problem}\index{Galileo!Problem}\footnote{See \aut{Benvenuto}\cite{benvenuto_1991}, p.177.}. 

The French scholar Edm\'e Mariotte (?-1684) also contributed to the mechanics of deformable bodies, but he is best known for his work on the properties of air, entitled \emph{Discourse on the Nature of Air}, published in 1676, in which he proposed the notorious inverse relation between volume and pressure: today, we call it the Boyle-Mariotte Law for gases. The early life of Mariotte is completely unknown and the first registered document of his existence is a letter he sent from Dijon in 1668 to the dutch physicist Christiaan Huygens, where he announced the discovery of the blind spot in human eye. It is also unclear how the Paris Academy of Sciences became acquainted with Mariotte's works on plant physiology, but the excellent impression that he caused on the academicians when invited to go to Paris in order to present his theories and experiments soon enabled his engagement at the Academy on 27 July 1667, as a \emph{physicien}. Following a usual practice of the scholars of his time, Mariotte wrote articles on a great variety of subjects: Medicine, Mathematics, Mechanics, Astronomy, hydrology, musical theory, plant biology, among others. His body of published work is extensive and the most important texts are the following: four essays, gathered under the title \emph{Essays on Pshyics}, of which the already mentioned \emph{Discourse on the Nature of Air} is part; a \emph{Treatise on the Motion of Water and Other Fluid Bodies}, published unfinished and postumously in 1686, where Mariotte studies natural springs, artificial fountains and the flow of water through pipes; and a \emph{Treatise on the Collision or Shock of Bodies}, first published in 1673, which covers the topic of elastic and inelastic collisions and shows Mariotte's abilities as a gifted experimenter. It is important to account that Mariotte's work heavily relied on the subjects commonly studied and produced by others at the Paris Academy and, apart from the volume-pressure relation, there is no relevant discoveries attributed to him. However, since rigorous and exhaustive experimentation is the fundamental basis of his efforts, some authors recognize him as the man who introduced the experimental physics into France. In corollary 6 of his third law of motion, \aut{Newton}\cite{newton_1999_1} cites Mariotte's book on collisions: ``\emph{But Wren additionally proved the truth of these rules before the Royal Society by means of an experiment with pendulums, which the eminent Mariotte soon after thought worthy to be made the subject of a whole book.}''
  

In part V of his \emph{Treatise on the Motion of Water and Other Fluid Bodies}, called \emph{On Water Motion and Pipe Strength}, Mariotte undertakes the study of Galileo's Problem in order to correctly design the dimensions of water pipes because he verified experimentally that Galileo's proposition of an axial strength on the bending beam was incorrect for the case of iron and wood. On the discourse II of this same part V, he starts by specifying that the axial strength of a beam will be measured not by a maximum load (absolute resistance to fracture), but by a maximum extension: the beam performs a certain extension in order to sustain a certain load; there is then a maximum extension over which the beam breaks. This is the first known quantitative consideration of deformation in the study of the resistance of deformable bodies. Still in part V, discourse II, Mariotte also describes in literal form a load-extension relation: ``\emph{... if a solid of wood needs to extend two lines to break, and a weight of 500 pounds make this extension, a weight of 125 pounds make it extend about half a line, 250 pounds, about one line, etc. Thus, each extension will balance with a certain weight.}\footnote{\aut{Benvenuto}\cite{benvenuto_1991}, p.266.}''. The drawing on the left in figure \ref{fg:mariotte} is a model conceived by Mariotte to study the behavior of the material fibers of a cantilever beam using identical chords $DI$, $GL$ and $HM$ subjected to tension by a lever $N$ with a given fulcrum $C$.     
\begin{figure}[!ht]
	\centering
	\begin{center}
		\scalebox{.72}{\includegraphics{partes/figs/mariotte.jpg}}
	\end{center}
	\titfigura{Extension of chords and Galileo's Problem (\aut{Mariotte}\cite{mariotte_1740_1}, pp.353-355).}\label{fg:mariotte}
\end{figure}
The rectangle $ACQP$ represents the part of the cantilever, shown on the right in this same figure, that is inside the wall and $AC=2.EC=4.BC$. Mariotte then measured the extensions $\epsilon_{DI}$, $\epsilon_{GL}$ and $\epsilon_{HM}$  of the chords until rupture for different applied loads $R$ and observed the relation $\epsilon_{DI}=2.\epsilon_{GL}=4.\epsilon_{HM}$, which is the same for the loads on the chords, assuming his load-extension law. For an infinite number of chords, the load distribution on face $AC$ decreases linearly to zero from $A$ to $C$ and therefore the resultant load acts at a distance $AC/3$ from $A$. Now considering the drawing on the right in figure \ref{fg:mariotte}, from his extensive experiments, Mariotte came to the conclusion that the beam fibers on the face $AD$ under the middle point $I$ are in compression and the fibers over $I$ are subjected to tension. These fibers in tension behave just like the chords of his model, that is, their extension decreases linearly to zero, from $A$ to $I$, and the resultant load\footnote{Points $G$ and $H$  do not refer to $AI/3$ and $DI/3$ but to another development by \aut{Mariotte}\cite{mariotte_1740_1}.} is at $AI/3$ from $A$. For the compressed fibers, Mariotte considered the same triangular distribution from his model of chords, applied to compressive loads, and thus the resultant compressive load is at $DI/3$ from $D$. He then considers that ``\emph{these extensions and these compressions will share the force of the weight $L$: adding one third of the thickness $IA$ to the third of the thickness $ID$, the whole will be equal to one third of the whole thickness $AD$; from which the same thing will follow as if all the parts extended.}\footnote{\aut{Benvenuto}\cite{benvenuto_1991}, p.268.}'' Since Mariotte did not express this vague reasoning in a literal or mathematical expression for the breaking load, it is at least improper to attribute to him the incorrect triangular distribution of loads on face $AD$, as is usually done. Despite the imprecision of Mariotte's proposition, we dare to interpret it as follows: the resistance of the extended fibers corresponds to half of the total resistance of the beam. In this context, isolating the upper part $AI$, we must consider a compressive longitudinal load concentrated on $I$ in order to balance the tensile loads on face $AI$. Thereby, let the beam be a prism, just like the case of Galileo's Problem, $h$ its thickness, $\tau_u.h.AD$ is Galileo's absolute resistance to fracture, $L_u$ the breaking load. The balance of \emph{momenta} relative to fulcrum I is  
\begin{eqnarray*}
\frac{L_u}{2}.DC = \frac{\tau_u.h.AD}{2} \dfrac{2.AI}{3}&\text{or}&L_u=\dfrac{\tau_u.h.AD^2}{3.DC}\,,
\end{eqnarray*}
which results in a breaking load two times greater than the ultimate load for rectangular cross sectional cantilever steel beams subjected to pure bending. 

Regarding his load-extension law, described previously, it is improbable that Mariotte knew the \emph{Lectures \emph{de Potentia Restitutiva} Or of Spring}, written by the member of the Royal Society of London Robert Hooke (1635-1703) and published in 1678, eight years before Mariotte's treatise on fluids. In this book, Hooke's approach on material deformation is broader than his French contemporary's: a linear load-extension law is presented as part of a general constitutive property of ``springing bodies'', as classified by him to express the behavior of deformed bodies that recover their initial shape after load removal, a feature currently known as elasticity. But before exploring this work, let's learn a little about its author's life. Hooke was born at the village of Freshwater, a peninsula on the west of the English Island of Wight, on July 18 and baptized eight days later by his own father, John Hooke, minister of that remote Anglican parish. Because of a weak constitution and recurrent illnesses until the age of seven, the family doubted Hooke would survive beyond childhood. After this period, but still suffering from frequent headaches, the boy did not show interest on the religious studies oriented by his father and preferred to spend most of hist time building mechanical toys. After his father's death in 1648, when he inherited a small sum of money, there are no records of Hooke's life until the age of twenty, when he went to live at the house of Richard Busby, the headmaster of the Westminster school. There, he became proficient on Latin and Greek, as well as Hebrew and other oriental languages. Under Busby orientation, he mastered Euclid's Elements by himself, ``\emph{and thence proceeded orderly from that sure Basis to the other parts of the Mathematicks, and after to the application thereof to Mechanicks, his first and last Mistress.}\footnote{\aut{Hooke}\cite{hooke_1705_1}, p.iii.}'' In 1653, he became a student of the Christ-Church, a constituent college of the University of Oxford, and received the degree of Master of Arts in 1662 or 1663. During this period at the college, he first worked as assistant to the medicine professor Dr. Thomas Willis and then to the eminent natural philosopher Dr. Robert Boyle from 1655 to 1662. In order to drive a pneumatic engine designed by Boyle, which enabled the publication of the famous Boyle's Law on gases (the same as Mariotte's) in 1662, Hooke contrived and perfected an air pump. In this same period, he attempted designs of structures attached to the human body in order to enable flying, but concluded that it was impossible because human muscles were not strong enough for the task. He took active part on the meetings with the group of scholars that would soon found the Royal Society of London in 1660. Already member of the Royal Society and famous for his incomparable skills as a mechanical experimenter, Hooke was nominated the curator of experiments of the Society in November 1661 and could take up his post with the blessings of Boyle, who released him. On the beginning, Hooke was in charge of providing every weekly meeting of the Royal Society with some new experiment or practical presentation on Mechanics, his expertise. However, since specialization at that time was not a virtue, Hooke published in 1665 the first relevant book on biological microscopic observations called \emph{Micrographia}, to which he built his own microscope and where the term ``cell'' was coined to refer to small biological structures. The following year, after the great fire of London, Hooke was appointed one of the surveyors to rebuild the city and was heavily demanded as an architect, which assured him a substantial financial income for the following ten years. During this period, the most productive of his career, Hooke kept working at the Society, giving lectures and publishing papers: the aforementioned \emph{Lectures \emph{de Potentia Restitutiva} Or of Spring} is one of them. 



In a very pragmatic style, Hooke does not waste time with preambles and at the very beginning of these lectures on springs, he starts describing the famous load-extension law that would bare his name until today: ``\emph{About two years since I printed this Theory in an Anagram at the end of my Book of the Descriptions of Helioscopes, \emph{viz, ceiiinosssttuu, id est, Ut tensio sic vis}; That is, the Power [force] of any Spring is in the same proportion with the [ex]Tension thereof: that is, if one power [force] stretch or bend it one space, two will bend it two, and three will bend it three, and so forward... And this is the Rule or Law of Nature, upon which all manner of Restituent or Springing motion does proceed... It is very evident that the Rule or Law of Nature in every springing body is, that the force or power thereof to restore itself to its natural position is always proportionate to the Distance or space it is removed therefrom...\footnote{\aut{Hooke}\cite{hooke_1678_1}, pp.1,2,4.}}'' In mathematical words, for each pair of applied load $F_i$ and extension $x_i$, value $k=F_i/x_i$ is a constant that is constitutive of the spring. Thereby, \textsb{Hooke's Law}\index{Law!Hooke's} can be described as the following: in other to attain an arbitrary extension $x$ of a spring with constant $k$, the applied force needs to be 
\begin{equation}
F=kx\,,
\end{equation}
and thus the restitutive force of the spring is obviously $-kx$. It is common to assume that when the spring is stretched, $x$ is positive; when compressed, it is negative. In the cited paragraph, Hooke also states that the load-extension relation of every springing body (elastic beam) is always linear, but this is incorrect since not all elastic bodies behave linearly. Therefore, in the context of elasticity, a body is called \textsb{Hookean}\index{body!Hookean} when it observes Hooke's Law, that is, when it deforms linearly. Figure \ref{fg:hooke} illustrates the two springs built by Hooke to obtain the results for his Law.  
\begin{figure}[!ht]
	\centering
	\begin{center}
		\scalebox{.72}{\includegraphics{partes/figs/hooke.jpg}}
	\end{center}
	\titfigura{Springs built by \aut{Hooke}\cite{hooke_1678_1} for his Law.}\label{fg:hooke}
\end{figure}

Concerning Galileo's Problem, Hooke did not apply his theory of springing bodies to solve it, but only to describe a ``compound way of springing\footnote{\aut{Hooke}\cite{hooke_1678_1}, p. 15.}'', resulting from the curvature of the deformed cantilever, where its inner elastic fibers are compressed and the external are stretched. Speaking of curvature, the mathematical description of the deflection or curvature of bending beams was first addressed by the swiss mathematician Jakob Bernoulli (1654-1705) on an article\footnote{See \aut{Truesdell}\cite{truesdell_1960}, p. 89.} in 1694, a study he called the problem of the \emph{curvatura laminae elasticae} (elastic band curvature), or simply \textsb{\emph{elastica}}\index{elastica}. Born in the city of Basel, Jakob was the eldest son of pharmacist Nikolaus Bernoulli and Margaretha Sch\"onauer, daughter of a banker. Nikolaus's father, also Jakob Bernoulli, was a merchant and emigrated from Amsterdam to Basel, where he became a swiss citizen through marriage. Jakob received his master of arts in philosophy in 1671 and, compelled by his father, graduated in theology in 1676, both at the University of Basel; but, to his parents dismay, during these university years, Jacob became strongly interested in Mathematics and astronomy. In the same year of 1676, working as a tutor of nobles, he traveled to Geneva and then to France, where he lived for two years and could study the works of Descartes and his followers. From 1681 to 1682, Jakob resumed his travels, but this time to contact scholars and mathematicians in Holland and England, where he met Robert Boyle and Robert Hooke. Back to Basel, during 1683, he started giving lectures on Mechanics and also publishing articles on Geometry and Algebra in the prominent scientific periodicals \emph{Journal des S\c{c}avans} and the \emph{Acta Eruditorum}. In 1684, Jakob married Judith Stupanus, daughter of a pharmacist, with whom he had two children. He kept working and publishing on Mathematics, as well as exchanging an intense scientific epistolary correspondence with acquaintances he made during his travels, including Leibniz, Huygens, L'Hospital, Varignon and others. In this period, his brother Johann Bernoulli (1667-1748) was studying medicine at the university, also compelled by Nikolaus, but Johann secretly started studying Mathematics under the orientation of Jakob. After the appointment of Jakob as professor of Mathematics at the University of Basel in 1687, both brothers started studying the differential calculus of Leibniz and adopted his differential notation for derivatives, that is, they became ``Leibnizians'', as opposed to british ``Newtonians'', who worked with Newton's fluxions and fluents, concepts that embodied an infinitely small number $o$. In a paper published in the 1690 edition of \emph{Acta Eruditorum}, concerning the solutions of the problem of the catenary curve proposed by Huygens and Leibniz, Jakob coined the term ``integral'' in the sense we currently use. Among his works, the main contributions deal with problems and propositions on Calculus, Probability and Series. According to \aut{Maugin}\cite{maugin_2014}, Jakob ``\emph{may have been less creative than John [Johann], but nonetheless played an essential role in the dissemination of integral calculus ..., in the establishment of the theory of probabilities, and in solving critical problems in mechanics (the isochronous curve, the \emph{elastica}).}''  

From 1690 to 1695, Jakob decided to study the so called flexible lines; mainly the catenary, the isochronous curve and the \emph{elastica}. Our interest here is his contribution on the deflection or curvature of the \emph{elastica}, which is an elastic line attached to a support by one of its ends and deformed by its own weight or by a load applied on the other end. Jakob had referred to this problem of the \emph{elastica}, proposed by Leibniz in private letters, in an article published 1691, announcing that he would present a solution the following year\footnote{See \aut{Truesdell}\cite{truesdell_1960}, p. 88.}; but it took three years until \emph{Curvatura Laminae Elasticae, \&c.} was finally submitted to \emph{Acta Eruditorum}. Jakob starts this article by saying that ``\emph{After a silence of three years I keep my word; but in such a way as right richly to compensate for that delay, which else the reader might have borne with annoyance, since I exhibit the curvature of springs not in one way only (as I had promised in the beginning) but generally for any hypothesis on the elongations; which, unless I err, I am the first to achieve, after the problem was attempted in vain by many.}\footnote{\aut{Jakob Bernoulli}\cite{jakob_1694}, pp. 262-263, translated from the Latin by \aut{Truesdell}\cite{truesdell_1960}.}'' In order to understand what he did, we reproduce on figure \ref{fg:jakob} one of the diagrams presented in the article.     
\begin{figure}[!ht]
	\centering
	\begin{center}
		\scalebox{.72}{\includegraphics{partes/figs/jakob.jpg}}
	\end{center}
	\titfigura{Study on the \emph{curva elastica} by \aut{Jakob Bernoulli}\cite{jakob_1694}.}\label{fg:jakob}
\end{figure}
Fixed to the support $SRYX$, he draws a beam $SVAR$ with uniform rectangular section, the inner fiber of which is the \emph{elastica} to be studied. The beam sustains a weight Z, whose chord is perpendicular to tangent $au$ on $A$. The curve $AC$ represents the stretching force-extension relation at an arbitrary point $y$, where the force, on the abscissa, increases from $A$ to $B$ and the extension increases from $B$ to $C$. And that's it: with the help of the shape and proportions of the geometry presented, the reader is supposed to understand and accept that the curvature $\kappa:=1/Qn$ at point $Q$ is directly proportional to distance $QP$. The following year to this article, Jakob submitted another to the same periodical, making things clearer: ``\emph{I consider a lever with fulcrum $Q$, in which the thickness $Qy$ of the band forms the shorter arm, the part of the curve $AQ$ the longer. Since $Qy$ and the attached weight $Z$ remain the same, it is clear that the force stretching the filament at $y$ ... is proportional to the segment $QP$.}\footnote{\aut{Jakob Bernoulli}\cite{jakob_1695}, p. 538, translated from the Latin by \aut{Truesdell}\cite{truesdell_1960}.}'' In other words, he assumes that the moment $Z.QP$ is balanced by $Qy.F$, where $F$ represents the force that stretches $y$ an extension $e$. Since $Qy$ and $Z$ are constants, $F$ is directly proportional to $QP$; but since $F$ is directly proportional to $e$, according to his stretching force-extension relation, then $e$ is directly proportional to $QP$; but since $e$ is inversely proportional to $Qn$, then $k$ is directly proportional to $QP$. Mathematically, equality $Qy.F=Z.QP$ implies that       




