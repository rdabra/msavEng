\chapter*{Synopsis of ``Mathematical Foundations of Elementary Continuum Mechanics''}

The main purpose of the book is to present the fundamental theoretical aspects of Continuum Mechanics in a comprehensive mathematical approach. In order to attain this goal, the study is divided in two parts. The first one presents the mathematical elements, starting with the basics of Algebra -- Abstract and Linear -- and then advancing to Tensor Algebra. After that, geometrical concepts are included in this algebraic context by presenting topics of Affine Geometry. Then, fundamental and advanced topics on the Calculus of Tensor Functions are studied. The second part deals with the mechanical introductory aspects of continua, using all the strong mathematical background presented in the first part. The study in this second part starts by introducing the subject of Continuum Mechanics, and then develops the Kinematics and Dynamics of continua. Finally, the subject of Constitutive Equations is exemplified by introducing Elasticity Theory. The book strives to be mathematically rigorous and as self-contained as possible, requiring the reader to be skilled on the basics of Linear Algebra and Calculus. There are many figures and a few tables.

\vspace*{3\baselineskip}


\noindent \textsb{Bras�lia, January 2018}

\noindent \textsb{Roberto Dias Algarte}

%\noindent \textsb{Bras�lia, Brazil}


