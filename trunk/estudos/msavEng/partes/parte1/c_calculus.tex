
\chapter{Basic Tensor Calculus}

Analysis is the branch of Mathematics that studies the behaviour of functions by using the idea of limit, considered fundamental, and its subordinate concepts. When such concepts involve  differentiation and integration, as well as their related definitions, Analysis is commonly known as Calculus. Intuitively speaking, up to this point, we have dealt with functions in a kind of ``static'' approach: on previous chapters, the sole definition of a mapping, with its associated function explicitly described or not, was sufficient for developing the concepts presented. Now, we are interested also in a ``dynamic'' approach for functions, where it is important not only to specify them but also to describe how their values evolve on their images, or how they behave, through derivatives and integrals. In this context, considering the study of tensors started on chapter \ref{cp:tensor}, the following pages introduce the Calculus of tensor functions, usually called Tensor Calculus. Since a tensor function is the most general type of function presented so far, we do not deviate from our main objective of being as abstract as possible on behalf of mathematical beauty.  



\section{Differentiation}

In the context of Banach tensor spaces, which are normed and complete, to differentiate a tensor function $\tnr{\psi}$ is to obtain its derivative, here understood as a tensor function $\tnr{\mathit{d}\psi}$ that measures qualitatively and quantitatively the sensitivity of $\tnr{\psi}$ on each element of the domain. Intuitively speaking, sensitivity refers to the intrinsic ``susceptibility'' of a function, in terms of its values, to an \emph{eventual change} on its argument. However, not all tensor functions can be differentiable: the following mathematical conditions and definitions will describe which functions can. Considering $\ete{\cam{F}}{U^{\times m}}$ and $\ete{\cam{F}}{V^{\times s}}$ Banach tensor spaces, let $\pmb{\mathcal{U}}\subset\ete{\cam{F}}{U^{\times m}}$ be an open set and element  $\tnr{H}\in\pmb{\mathcal{U}}$ an arbitrary tensor. Functions in $\map{\tnr{\psi}}{\pmb{\mathcal{U}}}{\ete{\cam{F}}{V^{\times s}}}$ and $\map{\tnr{\kappa}}{\pmb{\mathcal{U}}}{\ete{\cam{F}}{V^{\times s}}}$ are said to be \textsb{tangent}\index{function!tangent} to each other on $\tnr{H}$ if
\begin{equation}
\lim_{\|\tnr{X}\|\to 0} \dfrac{\fua{\tnr{\psi}}{\tnr{H}+\tnr{X}}-\fua{\tnr{\kappa}}{\tnr{H}+\tnr{X}}}{\|\tnr{X}\|}=\tnr{0}\,,
\end{equation} 
In other words, as the norm $\|\tnr{X}\|$, defined for the tensor space $\ete{\cam{F}}{U^{\times m}}$, tends to zero, equality $\fua{\tnr{\psi}}{\tnr{H}}=\fua{\tnr{\kappa}}{\tnr{H}}$ must be true. From this definition, it is possible to obtain that if two functions are tangent to $\tnr{\psi}$, they are tangent to each other.   

    