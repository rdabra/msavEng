
\chapter{Basic Tensor Calculus}

Analysis is the branch of Mathematics that studies the behaviour of functions by using the idea of limit, considered fundamental, and its subordinate concepts. When such concepts involve  differentiation and integration, as well as their related definitions, Analysis is commonly known as Calculus. Intuitively speaking, up to this point, we have dealt with functions in a kind of ``static'' approach: on previous chapters, the sole definition of a mapping, with its associated function explicitly described or not, was sufficient for developing the concepts presented. Now, we are interested also in a ``dynamic'' approach for functions, where it is important not only to specify them but also to describe how their values evolve on their images, or how they behave, through derivatives and integrals. In this context, considering the study of tensors started on chapter \ref{cp:tensor}, the following pages introduce the Calculus of tensor functions, usually called Tensor Calculus. Since a tensor function is the most general type of function presented so far, we do not deviate from our main objective of being as abstract as possible on behalf of mathematical beauty.  



\section{Differentiation}

In the context of Banach tensor spaces, which are normed and complete, to differentiate a tensor function $\tnr{\psi}$ is to obtain its derivative, here understood as a linear tensor function that measures qualitatively and quantitatively the sensitivity of $\tnr{\psi}$ on each element of the domain. Intuitively speaking, sensitivity refers to the intrinsic ``susceptibility'' of a function, in terms of its values, to an \emph{eventual change} on its argument. However, not all tensor functions can be differentiable: the following mathematical conditions and definitions will describe which functions can. Considering $\ete{\cam{F}}{U^{\times m}}$ and $\ete{\cam{F}}{V^{\times s}}$ Banach tensor spaces, both defined by the same field $\cam{F}$, let $\pmb{\mathcal{U}}\subset\ete{\cam{F}}{U^{\times m}}$ be an open set and element  $\tnr{X}_0\in\pmb{\mathcal{U}}$ an arbitrary tensor. Functions in $\map{\tnr{\psi}}{\pmb{\mathcal{U}}}{\ete{\cam{F}}{V^{\times s}}}$ and $\map{\tnr{\kappa}}{\pmb{\mathcal{U}}}{\ete{\cam{F}}{V^{\times s}}}$ are said to be \textsb{tangent}\index{function!tangent} to each other at $\tnr{X}_0$ if
\begin{equation}\label{eq:tangent}
\lim_{\tnr{X}\to \tnr{0}} \dfrac{\fua{\tnr{\psi}}{\tnr{X}_0+\tnr{X}}-\fua{\tnr{\kappa}}{\tnr{X}_0+\tnr{X}}}{\|\tnr{X}\|}=\tnr{0}\,,
\end{equation} 
In other words, as the norm $\|\tnr{X}\|$, defined for the tensor space $\ete{\cam{F}}{U^{\times m}}$, tends to zero, equality $\fua{\tnr{\psi}}{\tnr{X}_0}=\fua{\tnr{\kappa}}{\tnr{X}_0}$ must be true. From this definition, it is possible to obtain that if two functions are tangent to $\tnr{\psi}$ at $\tnr{X}_0$, they are tangent to each other at this tensor.

{\footnotesize
\begin{proof}
Since the sum of limits is the limit of sums, the previous statement can be verified by considering functions $\tnr{\kappa}_1$ and $\tnr{\kappa}_2$ tangent to $\tnr{\psi}$ at $\tnr{X}_0$ and subtracting both limit expressions described by \eqref{eq:tangent} : 
\begin{align*}
\tnr{0}&=\lim_{\tnr{X}\to \tnr{0}} \dfrac{\fua{\tnr{\psi}}{\tnr{X}_0+\tnr{X}}-\fua{\tnr{\kappa}_1}{\tnr{X}_0+\tnr{X}}-\fua{\tnr{\psi}}{\tnr{X}_0+\tnr{X}}+\fua{\tnr{\kappa}_2}{\tnr{X}_0+\tnr{X}}}{\|\tnr{X}\|}\\
&=\lim_{\tnr{X}\to \tnr{0}} \dfrac{\fua{\tnr{\kappa}_2}{\tnr{X}_0+\tnr{X}}-\fua{\tnr{\kappa}_1}{\tnr{X}_0+\tnr{X}}}{\|\tnr{X}\|}\,.
\end{align*} 
\end{proof}}

Proceeding with the above conditions, if there is a function $\tnr{\delta}$, tangent to $\tnr{\psi}$ at $\tnr{X}_0$, described by the rule
\begin{equation}
\fua{\tnr{\delta}}{\tnr{X}}=\fua{\tnr{\psi}}{\tnr{X}_0}+\fua{[\fua{\tnr{\psi}'}{\tnr{X}_0}]}{\tnr{X}-\tnr{X}_0}\,,
\end{equation}
where function $\fua{\tnr{\psi}'}{\tnr{X}_0}\in\evl{\cam{F}}{\pmb{\mathcal{U}}}{\ete{\cam{F}}{V^{\times s}}}$ is bounded, then $\tnr{\psi}$ is called \textsb{Fr�chet differentiable}\index{function! Fr�chet differentiable} or simply \textsb{differentiable}\index{function!differentiable} at $\tnr{X}_0$ while the linear function $\fua{\tnr{\psi}'}{\tnr{X}_0}$ is called the \textsb{derivative}\index{derivative} of $\tnr{\psi}$ at $\tnr{X}_0$. Moreover, the function in 
$\map{\tnr{\psi}'}{\pmb{\mathcal{U}}}{\evl{\cam{F}}{\pmb{\mathcal{U}}}{\ete{\cam{F}}{V^{\times s}}}}$ is said to be the \textsb{derivative function}\index{derivative!function} of $\tnr{\psi}$. Rewriting expression \eqref{eq:tangent} for $\tnr{\psi}$ and $\tnr{\delta}$ tangent at $\tnr{X}_0$, the result is
\begin{equation}
\lim_{\tnr{X}\to \tnr{0}} \dfrac{\fua{\tnr{\psi}}{\tnr{X}_0+\tnr{X}}-\fua{\tnr{\psi}}{\tnr{X}_0}-\fua{[\fua{\tnr{\psi}'}{\tnr{X}_0}]}{\tnr{X}}}{\|\tnr{X}\|}=\tnr{0}\,,
\end{equation} 
from which we can conclude that the numerator tends to the zero tensor faster than the denominator tends to zero, otherwise there would be no definite limit. If numerator tends to the zero tensor, the term $\fua{[\fua{\tnr{\psi}'}{\tnr{X}_0}]}{\tnr{X}}$ results a linear approximation for the difference $\fua{\tnr{\psi}}{\tnr{X}_0+\tnr{X}}-\fua{\tnr{\psi}}{\tnr{X}_0}$ and then the tensor function in $\map{r}{\pmb{\mathcal{U}}}{\ete{\cam{F}}{V^{\times s}}}$, where
\begin{equation}
\fua{r}{\tnr{X}}=\fua{\tnr{\psi}}{\tnr{X}_0+\tnr{X}}-\fua{\tnr{\psi}}{\tnr{X}_0}-\fua{[\fua{\tnr{\psi}'}{\tnr{X}_0}]}{\tnr{X}}\,,
\end{equation} 
can be called the \textsb{residue}\index{function!residue} of $\tnr{\psi}$ at $\tnr{X}_0$. Thereby, it is true that $\lim_{\tnr{X}\to \tnr{0}}\fua{r}{\tnr{X}}=\tnr{0}$


that We know that this residue is increasingly small when as tensor $\|\tnr{X}\|$ tends to zero or  



%LEMBRAR DE DEMONSTRAR A UNICIDADE DA FUNCAO 

% como todo espa�o tensorial � finitamente dimensional ent�o a diferen�a entre F-derivada e G-derivada reside somente na lineridade da derivada. A G-derivada adimite operador n�o linear, enquanto que a F-derivada exige.


% ap�s definir frechet derivative, mostra a fracao cl�ssica que obtem o frechet derivative. a partir dessa fracao, defina gateaux derivative. Colocar nota de rodap� explicando que se os espacos envolvidos n�o fossem dimensionalmente finitos as diferencas entre Gateaux e Frechet seriam maiores. 
    