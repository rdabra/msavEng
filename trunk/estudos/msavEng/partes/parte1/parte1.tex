%\pagenumbering{arabic}
{\let\newpage\relax \part{Mathematical Foundations}\label{par:matematica}}

\begin{comment}
\noindent\emph{Na
abordagem matem�tica que baseia a Teoria da Elasticidade, h� uma
grande variedade de metodologias e nota��es. Didaticamente, �
conveniente de\-fi\-nir\--se pre\-li\-mi\-nar\-men\-te, de forma
clara, os fun\-da\-men\-tos ma\-te\-m�\-ti\-cos essenciais a serem
utilizados: suas restri��es e suas formas. Para fins de clareza e
facilidade de compreens�o, esta parte centraliza todas as
defini��es e conceitos matem�ticos utilizados ao longo do texto.
Em resumo, apresenta-se os conceitos elementares da �lgebra
Abstrata, os fundamentos da �lgebra Linear e aspectos geom�tricos
envolvidos, �lgebra e C�lculo Tensorial. O escopo e o n�vel de
profundidade dos temas tratados abrange o necess�rio. Os t�picos
apresentados podem ser aprofundados consultando-se a bibliografia
utilizada, apresentada ao final desta parte. Ao leitor j�
familiarizado com os temas tratados, � recomend�vel passar
rapidamente pelos cap�tulos, ficando a par do enfoque e nota��o utilizados.}
\end{comment}
