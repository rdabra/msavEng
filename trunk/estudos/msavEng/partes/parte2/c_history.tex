\chapter{Presenting Continuum Mechanics}\label{ch:Collect}


This chapter starts with some historical aspects of Continuum Mechanics in order to situate the reader on the subject, which is a fundamental part of the science of Mechanics as a whole. By history of Continuum Mechanics we mean the direct contributions to the study of motion, strength and deformation of non rigid bodies; in other words, contributions to the mechanics of deformable bodies. Here, it is sufficient to rely on intuitive notions of all these concepts, which will be rigorously defined in the following chapters. Since our description is focused on non rigid bodies, works that contributed solely to the mechanics of material points will not be covered; otherwise, our history would be very, very long. Bla, bla, bla... 


\section{History}


An adequate historical account of the key contributors on the subject of Continuum Mechanics must start with the painter Leonardo Da Vinci (1452-1519). Born in the city of Vinci, a comune of Florence, in the italian region of Tuscany, Da Vinci became an apprentice, at the age of fourteen, in Andrea del Verrocchio's workshop, the most famous florentine artist of the time. Already informally educated on Latin, Geometry and Mathematics, at the workshop, he was taught, besides artistic abilities, a wide range of technical skills, including engineering, architecture, metallurgy and chemistry. Alem das famosas pinturas, Da vinci deixou tambem seus nao menos famosos notebooks.....

Ha muito debate entre estudiosos, célebres ou não, sobre o valor científico das anotacoes e desenhos de Da Vinci em seus notebooks. De qualquer forma, despite the ferocious criticism of \aut{Truesdell}\cite{truesdell_1968}, which claims, in his typical verbosity, that Da Vinci's sketches and annotations on Mechanics merely reproduce common technical knowledge of his time, it is important to inform the independent reader that the italian painter, in his notebook (CA, 82v-b), presents a study entitled ``Testing The Strength of Iron wires of Various Lengths'' and falar de barras.     