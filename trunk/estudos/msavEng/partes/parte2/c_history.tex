\chapter{Presenting Continuum Mechanics}\label{ch:Collect}


bla bla bla bla 


\section{History}

An adequate historical account of the key contributors on the subject of Continuum Mechanics must start with the painter Leonardo Da Vinci (1452-1519). Born in the city of Vinci, a comune of Florence, in the italian region of Tuscany, Da Vinci became an apprentice, at the age of fourteen, in Andrea del Verrocchio's workshop, the most famous florentine artist of the time. Already informally educated on Latin, Geometry and Mathematics, at the workshop, he was taught, besides artistic abilities, a wide range of technical skills, including engineering, metallurgy and chemistry.    Falar dos notebooks

Despite the ferocious criticism of \aut{Truesdell}\cite{truesdell_1968}, which claims that Da Vinci's sketches and annotations on Mechanics merely reproduce common knowledge of his time, it is important to inform the reader who does not accept subjective opinions as dogmas, including Truesdell's, that in notebook (CA, 82v-b) the italian painter presents a study entitled ``Testing The Strength of Iron wires of Various Lengths'' and falar de barras.     