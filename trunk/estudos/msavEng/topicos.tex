1. Na parte de equa�oes constitutivas falar de homogeneidade e isotropia
2. Ver no capitulo 2 como inserir o conceito de espa�o topol�gico antes dos conceitos de espa�os m�tricos.
3. Remover do conceito de paralelismo os subespa�os que ficam um dentro do outro e aquela propriedade que torna os espa�os puntuais iguais. Remover tamb�m as representa��es.
4. Colocar o quinto postulado de Euclides na parte de espa�os m�tricos para que ele possa ser definido em cima de um espa�o vetorial euclidinano. Tentar mostrar que o espa�o euclidiano promove que dada os pontos a diferente de b e um espa�o veto U, os espa�os puntuais gerados por a e b a partir de U ou s�o iguais ou s�o paralelos.






Meus Pensamentos

1. A necessidade humana de entender o mundo � mais intensa que a de model�-lo.

2. 
